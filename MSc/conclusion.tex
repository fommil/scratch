\section{Conclusions}
\begin{quotation}
	\raggedleft \it The universe is like a safe to which there is a combination, but the combination is locked up in the safe.\\
	-- Peter de Vries
\end{quotation}
We have shown that the motion of stars S0-1, S0-2 and S0-4, in the vicinity of the galactic centre, may be fitted by either a black hole
or fermion ball scenario, and that these two scenarios have nearly identical $\chi^2$ values. However, further observations should allow
discrimination between the two scenarios within a timeframe no longer than 40 years.

The phase space plots of star position and velocity components in our line of sight, reveal distinctive
regions which favour a particular scenario. We present the phase space analysis of these parameters as the best
technique to discriminate between the two scenarios.
A measurement of either $z$ or $v_z$ would dramatically reduce the required observational time to under a decade.
A measurement of both parameters could result in instant discrimination.

We have also shown that a simple Newtonian, optically thick, geometrically thin black hole accretion model cannot explain the observed
cut-off around $10^{13}$Hz in the spectrum of Sgr A*.
This model is however, known to be incorrectly implememted in the case of our galactic centre and a more accurate model, such as
an Advection Dominated Accretion Flow model would reveal that the black hole scenario does indeed predict the IR cut-off.
Following the same modelling technique, a fermion ball scenario explains the cut-off quite adequately and also
encourages material to be deposited at the centre, where there are no shear forces. This would be an excellent `breeding ground'
for new stars, and therefore could explain the dominance of young stars in that area.
