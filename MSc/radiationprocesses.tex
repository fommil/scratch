\section{Radiation Processes}
\label{app_opticallythickapprox}
An (unpolarised) radiation field is defined by the {\it specific intensity} ($I_\nu$), which gives the flux of energy per second per unit area
per solid angle per unit frequency interval
\begin{eqnarray*}
	dE=I_\nu dA \cos\theta d\nu d\Omega dt
\end{eqnarray*}
The {\it radiation flux} ($F_\nu$) is the rate at which energy crosses a unit area independent of direction and is obtained by integrating
over a solid angle
\begin{eqnarray*}
	F_\nu=\int I_\nu \cos\theta d\Omega
\end{eqnarray*}
The {\it specific luminosity} ($L_\nu$) of a source is the flux integrated over an area enclosing the source
\begin{eqnarray*}
	L_\nu = \int F_\nu dA
\end{eqnarray*}
the variation of $I_\nu$ in a medium which is emitting and absorbing is given by the {\it radiative transfer equation} which simply expresses
energy conservation for the radiation field. In the time independent case in a direction ${\bf n}$
\begin{eqnarray*}
	{\bf n}\cdot \nabla I_\nu = -\mu_\nu I_\nu + j_\nu
\end{eqnarray*}
where $\mu_\nu$ is the absorption coefficient and $j_\nu$ the emission coefficient, effectively the properties and state of the medium. It is
customary to characterise the medium by the {\it source function} ($S_\nu$)
\begin{eqnarray*}
	S_\nu = \frac{j_\nu}{\mu_\nu}
\end{eqnarray*}
The {\it optical depth} ($\tau_\nu$) along a path from source to observer is defined by
\begin{eqnarray*}
	d\tau_\nu=\mu_\nu ds
\end{eqnarray*}
Giving total specific intensity over an optical depth
\begin{eqnarray*}
	\frac{dI_\nu}{d\tau_\nu} = -I_\nu + S_\nu
\end{eqnarray*}
If the radiation field corresponds to thermal equilibrium at a temperature $T$, we know that $I_\nu=B_\nu(T)$, the black body spectrum.
Irrespective of the radiation mechanism, we can relate
\begin{eqnarray*}
	S_\nu=B_\nu(T)=\frac{2h\nu^3}{c^2\left(e^{\frac{h\nu}{kT}}-1\right)}
\end{eqnarray*}
If the medium can be characterised by a temperature $T$ (i.e. thermal emission) then we have Kirchhoff's law, independent of whether the
radiation field is Planckian.
\begin{eqnarray*}
	S_\nu = \frac{j_\nu}{\mu_\nu} = B_\nu(T)
\end{eqnarray*}
In general the source function is determined by the state populations in the medium. A necessary condition for a thermal radiation spectrum
is that the optical depth $\tau_\nu \rightarrow \infty$. In this case we say that the medium is {\it optically thick}, which in general,
means that $I_\nu=S_\nu$. On the other hand if $\tau_\nu \rightarrow 0$ we can neglect absorption and we get
\begin{eqnarray*}
	I_\nu=\int j_\nu ds
\end{eqnarray*}
Such a medium is said to be {\it optically thin}. Returning to the optically thick case, once we have a source function, we make the
following relations (as explained) to give us the observables $F_\nu$ or $L_\nu$, depending upon the preference of the experimentalist.
\begin{eqnarray*}
	I_\nu &=& S_\nu = \frac{2h\nu^3}{c^2\left(e^{\frac{h\nu}{KT}}-1\right)} \\
	F_\nu &=& \frac{4\pi h \nu^3\cos{i}}{c^2D^2} \int\frac{R}{e^{\frac{h\nu}{KT}}-1}dR \\
	L_\nu &=& 4\pi D^2 F_\nu = \frac{16\pi^2 h \nu^3\cos{i}}{c^2} \int\frac{R}{e^{\frac{h\nu}{KT}}-1}dR
\end{eqnarray*}

In the text, we have assumed an optically thick medium so that we may simply use a black body as our radiative mechanism. The absorption
coefficient is $\mu_\nu \propto \kappa \rho$, which decreases for a low density gas, and thus optical depth decreases.
$\kappa$ is the opacity, and is calculated from the radiative transfer equation
\begin{eqnarray*}
	\kappa(\nu ,T)=\frac{c^2}{8\pi \nu^2}A_{21}\frac{g_2}{g_1}n_1\left[1-e^{\frac{h\nu_0}{kT}}\right]\phi(\nu)
\end{eqnarray*}
As temperatures are generally low in the fermion ball case, the gas will not be ionised and therefore $\kappa$ will also aid in
decreasing the optical depth. The gas density is low under both scenarios.

This goes to show that an optically thin approximation may be more appropriate for the fermion ball scenario.
In an optically thin accretion disk, a different radiative mechanism must be used as the black body approximation is
no longer appropriate, such a valid mechanism could be a thermal bremsstrahlung spectrum.
