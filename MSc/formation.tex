\section{Formation of a Fermion Ball}
\begin{quotation}
	\raggedleft \it
	We are all agreed that your theory is crazy.\\
	The question which divides us is whether it is\\
	crazy enough to have a chance of being correct.\\
	-- Niels Bohr
\end{quotation}
Thus far, we have argued that a fermion ball is possible, provided a 16keV fermion exists in nature. Candidates for such a fermion
and their origins have also been briefly mentioned. Another requirement of the theory is that such a fermion ball may be formed. This
section aims to show that it is possible to create such an object by closely following the work of \cite{ref_formation}.

\subsection{Analogy to a Boson Star}
To understand the formation of a fermion ball, we first study the analysis of self-interacting scalar fields, often called boson stars
\cite{ref_bosonform}.
We then use scaling arguments to make an equivalence to a certain type of cold boson star.

Consider a complex scalar field $\Psi$ with a repulsive Lagrangian interaction term $U(|\Psi|^2)$. By introducing the dimensionless
parameter $\Lambda$, we define the length and mass scales as
\begin{equation}
	R_*=\frac{\Lambda}{m}=\frac{\lambda m_{pl}}{m^2} \qquad \qquad M_*=R_*m^2_{pl}
	\label{eqn_formscales}
\end{equation}
where $m_{pl}=\sqrt{G}^{-1}$ is the Planck mass. In the Newtonian limit, a self-gravitating boson star is governed by the
dimensionless Gross-Pitaevskii like equations
\begin{eqnarray}
	\Delta \phi &=& |\Psi|^2 \label{eqn_formpoisson} \\
	\frac{i}{\Lambda} \frac{\partial \Psi}{\partial t} &=& \left[ -\frac{\Delta}{2\Lambda^2} + \phi + V(|\Psi|^2) \right]
		\Psi \label{eqn_formbig} \\
	\rho &=& \frac{m^4}{4\pi \lambda^2} |\Psi|^2 \label{eqn_formrho}
\end{eqnarray}
where we have introduced the dimensionless potential
\begin{equation}
	V(|\Psi|^2) = \frac{4\pi \lambda^2}{m^4}\frac{dU}{d|\Psi|^2}
	\label{eqn_formv}
\end{equation}
The pressure tensor in these dimensionless units is given by
\begin{equation}
	P_{ij}= \frac{m^4}{4\pi\lambda^2 \Lambda^2}
		\left( \rm{Re}\frac{\partial\Psi}{\partial x_i} \frac{\partial\Psi^*}{\partial x_j}
			-\delta_{ij}\frac{\Delta|\Psi|^2}{4} \right)
		+\delta_{ij}\left(|\Psi|^2\frac{d U}{d |\Psi|^2} -U \right)
	\label{eqn_formpressuretensor}
\end{equation}
A static, spherically symmetric solution is obtained with the ansatz
\begin{equation}
	\Psi = e^{- i \epsilon R_* t} \Phi(r)
	\label{eqn_formpsiphi}
\end{equation}
where $\Phi(r)$ is a real function. It is clear that for large $\Lambda$, (\ref{eqn_formbig}) and (\ref{eqn_formpsiphi}) reduce to
\begin{equation}
	\frac{\epsilon}{m} - \phi - V = 0
	\label{eqn_formereduce}
\end{equation}
This solution is exact in the limit $\Lambda \rightarrow \infty$, which is the Thomas-Fermi limit \cite{ref_bosonformtf}. In this limit,
(\ref{eqn_formpressuretensor}) becomes diagonal with $P = P_{ii}$. We obtain an equation of state given by (\ref{eqn_formrho}) and
\begin{equation}
	P=\rho V(\rho) - U(\rho)
	\label{eqn_formpressure}
\end{equation}
Equations (\ref{eqn_formereduce}) and (\ref{eqn_formpressure}) yield the equation of hydrostatic equilibrium
\begin{equation}
	\frac{dP}{\rho}=-d\phi
	\label{eqn_formpressurephi}
\end{equation}
If the equation of state is given as the polytropic type
\begin{equation}
	P(\rho)=K\rho^{1+\frac{1}{n}}
	\label{eqn_formpolytropic}
\end{equation}
we can determine the potential $U$ and $V$ by integrating (\ref{eqn_formpressurephi}). It can therefore be shown that
Equation (\ref{eqn_formpressurephi}) is equal to $V$, which yields
\begin{equation}
	V=(n+1)K\rho^{\frac{1}{n}}
	\label{eqn_formvrho}
\end{equation}
If we fix $\lambda$ in a convenient manner, we get the potential
\begin{equation}
	V=|\Psi|^{\frac{n}{2}}
	\label{eqn_formpotential}
\end{equation}
The polytropic equation of state with $n=\frac{3}{2}$, together with (\ref{eqn_formpressurephi}) and (\ref{eqn_formpoisson}) (the dimensionless
Poisson equation), describe a fermion ball. This has demonstrated that a fermion ball is equivalent to a boson star in the limit
$\Lambda \rightarrow \infty$. It has been shown numerically \cite{ref_bosonform} that even for moderate values of $\Lambda$, the static
solutions are almost degenerate and are quite well approximated by the static solution for an infinite $\Lambda$.

\subsection{Time Evolution of the Collapse}
The basic regulating mechanism is the kinetic part of (\ref{eqn_formbig}), which penalises density spikes. The criterion for simulation
is that the ratio of kinetic and pressure contributions to the static energy functional should be small. Great care must be taken during
numeric simulation to ensure that the self-interaction term $V$ is not dominated by the other terms. If this occurs, there is a
crossover to mini boson star behaviour.

An evolution plot for collapsing fermionic matter ball is shown in Figure \ref{fig_robert}.
To prevent matter from being artificially reflected by the radius boundary, an $r$-dependant imaginary `sponge' has been introduced.
This sponge removes the ejected fermionic matter.
\begin{figure}[t]
	\begin{center}
	\includegraphics[angle=-90,width=0.9\textwidth]{eps/robert.eps}
	\caption{Combined contour-density plot for the evolution of $|r \Psi |^{2}$. Green contour lines denote levels from
	$10^{-5}$ to $10^{-4}$ while red lines denote levels below $10^{-5}$.
	Gravitational collapse is followed by an ejection of excess matter leaving a fermion ball at the centre
	\cite{ref_formation}.}
	\label{fig_robert}
	\end{center}
\end{figure}

In summary, using a bosonic representation of the dynamical Thomas-Fermi theory for a self-gravitating gas, it can be shown that
non-relativistic, degenerate fermionic matter will form a super-massive fermion ball through gravitational collapse accompanied by ejection.
