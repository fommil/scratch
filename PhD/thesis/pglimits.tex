\chapter{Penrose-G\"uven Limits and Isometric Embeddings}
\label{pglimits}
\section{The Penrose-G\"{u}ven Plane Wave Limit}
In \cite{Penrose1}, Penrose showed that any Lorentzian spacetime has a limiting
spacetime which is a plane wave. This limit can be thought of as a ``first order
approximation'' along a null geodesic. The limiting spacetime depends on the
choice of null geodesic, and hence any spacetime can have more than one Penrose
limit.

Let $(M,G)$ be a $d$-dimensional lorentzian spacetime. We can always introduce
local Penrose coordinates $(U,V,\mbf Y)$, $\mbf Y^\top=(Y^i)\in\R^{d-2}$ in the
neighbourhood of a segment of a null geodesic $\gamma\subset M$ which contains
no conjugate points, whereby the metric assumes the form \cite{LP1}
\begin{equation}
  \label{GPenrose}
  G = \left(2dU + \alpha dV + \beta_i dY^i \right)dV + C_{ij} dY^i dY^j
\end{equation}
where $\alpha$, $\beta_i$ and $C_{ij}$ are functions of the coordinates and
$C_{ij}$ is a symmetric positive-definite matrix. This coordinate system breaks
down when $\det C=0$, signalling the existence of a conjugate point. This
coordinate system has the advantage that a null geodesic congruence is singled
out by constant $V$ and $Y^i$, with $U$ being an affine parameter along
these geodesics. The geodesic $\gamma(U)$ is the one at $V=Y^i=0$.
 
In string backgrounds one also has to consider generic supergravity $p$-form
gauge potentials $A$ with $(p+1)$-form field strengths $F=dA$ in order to
compensate non-trivial spacetime curvature effects. The G\"uven extension of the
Penrose limit to general supergravity fields shows that any supergravity
background has plane wave limits which are also supergravity backgrounds
\cite{Guven1}. It requires the local temporal gauge choice
\begin{equation}
  \label{PGgaugechoice}
  A_{Ui_1\cdots i_{p-1}}=0
\end{equation}
in order to ensure well-defined potentials in the limit. With this
gauge choice, which can always be achieved via a gauge transformation
$A\mapsto A+d\Lambda$ leaving the flux $F$ invariant, we can write
general potentials and field strengths in the neighbourhood of a null
geodesic $\gamma$ on $M$ as
\begin{eqnarray}
  \label{Anull}
  A &=& a_{i_1 \ldots i_p} dV\wedge dY^{i_1}\wedge\ldots\wedge
  dY^{i_{p-1}} \\ \nonumber
  &&+b_{i_1\ldots i_p} dY^{i_1}\wedge\ldots\wedge dY^{i_p}\\ \nonumber
  &&+c_{i_1\ldots i_p} dU\wedge dV\wedge dY^{i_1}\wedge\ldots\wedge
  dY^{i_{p-2}} \\
  \label{Fnull}
  F &=& \left( \frac{\partial b_{i_1 \ldots i_{p+1}}}{\partial U}
  \right) dU\wedge dY^{i_1}\wedge\ldots\wedge dY^{i_{p}} \\ \nonumber
  &&+ d_{i_1 \ldots i_{p+1}} dY^{i_1}\wedge\ldots\wedge dY^{i_{p+1}} \\ \nonumber
  &&+ e_{i_1 \ldots i_{p+1}} dU\wedge dV\wedge dY^{i_1}\wedge\ldots\wedge
  dY^{i_{p-1}} \\ \nonumber
  &&+ f_{i_1 \ldots i_{p+1}} dV\wedge dY^{i_1}\wedge\ldots\wedge dY^{i_{p}}  
\end{eqnarray}
where $a$, $b$, $c$, $d$, $e$ and $f$ are functions of the coordinates. The
Penrose-G\"uven limit starts with the one-parameter family of local
diffeomorphisms $\psi_\lambda:M\to M$, $\lambda\in\R$ defined by a rescaling
of the Penrose coordinates as
\begin{equation}
  \label{psiOmegadef}
  \psi_\lambda(U,V,\mbf Y)=(u,\lambda^2v,\lambda\mbf y)
\end{equation}
One then defines new fields which are related to the original ones by a
diffeomorphism, a rescaling, and (in the case of potentials) possibly a gauge
transformation, by the well-defined limits
\begin{eqnarray}
  \label{PGlimitsdef}
  \widetilde{G}&=&\lim_{\lambda\to0}\lambda^{-2}\psi^*_\lambda G \\\nn
  \widetilde{A}&=&\lim_{\lambda\to0}\lambda^{-p}\psi^*_\lambda A \\\nn
  \widetilde{F}&=&\lim_{\lambda\to0}\lambda^{-p}\psi^*_\lambda F
\end{eqnarray}

Due to \eqref{psiOmegadef}, the only functions in \eqref{GPenrose},
\eqref{Anull} and \eqref{Fnull} which survive this limit are
$C_{ij}(u)=C_{ij}(U,0,\mbf0)$, $b_{i_1\cdots i_p}(u)=b_{i_1\cdots
  i_p}(U,0,\mbf0)$ and $c_{i_1\cdots i_{p-2}}(u)=c_{i_1\cdots
  i_{p-2}}(U,0,\mbf0)$, which are just the pull-backs of the tensor fields $C$,
$b$ and $c$ to the null geodesic $\gamma$. Explicitly, we obtain a pp-wave
metric and supergravity fields in Rosen coordinates $(u,v,\mbf y)$ \cite{Rosen1}
as
\begin{eqnarray}
  \label{GPGlim}
  \widetilde{G}&=& 2dudv + C_{ij}(u)dy^i dy^j \\
  \label{APGlim}
  \widetilde{A}&=& b_{i_1\ldots i_p}(u)dy^{i_1}
  \wedge\ldots\wedge dy^{i_p} \\ \nonumber
  &&+c_{i_1\ldots i_p}(u) du\wedge
  dv\wedge dy^{i_1}\wedge\ldots\wedge dy^{i_{p-2}}\\
  \label{FPGlim}
  \widetilde{F}&=&\frac{\partial b_{i_1\ldots i_{p+1}}(u)}{\partial u}
  du\wedge dy^{i_1}\wedge\ldots\wedge dy^{i_p}
\end{eqnarray}
The physical effect of this limit is to blow up a neighbourhood of the null
geodesic $\gamma$, giving the local background as seen by an observer moving at
the speed of light in $M$. It can be thought of as an infinite volume limit. We
may set $c_{i_1\cdots i_{p-2}}(u)=0$ in \eqref{APGlim} via the local gauge
transformation
\begin{eqnarray}
  \label{APGcset0}
  \widetilde{A} &\longmapsto& \widetilde{A}+d\widetilde{\Lambda}\\\nn
  \widetilde{\Lambda}&=&-\left(\mbox{$\int^u$}d u'~c_{i_1\cdots
      i_{p-2}}(u')\right)~d v\wedge d y^{i_1}\wedge\cdots\wedge
  d y^{i_{p-2}}
\end{eqnarray}

\section{Isometric Embedding Diagrams}
It is possible to generate a commutative isometric embedding diagram, whereby
isometric embeddings are denoted by vertical arrows and Penrose-G\"{u}ven limits
(PGL) by horizontal arrows:
\begin{equation}
  \label{isomembdiag}
  \begin{CD}
    @.\\
    M @>\text{PGL}>>                      \widetilde{M}\\
    \text{$\imath$}@AAA @AAA\text{$\widetilde{\imath}$}\\
    N @>\text{PGL}>>                      \widetilde{N}\\
    @.
  \end{CD}
\end{equation}
In order to ensure that the metric and fields commute in such a diagram (i.e.
that $M\rightarrow \widetilde{M} \rightarrow \widetilde{N}$ yields the same
result as $M\rightarrow N\rightarrow\widetilde{N}$), we must place some
restrictions on the kind of embeddings $\imath$, $\widetilde{\imath}$ we can
use.

We are interested in smooth, local isometric embeddings $\imath:W\subset
N\hookrightarrow M$ of a Lorentzian manifold $(N,g)$, also possibly supported by
non-trivial $p$-form fields, from an open subset $W$ of $N$ onto a sub-manifold
of $M$ in codimension $m\geq0$. Thus we require that $\imath:W\to\imath(W)$ be a
diffeomorphism in the induced ${\rm C}^\infty$ structure, and that the
derivative map $d\imath_x:T_xN\to T_{\imath(x)}M$ be one-to-one for all $x\in
W$. The Lorentzian metric $g$ of $N$ is related to that of $M$ through the
pull-back
\begin{equation}
  \label{gNpullback}
  g_x(\cdot, \cdot)=G_{\imath(x)}(d\imath(\cdot), d\imath(\cdot))
\end{equation}
on $T_xW\otimes T_xW\to\R$, and $p$-form fields $a$ on $N$ are similarly
related to those on $M$ by
\begin{equation}
  \label{aNpullback}
  a_x(\cdot,\ldots,\cdot)=A_{\imath(x)} (d\imath(\cdot),\ldots,d\imath(\cdot))
\end{equation}
on $\otimes^pT_xW\to\R$. In what follows we will usually write $\imath(N)$
for the projection.

We shall examine situations in which the Penrose-G\"uven limit will
simultaneously induce the Penrose-G\"uven limits of the ambient spacetime and of
the embedded submanifold. This automatically restricts the types of possible
embeddings. We will simplify matters somewhat by taking the Penrose-G\"uven
limit of $(N,g)$ along the same null geodesic $\gamma$ as that used on $(M,G)$.
The embedded submanifold $\imath(N)$ is then the intersection of $M$ with the
hypersurface $Y^i=0$, $\forall i\in{\mathcal I}:=\{j_1,\dots,j_m\}$. With the
additional requirement that the metric $G$ restricts non-degenerately on
$\imath(N)$, there is an orthogonal tangent space decomposition
\begin{equation}
  \label{taudecomp}
  T_xM=T_xN\oplus T_xN^\perp
\end{equation}
With $\d_i$ the elements of a local basis of tangent vectors dual to the
one-forms $d Y^i$ in an open neighbourhood of $x\in M$. The fibres of the normal
bundle $TN^\perp\to\imath(N)$ over the embedding are given by
$T_xN^\perp=\{\xi^\perp\in T_xM~|~G(\d_i,\xi^\perp)=0$, $\forall
i\notin{\mathcal I}\}$.

Suppose now that we are given another local isometric embedding
$\widetilde{\imath}:\widetilde{W}\subset\widetilde{N}\hookrightarrow
\widetilde{M}$ of lorentzian manifolds $(\widetilde{N},\widetilde{g})$ and
$(\widetilde{M},\widetilde{G})$, again possibly in the presence of other
supergravity fields. We are interested in the conditions under which the
isometric embedding diagram \eqref{isomembdiag} commutes. Such a commutative
diagram can only be written down under very exact symmetry constraints on the
geodesic restrictions of the transverse plane metric $C_{ij}$ of
\eqref{GPenrose} and supergravity tensor field $b_{i_1\cdots i_p}$ of
\eqref{Anull} in the directions normal to the embeddings $\imath(N)\subset M$
and $\widetilde{\imath}(\widetilde{N})\subset\widetilde{M}$.

To formulate these symmetry requirements, let
$\widetilde{\imath}(\widetilde{N})$ be realised as the intersection of
$\widetilde{M}$ with the hypersurface $y^i=0$, $\forall i\in\widetilde{\mathcal
  I}:=\{\tilde j_1,\dots,\tilde j_m\}$. This realisation of the isometric
embedding on the right-hand side of \eqref{isomembdiag} is dictated by the
embedding on the left-hand side and the Penrose limit.

Consider the submanifold, also denoted $\widetilde{N}$, defined by the
intersection of $M$ with the hypersurface $Y^i=0$, $\forall
i\in\widetilde{\mathcal I}$. Denoting the normal bundle fibres as $T_{\tilde
  x}\widetilde{N}^{\perp}:=\{\tilde\xi^\perp\in T_{\tilde
  x}M~|~G(\d_i,\tilde\xi^\perp)=0$, $\forall i\notin\widetilde{\mathcal I}\}$
for $\tilde x\in\widetilde{W}$, there is an orthogonal tangent space
decomposition
\begin{equation}
  \label{tautildedecomp}
  T_{\tilde x}M=T_{\tilde x}\widetilde{N}\oplus T_{\tilde x}
  \widetilde{N}^{\perp}
\end{equation}
analogous to \eqref{taudecomp}. Along the light-like null geodesic $\gamma$,
where $x=\tilde x=(U,0,\mbf0)$, we fix $p$ tangent vectors
$X,X_1,\dots,X_{p-1}\in T_{(U,0,\mbf0)}M$, and use the Lorentzian metric and
$p$-form gauge potentials to define the linear transformations
\begin{eqnarray}
  \label{Gelinmap}
  G_{(U,0,\mbf0)}(X,\cdot):T_{(U,0,\mbf0)}M&\longrightarrow&
  T_{(U,0,\mbf0)}M \\
  \label{Aelinmap}
  A_{(U,0,\mbf0)}(X_1,\dots,X_{p-1},\cdot):T_{(U,0,\mbf0)}M&
  \longrightarrow&T_{(U,0,\mbf0)}M
\end{eqnarray}

The isometric embedding diagram \eqref{isomembdiag} then commutes if, for every
collection of tangent vectors $X,X_1,\dots,X_{p-1}\in T_{(U,0,\mbf0)}M$, the
restrictions of the linear maps \eqref{Gelinmap} and \eqref{Aelinmap} to the
corresponding orthogonal projections in \eqref{taudecomp} and
\eqref{tautildedecomp} agree
\begin{eqnarray}
  \label{Ccommcond}
  \bigl.C_{(U,0,\mbf0)}(X,\cdot)\bigr|_{T_{(U,0,\mbf0)}N^\perp}&=&
  \bigl.C_{(U,0,\mbf0)}(X,\cdot)\bigr|_{T_{(U,0,\mbf0)}
    \widetilde{N}^{\perp}} \\
  \label{bcommcond}
  \bigl.b_{(U,0,\mbf0)}(X_1,\ldots,X_{p-1},\cdot)
  \bigr|_{T_{(U,0,\mbf0)}N^\perp}&=&
  \bigl.b_{(U,0,\mbf0)}(X_1,\ldots,X_{p-1},\cdot)\bigr|_{T_{(U,0,\mbf0)}
    \widetilde{N}^{\perp}}
\end{eqnarray}
in the following sense. The normal subspaces $T_{(U,0,\mbf0)}N^\perp$ and
$T_{(U,0,\mbf0)}\widetilde{N}^{\perp}$ of $T_{(U,0,\mbf0)}M$ are non-canonically
isomorphic as vector spaces. Fixing one such isomorphism, there is then a
one-to-one correspondence between normal vectors $\xi^\perp\in
T_{(U,0,\mbf0)}N^\perp$ and $\tilde\xi^\perp\in
T_{(U,0,\mbf0)}\widetilde{N}^{\perp}$, under which we require the transverse
plane metric and $p$-form fields to coincide
\begin{equation*}
  C_{(U,0,\mbf0)}(X,\xi^\perp)=C_{(U,0,\mbf0)}(X,\tilde\xi^\perp)  
\end{equation*}
and
\begin{equation*}
  b_{(U,0,\mbf0)}(X_1,\ldots,X_{p-1},\xi^\perp)=b_{(U,0,\mbf0)} (X_1,\ldots,X_{p-1},\tilde\xi^\perp)
\end{equation*}
These symmetry conditions together ensure that the same supergravity fields are
induced on $\widetilde{\imath}(\widetilde{N})$ along the two different paths of
the diagram \eqref{isomembdiag}, i.e. that the Penrose-G\"uven limit of $M$,
along the null geodesic described above, induces simultaneously the
Penrose-G\"uven limit of $N$. We stress that \eqref{Ccommcond} and
\eqref{bcommcond} are required to simultaneously hold under only a single
isomorphism of $m$-dimensional vector spaces, and in all there are
$\frac12m(m+1)$ such commuting isometric embedding diagrams that can potentially
be constructed for appropriate plane wave profiles.

These symmetry conditions are essentially just the simple statement that the
restrictions of the embeddings $\imath$ and $\widetilde{\imath}$ to the null
geodesic $\gamma(U)$ are equivalent. Nevertheless, there are many examples
whereby the Penrose limit of the metric carries through in \eqref{isomembdiag},
but not the G\"uven extension to generic $p$-form supergravity fields, i.e.
\eqref{Ccommcond} can hold with \eqref{bcommcond} being violated. Conversely,
there can be exotic isometric embeddings whereby the transverse metric violates
the requirement \eqref{Ccommcond}, leading to target spacetimes with distinct
pp-wave profiles induced by the Penrose limit of essentially the same lorentzian
structure. An interesting example of this would be a situation wherein the
metric is not preserved, but the other supergravity $p$-form fields are.

\section{Hpp-Wave Limits}
\label{PGLNW}
Let us now specialise the analysis of the previous section to a broad class of
examples that are important to the analysis that follows. Consider a real,
connected Lie group $\mathcal{G}$ possessing a bi-invariant metric. On the Lie
algebra $\mathfrak g$ of $\mathcal{G}$, this induces an invariant,
non-degenerate inner product
$\langle\cdot,\cdot\rangle:\mathfrak{g}\times\mathfrak{g}\to\R$. We
will also write $\mathcal{G}$ for the group manifold.

Symmetric D-branes wrapping submanifolds $D\subset \mathcal{G}$ preserve the
maximal (diagonal) symmetry group
$\mathcal{G}\subset\mathcal{G}\times\overline{\mathcal G}$ allowed by conformal
boundary conditions. They are described algebraically by twisted conjugacy
classes of the group \cite{AS1,FFFS1,FS1,Stanciu2}. Let
$\Omega:\mathfrak{g}\to\mathfrak{g}$ be an outer automorphism of the Lie algebra
of $\mathcal{G}$ preserving its inner product $\langle\cdot,\cdot\rangle$, and
let $\omega:\mathcal{G}\to \mathcal{G}$ be the corresponding Lie group
automorphism. The map $\omega$ is an isometry of the bi-invariant metric on
$\mathcal G$, and so it generates an orbit of any point $g\in \mathcal{G}$ under
the twisted adjoint action of the group as $g\mapsto{\rm
  Ad}_h^\omega(g)=hg\omega(h^{-1})$, $h\in \mathcal{G}$. We may thereby identify
$D$ with such an orbit as
\begin{equation}
  \label{Dtwistconj}
  D={\mathcal C}_g^\omega=\{hg\omega(h^{-1})|~h\in \mathcal{G}\}
\end{equation}
To each such $\omega$ we can associate an equivalence class of D-branes
foliating $\mathcal{G}$.

Since the metric on $\mathcal{G}$ is bi-invariant, the twisted conjugacy class
\eqref{Dtwistconj} may be exhibited as a homogeneous space
\begin{equation}
  \label{twistconjhom}
  {\mathcal C}_g^\omega=\mathcal{G}/{\mathcal Z}_g^\omega
\end{equation}
where ${\mathcal Z}_g^\omega\subset \mathcal{G}$ is the stabiliser subgroup of
the point $g$ under the twisted adjoint action of $\mathcal{G}$ defined by
\begin{equation}
  \label{stabdef}
  {\mathcal Z}_g^\omega=\{h\in \mathcal{G} |~hg\omega(h^{-1})=g\}
\end{equation}
By this homogeneity, it will always suffice to determine the geometry (and any
$\mathcal{G}$-invariant fluxes) at a single point, as all other points are
related by the twisted adjoint action of the group, which is an isometry. A
natural physical assumption is that the bi-invariant metric of $\mathcal{G}$
restricts non-degenerately to the twisted conjugacy classes
\eqref{twistconjhom}. The normal bundle over the D-submanifold then has fibres
given by
\begin{equation}
  \label{twistconjnorm}
  (T_g\mathcal{C}_g^\omega)^\perp=T_g\mathcal{Z}_g^\omega
\end{equation}
with $T_g\mathcal{C}_g^\omega\cap T_g\mathcal{Z}_g^\omega=\{0\}$, so that there
is an orthogonal direct sum decomposition of the tangent bundle of $\mathcal{G}$
as
\begin{equation}
  \label{TgGdecomp}
  T_g\mathcal{G}=T_g\mathcal{C}_g^\omega\oplus T_g\mathcal{Z}_g^\omega
\end{equation}
The space of normal vectors \eqref{twistconjnorm} may be identified with the Lie
algebra of the stabiliser subgroup \eqref{stabdef} given by
\begin{equation}
  \label{stabLiealg}
  \mathfrak{z}_g^\omega=\{X\in\mathfrak{g}|~g^{-1}Xg=\Omega(X)\}
\end{equation}
These D-submanifolds are also NS-supported with $B$-field \cite{Stanciu1}
\begin{equation}
  \label{Bfieldg0}
  B_{g_0}=-\langle d hh^{-1},{\rm Ad}_g\circ\Omega(d hh^{-1})\rangle
\end{equation}
at $g_0=hg\omega(h^{-1})\in{\mathcal C}_g^\omega$, and they are stabilised
against decay by the presence of a family of two-form Abelian gauge field
strengths $F_{g_0}^{(\zeta)}= d A_{g_0}^{(\zeta)}$,
$\zeta\in\mathfrak{z}_g^\omega$ given by \cite{BSR1}
\begin{equation}
  \label{2formfamily}
  F_{g_0}^{(\zeta)}=\langle\zeta,[ d hh^{-1}, d hh^{-1}]\rangle
\end{equation}
If $\zeta$ is a fractional symmetric weight of $\mathcal{G}$, then the integral
of \eqref{2formfamily} over any two-sphere in the worldvolume
$\mathcal{C}_g^\omega$ of the D-brane is an integer multiple of $2\pi$. The
two-forms
\begin{equation}
  \label{inv2forms}
  \mathcal{F}_{g_0}^{(\zeta)}:=B_{g_0}+F_{g_0}^{(\zeta)}
\end{equation}
are invariant under gauge transformations $B_{g_0}\mapsto B_{g_0}+ d\Lambda$,
$F_{g_0}^{(\zeta)}\mapsto F_{g_0}^{(\zeta)}- d\Lambda$ of the
$B$-field. They are the gauge invariant combinations (in string units
$\alpha'=1$) that appear in the Dirac-Born-Infeld action governing the
target space dynamics of the D-branes.

There are two ways in which the Penrose limit may be achieved within this
setting \cite{BFP1}. In both instances it can be understood as an
In\"on\"u-Wigner group contraction of $\mathcal{G}$ whose limit
$\widetilde{\mathcal{G}}$ is a non-compact, non-semisimple Lie group admitting a
bi-invariant metric. In the first case we assume $\mathcal G$ is simple and
consider a one-parameter subgroup $\mathcal{H}\subset\mathcal{G}$, which is
necessarily geodesic relative to the bi-invariant metric. If it is also null,
then it gives rise to a null geodesic and hence the Penrose limit can be taken
as prescribed in the previous subsection. In the second case we consider a
product group $\mathcal{G}=\mathcal{G}'\times\mathcal{H}$, where $\mathcal{G}'$
is a simple Lie group with a bi-invariant metric and $\mathcal{H}\subset
\mathcal{G}'$ is a compact subgroup, with Lie subalgebra
$\mathfrak{h}\subset\mathfrak{g}'$, which inherits a bi-invariant metric from
$\mathcal{G}$ by pull-back. The product group $\mathcal{G}'\times \mathcal{H}$
then carries the bi-invariant product metric corresponding to the bilinear form
$\langle\cdot,\cdot\rangle\oplus (-\langle\cdot,\cdot\rangle|_{\mathfrak{h}})$
on $\mathfrak{g}'\oplus\mathfrak{h}$. The submanifold $\mathcal{H}\subset
\mathcal{H}\times\mathcal{H}\subset\mathcal{G}'\times \mathcal{H}$ given by the
diagonal embedding is a Lie subgroup, and hence it is totally geodesic and
maximally isotropic. The (generalised) Penrose limit of
$\mathcal{G}'\times\mathcal{H}$ along $\mathcal{H}$ thereby yields a
non-semisimple Lie group with a bi-invariant metric. The related group
contraction can also be understood as an infinite volume limit along the compact
directions of $\mathcal{G}'\times \mathcal{H}$, giving a semi-classical picture
of the string dynamics in this background.

We may now attempt to specialise the isometric embedding diagram
\eqref{isomembdiag} to the diagram
\begin{equation}
  \label{twistconjembdiag}
  \begin{CD}
   @.\\
    \mathcal{G} @>\text{PGL}>>                      \widetilde{\mathcal{G}}\\
    \text{$\imath$}@AAA @AAA\text{$\widetilde{\imath}$}\\
    \mathcal{C}_g^\omega @>\text{PGL}>> \mathcal{C}_{\tilde g}^{\tilde\omega}\\
   @.
  \end{CD}
\end{equation}
specifying the Penrose-G\"uven limit between twisted D-branes. We have used the
fact that the Penrose limit of a maximally symmetric space is again a maximally
symmetric space \cite{BFP1}, so that the Penrose-G\"uven isometric embedding
diagrams preserve the symmetry of the embedded spaces and symmetric D-branes map
onto symmetric D-branes in the limit. To formulate the symmetry conditions
\eqref{Ccommcond} and \eqref{bcommcond} within this algebraic setting, we
restrict to the stabiliser algebra \eqref{stabLiealg}. For any $h\in\mathcal{H}$
and any Lie algebra element $X\in\mathfrak{g}$, the isometric embedding diagram
\eqref{twistconjembdiag} then commutes if and only if
\begin{eqnarray}
  \label{innprodtwistcomm}
  \langle X,\cdot\rangle|_{\mathfrak{z}_h^\omega}&=&
  \langle X,\cdot\rangle|_{\mathfrak{z}_h^{\tilde\omega}}\\
  \label{Bfieldtwistcomm}
  B_{h_0}(X,\cdot)|_{\mathfrak{z}_h^\omega}&=&
  B_{h_0}(X,\cdot)|_{\mathfrak{z}_h^{\tilde\omega}}\\
  \label{Ftwistcomm}
  F^{(\zeta)}_{h_0}(X,\cdot)|_{\mathfrak{z}_h^\omega}
  &=&F^{(\zeta)}_{h_0}(X,\cdot)|_{\mathfrak{z}_h^{\tilde\omega}}
\end{eqnarray}
with the analogous restrictions on higher-degree $p$-form fields.

\subsection{$\NW$ Limits}
\label{ApplNW}
We will now apply these considerations to examine the possible D-embeddings of
Nappi-Witten spacetimes. We start with the six-dimensional Lorentzian manifold
$M=\AdS_3\times\Sphere^3$ describing the near horizon geometry of a bound state
of fundamental strings and NS5-branes \cite{GKS1}, equivalent to a
two-dimensional superconformal field theory corresponding to the IR limit of the
dynamics of parallel D1-branes and D5-branes. This identification requires that
both factors share a common radius of curvature $R$. We can embed
$\AdS_3\times\Sphere^3$ in the pseudo-euclidean space $\E^{2,6}$ as the
intersection of the two quadrics
\begin{eqnarray}
  \label{AdS3quadric}
  \left(x^0\right)^2+\left(x^1\right)^2-\left(x^2\right)^2-\left(x^3\right)^2
  &=&R\\
  \label{S3quadric}
  \left(x^4\right)^2+\left(x^5\right)^2+\left(x^6\right)^2+\left(x^7\right)^2
  &=&R
\end{eqnarray}
with the induced metric. An explicit parametrisation is given by
\begin{eqnarray}
  \label{AdS3S3param}
  x^0&=R\sqrt{1+r^2}\cos\tau\cosh\beta \qquad &x^4=R\cos\phi \\\nn
  x^1&=R\sin\tau \qquad\qquad\qquad\qquad &x^5=R\chi\sin\phi\sin\psi \\\nn
  x^2&=R\sqrt{1+r^2}\cos\tau\sinh\beta \qquad
  &x^6=R\sqrt{1-\chi^2}\sin\phi\sin\psi \\\nn
  x^3&=Rr\cos\tau \qquad\qquad\qquad\qquad &x^7=R\sin\phi\cos\psi
\end{eqnarray}
In these coordinates the metric, NS-NS $B$-field and three-form flux on
$\AdS_3\times\Sphere^3$ are given by
\begin{eqnarray}
  \label{AdS3S3metric}
  \frac 1{R^2}G &=& - d\tau^2 + \cos^2{\tau}\left( \frac{dr^2}{1+r^2} +
    \left(1+r^2\right)d\beta^2\right) \\ \nonumber
  &&+d\phi^2+\sin^2{\phi}\left(\frac{d\chi^2}{1-\chi^2}+
    \left(1-\chi^2\right)d\psi^2\right)\\
  \label{AdS3S3flux}
  -\frac1{2R^2}H/2 &=&
  \cos^2{\tau} d\tau\wedge dr\wedge d\beta + \sin^2{\phi}d\phi\wedge
  d\chi\wedge d\psi\\
  \label{AdS3S3Bfield}
  -\frac2{R^2}2B &=& \left(\sin{2\tau}+2\tau\right)dr\wedge d\beta
  +\left(\sin{2\phi}-2\phi\right)d\chi\wedge d\psi
\end{eqnarray}
Viewing $M$ as the group manifold of the Lie group ${\rm SU}(1,1)\times{\rm
  SU}(2)$ with its usual bi-invariant metric induced by the Cartan-Killing form,
its embedded submanifolds wrapped by maximally symmetric D-branes are given by
twisted conjugacy classes of the group. Here we will focus on the family of
D3-branes which are isometric to $N=\AdS_2\times\Sphere^2$ and are given by the
intersections of the hyperboloids \eqref{AdS3quadric} and \eqref{S3quadric} with
the affine hyperplanes $x^3,x^5={\rm constant}$ \cite{BP1}. In this way we may
exhibit a foliation of $\AdS_3\times\S^3$ consisting of twisted D-branes, each
of which is isometric to $\AdS_2\times\S^2$. Within this family, the
intersection of $\AdS_3\times\S^3$ with the hyperplane defined by $x^3=x^5=0$ is
special for a variety of reasons. It corresponds to the fixed point set of the
reflection isometry of $\E^{2,6}$ defined by $x^3\mapsto-x^3$,
$x^5\mapsto-x^5$ while leaving fixed all other coordinates. This isometry
preserves the embedding \eqref{AdS3quadric}, \eqref{S3quadric} and hence induces
an isometry of $\AdS_3\times\S^3$. The metric \eqref{AdS3S3metric} restricts
nondegenerately to the corresponding $\AdS_2\times\S^2$ submanifold, which is
thereby totally geodesic and has equal radii of curvature $R$.

When we consider such embedded D-submanifolds, we should also add to the list of
supergravity fields \eqref{AdS3S3metric} through \eqref{AdS3S3Bfield}, the
constant ${\rm U}(1)$ gauge field flux \cite{BP1,BDS1}
\begin{equation}
  \label{AdS3S3U1flux}
  \frac2{R^2}F=\frac2{R^2}d A=4\pi\kappa_q d r\wedge d\beta-
  2\pi p d\chi\wedge d\psi
\end{equation}
where $p\in\Z$ is the magnetic monopole number through $\S^2$, while
$\kappa_q\in\R$ is related to the quantum number $q\in\Z$ giving the
Dirac-Born-Infeld electric flux through $\AdS_2$. The presence of this two-form
prevents the wrapped D3-branes from collapsing. The quantity ${\mathcal F}=B+F$
is invariant under two-form gauge transformations of the $B$-field, but it is
only the monopole flux and the dual Dirac-Born-Infeld electric displacement
which are quantised. The addition of such worldvolume electric fields
proportional to the volume form of $\AdS_2$ along with worldvolume magnetic
fields proportional to the volume form of $\S^2$ still preserves supersymmetry.
The sources of such fluxes are $(p,q)$ strings connecting the D3-branes to the
NS5/F1 black string background in the near horizon region \cite{BP1}. The
Dirac-Born-Infeld energy of the branes wrapping $\AdS_2\times\S^2$ is locally
minimised at $(\tau,\phi)=(2\pi\kappa_q,\pi p)$ \cite{BP1,BDS1}, and at those
values the gauge-invariant two-form is given by
\begin{equation}
  \label{gaugeinvAdS3S3}
  -\frac2{R^2}\mathcal{F}=\sin2\tau d r\wedge d\beta+\sin2\phi d\chi\wedge d\psi
\end{equation}

Let us now consider the Penrose-G\"uven limit of $M$ which produces the
six-dimen\-sional Nappi-Witten spacetime $\widetilde{M}=\NW_6$ \cite{BFP1}. As
Lie groups, the Penrose limit can be interpreted as an In\"on\"u-Wigner group
contraction of ${\rm SU}(1,1)\times{\rm SU}(2)$ onto $\mathcal{N}_6$ \cite{SF1}.
Geometrically, it can be achieved along any null geodesic which has a
non-vanishing velocity component tangent to the sphere $\S^3$. For this, we
change coordinates in the $(\tau,\phi)$ plane
\begin{eqnarray}
  \label{UVAdS3S3def}
  2\tau=U-V &\qquad& 2\phi=U+V \\ \nonumber
  r=Y^1\quad\beta=Y^2 &\qquad& \chi=Y^3\quad\psi=Y^4
\end{eqnarray}
This enables us to represent the fields in \eqref{AdS3S3metric} through
\eqref{AdS3S3Bfield} in the adapted coordinate forms \eqref{GPenrose},
\eqref{Anull} and \eqref{Fnull}, which thereby exhibits $\frac\partial{\partial
  U}=\frac12(\frac\partial{\partial\phi}+\frac\partial{\partial\tau})$ as the
null geodesic vector field with $G(\frac\partial{\partial
  U},\frac\partial{\partial U})=0$. After the Penrose-G\"uven limit, the metric
and Neveu-Schwarz fields along the geodesic $\gamma(U)$ are given by
\begin{eqnarray}
  \label{NW6tildemetric}
  \frac1{R^2}\widetilde{G}&=&2 dudv+\sin^2\frac u2d{\mbf y}^2\\
  \label{NW6tildeflux}
  \frac1{R^2}\widetilde{H}&=&\cos^2\frac u2du
  \wedge dy^1\wedge dy^2-\sin^2\frac u2du\wedge dy^3\wedge d y^4\\
  \label{NW6tildeBfield}
  \frac4{R^2}\widetilde{B}&=&-(u+\sin u)
  dy^1\wedge dy^2+(u-\sin u)dy^3\wedge dy^4
\end{eqnarray}
with ${\mbf y}^\top=(y^i)\in\R^4$. At this stage it is convenient to transform
from Rosen coordinates to Brinkman coordinates
\begin{equation}
  \label{eq:ex:jose:brinkman}
  u=2x^- \qquad v=x^+ + \frac{\cot{x^-}}{2}\vec{z}^2 \qquad
  y^i=\frac{z^i}{\sin{x^-}}
\end{equation}
It is then straightforward to compute that one recovers the standard
NS-supported geometry of $\NW_6$, with supergravity fields
$\frac1{R^2}\widetilde{G}= d s_6^2$, $\frac1{2R^2}\widetilde{H}=H_6$ and
$\frac2{R^2}\widetilde{B}=B_6$ given by \eqref{NW6metricBrink},
\eqref{NW6Bfield}.

The foliating hyperplanes $w=w_0\in\C$ isometrically embed $\NW_4$ in
$\NW_6$ with its standard geometry \eqref{NW4metricBrink}, \eqref{NS2formBrink}
and zero-point energy $b=-\frac{\theta^2}4|w_0|^2$. The Penrose-G\"uven limit of
the worldvolume flux \eqref{AdS3S3U1flux} vanishes, $\widetilde{F}=0$, because
it is the field strength of a ${\rm U}(1)$ gauge field. On the other hand, the
gauge invariant two-form \eqref{gaugeinvAdS3S3} transforms as a potential under
the Penrose-G\"uven limit and one finds
\begin{equation}
  \label{NW6tildegaugeinv}
  \mathcal{F}_6:=-\frac2{R^2}\widetilde{\mathcal{F}}
  =- i\cot\frac{\theta x^+}2\left[
    d\overline{\mz}^{\top}\wedge d\mz+\frac\theta2\cot\frac{\theta x^+}2
    dx^+\wedge(\mz^\top d\overline{\mz}-\overline{\mz}^{\top}d\mz)\right]
\end{equation}
Strictly speaking, this field is only defined on the constant time slices
$x^+=\frac\pi\theta(2\kappa_q+p)$ of $\NW_6$ induced by the energy minimising
configurations described above. However, with the scaling transformations
employed in the Penrose-G\"uven limit we can take the form
\eqref{NW6tildegaugeinv} to be valid at all times. It induces a worldvolume
two-form $\mathcal{F}_4:=\mathcal{F}_6|_{w=w_0}$ on $\NW_4$.

Let us now examine the isometric embedding of the totally geodesic
$\AdS_2\times\S^2$ D-brane in $\AdS_3\times\S^3$, which may be defined as the
hyperplane $Y^1=Y^3=0$. The geodetic property ensures that the Penrose limit of
$\AdS_3$ $\times$ $\S^3$ induces that of $\AdS_2$ $\times$ $\S^2$, which yields
the Cahen-Wallach symmetric space $\CW_4$ \cite{BFP1}. However, the G\"uven
extension breaks down. In this case the normal bundle fibre
$T_{(U,0,\mbf0)}N^\perp$ is locally spanned by the vector fields
$\partial_2,\partial_4$, while $T_{(U,0,\mbf0)}\widetilde{N}^{\perp}$ is spanned
by $\partial_3,\partial_4$. There is thus no gauge transformation such that
\eqref{bcommcond} is satisfied. On the other hand, since the transverse plane
metric $C_{ij}(u)=\sin^2\frac u2\delta_{ij}$ in \eqref{NW6tildemetric} is
proportional to the identity, \eqref{Ccommcond} is trivially satisfied and so
the Penrose limit of the $\AdS_2\times\S^2$ metric coincides with that of
$\CW_4$.

One cannot rectify this problem by choosing an alternative embedding of $\AdS_2$
$\times$ $\S^2$ for which the $B$-field is non-vanishing, such as that with
$\tau,\phi={\rm constant}$, however, we may still find an interesting $\CW_4$
spacetime by using such an alternative embedding. This case corresponds with the
minimal energy symmetric D3-branes and allows the Penrose limit of
$\AdS_2\times\S^2$ to be taken in the adapted coordinates $U=\chi+r$,
$V=\frac12(\chi-r)$, $Y^1=\beta$, $Y^2=\psi$. After another suitable change to
Brinkman coordinates it leads to the anticipated $\CW_4$ geometry. However, now
the $\CW_4$ branes carry a non-vanishing null worldvolume flux which may be
written in Brinkman coordinates as
\begin{equation}
  \label{tildefluxalt}
  \frac2{R^2}\widetilde{F}=\frac{\pi\theta}2\csc
  \frac{\theta x^+}2\bigl[(2\kappa_q- i p)
  dx^+\wedge dw+(2\kappa_q+ i p) dx^+\wedge d\overline{w}\bigr]
\end{equation}

\subsection{Embedding Diagrams for $\NW$ Spacetimes}
\label{DiagNW}
Let us now describe in more detail the two simple and obvious remedies to the
problem which we raised in the previous Section. The first one modifies the
embedding $\widetilde{\imath}$ on the right-hand side of the diagram
\eqref{isomembdiag} to be the intersection of $\widetilde{M}=\NW_6$ with the
hyperplane $y^1=y^3=0$, so that $T_{(U,0,\mbf0)}N^\perp =
T_{(U,0,\mbf0)}\widetilde{N}^{\perp}$ and the conditions \eqref{Ccommcond},
\eqref{bcommcond} are always trivially satisfied. Now the pull-backs of the
Neveu-Schwarz fields \eqref{NW6tildeflux}, \eqref{NW6tildeBfield} vanish, as
does that of the two-form \eqref{NW6tildegaugeinv}, while the pull-back of the
metric \eqref{NW6tildemetric} is still the standard metric on $\CW_4$. This
embedding thereby preserves the basic Cahen-Wallach structure
$\CW_4\subset\NW_6$, and the vanishing of the other supergravity form fields on
$\CW_4$ is indeed induced now by the Penrose-G\"uven limit from
$\AdS_2\times\S^2$. We may thereby write the commuting embedding diagram
\begin{equation}
  \label{CW4commdiag}
  \begin{CD}
   @.\\
    \AdS_3 \times \S^3             @>\text{PGL}>> \NW_6\\
    \text{$\imath~~$}@AAA @AAA\text{$\widetilde{\imath}$}\\
    \AdS_2 \times \S^2             @>\text{PGL}>> \CW_4\\
   @.
  \end{CD}
\end{equation}
which describes the Penrose-G\"uven limit between {\it commutative}, maximally
symmetric Lorentzian D3-branes in $\AdS_3\times\S^3$ and the six-dimensional
Nappi-Witten spacetime $\NW_6$. We may explicitly confirm that the path
$\AdS_3\times\S^3$ $\rightarrow$ $\AdS_2\times\S^2$ $\rightarrow\CW_4$ gives
\begin{eqnarray}
  \label{eq:bpg:ex:jose:path2:G}
  \widetilde{G}&=&2dx^-dx^+ +\frac{\vec{z}^2}{4}\left(d
    x^-\right)^2 + \left(d\vec{z}\right)^2 \\
  \label{eq:bpg:ex:jose:path2:fields}
  \widetilde{H}&=&d\widetilde{B}=0
\end{eqnarray}

Due to the vanishing of the worldvolume flux and the fact that $\pi_2(\S^3)=0$,
these D-branes can be unstable. They may shrink to zero sise completely
corresponding to point-like D-instantons, to euclidean D-strings induced at a
point in $\S^3$ with worldvolume geometries $\AdS_2\subset\AdS_3$, or to
euclidean D-strings sitting at a point in $\AdS_3$ and wrapping
$\S^2\subset\S^3$ on the left-hand side of the diagram \eqref{CW4commdiag}.
These symmetric decay products will be generically lost in the Penrose-G\"uven
limit, as will become evident in the ensuing sections.

Alternatively, we may choose to modify the embedding $\imath$ on the left-hand
side of the diagram \eqref{isomembdiag} to be the intersection of
$M=\AdS_3\times\S^3$ with the hyperplane $Y^1=Y^2=0$. Again
$T_{(U,0,\mbf0)}N^\perp = T_{(U,0,\mbf0)}\widetilde{N}^{\perp}$ and so the
conditions \eqref{Ccommcond} and \eqref{bcommcond} are trivially satisfied. This
hyperplane corresponds to the intersection of the hyperboloid
\eqref{AdS3quadric} with $x^2=x^3=0$, while \eqref{S3quadric} is left unchanged.
It thereby defines a totally geodesic embedding of $\S^{1,0}\times\S^3$ in
$\AdS_3\times\S^3$. This does not define a twisted conjugacy class of the Lie
group ${\rm SU}(1,1)\times{\rm SU}(2)$ and so does not correspond to a symmetric
D-brane \cite{BP1}. Instead, it arises in the near horizon geometry of a stack
of NS5-branes \cite{GO1}. The pull-backs of the NS--NS fields
\eqref{AdS3S3flux}, \eqref{AdS3S3Bfield} are non-vanishing, and the null
geodesic defined by \eqref{UVAdS3S3def} spins along an equator of the sphere
$\S^3$. The Penrose-G\"uven limit thus induces the complete NS-supported
geometry of $\NW_4$ \cite{DAK1}, and it can be thought of as a group contraction
of ${\rm U}(1)\times{\rm SU}(2)$ along ${\rm U}(1)$ onto $\mathcal{N}_4$. We may
thereby write the commuting embedding diagram
\begin{equation}
  \label{NW4commdiag}
  \begin{CD}
   @.\\
    \AdS_3 \times \S^3             @>\text{PGL}>> \NW_6\\
    \text{$\imath~~$}@AAA @AAA\text{$\widetilde{\imath}$}\\
    \S^{1,0}\times\S^3         @>\text{PGL}>> \NW_4\\
   @.
  \end{CD}
\end{equation}
describing {\it noncommutative} branes. The Lorentzian $\NW_4$ D3-brane,
supported by non-trivial worldvolume fields, is stabilised against decay by its
worldvolume two-form $\mathcal{F}_4=\mathcal{F}_6|_{z=0}$. One may confirm
explicitly that the path $\AdS_3\times\S^3\rightarrow\NW_6\rightarrow\NW_4$
yields
\begin{eqnarray}
  \label{eq:bpg:ex:jose:path1:G}
  \widetilde{G}&=&2dx^-dx^+ +\frac{\vec{z}^2}{4}\left(d
    x^-\right)^2 + \left(d\vec{z}\right)^2 \\
  \label{eq:bpg:ex:jose:path1:H}
  \widetilde{H}&=&-dx^-\wedge dz^1\wedge dz^2 \\
  \label{eq:bpg:ex:jose:path1:B}
  \widetilde{B}&=&-x^- dz^1\wedge dz^2
\end{eqnarray}
which is the Nappi-Witten WZW model.

These results are all consistent with the remark made just after
\eqref{twistconjembdiag}. The brane $\AdS_2\times\S^2$ is a symmetric
D-submanifold of $\AdS_3\times\S^3$. As we discuss in detail in Section
\ref{TwistedNCBranes}, $\NW_4$ is {\it not} a symmetric D-brane in $\NW_6$,
although $\CW_4$ is. Consistently, $\S^{1,0}\times\S^3\subset\AdS_3\times\S^3$
is not a maximally symmetric D-embedding, but rather a member of the hierarchy
of factorising symmetry-breaking D-branes in ${\rm SU}(1,1)\times{\rm SU}(2)$
which preserves the action of an $\R\times{\rm SU}(2)$ subgroup and is localised
along a product of images of twisted conjugacy classes \cite{Quella1}.
Similarly, both the $\AdS_3\times\S^3$ and $\NW_6$ branes are non-symmetric,
while $\CW_6$ is the worldvolume of a spacetime-filling twisted D5-brane in
$\NW_6$.

%%% Local Variables: 
%%% mode: latex
%%% TeX-master: "main.tex"
%%% End: 
