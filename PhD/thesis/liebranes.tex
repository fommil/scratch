\chapter{Lie Branes of $\NW_6$}
\label{liebranes}
\section{Noncommutative Branes in $\NW_6$}
\label{NCBranes}
In this section and the next we will identify which {\it symmetric} D-branes of
the six-dimensional Nappi-Witten spacetime $\NW_6$ support a noncommutative
worldvolume geometry. After identifying all supergravity fields living on the
respective worldvolumes, we will proceed to quantise the classical geometries
using standard techniques \cite{Schom1}. A potential obstruction to this
procedure is that, as in any curved string background, a non-vanishing $H$-flux
can lead to non-associative deformations of worldvolume algebras typically
giving rise to variants of quantum group algebras that are deformations of
standard noncommutative geometries for which there is no general notion of
quantisation.

In certain semi-classical limits the NS-flux vanishes, $H=0$, such as in the
conformal field theory description based on a compact group $\mathcal{G}$ in the
limit of infinite Kac-Moody level $k\to\infty$ \cite{ARS1}. Such limits
correspond to field theory limits of the string theory whereby the Neveu-Schwarz
$B$-field (or more precisely the gauge invariant two-form \eqref{inv2forms})
induces a symplectic structure on the brane worldvolume, which can be quantised
in principle. In our situation we can identify $k=\theta^{-2}$, and the
semi-classical limit corresponds to the limit in which the plane wave approaches
flat spacetime. Such limits will always arise for the branes that we encounter
in this thesis.

We will support our characterisations by comparison with known results from the
boundary conformal field theory of the Nappi-Witten model \cite{DK2,Hikida1}. In
this section we deal with branes described by conjugacy classes of the
Nappi-Witten group $\mathcal{N}_6$. Generally, for the ordinary conjugacy
classes ($\omega=\1$ in the notation of \eqref{Dtwistconj}), the worldvolumes
inherit a natural symplectic form via the exponential map from the usual
Kirillov-Kostant form on coadjoint orbits, which coincides with the symplectic
structure induced by the $B$-field when $H=0$. The conjugacy classes, given as
orbits of the adjoint action of the Lie group $\mathcal{G}$ on itself, are then
identified with representations of $\mathfrak{g}$, obtained from quantisation of
coadjoint orbits, via the non-degenerate invariant bilinear form
$\langle\cdot,\cdot\rangle$ on $\mathfrak{g}$. If $V_g$ is an irreducible module
over the Lie algebra $\mathfrak{g}$ corresponding to the D-brane
$D=\mathcal{C}_g:=\mathcal{C}_g^{\1}$, then the noncommutative algebra of
functions on the worldvolume is given by \cite{ARS1}
\begin{equation}
  \label{NCalgconjclass}
  \mathcal{A}(\mathcal{C}_g)={\rm End}(V_g)
\end{equation}
The worldvolume algebra \eqref{NCalgconjclass} carries a natural (adjoint)
action of the group $\mathcal{G}$. In the present case we will find that these
quantised conjugacy classes carry a noncommutative geometry which is completely
analogous to that carried by D3-branes in flat space $\E^4$ with a uniform
magnetic field on their worldvolume \cite{DNek1,SW1,Sz1,Sz2}, as expected from
the harmonic oscillator character of dynamics in Nappi-Witten spacetime
described at length in Section \ref{NWPW}. We shall also find an interesting
class of commutative null branes whose quantum geometry generically differs
significantly from that of the classical conjugacy classes, due to a transverse
${\rm U}(2)$ rotational symmetry of the $\NW_6$ background.

\subsection{General Construction}
\label{GenConstrUntwist}

It will prove convenient to introduce the doublet $(\underline{\P}^\pm)^\top :=
(\P^\pm,\Q ^\pm)$ of generators and write Brinkman coordinates on
$\mathcal{N}_6$ as
\begin{equation}
  \label{NW6globalcoords}
  g(x^+,x^-,\mz,\overline{\mz})=e^{\frac\theta2x^+\J}
  e^{\mz^\top\underline{\P}^++\overline{\mz}^\top\underline{\P}^-}
  e^{\frac\theta2x^+\J}
  e^{\theta^{-1}x^-\T}
\end{equation}
In these coordinates one can work out the adjoint action
\begin{equation*}
  {\rm Ad}_{g(x^+,x^-,\mz,\overline{\mz})}g(x^+_0,x^-_0,\mz_0, \overline{\mz}_0)
\end{equation*}
corresponding to a fixed point $(x_0^+,x_0^-,\mz_0)\in\NW_6$ \cite{FS1,SF1}, and
the conjugacy classes can be written explicitly as the submanifolds
\begin{eqnarray}
  \nn
  \mathcal{C}_{(x_0^+,x^-_0,\mz_0)}&=&
  \Biggl\{\Bigl(x^+_0,
  x^-_0-\frac\theta2 |\mz|^2\sin\theta x^+_0+\theta{\rm Im}\left[
    \mz_0^\top \overline{\mz} e^{\frac{i\theta}2x^+}
    \cos\frac{\theta x_0^+}2\right],\\
  &&\qquad e^{ i\theta x^+}\mz_0-2 i
  \sin\frac{\theta x_0^+}2e^{\frac{ i\theta}2x^+}\mz
  \Bigr)\Biggm| \genfrac{}{}{0pt}{}{x^+\in\S^1}{\mz\in\C^2}\Biggr\}
  \label{conjclassgen}
\end{eqnarray}
where we have used periodicity to restrict the light-cone time coordinate to
$x^+,x^+_0\in\S^1=\R/2\pi\theta^{-1}\Z$. The null planes $x^+=x_0^+$
are thus invariants of the conjugacy classes and can be used to distinguish the
different D-submanifolds of $\NW_6$. There are two types of euclidean branes
generically associated to these conjugacy classes that we shall now proceed to
describe in detail. In each case we first describe the classical geometry, and
then proceed to quantise the orbits.

\subsection{Null Branes}
\label{Null}
We begin with the ``degenerate'' cases where $x_0^+=0$.

When $\mz_0={\mbf0}$, the conjugacy class \eqref{conjclassgen} corresponds to a
D-instanton sitting at the point $(0,x_0^-,\mbf0)\in\NW_6$. When
$\mz_0^\top:=(z_0,w_0)\neq\mbf0$ we denote the conjugacy classes by
$\mathcal{C}_{|z_0|,|w_0|}:=\mathcal{C}_{(0,x_0^-,\mz_0)}$. Since
$\mathcal{C}_{|z_0|,|w_0|}=\{(0,x^-, e^{ i\theta\,x^+}\,\mz_0|
x^-\in\R,x^+\in\S^1\}\cong\R\times\S^1$, the resulting object may be
thought of as a cylindrical brane extended along the null light-cone direction
$x^-$ with fixed radii $|z_0|,|w_0|$ in the two transverse planes
$(z_0,w_0)\in\C^2$, and will therefore be refered to as a ``null'' brane.
However, the quantisation of the classical worldvolume will generally turn out
to depend crucially on the radii. We can understand this dependence
heuristically as follows. Generally, the Cahen-Wallach metric
\eqref{NW6metricBrink} possesses an ${\rm SO}(4)$ rotational symmetry of its
transverse space $\mz\in\C^2\cong\R^4$, which is broken to ${\rm U}(2)$
by the Neveu-Schwarz background \eqref{NW6Bfield}. When $|z_0|=0$ or $|w_0|=0$,
one has $\mathcal{C}_{|z_0|,0}\cong\mathcal{C}_{0,|w_0|}\cong\R\times\S^1$.
However, when both $|z_0|$ and $|w_0|$ are non-zero, we may use this ${\rm
  U}(2)$ symmetry to rotate $w_0\mapsto e^{ i\phi}\,w_0$ at fixed $z_0$ by an
arbitrary phase. In this case the cylindrical brane is parametrised effectively
by two independent periodic coordinates, and one has
\begin{eqnarray}
  \label{cylnullbrane}
  \mathcal{C}_{|z_0|,|w_0|}&=&\left\{(0,x^-,e^{ i\theta x^+}z_0,  e^{ i\theta y^+}
    w_0) \bigm| x^-\in\R, x^+, y^+ \in\S^1\right\} \nn\\
  &\cong&
  \begin{cases}
      {\rm point} &z_0=w_0=0 \\
      \R\times\S^1\times\S^1 &z_0,w_0\neq0 \\
      \R\times\S^1 &{\rm otherwise}
    \end{cases}
\end{eqnarray}
In each case it carries a degenerate metric
\begin{equation}
  \label{degmetricnull}
  d s_6^2\bigm|_{\mathcal{C}_{|z_0|,|w_0|}}=\theta^2 |z_0|^2\left( d
    x^+\right)^2+\theta^2 |w_0|^2\left( d y^+\right)^2
\end{equation}
Such a brane generically has no straightforward interpretation as a D-brane, as
the corresponding Dirac-Born-Infeld action is ill-defined. Nevertheless, it will
play a role in our ensuing analysis and so we shall analyse it in some detail.

When $\mz_0\neq{\mbf0}$, the centraliser of
\begin{equation*}
  g_0:=g(0,x^-_0, \mz_0,\overline{\mz}_0)
\end{equation*}
is the subgroup of $\mathcal{N}_6$ parametrised as the submanifold
\begin{equation}
  \label{nullcentr}
  \mathcal{Z}_{|z_0|,|w_0|}=\left\{(0 , x^- , \mz)\bigm|
  {\rm Im}\mz_0^\top \overline{\mz}=0\right\}\cong\R^4
\end{equation}
with a degenerate metric, while $\mathcal{Z}_{0,0}\cong\mathcal{N}_6$. The
Neveu-Schwarz fields vanish on the brane,
\begin{equation}
  \label{H6B6null0}
  H_6\bigm|_{\mathcal{C}_{|z_0|,|w_0|}}=
  B_6\bigm|_{\mathcal{C}_{|z_0|,|w_0|}}=0
\end{equation}
while an elementary calculation using \eqref{NW4algdef}, \eqref{NW4innerprod},
\eqref{NW4CMform} and \eqref{nullcentr} shows that the Abelian gauge field
fluxes \eqref{2formfamily} also vanish on the null brane worldvolume,
\begin{equation}
  \label{nullF60}
  F^{(\zeta)}_6\bigm|_{\mathcal{C}_{|z_0|,|w_0|}}=0
\end{equation}
for any $\zeta=\theta^{-1} x^- \T+\mz^\top \underline{\P}^++ \overline{\mz}^{
  \top} \underline{\P}^-$ in the tangent space to the centraliser
\eqref{nullcentr}. This suggests that these conjugacy classes should describe
{\it commutative} branes. Since the $H$-field vanishes on the null branes, we
can use the standard coadjoint orbit method as discussed earlier.

\subsubsection{Quantisation}
Let us now describe the algebra of functions \eqref{NCalgconjclass} on the null
brane worldvolume. The Lie algebra $\mathfrak{n}_6$ has three types of unitary
irreducible representations $\mathcal{D}^{p^+,p^-}:U(\mathfrak{n}_6)\to{\rm
  End}(V^{p^+,p^-})$ labelled by light-cone momenta $p^\pm\in\R$
\cite{BAKZ1,CFS1,KK1}. On each module $V^{p^+,p^-}$, elements of the centre of
the universal enveloping algebra $U(\mathfrak{n}_6)$ are proportional to the
identity operator $\1$. In particular, the central element $\T$ acts as
\begin{equation}
  \label{Tirrepgen}
  \mathcal{D}^{p^+,p^-}(\T)= i\theta p^+ \1
\end{equation}
According to \eqref{utaudef}, the modules corresponding to the null branes live
in the class $V^{0,p^-}_{\alpha,\beta}$, $\alpha,\beta\in[0,\infty)$ of
continuous representations with $p^+=0$. In this case, in addition to $\T$ there
are two other Casimir operators corresponding to the quadratic elements $\P^+
\P^-$ and $\Q^+ \Q^-$ of $U(\mathfrak{n}_6)$ with the eigenvalues
\begin{equation}
  \label{otherCasimirs}
  \mathcal{D}^{0,p^-}_{\alpha,\beta}\left(\P^+ \P^-\right)=-\alpha^2 \1 \ , ~~
  \mathcal{D}^{0,p^-}_{\alpha,\beta}\left(\Q^+ \Q^-\right)=-\beta^2 \1 \ .
\end{equation}

As a separable Hilbert space, the module $V^{0,p^-}_{\alpha,\beta}$ may be
expressed as the linear span
\begin{equation}
  \label{contrepbasis}
  V^{0,p^-}_{\alpha,\beta}=\bigoplus_{n,m\in\Z}\C\cdot
  \left|^n_\alpha {}^m_\beta ; p^-\right\rangle
\end{equation}
with the non-trivial actions of the generators of $\mathfrak{n}_6$
given by
\begin{eqnarray}
  \label{contrepactions}
  \mathcal{D}^{0,p^-}_{\alpha,\beta}\left(\P^\pm\right)
  \bigl| {}^{nm}_{\alpha\beta} ; p^-\bigr\rangle
  &=& i\alpha \bigl| {}^{n\mp1}_\alpha {}^m_\beta ; p^-\bigr\rangle \\\nn
  \mathcal{D}^{0,p^-}_{\alpha,\beta}\left(\Q^\pm\right)
  \bigl| {}^{nm}_{\alpha\beta} ; p^-\bigr\rangle
  &=& i\beta \bigl| {}^n_\alpha {}^{m\mp1}_\beta ; p^-\bigr\rangle \\\nn
  \mathcal{D}^{0,p^-}_{\alpha,\beta}\bigl(\J 
  \bigr)\bigl| {}^{nm}_{\alpha\beta} ; p^-\bigr\rangle
  &=& i\left(\theta^{-1} p^--n-m\right) \bigl| {}^n_\alpha {}^m_\beta
  ; p^-\bigr\rangle
\end{eqnarray}
The inner product is defined such that the basis \eqref{contrepbasis} is
orthonormal,
\begin{equation}
  \label{contreportho}
  \bigl\langle {}^n_\alpha {}^m_\beta ; p^-\bigm| 
  {}^{n'}_\alpha {}^{m'}_\beta ; p^-\bigr\rangle=
  \delta_{nn'} \delta_{mm'}
\end{equation}
These representations have no highest or lowest weight states. Note that
shifting the labels $n$ or $m$ by $1$ gives an equivalent representation
\begin{equation}
  \label{contrepinequiv}
  V^{0,p^-}_{\alpha,\beta}\cong V^{0,p^-+\theta}_{\alpha,\beta}
\end{equation}
In other words, only the representations $V^{0,p^-}_{\alpha,\beta}$ with
$p^-\in[0,\theta)$ are inequivalent. This is analogous to the periodicity
constraint imposed on the light-cone coordinate $x^+$ in \eqref{conjclassgen}.

Among these representations is the trivial one-dimensional representation
$V_{0,0}^{0,0}$ which corresponds to the D-instantons found above. To
generically associate them with the null brane worldvolumes obtained before, we
use the correspondence between Casimir elements and class functions on the
group, which are respectively constant in irreducible representations and on
conjugacy classes. Note that the geodesics in $\NW_6$ obey $p^+=\dot x^+$ and
$p^-=\dot x^--\frac{\theta^2}4 |\mz|^2 p^+$ (c.f. \eqref{masslessLag}). At $p^+=0$,
we may thereby identify the parameters $\alpha$ and $\beta$ with the radii
$|z_0|$ and $|w_0|$ in the two transverse planes $(z_0,w_0)$, while $p^-$ is
identified with a fixed value of the light-cone position $x^-$. The quantised
algebra of functions on the conjugacy class \eqref{cylnullbrane} is thus given
by
\begin{equation}
  \label{quantalgnullbrane}
  \mathcal{A}\left(\mathcal{C}_{|z_0|,|w_0|}\right)=
  {\rm End}\bigl(V_{|z_0|,|w_0|}^{0,p^-}\bigr)
\end{equation}
where a generic element $\hat f\in{\rm End}(V_{\alpha,\beta}^{0,p^-})$ is a
complex linear combination
\begin{equation}
  \label{contreplincomb}
  \hat
  f=\sum_{n,m,n',m'\in\Z}f_{n,m;n',m'}\bigm| {}^n_\alpha {}^m_\beta
  ; p^-\bigr\rangle\bigl\langle {}^{n'}_\alpha {}^{m'}_\beta
  ; p^-\bigr|
\end{equation}
In this context, we may appropriately regard the conjugacy class
\eqref{cylnullbrane} as the worldvolume of a symmetric D$p$-brane with
$p=-1,1,0$ and volume element \eqref{degmetricnull}. Because of
\eqref{contrepinequiv}, in the quantum geometry the light-cone position is
restricted to a finite interval $x^-\in[0,\theta)$.

This identification can be better understood by introducing the coherent states
\begin{equation}
  \label{contrepcohstates}
  \bigl| {}^{x^+}_\alpha {}^{y^+}_\beta ; p^-
  \bigr\rangle:=\sum_{n,m\in\Z}
   e^{ i n \theta x^++ i m \theta y^+}\bigl| {}^n_\alpha {}^m_\beta
  ; p^-\bigr\rangle \qquad\qquad x^+,y^+\in\S^1
\end{equation}
on which the non-trivial symmetry generators are represented as differential
operators
\begin{eqnarray}
  \label{contrepcohactions}
  \mathcal{D}^{0,p^-}_{\alpha,\beta}\left(\P^\pm\right)
  \bigl| {}^{x^+}_\alpha {}^{y^+}_\beta ; p^-\bigr\rangle
  &=& i\alpha e^{\pm i\theta x^+} \bigl| {}^{x^+}_\alpha {}^{y^+}_\beta
  ; p^-\bigr\rangle \\\nn
  \mathcal{D}^{0,p^-}_{\alpha,\beta}\left(\Q^\pm\right)
  \bigl| {}^{x^+}_\alpha {}^{y^+}_\beta ; p^-\bigr\rangle
  &=& i\beta~ e^{\pm i\theta y^+} \bigl| {}^{x^+}_\alpha {}^{y^+}_\beta
  ; p^-\bigr\rangle \\ \nn
  \mathcal{D}^{0,p^-}_{\alpha,\beta}\bigl(\J \bigr)\bigl| 
  {}^{x^+}_\alpha {}^{y^+}_\beta ; p^-\bigr\rangle
  &=& i\theta^{-1} \left(p^--\frac\partial{\partial x^+}-
      \frac\partial{\partial y^+}\right) \bigl| {}^{x^+}_\alpha 
  {}^{y^+}_\beta ; p^-\bigr\rangle
\end{eqnarray}
The conjugate state to \eqref{contrepcohstates} is given by
\begin{equation}
  \label{contrepconjstate}
  \bigl\langle {}^{x^+}_\alpha {}^{y^+}_\beta\bigr|=\sum_{n,m\in\Z}
  e^{- i n \theta x^+- i
    m \theta y^+}~\bigl\langle {}^n_\alpha {}^m_\beta ; p^-\bigr|
\end{equation}
so that the inner product is
\begin{equation}
  \label{contrepinnerprod}
  \bigl\langle {}^{x_1^+}_\alpha {}^{y_1^+}_\beta ; p^-
  \bigm|{}^{x_2^+}_\alpha {}^{y_2^+}_\beta ; p^-\bigr\rangle=\delta\left(
    x_1^+-x_2^+\right) \delta\left(y_1^+-y_2^+\right)
\end{equation}
while the resolution of unity is
\begin{equation}
  \label{contrepresunity}
  \1 = \theta^2 \int\limits_0^{2\pi \theta^{-1}}\frac{ d x^+}{2\pi}
  \int\limits_0^{2\pi \theta^{-1}}\frac{ d y^+}{2\pi}
  \bigl| {}^{x^+}_\alpha {}^{y^+}_\beta ; p^-\bigr\rangle\bigl
  \langle {}^{x^+}_\alpha {}^{y^+}_\beta ; p^-\bigr|
\end{equation}
Thus the states \eqref{contrepcohstates} form an over-complete basis for the
Fock space $V^{0,p^-}_{\alpha,\beta}$. The metric and Kirillov-Kostant
symplectic two-form on the orbit can be computed as the matrix elements
\eqref{contrepcohstates} of the operators
\begin{equation*}
  |\mathcal{D}_{|z_0|,|w_0|}^{0,p^-}(g^{-1} d g)|^2 
\end{equation*}
and
\begin{equation*}
  \mathcal{D}_{|z_0|,|w_0|}^{0,p^-}([g^{-1} d g , g^{-1} d g])
\end{equation*}
respectively. They coincide with those computed above as the pull-backs of the
Nappi-Witten geometry to the conjugacy classes \eqref{cylnullbrane}.

We may now attempt to view the worldvolume algebra \eqref{quantalgnullbrane} as
a deformation of the classical algebra of functions on the conjugacy class
\eqref{cylnullbrane}. On the null hyperplanes $x^-={\rm constant}$, we define an
isomorphism of underlying vector spaces
\begin{equation}
  \label{Deltanullbranes}
  \Delta : {\rm C}^\infty\left(\mathcal{C}_{|z_0|,|w_0|}
  \right)\longrightarrow{\rm End}\bigl(V_{|z_0|,|w_0|}^{0,p^-}\bigr)
\end{equation}
in the following manner. Decomposing any smooth function $f\in{\rm
  C}^\infty(\mathcal{C}_{|z_0|,|w_0|})$ as a Fourier series over
the torus $\S^1\times\S^1$ given by
\begin{equation}
  \label{Fourierfnull}
  f\left(x^+,y^+\right)=\sum_{n,m\in\Z}f_{n,m}~ e^{ i n \theta x^+
    + i m \theta y^+}
\end{equation}
we write
\begin{equation}
  \label{hatfDeltanull}
  \hat f=\Delta(f):=\sum_{n,m\in\Z}f_{n,m}
  \Bigl(\frac1{ i|z_0|} \mathcal{D}^{0,p^-}_{|z_0|,|w_0|}
  \bigl(\P^{\varepsilon(n)}\bigr)\Bigr)^{|n|} \Bigl(
  \frac1{ i|w_0|} \mathcal{D}^{0,p^-}_{|z_0|,|w_0|}
  \bigl(\Q^{\varepsilon(m)}\bigr)\Bigr)^{|m|}
\end{equation}
where the label $\varepsilon(n)=\pm$ corresponds to the sign of
$n\in\Z$. The inverse map is given by
\begin{eqnarray}
  \label{invDeltanull}
  f\left(x^+,y^+\right)&:=&\Delta^{-1}\bigl(\hat f \bigr)\left(x^+,y^+
  \right)\\\nn
  &=&\theta^2 \int\limits_0^{2\pi \theta^{-1}}\frac{ d\tilde x^+}{2\pi}
  \int\limits_0^{2\pi \theta^{-1}}\frac{ d\tilde y^+}{2\pi}
  \bigl\langle {}^{\tilde x^+}_{|z_0|} {}^{\tilde y^+}_{|w_0|} 
  ; p^-\bigr|\hat f\bigl| {}^{x^+}_{|z_0|} {}^{y^+}_{|w_0|} ;
  p^-\bigr\rangle
\end{eqnarray}
As expected, from \eqref{contrepcohactions} one finds that the functions
corresponding to the generators of $\mathfrak{n}_6$ on $V_{|z_0|,|w_0|}^{0,p^-}$
coincide with the coordinates of the conjugacy classes \eqref{cylnullbrane},
\begin{eqnarray}
  \label{nullfnsgens}
  \Delta^{-1}\Bigl(\mathcal{D}^{0,p^-}_{|z_0|,|w_0|}\left(\P^\pm\right)
  \Bigr)\left(x^+,y^+\right)&=& i|z_0|~ e^{\pm i \theta x^+} \\\nn
  \Delta^{-1}\Bigl(\mathcal{D}^{0,p^-}_{|z_0|,|w_0|}\left(\Q^\pm\right)
  \Bigr)\left(x^+,y^+\right)&=& i|w_0|~ e^{\pm i \theta y^+} \\\nn
  \Delta^{-1}\Bigl(\mathcal{D}^{0,p^-}_{|z_0|,|w_0|}\bigl(\J \bigr)
  \Bigr)\left(x^+,y^+\right)&=& i\theta^{-1} p^-
\end{eqnarray}
Generically, the classical conjugacy class \eqref{conjclassgen} in this case
corresponds to the diagonal subspace $x^+=y^+$ of the space spanned by the
coherent states \eqref{contrepcohstates}.

Since the operator \eqref{hatfDeltanull} is diagonal in the basis
\eqref{contrepcohstates} with eigenvalue $f(x^+,y^+)$, we easily establish that
in this case the map \eqref{Deltanullbranes} is in fact an {\it algebra}
isomorphism. Namely, the product of two operators $\hat f$ and $\hat g$ on
$V_{|z_0|,|w_0|}^{0,p^-}$ corresponds to the pointwise multiplication of the
associated functions on $\mathcal{C}_{|z_0|,|w_0|}$,
\begin{equation}
  \label{commprodnull}
  \Delta^{-1}\bigl(\hat f \hat g \bigr)\left(x^+,y^+\right)=
  f\left(x^+,y^+\right) g\left(x^+,y^+\right)
\end{equation}
Thus the worldvolume algebra \eqref{quantalgnullbrane} in the present case
describes a {\it commutative} geometry on the null branes, in agreement with the
classical analysis. This is also in accord with the fact that the $p^+=0$ sector
of the dynamics in Nappi-Witten spacetime does not feel the harmonic oscillator
potential of Section \ref{Dynamics}, and thereby describes free motion in the
transverse space.

\subsection{D3-Branes}
\label{CGED3B}

Let us now turn to the somewhat more interesting cases with $x_0^+\neq0$,
whereby the classical worldvolume geometries are nondegenerate.

In this case the conjugacy class \eqref{conjclassgen} can be written after a
trivial coordinate redefinition as
\begin{equation}
  \label{conjpieucl4}
  \mathcal{C}_{(x_0^+,x_0^-,\mz_0)}=
  \bigl\{(x_0^+ , x_0^-+\frac\theta4 \left(|\mz_0|^2-|\mz' |^2
  \right) \cot\frac{\theta x_0^+}2 , \mz' )\bigm|\mz'
  \in\C^2\bigr\}\cong\E^4
\end{equation}
It may be labelled as $\mathcal{C}_{x_0^+,\chi^{ }_0}$, with
$\chi_0:=x_0^-+\frac\theta4 |\mz_0|^2 \cot\frac{\theta x_0^+}2$, so that the
corresponding branes have four-dimensional worldvolumes located at
\begin{equation}
  \label{E4branesloc}
  x^+=x_0^+ \qquad\qquad\qquad x^-=\chi^{ }_0-\frac{\theta}4 |\mz|^2 \cot\frac{\theta
    x_0^+}2
\end{equation}
The worldvolume metric is non-degenerate and given by
\begin{equation}
  \label{metricpieucl4}
  d s_6^2\bigm|{}_{\mathcal{C}_{x_0^+,\chi_0}}=| d\mz|^2
\end{equation}
so that in this case the conjugacy classes are wrapped by flat euclidean
D3-branes. The NS fields on the worldvolume are given by flat space forms
\begin{eqnarray}
  \label{NSconj}
  H_6\bigl|_{\mathcal{C}_{x_0^+,\chi_0}}&=&0 \\
  \label{NSconjB6}
  B_6\bigl|_{\mathcal{C}_{x_0^+,\chi_0}}&=&-2 i\theta 
  x_0^+ d\overline{\mz}^{ \top}\wedge d\mz
\end{eqnarray}
The vanishing of the $H$-flux again means that we can apply standard
semi-classical quantisation techniques to these D-branes.

If $x_0^+=\frac\pi\theta$ then these branes, like the null branes, are
associated with the conjugate points of the Rosen plane wave geometry defined by
\eqref{Rosen} through \eqref{Cumatrix}. In this case the conjugacy class
$\mathcal{C}_{x_0^-}:=\mathcal{C}_{(\frac\pi\theta,x_0^-,\mz_0)}$ is a
four-plane labelled by the fixed light-cone position $x^-=x_0^-$. When
$x_0^+\neq0,\frac\pi\theta$, the brane worldvolume is a paraboloid corresponding
to a point-like object travelling at the speed of light while simultaneously
expanding or contracting in a three-sphere in the transverse space $\mz\in\C^2$
according to \eqref{E4branesloc}. Since these branes lie in the set of
conjugate-free points, we may analyse them using the Penrose-G\"uven limit of
Section \ref{ApplNW}, which yields a non-vanishing gauge invariant two-form
field via pull-back of \eqref{NW6tildegaugeinv} to the conjugacy classes as
\begin{equation}
  \label{E4F6PGL}
  \mathcal{F}_{x_0^+}:=\mathcal{F}_6\bigl|_{\mathcal{C}_{x_0^+,\chi_0}}
  =- i\cot\frac{\theta x_0^+}2 d\overline{\mz}^{ \top}\wedge d\mz
\end{equation}
Thus these D-branes are expected to carry a noncommutative geometry for
$x_0^+\neq0,\frac\pi\theta$. Extrapolating \eqref{E4F6PGL} to $x_0^+=\frac\pi\theta$
shows that the conjugacy classes $\mathcal{C}_{x_0^-}$ are expected to support a
commutative worldvolume geometry like the null branes, despite their
non-vanishing $B$-field.

The Penrose-G\"uven limit can also be used to understand the physical origin of
the euclidean D3-branes. They may be described through the commuting isometric
embedding diagram
\begin{equation}
  \label{E4commdiag}
  \begin{CD}
   @.\\
    \AdS_3 \times \S^3             @>\text{PGL}>> \NW_6\\
    \text{$\imath^{ \prime}  $}@AAA
@AAA\text{$\widetilde{\imath}^{ \prime}$}\\
    \AdS_2 \times \S^2             @>\text{PGL$^{ \prime}$}>> \E^4\\
   @.
  \end{CD}
\end{equation}
where the primes indicate that the limit is taken along a null geodesic which
does {\it not} pass through the $\AdS_2\times\S^2$ brane worldvolume, i.e. the
embeddings $\imath^{ \prime}$ and $\widetilde{\imath}^{ \prime}$ are defined by
the constant time slices $\tau,\phi={\rm constant}$ and $x^+={\rm constant}$,
respectively. Thus the resulting geometry is not of plane wave type. The brane
on the left-hand side of \eqref{E4commdiag} originates as a flat D3-brane
connected orthogonally to a distant NS5/F1 black string by a stretched $(p,q)$
string \cite{BP1}, the origin of the worldvolume flux \eqref{E4F6PGL}. As the
$(p,q)$ string pulls the flat D3-brane, it deforms its worldvolume geometry,
leading to an $\AdS_2\times\S^2$ brane in the near-horizon region of the black
string. The Penrose-G\"uven limit in \eqref{E4commdiag} pulls the deformed
branes away into the asymptotically flat region of the black string,
decompactifying it as $R\to\infty$ onto the flat euclidean D3-brane on the
right-hand side of the isometric embedding diagram. However, since the diagram
commutes, the standard fuzzy geometry of the $\AdS_2\times\S^2$ brane induces,
through the usual scaling limit \cite{CMS1}, a Moyal type noncommutative
geometry on the instantonic $\E^4$ brane. This argument is consistent with
the fact that the (modified) Penrose-G\"uven limit is also a map between
symmetric D-branes. We shall now substantiate this physical picture through
explicit computation of the quantised worldvolume geometry.

\subsubsection{Quantisation}
To describe the algebra of functions \eqref{NCalgconjclass} on the D3-brane
worldvolume, we start with the irreducible representations
$\mathcal{D}^{p^+,p^-}:U(\mathfrak{n}_6)\to{\rm End}(V^{p^+,p^-})$ having
$p^+>0$. In addition to \eqref{Tirrepgen}, the quadratic Casimir element ${\sf
  C}_6=2 \J \T+\frac12 [( \Pu^+)^\top \Pu^-+( \Pu^-)^\top \Pu^+]$ acts as a
scalar operator
\begin{equation}
  \label{C6irrep}
  \mathcal{D}^{p^+,p^-}(\C_6)=-2p^+ \left(p^-+\theta\right) \1
\end{equation}
This operator is positive for all $p^-\in(-\infty,-\theta)$. As a separable Hilbert
space, the module $V^{p^+,p^-}$ may be exhibited as the linear span
\begin{equation}
  \label{Vhighestex}
  V^{p^+,p^-}=\bigoplus_{n,m\in\N_0}\C\cdot\bigl|n,m;p^+,p^-
  \bigr\rangle
\end{equation}
with the non-trivial actions of the Nappi-Witten generators given by
\begin{eqnarray}
  \label{highestactions}
  \mcDp\left(\P^+\right)\bigl|n,m;p^+,p^-\bigr\rangle&=&
  2 i\theta p^+ n \bigl|n-1,m;p^+,p^-\bigr\rangle\\\nn
  \mcDp\left(\P^-\right)\bigl|n,m;p^+,p^-\bigr\rangle&=&
  i\bigl|n+1,m;p^+,p^-\bigr\rangle \\\nn
  \mcDp\left(\Q^+\right)\bigl|n,m;p^+,p^-\bigr\rangle&=&
  2 i\theta p^+ m \bigl|n,m-1;p^+,p^-\bigr\rangle \\\nn
  \mcDp\left(\Q^-\right)\bigl|n,m;p^+,p^-\bigr\rangle&=&
  i\bigl|n,m+1;p^+,p^-\bigr\rangle \\\nn
  \mcDp\bigl(\J \bigr)\bigl|n,m;p^+,p^-\bigr\rangle&=&
  i\left(\theta^{-1} p^--n-m\right)
  \bigl|n,m;p^+,p^-\bigr\rangle
\end{eqnarray}
The inner product on the basis \eqref{Vhighestex} is
\begin{equation}
  \label{highestortho}
  \bigl\langle n,m;p^+,p^-\bigm|n',m';p^+,p^-\bigr\rangle=
  \left(2\theta p^+\right)^{n+m} n! m! \delta_{nn'} \delta_{mm'}
\end{equation}
This representation admits a highest weight state $|0,0;p^+,p^-\rangle$ on which
$- i\mcDp(\J)$ has weight $\theta^{-1} p^-$.

To associate these representations with euclidean D3-brane worldvolumes, we note
that the constraint on the light-cone position in \eqref{E4branesloc} can be
written in the semi-classical limit $\theta\to0$ as
\begin{equation}
  \label{semiclasspos}
  2x_0^+ x^-+|\mz|^2=2x_0^+ \chi_0
\end{equation}
The relation \eqref{semiclasspos} agrees with the Casimir eigenvalue constraint
\eqref{C6irrep} under the identifications of the light-cone time $x_0^+$ with
momentum $p^+$ as before and the class variable $\chi_0$ with $p^-+\theta$. We will
soon identify the worldvolume coordinates $\mz\in\C^2$ with the operators
$\mcDp( \Pu^+)$. Thus the quantised algebra of functions on the conjugacy class
is given by
\begin{equation}
  \label{quantalgD3}
  \mathcal{A}\bigl(\mathcal{C}_{x_0^+,\chi_0}\bigr)=
  {\rm End}\bigl(V^{p^+,p^-}\bigr)
\end{equation}
where a generic element $\hat f\in{\rm End}(V^{p^+,p^-})$ is a complex
linear combination
\begin{equation}
  \label{genlincombD3}
  \hat f=\sum_{n,m,n',m'\in\N_0}f_{n,m;n',m'} \bigl|n,m;p^+,p^-
  \bigr\rangle\bigl\langle n',m';p^+,p^-\bigr|
\end{equation}

As before, it is convenient to work in the conventional coherent state basis of
the Fock module $V^{p^+,p^-}$ defined for $\mz^\top=(z,w)\in\C^2$ by
\begin{equation}
  \label{D3cohstates}
  \bigl|\mz;p^+,p^-\bigr\rangle:= e^{-\mz^\top \Pu^-}
  \bigl|0,0;p^+,p^-\bigr\rangle=\sum_{n,m\in\N_0}
  \frac{z^n w^m}{ i^{n+m} n! m!} \bigl|n,m;p^+,p^-
  \bigr\rangle
\end{equation}
on which the non-trivial symmetry generators are represented by differential
operators
\begin{eqnarray}
  \label{symactioncohD3}
  \mcDp\left( \Pu^+\right)\bigl|\mz;p^+,p^-\bigr\rangle&=&
  2\theta p^+ \mz \bigl|\mz;p^+,p^-\bigr\rangle\\\nn
  \mcDp\left( \Pu^-\right)\bigl|\mz;p^+,p^-\bigr\rangle&=&
  -\bfd\bigl|\mz;p^+,p^-\bigr\rangle \\\nn
  \mcDp\bigl(\J \bigr)\bigl|\mz;p^+,p^-\bigr\rangle&=&
  i\left(\theta^{-1} p^--\mz^\top \bfd\right)
  \bigl|\mz;p^+,p^-\bigr\rangle
\end{eqnarray}
with $\bfd^\top:=(\frac\partial{\partial z},\frac\partial{\partial
  w})$. The conjugate state to \eqref{D3cohstates} is given by
\begin{equation}
  \label{conjD3cohstates}
  \bigl\langle\mz;p^+,p^-\bigr|=\bigl\langle0,0;p^+,p^-\bigr|
  e^{\overline{\mz}{}^{ \top} \Pu^+}
\end{equation}
so that the inner product of coherent states is
\begin{equation}
  \label{innprodcohD3}
  \bigl\langle\mz;p^+,p^-\bigm|\mz';p^+,p^-\bigr\rangle=
  e^{2\theta p^+ \overline{\mz}{}^{ \top} \mz'}
\end{equation}
while their completeness relation is
\begin{equation}
  \label{D3complrel}
  \1 = \left(2\theta p^+\right)^2 \int\limits_{\C^2}
  d\varrho\left(\mz,\overline{\mz} \right)  e^{-2\theta p^+ |\mz|^2} 
  \bigl|\mz;p^+,p^-\bigr\rangle\bigl\langle\mz;p^+,p^-\bigr|
\end{equation}
where $ d\varrho(\mz,\overline{\mz} )=\frac1{\pi^2} | d z\wedge
d\overline{z}\wedge d w\wedge d\overline{w} |$ is the standard flat measure on
$\E^4\cong\C^2$. We will now use these states to describe the
worldvolume algebra \eqref{quantalgD3} as a deformation of the classical algebra
of functions on the conjugacy class $\mathcal{C}_{x_0^+,\chi_0}$.

For this, we construct an isomorphism of underlying vector spaces
\begin{equation}
  \label{Deltastar}
  \Delta_* : {\rm C}^\infty\bigl(\mathcal{C}_{x_0^+,\chi_0}\bigr)
  \longrightarrow {\rm End}\bigl(V^{p^+,p^-}\bigr)
\end{equation}
by expanding any smooth function $f\in{\rm C}^\infty
(\mathcal{C}_{x_0^+,\chi_0})$ as a Taylor series over $\C^2$ given by
\begin{equation}
  \label{TaylorD3}
  f\left(\mz,\overline{\mz} \right)=\sum_{n,m,n',m'\in\N_0}f_{n,m;n',m'} 
  z^n w^m \overline{z}{}^{ n'} \overline{w}^{ m'}
\end{equation}
and defining
\begin{eqnarray}
  \label{DeltafD3}
  \hat f = \Delta_*(f)&:=&\sum_{n,m,n',m'\in\N_0} \frac{f_{n,m;n',m'}}
  {\left(2\theta p^+\right)^{n+m+n'+m'}} \mcDp\left(\P^-\right)^{n'} 
  \mcDp\left(\Q^-\right)^{m'}\nn\\ &&\times \mcDp\left(\P^+\right)^{n} 
  \mcDp\left(\Q^+\right)^{m}
\end{eqnarray}
The inverse map is provided by the normalised matrix elements of
operators as
\begin{equation}
  \label{DeltainvD3}
  f\left(\mz,\overline{\mz} \right):=\Delta_*^{-1}\bigl(\hat f 
  \bigr)\left(\mz,\overline{\mz} \right)= e^{-2\theta p^+ |\mz|^2} 
  \bigl\langle\mz;p^+,p^-\bigl| \hat f \bigr|\mz;p^+,p^-\bigr\rangle
\end{equation}
As before, the functions corresponding to the generators of the
Nappi-Witten algebra $\mathfrak{n}_6$ coincide with the coordinates of
the conjugacy classes $\mathcal{C}_{x_0^+,\chi_0}$,
\begin{eqnarray}
  \label{DeltainvD3gens}
  \Delta_*^{-1}\Bigl(\mcDp\left( \Pu^+\right)\Bigr)\left(\mz,
    \overline{\mz} \right)&=&2\theta p^+ \mz \\\nn
  \Delta_*^{-1}\Bigl(\mcDp\left( \Pu^-\right)\Bigr)\left(\mz,
    \overline{\mz} \right)&=&2\theta p^+ \overline{\mz} \\\nn
  \Delta_*^{-1}\Bigl(\mcDp\bigl(\J \bigr)\Bigr)\left(\mz,
    \overline{\mz} \right)&=&
  i\left(\theta^{-1} p^--2\theta p^+ |\mz|^2\right)
\end{eqnarray}

The map \eqref{Deltastar} in this case is {\it not} an algebra homomorphism and
it can be used to deform the pointwise multiplication of functions in the
algebra ${\rm C}^\infty(\mathcal{C}_{x_0^+,\chi_0})$, giving an associative
$\ast$-product defined by
\begin{eqnarray}
  \label{D3stardef}
  (f*g)\left(\mz,\overline{\mz} \right)&:=&
  \Delta_*^{-1}\bigl(\hat f \hat g \bigr)
  \left(\mz,\overline{\mz} \right) =  e^{-2\theta p^+ |\mz|^2} 
  \bigl\langle\mz;p^+,p^-\bigl| \hat f \hat g \bigr|\mz;p^+,p^-
  \bigr\rangle\nn\\ &=&\left(2\theta p^+\right)^2 
  \int\limits_{\C^2} d\varrho\left(\mz',\overline{\mz}{}^{ \prime} 
  \right)  e^{-2\theta p^+ (|\mz'|^2+|\mz|^2)}\nn\\
  &&\times \bigl\langle\mz;p^+,p^-
  \bigl| \hat f \bigr|\mz';p^+,p^-\bigr\rangle \bigl\langle
  \mz';p^+,p^-\bigl| \hat g \bigr|\mz;p^+,p^-\bigr\rangle
\end{eqnarray}
We can express this $\ast$-product more explicitly in terms of a bi-differential
operator acting on ${\rm C}^\infty(\mathcal{C}_{x_0^+,\chi_0})\otimes{\rm
  C}^\infty(\mathcal{C}_{x_0^+,\chi_0})\to{\rm
  C}^\infty(\mathcal{C}_{x_0^+,\chi_0})$ by writing the normalised matrix
elements in \eqref{D3stardef} using translation operators as \cite{APS1}
\begin{eqnarray}
  \label{matrixelttransl}
  e^{-\mz^\top \bfd'}  e^{\mz^{\prime \top} \bfd}
  f\left(\mz,\overline{\mz} \right)&=& e^{-\mz^\top \bfd'}
  \frac{\bigl\langle\mz;p^+,p^-
    \bigl| \hat f \bigr|\mz+\mz';p^+,p^-\bigr\rangle}{
    \bigl\langle\mz;p^+,p^-\bigm|\mz+\mz';p^+,p^-\bigr\rangle}\\\nn
  &=&
  \frac{\bigl\langle\mz;p^+,p^-
    \bigl| \hat f \bigr|\mz';p^+,p^-\bigr\rangle}{
    \bigl\langle\mz;p^+,p^-\bigm|\mz';p^+,p^-\bigr\rangle}
\end{eqnarray}
The translation operator $ e^{-\mz^\top \bfd'} e^{\mz^{\prime \top} \bfd}$,
acting on $\mz'$-independent functions, can be expressed as a normal ordered
exponential $\NO \exp(\mz'-\mz)^\top \overrightarrow{\bfd} \NO$ with
derivatives ordered to the right in each monomial of the Taylor series expansion
of the exponential function. In this way we may write the $\ast$-product
\eqref{D3stardef} as
\begin{eqnarray}
  \nn
  (f*g)\left(\mz,\overline{\mz} \right)&=&\left(2\theta p^+\right)^2 
  \int\limits_{\C^2} d\varrho\left(\mz',\overline{\mz}{}^{ \prime} 
  \right) f\left(\mz,\overline{\mz} \right) \NO \exp
  \overleftarrow{\bfd}{}^{ \top} \left(\mz'-\mz\right) \NO\\
  &&\times  e^{-2\theta p^+ |\mz'-\mz|^2} \NO \exp\left(
    \overline{\mz}{}^{ \prime}-\overline{\mz} \right)^\top 
  \overrightarrow{\overline{\bfd}} \NO g\left(\mz,\overline{\mz} \right)
  \label{D3starform}
\end{eqnarray}
and performing the Gaussian integral in \eqref{D3starform} leads to our final
form
\begin{equation}
  \label{D3starfinal}
  (f*g)\left(\mz,\overline{\mz} \right)=f\left(\mz,
    \overline{\mz} \right) \exp\Bigl(\frac1{2\theta p^+}
  \overleftarrow{\bfd}{}^{ \top} \overrightarrow{\overline{\bfd}}
  \Bigr) g\left(\mz,\overline{\mz} \right)
\end{equation}

The $\ast$-product \eqref{D3starfinal} is the Voros product \cite{Voros1} on
four-dimensional noncommutative euclidean space $\E_\Theta^4$ corresponding
to the Poisson bi-vector
\begin{equation}
  \label{NCparD3}
  \Theta=-\frac i{2\theta p^+} \overline{\bfd}{}^{ \top}
  \wedge\bfd
\end{equation}
whose components are proportional to the inverse of the magnetic field (or
equivalently harmonic oscillator frequency) $\omega=\frac12 \theta p^+$ in the
effective particle dynamics in Nappi-Witten spacetime described in Section
\ref{Dynamics}. This product is {\it not} the same as the standard Moyal product
\begin{equation}
  \label{MoyalD3def}
  (f\star g)\left(\mz,\overline{\mz} \right):=f\left(\mz,
    \overline{\mz} \right) \exp\left[\frac1{4\theta p^+}
    \Bigl(\overleftarrow{\bfd}{}^{ \top} \overrightarrow{\overline{\bfd}}
    -\overleftarrow{\overline{\bfd}}{}^{ \top} 
    \overrightarrow{\bfd} \Bigr)\right] g\left(\mz,\overline{\mz} \right)
\end{equation}
which arises from Weyl operator ordering, rather than the normal ordering
prescription employed in \eqref{DeltafD3}. Although different, these two
products are cohomologically {\it equivalent} \cite{Voros1}, because the
invertible differential operator $\mathcal{T}:=\exp\frac1{4\theta p^+} |\bfd|^2$
gives an algebra isomorphism $\mathcal{T}:({\rm
  C}^\infty(\mathcal{C}_{x_0^+,\chi_0}) , \star)\to({\rm
  C}^\infty(\mathcal{C}_{x_0^+,\chi_0}) , *)$, i.e. $\mathcal{T}(f\star
g)=\mathcal{T}(f)*\mathcal{T}(g) \forall f,g\in{\rm
  C}^\infty(\mathcal{C}_{x_0^+,\chi_0})$.

Finally, to deal with the geometry in the case $p^+<0$, we construct a
lowest-weight module $\widetilde{V}^{p^+,p^-}$ which defines the representation
conjugate to $V^{p^+,-p^-}$ above by interchanging the roles of the generators
$\Pu^+\leftrightarrow\Pu^-$ and replacing the generators $\J,\T$ with their
reflections $-\J,-\T$. The two representations have the same quadratic Casimir
eigenvalue \eqref{C6irrep} and are dual to each other as
\begin{equation}
  \label{dualreps}
  \widetilde{V}^{p^+,p^-} \cong \bigl(V^{p^+,p^-}\bigr)^*
\end{equation}
The $\star$-product is generically given as \eqref{D3starfinal} or
\eqref{MoyalD3def} with $p^+$ replaced by $|p^+|$, and the worldvolume algebra
\eqref{quantalgD3} is canonically isomorphic as a vector space to
\begin{equation}
  \label{quantalgD3duals}
  \mathcal{A}\bigl(\mathcal{C}_{x_0^+,\chi_0}\bigr)=V^{p^+,p^-}
  \otimes\widetilde{V}^{p^+,p^-} \qquad\qquad p^+>0
\end{equation}
The $\mathcal{N}_6$-module structure is then determined by the Clebsch-Gordan
decomposition of \eqref{quantalgD3duals} into the irreducible continuous
representations of the Nappi-Witten algebra as
\begin{equation}
  \label{D3module}
  \mathcal{A}\bigl(\mathcal{C}_{x_0^+,\chi_0}\bigr)=
  \hspace{4pt}\int\limits_0^\infty
  \hspace{-15.6pt}\bigodot\hspace{-19pt}\mbf-\hspace{-4pt}\mbf-\hspace{4pt}
  d\alpha \alpha
  \hspace{4pt}\int\limits_0^\infty
  \hspace{-15.6pt}\bigodot\hspace{-19pt}\mbf-\hspace{-4pt}\mbf-\hspace{4pt}
  d\beta \beta  V_{\alpha,\beta}^{0,0}
\end{equation}
The organisation of the quantised worldvolume algebra into irreducible
representations associated with null branes owes to the fact that the isometry
generators of the noncommutative D3-branes in $\NW_6$ are given by the Killing
vectors $P^\pm$, $Q^\pm$ and $J+\overline{J}$ in \eqref{6CWKilling},
\ref{extraHKilling} with $T=0$, corresponding to translations in each transverse
plane $z,w\in\C$ along with simultaneous rotations of the two planes.
These isometries generate the subgroup \eqref{S5subgp} which coincides with the
group ${\rm Inn}(\mathfrak{n}_6)$ of inner automorphisms of the Nappi-Witten Lie
algebra. The symmetric untwisted D3-brane breaks the generic rotational symmetry
${\rm U}(2)={\rm SU}(2)\times{\rm U}(1)$ to ${\rm U}(1)\cong{\rm SO}(2)$,
leaving the overall isometry subgroup \eqref{S5subgp} which is precisely the
symmetry group of noncommutative euclidean space in four dimensions with equal
magnetic fields through each parallel plane \cite{AlVaz1,CMNTV1}. Thus the
embedding of flat branes into $\NW_6$ realises explicitly the usual breaking of
${\rm ISO}(4)$ invariance in passing to the noncommutative space
$\E_\Theta^4$.

\subsection{Open String Description}
\label{D3OSD}
Let us now compare the semi-classical results obtained above with the
predictions from the open string dynamics on the NS supported D-branes in the
Seiberg-Witten decoupling limit. Generally, let $G$ be the closed string metric
on a D-brane and $\mathcal{F}=B+F$ the gauge-invariant two-form which we assume
is non-degenerate. Whenever $ d(\mathcal{F}-G \mathcal{F}^{-1} G)=0$, the
momentum of an open string attached to the D-brane is small in the low-energy
limit \cite{HY1}. This is just the requirement $H=0$ of a vanishing NS flux in
the limit of large $B$-field. The strings are then very short and see only a
small portion of the worldvolume, which is approximately flat. The same
expressions that apply to flat backgrounds can thereby be applied in these
instances \cite{ARS1,HY1}. In particular, the noncommutativity parameters and
the open string metric may be computed from the usual Seiberg-Witten formulas
\cite{SW1}
\begin{eqnarray}
  \label{SWThetagen}
  \Theta&=&\frac1{\mathcal{F}-G \mathcal{F}^{-1} G} = -\frac1{G+\mathcal{F}}
  \mathcal{F} \frac1{G-\mathcal{F}}
  \\\label{SWopenmetgen}
  G_{\rm o}&=&G-\mathcal{F} G^{-1} \mathcal{F}
\end{eqnarray}
For the flat euclidean D3-branes of the previous subsection, we substitute
\eqref{metricpieucl4} and \eqref{E4F6PGL} in \eqref{SWThetagen} to get the
bi-vector
\begin{equation}
  \label{ThetaD3}
  \Theta=-\frac i{(\mathcal{F}_{x_0^+})_{z \overline{z}}+
    (\mathcal{F}_{x_0^+})^{-1}_{z \overline{z}}}
  \overline{\bfd}{}^{ \top}\wedge\bfd=
  -\frac i2 \sin\theta x_0^+ \overline{\bfd}^{ \top}
  \wedge\bfd
\end{equation}
This does not agree with \eqref{NCparD3} in the limit $\theta\to0$, as one would
have naively expected, but the reason for the discrepancy is simple. The
semi-classical analysis of the previous subsection is strictly speaking valid
only in the limit of large $B$-field, for which the formula \eqref{SWThetagen}
reduces to $\Theta=B^{-1}$ and coincides with \eqref{NCparD3} on the D3-branes.
This is equivalent to the zero-slope field theory limit ($\alpha'\to0$) of the
open string dynamics, and it yields the Kirillov-Kostant Poisson bi-vector on
the coadjoint orbits corresponding to the symplectic two-form $B$. This
situation is characteristic of branes in curved backgrounds \cite{ARS1,HY1}, and
we will regard the $\theta\to0$ limit of \eqref{ThetaD3}, for which
$\Theta=-\frac{ i\theta p^+}2 \overline{\bfd}^{ \top}\wedge\bfd$, as ``dual'' to
the field theoretic quantity \eqref{NCparD3} with respect to the inner product
$\langle \cdot , \cdot \rangle$ on $\mathfrak{n}_6$.

In \cite{Halliday:2005zt} we find the intrinsic ${\rm SU}(2)$ symmetry of the NS
background implies long strings, of light-cone momenta $p^+>1$, can move freely
in the two transverse planes to the Nappi-Witten pp-wave and correspond to
spectral-flowed null brane states \cite{BAKZ1}. The long strings in $\NW_6$ thus
correspond to fundamental string states. The spectral flow of long string states
also implies that the strong NS-field limit gives a flat space theory
\cite{DAK1}, just like the semiclassical limit.

\section{Twisted Noncommutative Branes in $\NW_6$}
\label{TwistedNCBranes}
The quantisation of twisted conjugacy classes is more intricate \cite{AFQS1}.
The $\omega$-twisted D-branes in the Lie group $\mathcal{G}$ are labelled by
representations of the invariant subgroup
\begin{equation}
  \label{Gomegagen}
  \mathcal{G}^\omega:=\bigl\{g\in\mathcal{G} \bigm| \omega(g)=g
  \bigr\}
\end{equation}
In a neighbourhood of the identity of $\mathcal G$, a twisted conjugacy class
$\mathcal{C}_g^\omega$ may be regarded as a fibration
\begin{equation}
  \label{conjfibre}
  \begin{CD}
   @.\\
    \breve{\mathcal C}_g        @>>> \mathcal{C}_g^\omega\\
    @.    @VVV\\ @.
     \mathcal{G} / \mathcal{G}^\omega  \\
   @.
  \end{CD}
\end{equation}
where $\breve{\mathcal C}_g$ is an ordinary conjugacy class of
$\mathcal{G}^\omega$ and $\mathcal{G}^\omega$ acts on $\mathcal{G}$ by right
multiplication. In particular, $\breve{\mathcal C}_g$ can be identified with a
coadjoint orbit of the subgroup \eqref{Gomegagen}, with the standard linear
Poisson structure coinciding with that induced by pull-back
$B|_{\mathcal{G}^\omega}$ of the Neveu-Schwarz $B$-field when $H=0$.
Semiclassically, after quantisation $\mathcal{C}_g^\omega=\mathcal{G}
\times_{\mathcal{G}^\omega}\breve{\mathcal C}_g$ becomes a trivial bundle with
noncommutative fibers labelled by irreducible modules $\breve{V}_g^\omega$ over
the group $\mathcal{G}^\omega$ but with a classical base space $\mathcal{G} /
\mathcal{G}^\omega$ \cite{AFQS1}. The associative noncommutative algebra of
functions on the worldvolume in this case is thus given by
\begin{equation}
  \label{NCtwistalg}
  \mathcal{A}\left(\mathcal{C}_g^\omega\right)=\Bigl({\rm C}^\infty\bigl(
  \mathcal{G}\bigr)\otimes{\rm End}\bigl(\breve{V}_g^\omega\bigr)
  \Bigr)^{\mathcal{G}^\omega}
\end{equation}
where the superscript denotes the $\mathcal{G}^\omega$-invariant part and
$\mathcal{G}^\omega$ acts on ${\rm C}^\infty(\mathcal{G})$ through the induced
derivative action of right isometries of $\mathcal{G}$. Again, the worldvolume
algebra \eqref{NCtwistalg} carries a natural action of the group $\mathcal{G}$,
now through the induced action on ${\rm C}^\infty(\mathcal{G})$ by right
isometries.

If $\omega=\1$, then $\mathcal{G}^\omega=\mathcal{G}$ and \eqref{NCtwistalg}
reduces to the definition \eqref{NCalgconjclass}. If the module
$\breve{V}_g^\omega$ is one-dimensional (e.g. the trivial representation of
$\mathcal{G}^\omega$), then ${\rm End}(\breve{V}_g^\omega)\cong\C$ and the
algebra \eqref{NCtwistalg} becomes the commutative algebra of functions
$\mathcal{A}(\mathcal{C}_g^\omega)\cong{\rm C}^\infty(\mathcal{G} /
\mathcal{G}^\omega)$ on $\mathcal{G}$ invariant under the action of the subgroup
$\mathcal{G}^\omega\subset\mathcal{G}$ by right isometries. It is this latter
situation that we shall discover in the present case. All twisted D-branes in
$\NW_6$ support {\it commutative} worldvolume geometries, given by the algebra
of functions on the classical twisted conjugacy class, consistent with the
remarks made in Section \ref{DiagNW}. This result will be supported by the fact
that all twisted branes have vanishing NS flux $H=0$, so that again
semi-classical quantisation applies. However, for certain lorentzian D-branes
that we shall encounter, there are some important subtleties hidden in the
structure of the worldvolume two-form fields that are invisible in the field
theoretic analysis outlined here.

\subsection{General Construction}
\label{GenConstrTwist}
The group of outer automorphisms of the Lie algebra $\mathfrak{n}_6$ is given by
${\rm Out}(\mathfrak{n}_6)=\Z_2\ltimes{\rm SU}(2) / \Z_2$ \cite{SF1}, where
the ${\rm SU}(2)$ factor is the transverse space rotational symmetry described
in the previous section. There are thus two families of outer automorphisms
$\omega_\pm^S$ parametrised by a matrix
\begin{equation}
  \label{SU2matrixouter}
  S=\begin{pmatrix}a&b\\-\overline{b}&\overline{a}\end{pmatrix} \in 
  {\rm SU}(2)
\end{equation}
with $a,b\in\C$ obeying $|a|^2+|b|^2=1$ and with the identification
$S\sim-S$. Corresponding to the identity element of $\Z_2$, the automorphism
$\omega_+^S$ acts on a group element \eqref{NW6globalcoords} via a rotation in
the transverse space as
\begin{equation}
  \label{omegapdef}
  \omega_+^S(g)=g\left(x^+,x^-,S\mz,\overline{S} \overline{\mz} \right)
\end{equation}
with $S\neq\1$ (or else the automorphism is trivial). As a consequence, the
invariant subgroup \eqref{Gomegagen} in this case is given by
\begin{equation}
  \label{invsubgpp}
  \mathcal{N}_6^{ \omega_+^S} = \bigl\{g\left(x^+,x^-,\mbf0,\mbf0\right)
  \in\mathcal{N}_6 \bigm| x^\pm\in\R\bigr\} \cong \R^2
\end{equation}
This is the two-dimensional abelian group of translations of the light-cone in
$\NW_6$, and its irreducible representations are all one-dimensional. Thus the
quantised worldvolume algebra \eqref{NCtwistalg} in this case is
$\mathcal{A}(\mathcal{C}_g^{\omega_+^S})\cong{\rm C}^\infty(\E^4)$
corresponding to {\it commutative} euclidean D3-branes. We shall explicitly
construct these branes in what follows, showing that the group \eqref{invsubgpp}
is isomorphic to the stabiliser of the twisted conjugacy class and hence its
points (corresponding to its conjugacy classes) label the locations of the
various D3-branes in $\NW_6$. This characterisation is consistent with the fact
that only the (trivial) one-dimensional subspace of $V^{p^+,p^-}$ spanned by the
ground state vector $|0,0;p^+,p^-\rangle$ is invariant under the action of the
automorphism \eqref{omegapdef}.

The other class of automorphisms $\omega_-^S$ combines the transverse space
rotations with the non-trivial element of $\Z_2$ which acts by charge
conjugation on the generators of $\mathfrak{n}_6$ to give
\begin{equation}
  \label{omegamdef}
  \omega_-^S(g)=g\left(-x^+,-x^-,S \overline{\mz},\overline{S} \mz \right)
\end{equation}
It defines an isomorphism $V^{p^+,p^-}\leftrightarrow\widetilde{V}{}^{p^+,p^-}$
on irreducible representations. The invariant subgroup \eqref{Gomegagen} in this
case is determined by the equation $S \overline{\mz}=\mz$ for
$\mz\in\C^2$. The only solution of this equation is $\mz=\mbf0$, unless
the parameter $b$ in \eqref{SU2matrixouter} is purely imaginary in which case
the subgroup reduces to the two-dimensional abelian group generated by the
elements
\begin{equation}
  \label{2Dabgp}
  \P_S=a \P^++\P^-+b \Q^+ \qquad\qquad
  \Q_S=\overline{a} \Q^++\Q^-+b \P^+
\end{equation}
Thus
\begin{equation}
  \label{invsubgp}
  \mathcal{N}_6^{ \omega_-^S} = \bigl\{g\left(0,0,\mz,\overline{\mz} \right)
  \in\mathcal{N}_6 \bigm| \mz=S \overline{\mz}\in\C^2\bigr\} \cong
  \begin{cases}
    \R^2  &b\in i\R\\
    \{\1\}  &{\rm otherwise}
  \end{cases}
\end{equation}
is again always an abelian group (of transverse space translations) with only
one-dimensional irreducible representations, corresponding to commutative
worldvolume geometries \eqref{NCtwistalg}. As we show explicitly below, the
first instance corresponds to a class of commutative, Lorentzian D3-branes
wrapping $\CW_4$ $\subset$ $\NW_6$, i.e.
$\mathcal{A}(\mathcal{C}_g^{\omega_-^S})$ $\cong$ ${\rm C}^\infty(\CW_4)$, with
the points of \eqref{invsubgp} again parametrising the locations of the branes
in $\NW_6$. The second case represents a family of commutative, spacetime
filling lorentzian D5-branes each isometric to $\CW_6$, i.e.
$\mathcal{A}(\mathcal{C}_g^{\omega_-^S})$ $\cong$ ${\rm
  C}^\infty(\mathcal{N}_6)$. These results are consistent with the fact that
only the self-dual null brane modules $V_{\alpha,\beta}^{0,0}$ and
$V_{\alpha,\beta}^{0,\frac\theta2}$ are invariant under the action of the
automorphism \eqref{omegamdef}.

As in Section \ref{GenConstrUntwist}, one can now straightforwardly work out the
twisted adjoint actions ${\rm Ad}^{\omega_\pm^S}_{g(x^+,x^-,\mz,\overline{\mz}
  )} g(x^+_0,x^-_0,\mz_0, \overline{\mz}_0)$ corresponding to a fixed point
$(x_0^+,x_0^-,\mz_0)\in\NW_6$ \cite{FS1,SF1}. For the automorphism
\eqref{omegapdef} the twisted conjugacy classes can be written explicitly as the
submanifolds
\begin{eqnarray}
  \label{conjclassgenp}
  \mathcal{C}^{\omega_+^S}_{(x_0^+,x^-_0,\mz_0)}&=&
  \Bigl\{\bigl(x^+_0 , x^-_0+\frac\theta2
    {\rm Im}\bigl[\overline{\mz}_0^{ \top} ( e^{-\frac{ i\theta}2 x_0^+}
    S+ e^{\frac{ i\theta}2 x_0^+} \1)\mz  e^{-\frac{ i\theta}2 x^+}\\\nn
    &&\qquad\qquad\qquad\qquad+\frac12 \mz^\top
    ( e^{- i\theta x_0^+} S- e^{ i\theta x_0^+} \1)
    \overline{\mz} \bigr] ,\\\nn
  &&\qquad e^{\frac{ i\theta}2 (x^+-x_0^+)} (S- e^{ i\theta x_0^+} \1)\mz+
      e^{ i\theta x^+} \mz_0 \bigr) \biggm| x^+\in\S^1 , \mz\in\C^2\Bigr\}
\end{eqnarray}
Again $x_0^+$ is an orbit invariant, and the corresponding branes are
thus euclidean. In fact, except for the nature of their noncommutative
worldvolume geometry, these branes are completely analogous to the
branes described in the previous section. For the automorphism
\eqref{omegamdef}) one finds instead the submanifolds
\begin{eqnarray}
  \label{conjclassgenm}
  \mathcal{C}^{\omega_-^S}_{(x_0^+,x^-_0,\mz_0)}&=&
  \Bigl\{\bigl(x^+_0+2x^+ , x^-_0+2x^--\theta
  {\rm Im}\bigl[\overline{\mz}{}_0^{ \top} \mz  e^{\frac{ i\theta}2 
    (x_0^++x^+)}\\\nn
  &&\qquad\qquad\qquad\qquad\qquad\qquad\qquad-( \overline{\mz}{}_0^{ \top}+
  e^{\frac{ i\theta}2 (x^++x_0^+)} \overline{\mz}^{ \top}) S
  \overline{\mz}  e^{\frac{ i\theta}2 (x^++x_0^+)}
  \bigr], \\\nn
  &&\qquad\mz_0-S \overline{\mz}  e^{-\frac{ i\theta}2 (x^++x_0^+)}+\mz 
  e^{\frac{ i\theta}2 (x^++x_0^+)}\bigr) \biggm|
  x^+\in\S^1 , x^-\in\R , \mz\in\C^2\Bigr\}
\end{eqnarray}
wrapped by lorentzian branes. We will now briefly describe the
supergravity fields supported by each of these classes of branes.

\subsection{Euclidean D3-Branes}
\label{EuclD3Twist}
The geometry of the twisted conjugacy classes \eqref{conjclassgenp} is
determined by the complex map on $\C^2\to\C^2$ defined by
\begin{equation}
  \label{complexmapeucl}
  \mz':= e^{\frac{ i\theta}2 (x^+-x_0^+)} \left(S- e^{ i\theta x_0^+} \1
  \right)\mz+ e^{ i\theta x^+} \mz_0
\end{equation}
Whenever the $2\times2$ matrix $S- e^{ i\theta x_0^+} \1$ is invertible, this map
defines an isomorphism of linear spaces and one has
$\mathcal{C}^{\omega_+^S}_{(x_0^+,x^-_0,\mz_0)}\cong\E^4$ with the same
worldvolume fields as in \eqref{metricpieucl4} and \eqref{NSconj}, the
$x^+$-dependence of \eqref{conjclassgenp} cancelling out. However, although
they are qualitatively the same as the euclidean D3-branes of Section
\ref{CGED3B}, the worldvolume two-forms on these branes are different. The
stabiliser of the twisted conjugacy class in this case is parametrised as the
cylindrical submanifold
\begin{equation}
  \label{stabE4twist}
  \mathcal{Z}^{\omega_+^S}_{(x_0^+,x^-_0,\mz_0)}=\Biggl\{
      \left(x^+,x^-, e^{\frac{ i\theta}2 (x_0^+-x^+)} 
      (1- e^{ i\theta x^+}) (S- e^{ i\theta x_0^+} 
      \1)^{  -1}\mz_0\right) \Biggm|
      \genfrac{}{}{0pt}{}{x^+\in\S^1}{x^-\in\R}
      \Biggr\}
\end{equation}
from which we may compute the abelian gauge field fluxes \eqref{2formfamily} to
be
\begin{equation}
  \label{E4F6twist}
  F^{(\zeta)}_6\bigm|_{\mathcal{C}^{\omega_+^S}_{(x_0^+,x^-_0,\mz_0)}}=
  2 i x_0^+  d\overline{\mz}{}^{ \top}\wedge d\mz
\end{equation}
Thus the gauge-invariant two-forms $\mathcal{F}_6^{(\zeta)}$ vanish on these
twisted conjugacy classes, leading to the anticipated commutative worldvolume
geometry.

We can also understand the origin of these branes through the Penrose-G\"uven
limit of Section \ref{ApplNW} by considering the commuting isometric embedding
diagram
\begin{equation}
  \label{E4commdiagH2}
  \begin{CD}
   @.\\
    \AdS_3 \times \S^3             @>\text{PGL}>> \NW_6\\
    \text{$\imath^{ \prime}  $}@AAA
@AAA\text{$\widetilde{\imath}^{ \prime}$}\\
           \Hyp^2\times\S^2    @>\text{PGL$^{ \prime}$}>> \E^4\\
   @.
  \end{CD}
\end{equation}
analogous to \eqref{E4commdiag}. The hyperbolic plane $\Hyp^2$ is wrapped by
symmetric instantonic D-strings in $\AdS_3$ \cite{Stanciu3}, corresponding to
conjugacy classes of the group ${\rm SU}(1,1)$, and it is obtained from the
intersection of the three-dimensional hyperboloid \eqref{AdS3quadric} with the
affine hyperplane $x^1=\tau=0$. Now the pull-back of the $B$-field
\eqref{AdS3S3Bfield} to the $\Hyp^2$ worldvolume vanishes, leading to a
vanishing gauge-invariant two-form.

\subsection{Spacetime Filling D5-Branes}
\label{LorD5}
Let us now examine the twisted conjugacy classes \eqref{conjclassgenm}. By
defining $y^+:=x_0^++2x^+$ and $\mw:= e^{\frac{ i\theta}4 (y^++x_0^+)} \mz$ we may
express them as the submanifolds
\begin{equation}
  \label{conjmrew}
  \mathcal{C}_{\mz_0}^S=\left.\left\{(y^+,y^-,\mz_0-S \overline{\mw}+\mw) 
    \right| y^+\in\S^1 , y^-\in\R , \mw\in\C^2\right\}
\end{equation}
The geometry of the twisted conjugacy class \eqref{conjmrew} is
thereby determined by the {\it real} linear transformation on
$\R^4\to\R^4$ defined by
\begin{equation}
  \label{reallinmapm}
  \mw':=S \overline{\mw}-\mw
\end{equation}
whose determinant is straightforwardly worked out to be $4({\rm Re} b)^2$. As
long as the parameter $b$ in \eqref{SU2matrixouter} has a non-zero real part,
the map \eqref{reallinmapm} is a linear isomorphism and we can write the induced
geometry in terms of the new transverse space coordinates $\mw'\in\C^2$.
After an irrelevant shift, the pull-back of the plane wave metric
\eqref{NW6metricBrink} to the twisted conjugacy class \eqref{conjmrew} is given
by
\begin{equation}
  \label{D5metricpull}
  d s_6^2\bigm|_{\mathcal{C}_{\mz_0}^S}=2  d y^+  d y^-+\left| d
    \mw' \right|^2-\frac{\theta^2}4 \left|\mw' 
  \right|^2 \left( d y^+\right)^2
\end{equation}
and the corresponding worldvolumes are therefore wrapped by spacetime
filling D-branes isometric to $\CW_6$. From the identity
\begin{equation}
  \label{dwidentity}
  d\overline{\mw}{}^{ \prime \top}\wedge d\mw'= d\overline{\mw}{}^{ \top}
  \wedge\left(\1-\overline{S}{}^{ \top} S\right)  d\mw
\end{equation}
and the unitarity of the matrix $S\in{\rm SU}(2)$ it follows that the pull-back
of the NS background \eqref{NW6Bfield} is trivial,
\begin{equation}
  \label{D5Bpull}
  H_6\bigm|_{\mathcal{C}_{\mz_0}^S} = 0 = 
  B_6\bigm|_{\mathcal{C}_{\mz_0}^S}
\end{equation}
Furthermore, since the corresponding stabiliser in this case is euclidean,
having $x^+=0$, it is elementary to see that the pull-backs of the
gauge-invariant two-forms also vanish,
\begin{equation}
  \label{D5F6pull}
  \mathcal{F}^{(\zeta)}_6\bigm|_{\mathcal{C}_{\mz_0}^S} = 0
\end{equation}
In perfect agreement with the general analysis of Section \ref{GenConstrTwist},
these D5-branes thus all carry commutative worldvolume geometries.

As described in Section \ref{ApplNW}, these D-branes originate through the
Penrose limit of D5-branes isometric to $\AdS_3\times\S^3$. The latter do {\it
  not} correspond to symmetric D-branes of the ${\rm SU}(1,1)\times{\rm SU}(2)$
group and, unlike the symmetric D5-branes obtained here which are spacetime
filling branes of the six-dimensional plane wave, they do not fill the whole
$\AdS_3\times\S^3$ target space. They wrap only a three-dimensional submanifold
of the $\S^3$ component given by the complement in $\S^3$ of the disjoint union
of two circles \cite{Stanciu3}. While the Penrose limit de-compactifies the
branes to spacetime filling ones, the Neveu-Schwarz background vanishes due to
the non-trivial embedding $\imath$ represented through the isometric embedding
diagram
\begin{equation}
  \label{CW6commdiag}
  \begin{CD}
   @.\\
    \AdS_3 \times \S^3             @>\text{PGL}>> \NW_6\\
    \text{$\imath$}   @AAA @AAA\text{$\widetilde{\imath}$} \\
    \AdS_3 \times \S^3             @>\text{PGL}>> \CW_6\\
   @.
  \end{CD}
\end{equation}
with $\widetilde{\imath}$ constructed as above.

\subsection{Lorentzian D3-Branes}
\label{LorD3}
We now describe the situations wherein the linear maps defined above are not of
maximal rank, beginning with \eqref{reallinmapm}. In this case the parameter
$b\in i\R$ is purely imaginary, and the map has real rank $2$ so that the
variable $\mw^{\prime \top}=(w_1',w_2')$ lives in a one-dimensional complex
subspace of $\C^2$. When ${\rm Re} a=a=1$, one has $b=0$ and $\mw'\in
i\R^2$. Defining $z\in\C$ by $z:=2 {\rm Im} w_1+2 i {\rm Im} w_2$, it
follows that $|\mw' |=|z|$ and the metric \eqref{D5metricpull} coincides with
the usual Cahen-Wallach geometry \eqref{NW4metricBrink} of $\CW_4$. When ${\rm
  Re} a<1$, we may coordinatise the twisted conjugacy class \eqref{conjmrew} by
$\mw\in\R^2$. Then by defining
\begin{equation}
  \label{zD3def}
  z:=\sqrt{2(1-{\rm Re} a)} \left(w_1- \frac b{2(1-{\rm Re} a)} 
    w_2\right)+ i\sqrt{2(1-{\rm Re} a)- \frac{b^2} {2(1-{\rm Re} a)} } w_2
\end{equation}
one finds $|\mw' |=|z|$ and \eqref{D5metricpull} again reduces to the form
\eqref{NW4metricBrink}. The identity \eqref{dwidentity} once more implies the
vanishing \eqref{D5Bpull} of the NS background, so that the twisted conjugacy
classes \eqref{conjmrew} are wrapped by curved lorentzian D3-branes isometric to
$\CW_4$. All of this is in perfect harmony with the general findings of Section
\ref{GenConstrTwist}.

The $\AdS_3\times\S^3$ origin of these D3-branes is described by
\eqref{CW4commdiag}. However, now an apparent discrepency arises. Using
\eqref{dwidentity} and the fact that the matrix \eqref{SU2matrixouter} is
symmetric whenever $b\in i\R$, one finds a non-vanishing pull-back of the
worldvolume flux \eqref{NW6tildegaugeinv} given by
\begin{equation}
  \label{non0D3flux}
  \mathcal{F}_6\bigm|_{\mathcal{C}_{\mz_0}^S}=- \frac{ i\theta}2  
  \cot^2 \frac{\theta y^+}2   d y^+\wedge\left(\mz_0^\top 
    d\overline{\mw}^{ \prime}-\overline{\mz}_0^{ \top} 
    d\mw' \right)
\end{equation}
This worldvolume two-form depends on the locations of the branes in $\NW_6$, and
it should induce a spacetime noncommutative geometry on the D3-brane
worldvolume. However, this flux corresponds to a worldvolume {\it electric}
field, rather than a magnetic field, and D-branes in electric backgrounds do not
exhibit a well-defined decoupling of the massive string states. Instead of field
theories, such backgrounds lead to noncommutative open string theories
\cite{GMMS1,SST1} which lie beyond the scope of the semi-classical analysis at
the beginning of this section. Such a noncommutativity agrees with the
non-trivial boundary three-point couplings \cite{DK2} on the symmetric
lorentzian D-membranes \cite{FS1} wrapping the pp-waves
$\CW_3\hookrightarrow\NW_4$ defined by the volumes ${\rm Im} z={\rm constant}$
in the above. Given the origin of the null worldvolume flux \eqref{non0D3flux}
from Section \ref{ApplNW} and the mechanism described after \eqref{E4commdiag},
this noncommutative open string theory could provide a curved space analog of
the expected flat space S-duality with $(p,q)$ strings.

\subsection{D-Strings}
\label{DStrings}
Let us now turn to the degenerate cases of the map \eqref{complexmapeucl}. The
vanishing of the determinant of the $2\times2$ matrix $S- e^{ i\theta x_0^+} \1$
shows that these branes arise when the fixed light-cone time coordinate obeys
the equation
\begin{equation}
  \label{nullcoordeq}
  \cos\theta x_0^+={\rm Re} a
\end{equation}
Since $|a|<1$ in this case, these worldvolumes lie in the set of conjugate free
points $x_0^+\neq0,\frac\pi\theta$ of the Rosen plane wave geometry and $S- e^{
  i\theta x_0^+} \1$ is of complex rank $1$. Explicitly, with $\mz^\top=(z,w)$ and
$\xi\in\C$ defined by $\xi:=b w- i(\sin\theta x_0^+-{\rm Im} a) z$, one finds
using \eqref{nullcoordeq} that
\begin{equation}
  \label{Sdegexpl}
  \left(S- e^{ i\theta x_0^+} \1\right)\mz=
  \begin{pmatrix}
    \xi\\[4mm]
    \frac{ i\overline{b}}{\sin\theta x_0^+-{\rm Im} a} \xi
  \end{pmatrix}
\end{equation}
If $\mz_0=(S- e^{ i\theta x_0^+} \1)\mw_0$ for some fixed $\mw_0\in\C^2$,
then a simple coordinate redefinition in \eqref{complexmapeucl} enables us to
parametrise the twisted conjugacy class by a single complex variable. The
pull-back of the $\NW_6$ metric is non-degenerate and the classes in this case
are wrapped by euclidean D1-branes. As in Section \ref{EuclD3Twist}, all
worldvolume fields in this case are easily found to be trivial. These branes do
not explicitly appear in the general analysis of Section \ref{GenConstrTwist},
and like the null branes of Section \ref{Null} their quantised worldvolume
geometry, although commutative, differs from the classical one. They can
nevertheless be regarded as subclasses of the twisted euclidean D3-branes
constructed in Section \ref{EuclD3Twist}. In particular, they originate from
symmetric euclidean D-strings in $\AdS_3\times\S^3$, either wrapping
$\Hyp^2\subset\AdS_3$ and sitting at a point in $\S^3$ or wrapping
$\S^2\subset\S^3$ and sitting at a point in $\AdS_3$.

\subsection{D-Membranes}
\label{DMem}
Finally, if $\mz_0\notin{\rm im}(S- e^{ i\theta x_0^+} \1)$ on
$\C^2$, then from \eqref{complexmapeucl} it follows that the
twisted conjugacy class \eqref{conjclassgenp} is isometric to
$\S^1\times\E^2$. The metric $ d s_6^2$ restricts
non-degenerately, and once again all worldvolume form fields are
trivial. The orbits are thus wrapped by euclidean D2-branes, which can
likewise be regarded as subclasses of the D3-branes in
Section \ref{EuclD3Twist}. Like the spacetime filling D-branes of
Section \ref{LorD5}, these branes do not originate from symmetric
D-branes in $\AdS_3\times\S^3$, but rather from either of the
trivially embedded $\AdS_3$ or $\S^3$ submanifolds.

\section{The Dolan-Nappi Model}
\label{DNModel}
Introducing to $\NW_6$ the one-form
\begin{equation}
  \label{Lambdadef}
  \Lambda:=- i\left(\theta^{-1} x_0^-+\theta x^+\right)
  \left(\mz^\top  d\overline{\mz}-\overline{\mz}^{\top}d\mz\right)
\end{equation}
on the null hypersurfaces of constant $x^-=x_0^-$, we may compute the
corresponding two-form gauge transformation of the $B$-field in
\eqref{NW6Bfield} to get
\begin{equation}
  \label{B6gaugeequiv}
  B_6^\Lambda:=B_6+ d\Lambda=- i\theta  d x^+\wedge\left(
    \mz^\top  d\overline{\mz}-\overline{\mz}{}^{ \top}  d\mz
  \right)+2 i\theta^{-1} x_0^-  d\overline{\mz}{}^{ \top}
  \wedge d\mz
\end{equation}
With $x_0^+=0$ and restricted to the four-dimensional hypersurface defined by
$\mz^\top=(z,0)$, the metric \eqref{NW4metricBrink} and NS potential
\eqref{B6gaugeequiv} coincide with those of the Dolan-Nappi model \cite{DN1}
describing a (non-symmetric) D3-brane with the complete NS-supported geometry of
$\NW_4$.

In \cite{HT1} this geometry is realised as a null Melvin twist of a flat
commutative D3-brane with twist parameter $\frac\theta2$ (in string units
$\alpha'=1$), leading to the Melvin universe with a boost. Despite the
non-vanishing null NS three-form flux of $\NW_4$, it can be argued from this
realisation that the usual flat space Seiberg-Witten formulae
\eqref{SWThetagen}, \eqref{SWopenmetgen} hold in this closed string background
with $F=0$ a consistent solution to the corresponding Dirac-Born-Infeld
equations of motion. The isometry with respect to which the background is
twisted corresponds to the R-symmetry of the D3-brane worldvolume field theory,
which thereby becomes a non-local theory of dipoles whose length is proportional
to the R-charge. The open string metric \eqref{SWopenmetgen} correctly captures
the non-local dipole-like open string dynamics on the D3-brane.

Extrapolating this argument to $x_0^-\neq0$ and to the full six-dimensional
spacetime $\NW_6$, a straightforward calculation gives the Seiberg-Witten
bi-vector \eqref{SWThetagen} with $F=0$ for the background
\eqref{NW6metricBrink}, \eqref{B6gaugeequiv} as
\begin{equation}
  \label{ThetaLambda}
  \Theta^\Lambda=- \frac{2 i\theta}{\theta^2+\left(x_0^-\right)^2}  
  \left[\theta^2 \partial_-\wedge\left(\mz^\top \bfd-
      \overline{\mz}{}^{ \top} \overline{\bfd} \right)+4x_0^- 
    \bfd^\top\wedge\overline{\bfd} \right]
\end{equation}
while the corresponding open string metric \eqref{SWopenmetgen} is given by
\begin{equation}
  \label{GopenLambda}
  G_{\rm o}^\Lambda=2  d x^+  d x^-+ \frac{\theta^2+\left(x_0^-
    \right)^2}{\theta^2}  | d\mz|^2+2 i x_0^- \left(\mz^\top  d
    \overline{\mz}-\overline{\mz}{}^{ \top}  d\mz\right)  d x^+
\end{equation}
Since \eqref{ThetaLambda} is degenerate on the whole $\NW_6$ spacetime, it does
not define a symplectic structure. Generally, if the components of a bi-vector
$\Theta:=\Theta^{ij} \partial_i\wedge\partial_j$ obey
\begin{equation}
  \label{ThetaPoissoncondn}
  \Theta^{il} \partial_l\Theta^{jk}+\Theta^{jl} \partial_l\Theta^{ki}+
  \Theta^{kl} \partial_l\Theta^{ij}=0
\end{equation}
for all $i,j,k$, then $\Theta$ defines a Poisson structure, i.e. it is a Poisson
bi-vector and \eqref{ThetaPoissoncondn} is equivalent to the Jacobi identity for
the corresponding Poisson brackets. If in addition $\Theta$ is invertible, then
\eqref{ThetaPoissoncondn} is equivalent to the required closure condition $
d(\Theta^{-1})=0$ for a symplectic two-form. It is easily checked that
\eqref{ThetaLambda} satisfies \eqref{ThetaPoissoncondn} and hence that it
defines a Poisson bi-vector. In the flat space limit $x_0^-\to0$ of
\eqref{GopenLambda}, the corresponding quantisation of $\NW_6$ is given by the
associative Kontsevich $\star$-product \cite{Kont1} in this case.

The important feature of the noncommutativity parameter \eqref{ThetaLambda} is
that it is time independent, though non-constant. If we think of the light-cone
position $x^-$ as being dual to the Nappi-Witten generator $\J$, then the form
of \eqref{ThetaLambda} agrees with its representation in \eqref{nullfnsgens} and
\eqref{DeltainvD3gens}. On the other hand, the calculation of \cite{DN1}
provides evidence for a time-dependent Poisson bi-vector in the original closed
string background \eqref{NW6Bfield}. To make this precise, however, one would
require a detailed understanding of the worldvolume stabilising flux on the
$\NW_6$ brane, which is difficult to determine for non-symmetric D-branes. The
noncommutativity parameter and open string metric in the decoupling limit of
D5-branes in Nappi-Witten spacetime $\NW_6$ are thereby presently given by
\eqref{ThetaLambda} and \eqref{GopenLambda}. In particular, at the special value
$x_0^-=\theta$ and with the rescaling $\mz\to\sqrt{2/\theta \tau} \mz$, the metric
\eqref{GopenLambda} becomes that of $\CW_6$ in global coordinates analogous to
\eqref{NW4metricNW}, while the non-vanishing Poisson brackets corresponding to
\eqref{ThetaLambda} read
\begin{eqnarray}
  \label{Poissonspecial}
  \left\{z_a, \overline{z}_b\right\}&=&2 i\theta \tau \delta_{ab}\\\nn
  \left\{x^-, z_a\right\}&=&- i\theta z_a\\\nn
  \left\{x^-, \overline{z}_a\right\}&=& i\theta \overline{z}_a
\end{eqnarray}
for $a,b=1,2$. The Poisson algebra thereby coincides with the Nappi-Witten Lie
algebra $\mathfrak{n}_6$ in this case and the metric on the branes with the
standard curved geometry of the pp-wave. In the semi-classical flat space limit
$\theta\to0$, the quantisation of the brackets \eqref{Poissonspecial} thereby
yields a noncommutative worldvolume geometry on D5-branes wrapping $\NW_6$ which
can be associated with a quantisation of $\mathfrak{n}_6$ (or more precisely of
its dual $\mathfrak{n}_6^*$).

With a slight abuse of notation, we will denote the central coordinate $\tau$ as
the plane wave time coordinate $x^+$. Our semi-classical quantisation will then
be valid in the small time limit $x^+\to0$.

%%% Local Variables: 
%%% TeX-command-default: "LaTeX"
%%% TeX-master: "main.tex"
%%% End:
