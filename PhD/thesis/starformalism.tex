\chapter{$\star$-Products, Derivatives and Integrals}
\label{star}

\section{Quantisation}
\label{NWQuant}
Our starting point in describing the noncommutative geometry of $\NW_6$ will be
at the algebraic level. We will consider the deformation quantisation of the
dual $\mathfrak{n}^\vee$ to the Lie algebra $\mathfrak{n}$. Naively, one may
think that the easiest way to carry this out is to compute $\star$-products on
the pp-wave by taking the Penrose limits of the standard ones on $\S^3$ and
$\AdS_3$ (or equivalently by contracting the standard quantisations of the Lie
algebras ${\rm su}(2)$ and ${\rm sl}(2,\R)$). However, some quick calculations
show that the induced $\star$-products obtained in this way are divergent in the
infinite volume limit, and the reason why is simple. While the standard
In\"on\"u-Wigner contractions hold at the level of the Lie algebras
\cite{SF1}, they need not necessarily map the corresponding universal
enveloping algebras, on which the quantisations are performed. This is connected
to the phenomenon that twisted conjugacy classes of branes are not necessarily
related by the Penrose-G\"uven limit \cite{Halliday:2005zt}. We must therefore resort to a
more direct approach to quantising the spacetime $\NW_6$.

For notational ease, we will write the algebra $\mathfrak{n}$ in the
generic form
\begin{equation}
  \label{n6genform}
  [\X_a,\X_b]= i \theta C_{ab}^{  c} \X_c  
\end{equation}
where $(\X_a):=\theta (\J,\T,\P^i_\pm)$ are the generators of $\mathfrak{n}$ and
the structure constants $C_{ab}^{ c}$ can be gleamed off from \eqref{NW4algdef}.
The algebra \eqref{n6genform} can be regarded as a formal deformation
quantisation of the Kirillov-Kostant Poisson bracket on $\mathfrak{n}^\vee$ in
the standard coadjoint orbit method. Let us identify $\mathfrak{n}^\vee$ as the
vector space $\R^6$ with basis $\X_a^\vee:=\langle\X_a,-\rangle:\mfn\to\R$ dual
to the $\X_a$. In the algebra of polynomial functions
$\C(\mathfrak{n}^\vee)=\C(\R^6)$, we may then identify the generators $\X_a$
themselves with the coordinate functions
\begin{eqnarray}
  \label{Xacoordfns}
  \X_\J(\mx)&=&x_\T = x^-   \nn\\ \X_\T(\mx)
  &=&x_\J = x^+   \nn\\ 
  \X_{\P_+^i}(\mx)&=&2x_{\P_-^i} = 2\overline{z}_i  
  \nn\\ \X_{\P_-^i}(\mx)&=&2x_{\P_+^i} = 2z_i
\end{eqnarray}
for any $\mx\in\mathfrak{n}^\vee$ with component $x_a$ in the $\X_a^\vee$
direction. These functions generate the whole coordinate algebra and their
Poisson bracket $\Theta$ is defined by
\begin{equation}
  \label{KKXadef}
  \Theta(\X_a,\X_b)(\mx)=\mx\bigl([\X_a,\X_b]\bigr)
  \qquad\qquad\forall\mx\in\mathfrak{n}^\vee  
\end{equation}
Therefore, when viewed as functions on $\R^6$ the Lie algebra generators have
a Poisson bracket given by the Lie bracket, and their quantisation is provided
by \eqref{n6genform} with deformation parameter $\theta$. We will explore
various aspects of this quantisation and derive several (equivalent) star
products on $\mfn^\vee$.

\section{Gutt Products}
\label{StarProds}
The formal completion of the space of polynomials $\C(\mfn^\vee)$ is the algebra
${\rm C}^\infty(\mathfrak{n}^\vee)$ of smooth functions on $\mathfrak{n}^\vee$.
There is a natural way to construct a $\star$-product on the cotangent bundle
$T^\vee\mathcal{N}\cong\mathcal{N}\times\mathfrak{n}^\vee$, which naturally
induces an associative product on ${\rm C}^\infty(\mfn^\vee)$. This induced
product is called the Gutt product. The Poisson bracket defined by
\eqref{KKXadef} naturally extends to a Poisson structure
$\Theta:\CC^\infty(\mfn^\vee)\times\CC^\infty(\mfn^\vee)\to\CC^\infty(\mfn^\vee)$
defined by the Kirillov-Kostant bi-vector
\begin{equation}
  \label{KKbivector}
  \Theta=\frac12C_{ab}^{  c} x_c \d^a\wedge\d^b
\end{equation}
where $\d^a:=\frac{\d}{\d x_a}$. The Gutt product constructs a quantisation of
this Poisson structure. It is equivalent to the Kontsevich $\star$-product in
this case \cite{Dito1}, and by construction it keeps that part of the Kontsevich
formula which is associative \cite{Shoikhet1}. In general, within the present
context, the Gutt and Kontsevich deformation quantisations are only identical
for nilpotent Lie algebras.

The algebra $\C(\mfn^\vee)$ of polynomial functions on the dual to the Lie
algebra is naturally isomorphic to the symmetric tensor algebra $S(\mfn)$ of
$\mfn$. By the Poincar\'e-Birkhoff-Witt theorem, there is a natural isomorphism
$\Omega:S(\mfn)\to U(\mfn)$ with the universal enveloping algebra $U(\mfn)$ of
$\mfn$. Using the above identifications, this extends to a canonical isomorphism
\begin{equation}
  \label{Sigmaiso}
  \Omega : \CC^\infty\left(\R^6\right) \longrightarrow \overline{
    U(\mfn)^\C}
\end{equation}
defined by specifying an ordering for the elements of the basis of monomials for
$S(\mfn)$, where $\overline{U(\mfn)^\C}$ denotes a formal completion of
the complexified universal enveloping algebra
$U(\mathfrak{n})^\C:=U(\mathfrak{n})\otimes\C$. Denoting this ordering
by $\NO-\NO$, we may write this isomorphism symbolically as
\begin{equation}
  \label{Sigmasymbol}
  \Omega(x_{a_1}\cdots x_{a_n})=\NO \X_{a_1}\cdots\X_{a_n} \NO
\end{equation}
The original Gutt construction \cite{Gutt1} takes the isomorphism $\Omega$ on
$S(\mfn)$ to be symmetrisation of monomials. In this case $\Omega(f)$ is usually
called the Weyl symbol of $f\in\CC^\infty(\R^6)$ and the symmetric ordering
$\NO-\NO$ of symbols $\Omega(f)$ is called Weyl ordering. In the following we
shall work with three natural orderings appropriate to the algebra $\mfn$.

The isomorphism \eqref{Sigmaiso} can be used to transport the algebraic
structure on the universal enveloping algebra $U(\mfn)$ of $\mfn$ to the algebra
of smooth functions on $\mfn^\vee\cong\R^6$ and give the $\star$-product defined
by
\begin{equation}
  \label{fstargSigma}
  f\star g:=\Omega^{-1}\bigl( \NO \Omega(f)\cdot\Omega(g) \NO
  \bigr) \qquad\qquad f,g\in\CC^\infty\left(\R^6\right)
\end{equation}
The product on the right-hand side of the formula \eqref{fstargSigma} is taken
in $U(\mfn)$, and it follows that $\star$ defines an associative, noncommutative
product. Moreover, it represents a deformation quantisation of the
Kirillov-Kostant Poisson structure on $\mfn^\vee$, in the sense that
\begin{equation}
  \label{xyPoisson}
  [x,y]_\star:=x\star y-y\star x= i
  \theta \Theta(x,y) \qquad\qquad x,y\in\C_{(1)}\left(\mfn^\vee\right)
\end{equation}
where $\C_{(1)}(\mfn^\vee)$ is the subspace of homogeneous polynomials of
degree $1$ on $\mfn^\vee$. In particular, the Lie algebra relations
\eqref{n6genform} are reproduced by $\star$-commutators of the coordinate
functions as
\begin{equation}
  \label{xaxbstarcomm}
  [x_a,x_b]_\star= i\theta C_{ab}^{  c} x_c
\end{equation}
in accordance with the Poisson brackets \eqref{Poissonspecial} and the
definition \eqref{KKXadef}.

Let us now describe how to write the $\star$-product \eqref{fstargSigma}
explicitly in terms of a bi-differential operator $\hat{\mathcal{D}}:
\CC^\infty(\mfn^\vee)\times\CC^\infty(\mfn^\vee)\to \CC^\infty(\mfn^\vee)$
\cite{Kathotia1}. Using the Kirillov-Kostant Poisson structure as before, we
identify the generators of $\mfn$ as coordinates on $\mfn^\vee$. This
establishes, for small $s\in\R$, a one-to-one correspondence between group
elements $ e^{s \X}$, $\X\in\mfn$ and functions $ e^{s x}$ on $\mfn^\vee$.
Pulling back the group multiplication of elements $ e^{s \X}\in\mathcal{N}$ via
this correspondence induces a bi-differential operator $\hat{\mathcal{D}}$
acting on the functions $ e^{s x}$. Since these functions separate the points on
$\mfn^\vee$, this extends to an operator on the whole of
$\CC^\infty(\mfn^\vee)$.

To apply this construction explicitly, we use the following trick
\cite{MSSW1,BehrSyk1} which will also be useful for later considerations. By
restricting to an appropriate Schwartz subspace of functions
$f\in\CC^\infty(\R^6)$, we may use a Fourier representation
\begin{equation}
  \label{Fouriertransfdef}
  f(\mx)=\int\limits_{\R^6}\frac{\dd\mk}{(2\pi)^6} \tilde f(\mk) 
  e^{ i\mk\cdot\mx}
\end{equation}
This establishes a correspondence between (Schwartz) functions on $\mfn^\vee$ and
elements of the complexified group $\mathcal{N}^\C:=
\mathcal{N}\otimes\C$. The products of symbols $\Omega(f)$ may be computed
using \eqref{Sigmasymbol}, and the $\star$-product \eqref{fstargSigma} can be
represented in terms of a product of group elements in $\mathcal{N}^\C$ as
\begin{equation}
  \label{fstargFourier}
  f\star g=\int\limits_{\R^6}\frac{\dd\mk}{(2\pi)^6} 
  \int\limits_{\R^6}\frac{\dd\mq}{(2\pi)^6} \tilde f(\mk) 
  \tilde g(\mq) \Omega^{-1}\left( \NO  \NO  e^{ i k^a \X_a} \NO\cdot
    \NO  e^{ i q^a \X_a} \NO  \NO \right)
\end{equation}
Using the Baker-Campbell-Hausdorff formula, to be discussed below, we
may write
\begin{equation}
  \label{NOproductsBCH}
  \NO  \NO  e^{ i k^a \X_a} \NO\cdot\NO  e^{ i q^a \X_a} \NO  \NO=
  \NO  e^{ i D^a(\mk,\mq) \X_a} \NO
\end{equation}
for some function $\mD=(D^a):\R^6\times\R^6\to\R^6$. This enables us to
rewrite the $\star$-product \eqref{fstargFourier} in terms of a bi-differential
operator $f\star g:=\hat{\mathcal{D}}(f,g)$ given explicitly by
\begin{equation}
  \label{fstargbidiff}
  f\star g=f  e^{ i\mx\cdot[\mD( - i\overleftarrow{\bfd}
    , - i\overrightarrow{\bfd} )+ i\overleftarrow{\bfd}+ i
    \overrightarrow{\bfd} ]} g
\end{equation}
with $\bfd:=(\d^a)$. In particular, the $\star$-products of the coordinate
functions themselves may be computed from the formula
\begin{equation}
  \label{xastarxb}
  x_a\star x_b=\left.-\frac{\d}{\d k^a}\frac\d
    {\d q^b} e^{ i\mD(\mk,\mq)\cdot\mx}\right|_{\mk=\mq=\mbf0}
\end{equation}

Finally, let us describe how to explicitly compute the functions $D^a(\mk,\mq)$
in \eqref{NOproductsBCH}. For this, we consider the Dynkin form of the
Baker-Campbell-Hausdorff formula which is given for $\X,\Y\in\mfn$ by
\begin{equation}
  \label{eq:BCH:define}
  e^\X  e^\Y= e^{\mathrm{H}(\X:\Y)}
\end{equation}
where $\mathrm{H}(\X:\Y)=\sum_{n\geq1}\mathrm{H}_n(\X:\Y)\in\mfn$ is
generically an infinite series whose terms may be calculated through the
recurrence relation
\begin{eqnarray}
  \label{eq:BCH}
  &&(n+1) \mathrm{H}_{n+1}(\X:\Y) = \frac 12 \bigl[\X-\Y , 
  \mathrm{H}_n(\X:\Y)\bigr] \\\nn
  &&\qquad\qquad+\sum_{p=1}^{\lfloor n/2\rfloor}\frac{B_{2p}}{(2p)!} 
  \sum_{\substack{k_1,\ldots,k_{2p}> 0 \\ k_1+\ldots+k_{2p}=n }}
  \bigl[\mathrm{H}_{k_1}(\X:\Y) , \bigl[ \ldots , \bigl[
  \mathrm{H}_{k_{2p}}(\X:\Y) , \X+\Y\bigr]\ldots\bigr] \bigr]
\end{eqnarray}
with $\mathrm{H}_1(\X:\Y):=\X+\Y$. The coefficients $B_{2p}$ are the
Bernoulli numbers which are defined by the generating function
\begin{equation}
  \label{eq:BCH:K}
  \frac{s}{1- e^{-s}}-\frac s2-1=\sum_{p=1}^\infty\frac{B_{2p}}{(2p)!}
   s^{2p}
\end{equation}
The first few terms of the formula \eqref{eq:BCH:define} may be written
explicitly as
\begin{eqnarray}
  \label{eq:BCH:1}
  \mathrm{H}_1(\X:\Y)&=& \X+\Y \\\nn
  \mathrm{H}_2(\X:\Y)&=&\frac 12 \cb \X\Y \\\nn
  \mathrm{H}_3(\X:\Y)&=&\frac 1{12} \bigl[\X , \cb \X\Y \bigr]
  -\frac 1{12} \bigl[\Y , \cb \X\Y \bigr] \\\nn
  \mathrm{H}_4(\X:\Y)&=& -\frac 1{24} \bigl[\Y , \bigl[\X
   , \cb \X\Y \bigr] \bigr]
\end{eqnarray}
Terms in the series grow increasingly complicated due to the sum over partitions
in \eqref{eq:BCH}, and in general there is no closed symbolic form, as in the
case of the Moyal product based on the ordinary Heisenberg algebra, for the
functions $D^a(\mk,\mq)$ appearing in \eqref{NOproductsBCH}. However, at least
for certain ordering prescriptions, the solvability of the Lie algebra $\mfn$
enables one to find explicit expressions for the $\star$-product
\eqref{fstargbidiff} in this fashion. We will now proceed to construct three
such products.

\subsection{Time Ordering}
\label{TOP}

The simplest Gutt product is obtained by choosing a ``time ordering''
prescription in \eqref{Sigmasymbol} whereby all factors of the time translation
generator $\J$ occur to the far right in any monomial in $U(\mfn)$. It coincides
precisely with the global coordinatisation \eqref{NW4coords} of the
Cahen-Wallach spacetime, and written on elements of the complexified group
$\mathcal{N}^\C$ it is defined by
\begin{equation}
  \label{eq:time:defn}
  \Omega_*\left( e^{ i\mk\cdot\mx}\right)=
  \NOa  e^{ i k^a \X_a} \NOa:= e^{ i(p_i^+ \P^i_+
    +p_i^- \P^i_-)}  e^{ i j \J}  e^{ i t \T}
\end{equation}
where we have denoted $\mk:=(j,t,\mbp^\pm)$ with $j,t\in\R$ and
$\mbp^\pm=\overline{\mbp^\mp}=(p_1^\pm,p_2^\pm)\in\C^2$. To calculate the
corresponding $\star$-product $*$, we have to compute the group products
\begin{eqnarray}
  \label{TOgpprods}
  \NOa  \NOa  e^{ i k^a \X_a} \NOa\cdot\NOa  e^{ i k^{\prime a} \X_a}
  \NOa  \NOa
\end{eqnarray}
The simplest way to compute these products is to realise the six-dimensional Lie
algebra $\mfn$ as a central extension of the subalgebra $\mfs={\rm so}(2)$
$\ltimes$ $\R^4$ of the four-di\-mensional Euclidean algebra ${\rm iso}(4)$ $=$
${\rm so}(4)$ $\ltimes$ $\R^4$ \cite{SF1,FS1}. Regarding $\R^4$ as $\C^2$ (with
respect to a chosen complex structure), for generic $\theta\neq0$ the generators
of $\mfn$ act on $\mw\in\C^2$ according to the affine transformations $ e^{ i j
  \J}\cdot\mw= e^{-\theta j} \mw$ and $ e^{ i(p_i^+ \P^i_++p_i^-
  \P^i_-)}\cdot\mw=\mw+ i\theta \mbp^-$, corresponding to a combined rotation in
the $(12)$, $(34)$ planes and translations in $\R^4\cong\C^2$. The central
element generates an abstract one-parameter subgroup acting as $ e^{ i t
  \T}\cdot\mw= e^{-\theta t} \mw$ in this representation. From this action we
can read off the group multiplication laws
\begin{eqnarray}
  \label{JJgpmultlaw}
  e^{ i j \J}  e^{ i j' \J}&=& e^{ i(j+j' ) \J}
  \\\label{JQgpmultlaw}
  e^{ i j \J}  e^{ i(p_i^+ \P^i_++p_i^- \P^i_-)}
  &=& e^{ i(  e^{-\theta j} p_i^+ \P^i_+ +e^{\theta j} p_i^- \P^i_-)}e^{ ij\J}
  \\\label{QQgpmultlaw}
  e^{ i(p_i^+ \P^i_++p_i^- \P^i_-)}e^{ i(p_i^{\prime +} \P^i_++p_i^{\prime -} \P^i_-)} 
  &=& e^{ i[(p_i^++p_i)^{\prime +} \P^i_+ +(p_i^-+p_i)^{\prime -} \P^i_-]}
  e^{2\theta {\rm Im}(\mbp^+\cdot\mbp^{\prime -}) \T}
\end{eqnarray}
where the formula \eqref{JQgpmultlaw} displays the semi-direct product nature of
the euclidean group, while \eqref{QQgpmultlaw} displays the group cocycle of the
projective representation of the subgroup $\mathcal S$ of ${\rm ISO}(4)$,
arising from the central extension, which makes the translation algebra
noncommutative and is computed from the Baker-Campbell-Hausdorff formula.

Using \eqref{JJgpmultlaw} through \eqref{QQgpmultlaw} we may now compute the
products \eqref{TOgpprods}
\begin{eqnarray}
  \label{TOgpprodexpl}
  \NOa  \NOa  e^{ i k^a \X_a} \NOa\cdot\NOa  e^{ i k^{\prime a} \X_a}
  \NOa  \NOa &=& e^{i{\bf F}^\ast\cdot\X}\\\nn
  F^{\ast+}_i&=& p_i^++ e^{-\theta j} p_i^{\prime +}\\\nn
  F^{\ast-}_i&=& p_i^-+ e^{\theta j} p_i^{\prime -}\\\nn
  F^\ast_j&=& j+j'\\\nn
  F^\ast_t&=& t+t'-\theta ( e^{\theta j} \mbp^+\cdot\mbp^{\prime -}- e^{-\theta j}
  \mbp^-\cdot\mbp^{\prime +})
\end{eqnarray}
From \eqref{xastarxb} we may compute the $\star$-products between the coordinate
functions on $\mfn^\vee$ and easily verify the commutation relations of the algebra
$\mfn$,
\begin{eqnarray}
  \label{TOcoordstarprods}
  x_a*x_a&=&(x_a)^2\\\nn
  x_a*x^+&=&x^+*x_a = x_a x^+\\\nn
  z_1*z_2&=&z_2*z_1 = z_1 z_2\\\nn
  \overline{z}_1*\overline{z}_2&=&\overline{z}_2*\overline{z}_1
  = \overline{z}_1 \overline{z}_2\\\nn
  x^-*z_i&=&x^- z_i- i\theta z_i\\\nn
  z_i*x^-&=&x^- z_i\\\nn
  x^-*\overline{z}_i&=&x^- \overline{z}_i+ i\theta \overline{z}_i\\\nn
  \overline{z}_i*x^-&=&x^- \overline{z}_i \\\nn
  z_i*\overline{z}_i&=&z_i \overline{z}_i- i\theta x^+\\\nn
  \overline{z}_i*z_i&=&z_i \overline{z}_i+ i\theta x^+
\end{eqnarray}
with $a=1,\dots,6$ and $i=1,2$.

From \eqref{NOproductsBCH}, \eqref{fstargbidiff} we find the $*$-product of
generic functions $f,g\in\CC^{\infty}(\mfn^*)$ given by
\begin{eqnarray}
  \label{TOstargen}
  f*g&=&\mu\circ\exp\left[ i\theta x^+ \left( e^{- i\theta \d_-} 
      \bfd^\top\otimes\overline{\bfd}- e^{ i\theta \d_-} 
      {\overline{\bfd}}^{ \top}\otimes\bfd\right)\right.\\\nn
  &&\qquad\qquad+\left.\overline{z}_i 
    \left( e^{ i\theta \d_-}-1\right)\otimes
    \d^i+z_i \left( e^{- i\theta \d_-}-1\right)
    \otimes\overline{\d}^{ i}\right]f\otimes g
\end{eqnarray}
where $\mu(f\otimes g)=f g$ is the pointwise product. To second order
in the deformation parameter $\theta$ we obtain
\begin{eqnarray}
  \label{eq:time:positionspace}
  f\ast g&=&f g
  - i\theta \left[
    x^+ \left( \overline{\bfd}f\cdot\bfd g
    -\bfd f\cdot\overline{\bfd}g\right)
    -\overline{\mz}\cdot\d_-f \bfd g
    +\mz\cdot\d_-f \overline{\bfd}g
  \right]\\\nn
  &&- \theta^2 \sum\limits_{i=1,2} \left[ 
  \frac12 \left(x^+\right)^2 \left((\d^i)^2f 
  ( {\overline{\d}}^{ i})^2g
  -2{\overline{\d}}^{ i}\d^if {\overline{\d}}^{ i}\d^ig
  +({\overline{\d}}^{ i})^2f (\d^i)^2g\right)\right.
  \\\nn&&\qquad\qquad\quad- x^+ \left(\d^i\d_- f 
  {\overline{\d}}^{ i}g
  -{\overline{\d}}^{ i}\d_- f \d^ig\right)\\\nn
&&\qquad\qquad\quad
-x^+ \overline{z}_i \left( {\overline{\d}}^{ i}\d_- f (\d^i)^2g
  -\d^i\d_- f {\overline{\d}}^{ i}\d^ig\right)\\
  \nn &&\qquad\qquad\quad
  + x^+ z_i \left( {\overline{\d}}^{ i}\d_-
    f {\overline{\d}}^{ i}\d^ig
  -\d^i\d_- f ( {\overline{\d}}^{ i})^2g\right)
  -\overline{z}_i z_i \d_-^2f {\overline{\d}}^{ i}\d^ig
  \\ \nn &&\qquad\qquad\quad
  +\Bigl.\frac12 \left(\overline{z}_i^{ 2} \d_-^2f (\d^i)^2g
  +\overline{z}_i \d_-^2f \d^ig
  +z_i \d_-^2f {\overline{\d}}^{ i}g
  +z_i^2 \d_-^2f ( {\overline{\d}}^{ i})^2g\right)
\Bigr]\\\nn
&& + \mathcal{O}\left(\theta^3\right)
\end{eqnarray}

\subsection{Symmetric Time Ordering}
\label{TSOP}
Our next Gutt product is obtained by taking a ``symmetric time ordering''
whereby any monomial in $U(\mfn)$ is the symmetric sum over the two time
orderings obtained by placing $\J$ to the far right and to the far left. This
ordering is induced by the group contraction of ${\rm U}(1)\times{\rm SU}(2)$
onto the Nappi-Witten group $\mathcal{N}_0$ \cite{DAK1}, and it thereby induces
the coordinatisation of $\NW_4$ that is obtained from the Penrose-G\"uven limit
of the spacetime $\S^{1,0}\times\S^3$, i.e. it coincides with the Brinkman
coordinatisation of the Cahen-Wallach spacetime. On elements of $\mathcal{N}^\C$
it is defined by
\begin{equation}
  \label{TOsymgpprods}
  \Omega_\bullet\left( e^{ i\mk\cdot\mx}\right)=
  \NOb  e^{ i k^a \X_a} \NOb:= e^{\frac i2 j \J} 
  e^{ i(p_i^+ \P^i_+
    +p_i^- \P^i_-)}  e^{\frac i2 j \J}  e^{ i t \T}
\end{equation}
From \eqref{JJgpmultlaw} through \eqref{QQgpmultlaw} we can again easily compute
the required group products to get
\begin{eqnarray}
  \label{TOsymgpprodexpl}
  \NOb  \NOb  e^{ i k^a \X_a} \NOb\cdot\NOb  e^{ i k^{\prime a} \X_a}
  \NOb  \NOb &=& e^{i{\bf F}^\bullet\cdot\X}\\\nn
  F^{\bullet+}_i&=& e^{\frac{\theta}2 j'} p_i^++
        e^{-\frac{\theta}2 j} p_i^{\prime +}\\\nn
  F^{\bullet-}_i&=&e^{-\frac{\theta}2 j'} p_i^-+
        e^{\frac{\theta}2 j} p_i^{\prime -}\\\nn
  F^\bullet_j&=&j+j'\\\nn
  F^\bullet_t&=&t+t'-\theta e^{\frac{\theta}2 (j+j' )} 
        \mbp^+\cdot\mbp^{\prime -}- \theta e^{-\frac{\theta}2 (j+j' )} 
        \mbp^-\cdot\mbp^{\prime +}
\end{eqnarray}
With the same conventions as above, from \eqref{xastarxb} we may now compute the
$\star$-products $\bullet$ between the coordinate functions on $\mfn^\vee$ and again
verify the commutation relations of the algebra $\mfn$,
\begin{eqnarray}
  \label{TOsymcoordstarprods}
  x_a\bullet x_a&=&(x_a)^2 \\\nn
  x_a\bullet x^+&=&x^+\bullet x_a = x_a x^+ \\\nn
  z_1\bullet z_2&=&z_2\bullet z_1= z_1 z_2 \\\nn
  \overline{z}_1\bullet \overline{z}_2&=&\overline{z}_2\bullet \overline{z}_1
        = \overline{z}_1 \overline{z}_2 \\\nn
  x^-\bullet z_i&=&x^- z_i-\frac i2\theta z_i \\\nn
  z_i\bullet x^-&=&x^- z_i+\frac i2\theta z_i \\\nn
  x^-\bullet \overline{z}_i&=&x^- \overline{z}_i+\frac i2
  \theta \overline{z}_i \\\nn
  \overline{z}_i\bullet x^-&=&x^- \overline{z}_i
  -\frac i2\theta \overline{z}_i \\\nn
  z_i\bullet \overline{z}_i&=&z_i \overline{z}_i- i\theta x^+ \\\nn
  \overline{z}_i\bullet z_i &=&z_i \overline{z}_i+ i\theta x^+
\end{eqnarray}
From \eqref{NOproductsBCH} through \eqref{fstargbidiff} we find for generic
functions the formula
\begin{eqnarray}
  \label{TOsymstargen}
  f\bullet g&=&\mu\circ\exp\left\{ i\theta x^+ \left( e^{-\frac{ i\theta}2
        \d_-} \bfd^\top\otimes e^{-\frac{ i\theta}2 \d_-} 
      \overline{\bfd}- e^{\frac{ i\theta}2 \d_-} 
      \overline{\bfd}^{ \top}\otimes e^{\frac{ i\theta}2 \d_-} 
      \bfd\right)\right.\\\nn
  &&\qquad\qquad+ \overline{z}_i \left[\d^i\otimes
    \left( e^{-\frac{ i\theta}2 \d_-}-1\right)
    +\left( e^{\frac{ i\theta}2 \d_-}-1\right)\otimes
    \d^i\right]\\\nn
  &&\qquad\qquad+\left.
    z_i \left[ \overline{\d}^{ i}\otimes
      \left( e^{\frac{ i\theta}2 \d_-}-1\right)
      +\left( e^{-\frac{ i\theta}2 \d_-}-1\right)
      \otimes\overline{\d}^{ i}\right]\right\}f\otimes g
\end{eqnarray}
To second order in $\theta$ we obtain
\begin{eqnarray}
  \label{eq:symtime:positionspace}\nn
  f\bullet g&=&f g- \frac{ i}2  \theta \left[
  2x^+ \left( \overline{\bfd}f\cdot\bfd g
  - \bfd f\cdot\overline{\bfd}g\right)\right.\\\nn
  &&\qquad\qquad\quad\quad+\left.\overline{\mz}\cdot\left(\bfd f \d_- g
  - \d_- f \bfd g\right)+ \mz\cdot\left(\d_- f \overline{\bfd}g
   - \overline{\bfd}f \d_- g\right)\right]\\ \nn
  &&-  \frac1{2}  \theta^2  \sum\limits_{i=1,2}  
  \left[\left(x^+\right)^2 \left(( \overline{\d}^{ i})^2f (\d^i)^2g
  +(\d^i)^2f ( \overline{\d}^{ i})^2g
  -2\overline{\d}^{ i}\d^if \overline{\d}^{ i}\d^ig
  \right)\right.\\ \nn &&\qquad
  - x^+ \bigl(\d^if \overline{\d}^{ i}\d_- g
  +\overline{\d}^{ i}f \d^i\d_- g
  +\overline{\d}^{ i}\d_- f \d^ig
  +\d^i\d_- f \overline{\d}^{ i}g\bigr)\\ \nn
  &&\qquad
  + x^+ \overline{z}_i \bigl( \overline{\d}^{ i}\d^if \d^i\d_- g
  -\overline{\d}^{ i}\d_- f (\d^i)^2g
  +\d^i\d_- f \overline{\d}^{ i}\d^ig
  -(\d^i)^2f \overline{\d}^{ i}\d_- g\bigr)
  \\ \nn &&\qquad
  + x^+ z_i \left( \overline{\d}^{ i}\d^if \overline{\d}^{ i}\d_- g
  -\d^i\d_- f ( \overline{\d}^{ i})^2g
  +\overline{\d}^{ i}\d_- f \overline{\d}^i\d^ig
  -( \overline{\d}^{ i})^2f \d^i\d_- g\right)\\ \nn
  &&\qquad
  +  \frac12  \overline{z}_i z_i \left( 
  \overline{\d}^{ i}\d_- f \d^i\d_- g
  +\d^i\d_- f \overline{\d}^{ i}\d_- g
  -\d_- ^2f \overline{\d}^{ i}\d^ig
  -\overline{\d}^{ i}\d^if \d_-^2g\right)
  \\ \nn &&\qquad
  +  \frac14  \overline{z}_i^{ 2} 
  \left((\d^i)^2f \d_-^2g
  -2\d^i\d_- f \d^i\d_- g
  +\d_-^2f (\d^i)^2g\right)\\ \nn &&\qquad
  +  \frac14  z_i^2 
  \left(( \overline{\d}^{ i})^2f \d_-^2g
  -2\overline{\d}^{ i}\d_- f \overline{\d}^{ i}\d_- g
  +\d_-^2f ( \overline{\d}^{ i})^2g\right)
  \\ \nn &&\qquad
  +\Bigl. \frac14  \overline{z}_i \left(\d^if \d_-^2g
  +\d_-^2f \d^ig\right)
  + \frac14  z_i \left(\d_-^2f \overline{\d}^{ i}g
  +\overline{\d}^{ i}f \d_-^2g\right)
  \Bigr]\\\nn&&+{\mathcal O}\left(\theta^3\right)
\end{eqnarray}

\subsection{Weyl Ordering}
\label{WOP}
The original Gutt product \cite{Gutt1} is based on the ``Weyl ordering''
prescription by which all monomials in $U(\mfn)$ are completely symmetrised over
all elements of $\mfn$. On $\mathcal{N}^\C$ it is defined by
\begin{equation}
  \label{Weylgpprods}
  \Omega_\star\left( e^{ i\mk\cdot\mx}\right)=
  \NO  e^{ i k^a \X_a} \NO:= e^{ i k^a \X_a}
\end{equation}
While this ordering is usually thought of as the ``canonical'' ordering for the
construction of $\star$-products, in our case it turns out to be drastically more
complicated than the other orderings. Nevertheless, we shall present here its
explicit construction for the sake of completeness and for later comparisons.

It is an extremely arduous task to compute products of the group elements
\eqref{Weylgpprods} directly from the Baker-Campbell-Hausdorff formula
\eqref{eq:BCH}. Instead, we shall construct an isomorphism
$\mathcal{G}:\overline{U(\mfn)^\C}\to \overline{U(\mfn)^\C}$ which
sends the time-ordered product defined by \eqref{TOgpprods} into the
Weyl-ordered product defined by \eqref{Weylgpprods}, i.e.
\begin{equation}
  \label{1ststudy}
  \mathcal{G}\circ\Omega_*=\Omega_\star
\end{equation}
Then by defining
$\mathcal{G}_\Omega:=\Omega_*^{-1}\circ\mathcal{G}\circ\Omega_\star$,
the $\star$-product $\star$ associated with the Weyl ordering
prescription \eqref{Weylgpprods} may be computed as
\begin{equation}
  \label{WeylTOrel}
  f\star g=\mathcal{G}_\Omega\bigl(\mathcal{G}_\Omega^{-1}(f)*
  \mathcal{G}_\Omega^{-1}(g)\bigr) \qquad\qquad\qquad f,g\in\CC^\infty(\mfn^\vee)
\end{equation}
Explicitly, if
\begin{equation}
  \label{1ststudyexpl}
  \NOa  e^{ i k^a \X_a} \NOa= e^{ i G^a(\mk) \X_a}
\end{equation}
for some function $\mG=(G^a):\R^6\to\R^6$, then the isomorphism
$\mathcal{G}_\Omega:\CC^\infty(\mfn^\vee)\to\CC^\infty(\mfn^\vee)$ may be represented
as the invertible differential operator
\begin{equation}
  \label{Gdiffop}
  \mathcal{G}_\Omega= e^{ i\mx\cdot[\mG(- i\mbf\d)+ i\mbf\d]}
\end{equation}
This relation just reflects the fact that the time-ordered and Weyl-ordered
$\star$-products simply represent different ordering prescriptions for the same
algebra and are therefore (cohomologically) {\it equivalent}. We will elucidate
this property more thoroughly in section \ref{WeylSystems}. Thus once the map
\eqref{1ststudyexpl} is known, the Weyl ordered $\star$-product $\star$ can be
computed in terms of the time-ordered $\star$-product $*$ of Section \ref{TOP}.

The functions $G^a(\mk)$ appearing in \eqref{1ststudyexpl} are readily
calculable through the Baker-Campbell-Hausdorff formula. It is clear from
\eqref{TOgpprods} that the coefficient of the time translation generator
$\J\in\mfn$ is simply
\begin{equation}
  \label{Gj}
  G^j(j,t,\mbp^\pm)=j
\end{equation}
From \eqref{eq:BCH} it is also clear that the only terms proportional to
$\P^i_+$ come from commutators of the form $[\J,[\dots,[\J,\P^i_+] ]\dots]$, and
gathering all terms we find
\begin{eqnarray}
  \label{GomzBn}
  \sum\limits_{i=1,2}  
  G^{p_i^+}(j,t,\mbp^\pm) \P^i_+&=&- i\sum_{n=0}^\infty
  \frac{B_n}{n!} \bigl[ \underbrace{ i j \J , \bigl[\dots , 
    \bigl[ i j\J}_n ,  i p_i^+ \P^i_+ \bigr] \bigr]
  \dots\bigr]\\\nn &=&
  p_i^+ \sum_{n=0}^\infty\frac{B_n}{n!} 
  (-\theta j)^n \P^i_+
\end{eqnarray}
Since $B_0=1$, $B_1=-\frac12$ and $B_{2k+1}=0$ $\forall k\geq1$, from
\eqref{eq:BCH:K} we thereby find
\begin{equation}
  \label{Gomz}
  G^{\mbp^+}(j,t,\mbp^\pm)=\frac{\mbp^+}{\phi_\theta(j)}
\end{equation}
where we have introduced the function
\begin{equation}
  \label{phithetadef}
  \phi_\theta(j)=\frac{1- e^{-\theta j}}{\theta j}
\end{equation}
obeying the identities
\begin{eqnarray}
  \phi_\theta(j)e^{\theta j}=\phi_{-\theta}(j) \qquad
  \phi_\theta(j)\phi_{-\theta}(j)=-\frac2{(\theta j)^2}\bigl(1-\cos(\theta j)\bigr)
\label{phithetaids}
\end{eqnarray}
In a completely analogous way one finds the coefficient of the $\P^i_-$ term to
be given by
\begin{equation}
  \label{Gmz}
  G^{\mbp^-}(j,t,\mbp^\pm)=\frac{\mbp^-}{{\phi_{-\theta}(j)}}
\end{equation}
Finally, the non-vanishing contributions to the central
element $\T\in\mfn$ are given by
\begin{eqnarray}
  \nn
  G^t(j,t,\mbp^\pm) \T&=&
  t \T- i\sum_{n=1}^\infty\frac{B_{n+1}}{n!} 
  \left(\bigl[ i p_i^+ \P^i_+ , \bigl[ 
    \underbrace{ i j \J , \dots\bigl[ i j \J}_n ,  i p_i^- 
    \P^i_- \bigr]\dots\bigr] \bigr]\right.\\\nn &&\qquad\qquad\qquad
  +\left.\bigl[ i p_i^- \P^i_- , \bigl[ 
    \underbrace{ i j \J , \dots\bigl[ i j \J}_n , 
    i p_i^+ \P^i_+ \bigr]\dots\bigr] \bigr]\right)\\
  \label{GtBn}
  &=&t \T+4\theta \mbp^+\cdot\mbp^- \sum_{n=1}^\infty\frac{B_{n+1}}
  {n!} (-\theta j)^n \T
\end{eqnarray}
By differentiating \eqref{GomzBn} and \eqref{phithetadef} with respect
to $s=-\theta j$ we arrive finally at
\begin{equation}
  \label{Gt}
  G^t(j,t,\mbp^\pm)=t+4\theta \mbp^+\cdot\mbp^-\gamma_\theta(j)
\end{equation}
where we have introduced the function
\begin{equation}
  \label{gammathetadef}
  \gamma_\theta(j)=\frac12+\frac{(1+\theta j)  e^{-\theta j}-1}
  {\left( e^{-\theta j}-1\right)^2}
\end{equation}

From \eqref{Gdiffop} we may now write down the explicit form of the
differential operator implementing the equivalence between the
$\star$-products $*$ and $\star$ as
\begin{eqnarray}
  \label{Gdiffopexpl}
  \mathcal{G}_\Omega&=&\exp\left[-2 i\theta x^+ \overline{\bfd}
    \cdot\bfd\left(1+\frac{2(1- i\theta \d_-) 
        e^{ i\theta \d_-}-1}{\left( e^{ i\theta \d_-}-1
        \right)^2}\right)\right.\\\nn
  &&\qquad\quad+\left.
    \overline{\mz}\cdot\bfd\left(\frac{ i\theta \d_-}
      { e^{ i\theta \d_-}-1}-1\right)-\mz\cdot\overline{\bfd}
    \left(\frac{ i\theta \d_-}
      { e^{- i\theta \d_-}-1}+1\right)\right]
\end{eqnarray}
From \eqref{TOgpprodexpl} and \eqref{1ststudyexpl} we may readily compute the
products of Weyl symbols with the result
\begin{eqnarray}
  \label{Weylgpprodexpl}
  \NO  \NO  e^{ i k^a \X_a} \NO\cdot\NO  e^{ i k^{\prime a} \X_a}
  \NO  \NO &=& e^{i{\bf F}^\star\cdot\X}\\\nn
  F^{\star+}_i&=& \frac{\phi_\theta(j) p_i^++
        e^{-\theta j}\phi_\theta(j') p_i^{ \prime +}}{\phi_\theta(j+j')}\\\nn
  F^{\star-}_i&=& \frac{{\phi_{-\theta}(j)} p_i^-+
    e^{\theta j}{\phi_{-\theta}(j')} p_i^{\prime -}}{\phi_{-\theta}(j+j')}\\\nn
  F^\star_j&=& j+j'\\\nn
  F^\star_t&=&t+t'+\theta \bigl( \phi_{-\theta}(j) 
    \phi_{-\theta}(j' ) \mbp^+\cdot\mbp^{\prime -}-\phi_{\theta}(j) 
    \phi_{\theta}(j' ) \mbp^-\cdot\mbp^{\prime +}\bigr)\\\nn
    &&- 4\theta \biggl(\gamma_\theta(j+j' )
      \bigl({\phi_{-\theta}(j)} 
      \mbp^++ e^{\theta j} {\phi_{-\theta}(j' )} \mbp^{\prime +} \bigr)\\\nn
    &&\qquad\quad\times\bigl( \phi_\theta(j)
    \mbp^-+ e^{-\theta j} \phi_\theta(j' ) \mbp^{\prime -} \bigr)\\\nn
    &&-\gamma_\theta(j) \phi_\theta(j) \phi_{-\theta}(j) \mbp^+\cdot\mbp^--
    \gamma_\theta(j' ) \phi_\theta(j' ) \phi_{-\theta}(j' )
    \mbp^{\prime +}\cdot\mbp^{\prime -}\biggr)
\end{eqnarray}
The $x_a \star x_b$ products are identical to those of the symmetric time
ordering prescription \eqref{TOsymcoordstarprods}. After some computation, from
\eqref{NOproductsBCH}, \eqref{fstargbidiff} we find for generic functions
$f,g\in\CC^\infty(\mfn^\vee)$ the formula
\begin{eqnarray}
\nn
f\star g&=&\mu\circ\exp\left\{\theta x^+ \left[  
\frac{1\otimes1+\bigl( i
\theta (\d_-\otimes1+1\otimes\d_-)-1\otimes1
\bigr)  e^{ i\theta \d_-}\otimes e^{ i\theta \d_-}}
{\left( e^{ i\theta \d_-}\otimes e^{ i\theta \d_-}-1\otimes1
\right)^2}\right.\right.\\ &&\times 
\left(\frac{4\bfd^\top\left( e^{- i\theta \d_-}-1
\right)}{\theta \d_-}\otimes
\frac{\overline{\bfd}\left( e^{- i\theta \d_-}-1
\right)}{\theta \d_-}-\frac{3\overline{\bfd}^{ \top}
\left( e^{ i\theta \d_-}-1
\right)}{\theta \d_-}\otimes
\frac{\bfd\left( e^{ i\theta \d_-}-1
\right)}{\theta \d_-}\right.\nn\\ &&+\left.
\frac{4\overline{\bfd}\cdot\bfd\sin^2\left(\frac\theta2
 \d_-\right)}{\theta^2 \d_-^2}\otimes1-1\otimes
\frac{4\overline{\bfd}\cdot\bfd\sin^2\left(\frac\theta2
 \d_-\right)}{\theta^2 \d_-^2}\right)
\nn\\ &&+ 
\frac{4 i\overline{\bfd}\cdot\bfd}{\theta \d_-}
\left(\frac{ i\sin^2\left(\frac\theta2
 \d_-\right)}{\theta \d_-\left(
 e^{ i\theta \d_-}-1\right)}-1\right)\otimes1+1\otimes
\frac{4 i\overline{\bfd}\cdot\bfd}{\theta \d_-}
\left(\frac{ i\sin^2\left(\frac\theta2
 \d_-\right)}{\theta \d_-\left(
 e^{ i\theta \d_-}-1\right)}-1\right)\nn\\ &&
+\left.\frac{3\overline{\bfd}^{ \top}\left( e^{ i\theta \d_-}-1
\right)}{\theta \d_-}\otimes\frac{
\bfd\left( e^{ i\theta \d_-}-1
\right)}{\theta \d_-}+\frac{\bfd^\top
\left( e^{- i\theta \d_-}-1
\right)}{\theta \d_-}\otimes\frac{\overline{\bfd}
\left( e^{- i\theta \d_-}-1\right)}{\theta \d_-}  \right]
\nn\\&&
+ \frac{\overline{z}_i}{1\otimes e^{- i\theta 
\d_-}- e^{ i\theta \d_-}\otimes1} \left[  
\frac{\d^i}{\d_-} \left(1- e^{ i\theta \d_-}
\right)\otimes\d_--\d_-\otimes\frac{\d^i}
{\d_-} \left(1- e^{- i\theta \d_-}\right)\right.\nn\\
&& +\Biggl. 1\otimes\d^i 
 e^{- i\theta \d_-}-\d^i  e^{ i\theta \d_-}
\otimes1-1\otimes2\d^i  \Biggr]\nn\\ &&
+ \frac{z_i}{1\otimes e^{ i\theta 
\d_-}- e^{- i\theta \d_-}\otimes1} \left[  
\frac{\overline{\d}^{ i}}{\d_-} \left(1- e^{- i\theta \d_-}
\right)\otimes\d_--\d_-\otimes\frac{\overline{\d}^{ i}}
{\d_-} \left(1- e^{ i\theta \d_-}\right)\right.\nn\\
&& +\left.\Biggl.1\otimes\overline{\d}^{ i} 
 e^{ i\theta \d_-}-\overline{\d}^{ i}  e^{- i\theta \d_-}
 \otimes1-1\otimes2\overline{\d}^{ i}  \Biggr]\right\}f\otimes g
  \label{Weylstargen}
\end{eqnarray}
To second order in the deformation parameter $\theta$ we obtain
\begin{eqnarray}
  \label{eq:weyl:positionspace}
  f\star g&=&f g- \frac{ i}2  \theta \left[
  2x^+ \left( \overline{\bfd}f\cdot\bfd g
  - \bfd f\cdot\overline{\bfd}g\right)\right.\\\nn
  &&\qquad\qquad\qquad+\left.\overline{\mz}\cdot\left(\bfd f \d_- g
  - \d_- f \bfd g\right)+ \mz\cdot\left(\d_- f \overline{\bfd}g
   - \overline{\bfd}f \d_- g\right)\right]\\ \nn
  &&-  \frac1{2}  \theta^2  \sum\limits_{i=1,2}  \left[
  \left(x^+\right)^2 
  \left(( \overline{\d}^{ i})^2f (\d^i)^2g
  +(\d^i)^2f ( \overline{\d}^{ i})^2g
  -2\overline{\d}^{ i}\d^if \overline{\d}^{ i}\d^ig
  \right)\right.\\ \nn &&\qquad\quad
  -  \frac13  x^+ \left(\d^if \overline{\d}^{ i}\d_- g
  +\overline{\d}^{ i}f \d^i\d_- g
  +\overline{\d}^{ i}\d_- f \d^ig
  +\d^i\d_- f \overline{\d}^{ i}g\right.\\ \nn
  &&\qquad\qquad\qquad\qquad\quad
  -\left.2\d_-f \overline{\d}^{ i}
  \d^ig-2\overline{\d}^{ i}\d^if 
  \d_-g\right)\\\nn&&\qquad\quad
  + x^+ \overline{z}_i \left( \overline{\d}^{ i}
  \d^if \d^i\d_- g
  -\overline{\d}^{ i}\d_- f (\d^i)^2g
  +\d^i\d_- f \overline{\d}^{ i}\d^ig
  -(\d^i)^2f \overline{\d}^{ i}\d_- g\right)
  \\ \nn &&\qquad\quad
  + x^+ z_i \left( \overline{\d}^{ i}\d^if \overline{\d}^{ i}\d_- g
  -\d^i\d_- f ( \overline{\d}^{ i})^2g
  +\overline{\d}^{ i}\d_- f \overline{\d}^{ i}\d^ig
  -( \overline{\d}^{ i})^2f \d^i\d_- g\right)\\ \nn
  &&\qquad\quad
  +  \frac12  \overline{z}_i z_i \left( 
  \overline{\d}^{ i}\d_- f \d^i\d_- g
  +\d^i\d_- f \overline{\d}^{ i}\d_- g
  -\d_- ^2f \overline{\d}^{ i}\d^ig
  -\overline{\d}^{ i}\d^if \d_-^2g\right)
  \\ \nn &&\qquad\quad
  +  \frac14  \overline{z}_i^{ 2} 
  \left((\d^i)^2f \d_-^2g
  -2\d^i\d_- f \d^i\d_- g
  +\d_-^2f (\d^i)^2g\right)\\ \nn &&\qquad\quad
  +  \frac14  z_i^2 
  \left(( \overline{\d}^{ i})^2f \d_-^2g
  -2\overline{\d}^{ i}\d- f \overline{\d}^{ i}\d_- g
  +\d_-^2f ( \overline{\d}^{ i})^2g\right)
  \\ \nn &&\qquad\quad
  +  \frac16  \overline{z}_i \left(\d^if \d_-^2g
  +\d_-^2f \d^ig-\d_-f \d^i
  \d_-g-\d^i\d_-f \d_-g\right)
  \\&&\qquad\quad
  +\Bigl. \frac16  z_i \left(\d_-^2f \overline{\d}^{ i}g
  +\overline{\d}^{ i}f \d_-^2g-\d_-f \overline{\d}^{ i}\d_-g-
  \overline{\d}^{ i}\d_-f \d_-g\right)
\Bigr]\\\nn
&&+{\mathcal O}\left(\theta^3\right)
\end{eqnarray}

Although extremely cumbersome in form, the Weyl-ordered product has several
desirable features over the simpler time-ordered products. For instance, it is
Hermitean owing to the property
\begin{equation}
  \label{Weylstarherm}
  \overline{f\star g}=\overline{g}\star\overline{f}
\end{equation}
Moreover, while the $\mfn$-covariance condition \eqref{xyPoisson} holds for all
of our $\star$-products, the Weyl product is in fact $\mfn$-invariant, because for
any $x\in\C_{(1)}(\mfn^\vee)$ one has the stronger compatibility condition
\begin{equation}
  \label{strongcompcond}
  [x,f]_\star= i\theta \Theta(x,f)\qquad\qquad\forall f\in\CC^\infty(\mfn^\vee)
\end{equation}
with the action of the Lie algebra $\mfn$. In the next section we shall see that
the Weyl-ordered $\star$-product is, in a certain sense, the generator of all other
$\star$-products making it the ``universal'' product for the quantisation of the
spacetime $\NW_6$.

\section{Weyl Systems}
\label{WeylSystems}

In this section we will use the notion of a generalised Weyl system introduced
in \cite{ALZ1} to describe some more formal aspects of the $\star$-products that
we have constructed and to analyse the interplay between them. This generalises
the standard Weyl systems \cite{Sz1} which may be used to provide a purely
operator theoretic characterisation of the Moyal product, associated to the
(untwisted) Heisenberg algebra. In that case, it can be regarded as a projective
representation of the translation group in an even-dimensional real vector
space. However, for the twisted Heisenberg algebra such a representation is not
possible, since by definition the appropriate arena should be a central
extension of the non-Abelian subgroup $\mathcal S_5$ \eqref{S5subgp} of the full
euclidean group ${\rm ISO}(4)$. This requires a generalisation of the standard
notion which we will now describe and use it to obtain a very useful
characterisation of the noncommutative geometry induced by the algebra $\mfn$.

Let $\mathbb{V}$ be a five-dimensional real vector space. In a suitable
(canonical) basis, vectors $\mk\in\mathbb{V}\cong \R\times\C^2$ will be
denoted (with respect to a chosen complex structure) as
\begin{equation}
  \label{Svectors}
  \mk=\begin{pmatrix}j\\\mbp^+ \\\mbp^-\end{pmatrix}
\end{equation}
with $j\in\R$ and $\mbp^\pm=\overline{\mbp^\mp}\in\C^2$. As the notation
suggests, we regard $\mathbb{V}$ as the ``momentum space'' of the dual
$\mfn^\vee$. Note that we do not explicitly incorporate the component
corresponding to the central element $\T$, as it will instead appear through the
appropriate projective representation that we will construct, similarly to the
Moyal case. As an Abelian group, $\mathbb{V}\cong\R^5$ with the usual addition
$+$ and identity $\mbf0$. Corresponding to a deformation parameter
$\theta\in\R$, we deform this Abelian Lie group structure to a generically
non-Abelian one. The deformed composition law is denoted $\comp$. It is
associative and in general will depend on $\theta$. The identity element with
respect to $\comp$ is still defined to be $\mbf0$, and the inverse of any
element $\mk\in\mathbb{V}$ is denoted $\underline{\mk}$, so that
\begin{equation}
  \label{compinverse}
  \mk\comp\underline{\mk}=\underline{\mk}\comp\mk=\mbf0
\end{equation}
Being a deformation of the underlying Abelian group structure on $\mathbb{V}$
means that the composition of any two vectors $\mk,\mq\in\mathbb{V}$ has a
formal small $\theta$ expansion of the form
\begin{equation}
  \label{compsmalltheta}
  \mk\comp\mq=\mk+\mq+{\mathcal O}(\theta)
\end{equation}
from which it follows that
\begin{equation}
  \label{compinvsmalltheta}
  \underline{\mk}=-\mk+{\mathcal O}(\theta)
\end{equation}
In other words, rather than introducing $\star$-products that deform the
pointwise multiplication of functions on $\mfn^\vee$, we now deform the ``momentum
space'' of $\mfn^\vee$ to a non-Abelian Lie group. We will see below that the
five-dimensional group $(\mathbb{V},\comp)$ is isomorphic to the original
subgroup $\mathcal{S}\subset{\rm ISO}(4)$, and that the two notions of
quantisation are in fact the same.

Given such a group, we now define a (generalised) Weyl system for the algebra
$\mfn$ as a quadruple $(\mathbb{V},\comp,\weyl,\omega)$, where the map
\begin{equation}
  \label{genweylmapdef}
  \weyl : \mathbb{V} \longrightarrow \overline{U(\mfn)^\C}
\end{equation}
is a projective representation of the group $(\mathbb{V},\comp)$ with projective
phase $\omega:\mathbb{V}\times\mathbb{V}\to\C$. This means that for every
pair of elements $\mk,\mq\in\mathbb{V}$ one has the composition rule
\begin{equation}
  \label{Weylcomprule}
  \weyl(\mk)\cdot\weyl(\mq)= e^{\frac i2 \omega(\mk,\mq) \T} \cdot 
  \weyl(\mk\comp\mq)
\end{equation}
in the completed, complexified universal enveloping algebra of $\mfn$. The
associativity of $\comp$ and the relation \eqref{Weylcomprule} imply that the
subalgebra $\weyl(\mathbb{V})\subset\overline{U(\mfn)^\C}$ is associative
if and only if
\begin{equation}
  \label{cocyclecond}
  \omega(\mk\comp\mbf p,\mq)=\omega(\mk,\mbf p\comp\mq)+\omega(\mbf p,\mq)-
  \omega(\mk,\mbf p)
\end{equation}
for all vectors $\mk,\mq,\mbf p\in\mbbV$. This condition means that $\omega$
defines a one-cocycle in the group cohomology of $(\mathbb{V},\comp)$. It is
automatically satisfied if $\omega$ is a bilinear form with respect to $\comp$.
We will in addition require that $\omega(\mk,\mq)={\mathcal O}(\theta)$ $\forall
\mk,\mq\in\mathbb{V}$ for consistency with \eqref{compsmalltheta}. The identity
element of $\weyl(\mathbb{V})$ is $\weyl(\mbf0)$ while the inverse of
$\weyl(\mk)$ is given by
\begin{equation}
  \label{weylinverse}
  \weyl(\mk)^{-1}=\weyl( \underline{\mk} )
\end{equation}
The standard Weyl system on $\R^{2n}$ takes $\comp$ to be ordinary addition
and $\omega$ to be the Darboux symplectic two-form, so that $\weyl(\R^{2n})$
is a projective representation of the translation group, as is appropriate to
the Moyal product.

Given a Weyl system defined as above, we can now introduce another isomorphism
\begin{equation}
  \label{Omegaquantmap}
  \Pi : \CC^\infty\left(\R^5\right) \longrightarrow 
  \weyl\bigl(\mathbb{V}\bigr)
\end{equation}
defined by the symbol
\begin{equation}
  \label{Omegafdef}
  \Pi(f):=\int\limits_{\R^5}\frac{\dd\mk}{(2\pi)^5} 
  \tilde{f}(\mk) \weyl(\mk)
\end{equation}
where as before $\tilde f$ denotes the Fourier transform of
$f\in\CC^\infty(\R^5)$. This definition implies that
\begin{equation}
  \label{weylmkDelta}
  \Pi\bigl( e^{ i\mk\cdot\mx}\bigr)=\weyl(\mk)
\end{equation}
and that we may introduce a $\star$-involution $\dag$ on both algebras
$\CC^\infty(\R^5)$ and $\weyl(\mbbV)$ by the formula
\begin{equation}
  \label{involdef}
  \Pi\bigl(f^\dag\bigr)=\Pi\bigl(f\bigr)^\dag:=
  \int\limits_{\R^5}\frac{\dd\mk}{(2\pi)^5} \overline{
    \tilde f( \underline{\mk} )} \weyl(\mk)
\end{equation}
The compatibility condition
\begin{equation}
  \label{compconddag}
  \bigl(\Pi(f)\cdot\Pi(g)\bigr)^\dag=\Pi(g)^\dag\cdot
  \Pi(f)^\dag
\end{equation}
with the product in $\overline{U(\mfn)^\C}$ imposes further constraints on the
group composition law $\comp$ and cocycle $\omega$ \cite{ALZ1}. From
\eqref{Weylcomprule} we may thereby define a $\dag$-hermitean $\star$-product of
$f,g\in\CC^\infty(\R^5)$ by the formula
\begin{equation}
  \label{fstargweyl}
  f\star g:=\Pi^{-1}\bigl(\Pi(f)\cdot\Pi(g)\bigr)
  =\int\limits_{\R^5}\frac{\dd\mk}{(2\pi)^5} 
  \int\limits_{\R^5}\frac{\dd\mq}{(2\pi)^5} \tilde f(\mk) 
  \tilde g(\mq)  e^{\frac i2 \omega(\mk,\mq)} 
  \Pi^{-1}\circ\weyl(\mk\comp\mq)
\end{equation}
and in this way we have constructed a quantisation of the algebra $\mfn$ solely
from the formal notion of a Weyl system. The associativity of $\star$ follows
from associativity of $\comp$. We may also rewrite the $\star$-product
\eqref{fstargweyl} in terms of a bi-differential operator as
\begin{equation}
  \label{bidiffweyl}
  f\star g=f  e^{\frac i2 \omega( - i\overleftarrow{\mbf\d} , 
    - i\overrightarrow{\mbf\d} )+ i\mx\cdot(- i\overleftarrow{\mbf\d} \comp
    - i\overrightarrow{\mbf\d}\comp + i\overleftarrow{\mbf\d}+ i
    \overrightarrow{\mbf\d} )} g
\end{equation}

This deformation is completely characterised in terms of the new algebraic
structure and its projective representation provided by the Weyl system. It is
clear that the Lie algebra of $(\mathbb{V},\comp)$ coincides precisely with the
original subalgebra $\mfs\subset{\rm iso}(4)$, while the cocycle $\omega$
generates the central extension of $\mfs$ to $\mfn$ in the usual way. From
\eqref{fstargweyl} one may compute the $\star$-products of coordinate functions
on $\R^5$ as
\begin{equation}
  \label{xastarxbweyl}
  x_a\star x_b=x_a x_b- i\mx\cdot\left.\frac\d{\d k^a}\frac\d
    {\d q^b}(\mk\comp\mq)\right|_{\mk=\mq=\mbf0}-\left.\frac i2 
    \frac\d{\d k^a}\frac\d
    {\d q^b}\omega(\mk,\mq)\right|_{\mk=\mq=\mbf0}
\end{equation}
The corresponding $\star$-commutator may thereby be written as
\begin{equation}
  \label{xaxbstarcommxi}
  [x_a,x_b]_\star= i\theta C_{ab}^{  c} x_c+ i\theta \xi_{ab}
\end{equation}
where the relation
\begin{equation}
  \label{Cabccomp}
  \theta C_{ab}^{  c}=-\left.\left(\frac\d{\d k^a}\frac\d
      {\d q^b}-\frac\d{\d k^b}\frac\d
      {\d q^a}\right)(\mk\comp\mq)^c\right|_{\mk=\mq=\mbf0}
\end{equation}
gives the structure constants of the Lie algebra defined by the Lie group
$(\mathbb{V},\comp)$, while the cocycle term
\begin{equation}
  \label{cocycleterm}
  \theta \xi_{ab}=-\frac12 \left.\left(\frac\d{\d k^a}\frac\d
      {\d q^b}-\frac\d{\d k^b}\frac\d
      {\d q^a}\right)\omega(\mk,\mq)\right|_{\mk=\mq=\mbf0}
\end{equation}
gives the usual form of a central extension of this Lie algebra. Demanding that
this yield a deformation quantisation of the Kirillov-Kostant Poisson structure
on $\mfn^\vee$ requires that $C_{ab}^{ c}$ coincide with the structure constants of
the subalgebra $\mfs\subset{\rm iso}(4)$ of $\mfn$, and also that $\xi_{\mbp^-,
  \mbp^+} = -\xi_{\mbp^+,\mbp^-}=2t$ be the only non-vanishing components of the
central extension.

It is thus possible to define a broad class of deformation quantisations of
$\mfn^\vee$ solely in terms of an abstract Weyl system
$(\mbbV,\comp,\weyl,\omega)$, without explicit realisation of the operators
$\weyl(\mk)$. In the remainder of this section we will set $\Pi=\Omega$ above
and describe the Weyl systems underpinning the various products that we
constructed previously. This entails identifying the appropriate maps
\eqref{genweylmapdef}, which enables the calculation of the projective
representations \eqref{Weylcomprule} and hence explicit realisations of the
group composition laws $\comp$ in the various instances. This unveils a purely
algebraic description of the $\star$-products which will be particularly useful
for our later constructions, and enables one to make the equivalences between
these products explicit.

\subsection{Time Ordering}
\label{TOPGWS}
Setting $t=t'=0$ in \eqref{TOgpprodexpl}, we find the ``time-ordered''
non-Abelian group composition law $\compa$ for any two elements of the
form \eqref{Svectors} to be given by
\begin{equation}
  \label{TOcomplaw}
  \mk\compa\mk'=\begin{pmatrix}j+j' \\\mbp^++ e^{-\theta j} 
    \mbp^{\prime +} \\\mbp^-+ e^{\theta j} \mbp^{\prime -}
  \end{pmatrix}
\end{equation}
From \eqref{TOcomplaw} it is straightforward to compute the inverse
$\underline{\mk}$ of a group element \eqref{Svectors}, satisfying
\eqref{compinverse}, to be
\begin{equation}
  \label{TOinverse}
  \underline{\mk}=-\begin{pmatrix}j\\ e^{\theta j} \mbp^+\\
    e^{-\theta j} \mbp^- \end{pmatrix}
\end{equation}
The group cocycle is given by
\begin{equation}
  \label{TOcocycle}
  \omega_*(\mk,\mk' )=2 i\theta \left( e^{\theta j} 
    \mbp^+\cdot\mbp^{\prime -}- e^{-\theta j} 
    \mbp^-\cdot\mbp^{\prime +}\right)
\end{equation}
and it defines the canonical symplectic structure on the $j={\rm constant}$
subspaces $\C^2\subset\mathbb{V}$. Note that in this representation, the
central coordinate function $x^+$ is not written explicitly and is simply
understood as the unit element of $\C(\R^5)$, as is conventional in the
case of the Moyal product. For $\mk\in\mathbb{V}$ and $\X_a\in\mfs$ the
projective representation \eqref{Weylcomprule} is generated by the time-ordered
group elements
\begin{equation}
  \label{TOweylop}
  \weyl_*(\mk)=\NOa  e^{ i k^a \X_a} \NOa
\end{equation}
defined in \eqref{eq:time:defn}.

\subsection{Symmetric Time Ordering}
\label{TSOPGWS}

In a completely analogous manner, inspection of \eqref{TOsymgpprodexpl} reveals
the ``symmetric time-ordered'' non-Abelian group composition law $\compb$
defined by
\begin{equation}
  \label{TOsymcomplaw}
  \mk\compb\mk'=\begin{pmatrix}j+j' \\ e^{\frac{\theta}2 j'} \mbp^++
    e^{-\frac{\theta}2 j} \mbp^{\prime +} \\ e^{-\frac{\theta}2 j'} 
    \mbp^-+ e^{\frac{\theta}2 j} \mbp^{\prime -} \end{pmatrix}
\end{equation}
for which the inverse $\underline{\mk}$ of a group element \eqref{Svectors} is
simply given by
\begin{equation}
  \label{TOsyminverse}
  \underline{\mk}=-\mk
\end{equation}
The group cocycle is
\begin{equation}
  \label{TOsymcocycle}
  \omega_\bullet(\mk,\mk' )=2 i\theta \left( e^{\frac{\theta}2 
      (j+j' )} \mbp^+\cdot\mbp^{\prime -}- e^{-\frac{\theta}2 
      (j+j' )} \mbp^-\cdot\mbp^{\prime +}\right)
\end{equation}
and it again induces the canonical symplectic structure on
$\C^2\subset\mbbV$. The corresponding projective representation of
$(\mbbV,\compb)$ is generated by the symmetric time-ordered group elements
\begin{equation}
  \label{TOsymweylop}
  \weyl_\bullet(\mk)=\NOb  e^{ i k^a \X_a} \NOb
\end{equation}
defined in \eqref{TOsymgpprods}.

\subsection{Weyl Ordering}
\label{WOPGWS}
Finally, we construct the Weyl system $(\mbbV,\compc,\weyl_\star,\omega_\star)$
associated with the Weyl-ordered $\star$-product of Section \ref{WOP}. Starting
from \eqref{Weylgpprodexpl} we introduce the non-Abelian group composition law
$\compc$ by
\begin{equation}
  \label{Weylcomplaw}
  \mk\compc\mk'=\begin{pmatrix}j+j' \\[2mm]
    \frac{\phi_\theta(j) \mbp^++
      e^{-\theta j} \phi_\theta(j' ) \mbp^{\prime +}}
    {\phi_\theta(j+j' )}\\[3mm]
    \frac{\phi_{-\theta}(j) \mbp^-+
      e^{\theta j} \phi_{-\theta}(j' ) \mbp^{\prime -}}
    {\phi_{-\theta}(j+j' )}\end{pmatrix}
\end{equation}
from which we may again straightforwardly compute the inverse $\underline{\mk}$
of a group element \eqref{Svectors} simply as
\begin{equation}
  \label{Weylinverse}
  \underline{\mk}=-\mk
\end{equation}
When combined with the definition \eqref{involdef}, one has $f^\dag=\overline{f}$
$\forall f\in\CC^\infty(\R^5)$ and this explains the hermitean property
\eqref{Weylstarherm} of the Weyl-ordered $\star$-product $\star$. This is also
true of the product $\bullet$, whereas $*$ is only hermitean with respect to the
modified involution $\dag$ defined by \eqref{involdef} and \eqref{TOinverse}.
The group cocycle is given by
\begin{eqnarray}
  \label{Weylcocycle}
  &&\omega_\star(\mk,\mk' )=-2 i\theta \Bigl(
\phi_{-\theta}(j) \phi_{-\theta}(j' ) 
\mbp^+\cdot\mbp^{\prime -}-\phi_{\theta}(j) \phi_{\theta}(j' ) 
\mbp^-\cdot\mbp^{\prime +}\Bigr.\\\nn &&\qquad\qquad\qquad
- \gamma_\theta(j+j' ) \bigl(\phi_\theta(j) 
\mbp^++ e^{-\theta j} \phi_\theta(j' ) \mbp^{\prime +}\bigr)
\cdot\bigl(\phi_{-\theta}(j) 
\mbp^-+ e^{\theta j} \phi_{-\theta}(j' ) 
\mbp^{\prime -}\bigr)\\\nn &&
\qquad\qquad\qquad+\Bigl.
\gamma_\theta(j) \phi_\theta(j) \phi_{-\theta}(j) \mbp^+\cdot\mbp^-+
\gamma_\theta(j' ) \phi_\theta(j' ) \phi_{-\theta}(j' ) 
\mbp^{\prime +}\cdot\mbp^{\prime -}
\Bigr)
\end{eqnarray}
In contrast to the other cocycles, this does {\it not} induce any symplectic
structure, at least not in the manner described earlier. The corresponding
projective representation \eqref{Weylcomprule} is generated by the completely
symmetrised group elements
\begin{equation}
  \label{Weylweylop}
  \weyl_\star(\mk)= e^{ i k^a \X_a}
\end{equation}
with $\mk\in\mbbV$ and $\X_a\in\mfs$.

The Weyl system $(\mbbV,\compc,\weyl_\star,\omega_\star)$ can be used to
generate the other Weyl systems that we have found \cite{ALZ1}. From
\eqref{1ststudyexpl} and \eqref{Weylgpprodexpl} one has the identity
\begin{equation}
  \label{weylTOWeylDelta}
  \weyl_*(j,\mbp^\pm)=\Omega_\star\left(
    e^{ i(\mbf p^+\cdot\overline{\mz}
      +\mbp^-\cdot\mz)}\star e^{ i j  x^-}\right)
\end{equation}
which implies that the time-ordered $\star$-product $*$ can be expressed by
means of a choice of different Weyl system generating the product $\star$. Since
$\Omega_\star$ is an algebra isomorphism, one has
\begin{equation}
  \label{weylTOWeylprods}
  \weyl_*(j,\mbp^\pm)=\weyl_\star(0,\mbp^\pm)
  \cdot\weyl_\star(j,\mbf0)
\end{equation}
This explicit relationship between the Weyl systems for the $\star$-products $*$
and $\star$ is another formulation of the statement of their cohomological
equivalence, as established by other means in Section \ref{WOP}. Similarly, the
symmetric time-ordered $\star$-product $\bullet$ can be expressed in terms of
$\star$ through the identity
\begin{equation}
  \label{WeylTOsymWeylDelta}
  \weyl_\bullet(j,\mbp^\pm)=\Omega_\star\left(
    e^{\frac i2 j x^-}\star e^{ i(\mbp^+\cdot\overline{\mz}
      +\mbp^-\cdot\mz)}\star e^{\frac i2 j x^-}\right)
\end{equation}
which implies the relationship
\begin{equation}
  \label{WeylTOsymWeylprods}
  \weyl_\bullet\bigl(j,\mbp^\pm\bigr)=\weyl_\star
  \bigl( \frac j2 ,\mbf0\bigr)\cdot\weyl_\star\bigl(0,\mbp^\pm\bigr)
  \cdot\weyl_\star\bigl( \frac j2 ,\mbf0\bigr)
\end{equation}
between the corresponding Weyl systems. This shows explicitly that the
$\star$-products $\bullet$ and $\star$ are also equivalent.

\section{Twisted Isometries}
\label{Coprod}
We will now start working our way towards the explicit construction of the
geometric quantities required to define field theories on the noncommutative
plane wave $\NW_6$. We will begin with a systematic construction of derivative
operators on the present noncommutative geometry, which will be used later on to
write down kinetic terms for scalar field actions. In this section we will study
some of the basic spacetime symmetries of the $\star$-products that we
constructed in Section \ref{StarProds}, as they are directly related to the
actions of derivations on the noncommutative algebras of functions.

Classically, the isometry group of the gravitational wave $\NW_6$ is the group
$\mcN_{\rm L}\times\mcN_{\rm R}$ induced by the left and right regular actions
of the Lie group $\mcN$ on itself. The corresponding Killing vector fields live
in the 11-dimensional Lie algebra $\mfg:=\mfn_{\rm L}\oplus\mfn_{\rm R}$ (The
left and right actions generated by the central element $\T$ coincide). This
isometry group contains an ${\rm SO}(4)$ subgroup acting by rotations in the
transverse space $\mz\in\C^2\cong\R^4$, which is broken to ${\rm U}(2)$ by the
Neveu-Schwarz background \eqref{NS2formBrink}. This symmetry can be restored
upon quantisation by instead letting the generators of $\mfg$ act in a twisted
fashion \cite{CPT1,CKNT1,Wess1}, as we now proceed to describe.

The action of an element $\nabla\in U(\mfg)$ as an algebra automorphism
$\CC^\infty(\mfn^\vee)\to\CC^\infty(\mfn^\vee)$ will be denoted
$f\mapsto\nabla\triangleright f$. The universal enveloping algebra $U(\mfg)$ is
given the structure of a cocommutative bialgebra by introducing the ``trivial''
coproduct $\Delta:U(\mfg)\to U(\mfg)\otimes U(\mfg)$ defined by the homomorphism
\begin{equation}
  \label{trivialcoprod}
  \Delta(\nabla)=\nabla\otimes1+1\otimes\nabla
\end{equation}
which generates the action of $U(\mfg)$ on the tensor product
$\CC^\infty(\mfn^\vee)\otimes\CC^\infty(\mfn^\vee)$. Since $\nabla$ is an automorphism
of $\CC^\infty(\mfn^\vee)$, the action of the coproduct is compatible with the
pointwise (commutative) product of functions
$\mu:\CC^\infty(\mfn^\vee)\otimes\CC^\infty(\mfn^\vee)\to\CC^\infty(\mfn^\vee)$ in the
sense that
\begin{equation}
  \label{commcoprodcomp}
  \nabla\triangleright\mu(f\otimes g)=\mu\circ\Delta(\nabla)
  \triangleright(f\otimes g)
\end{equation}
For example, the standard action of spacetime translations is given by
\begin{equation}
  \label{commtranslaction}
  \d^a\triangleright f=\d^af
\end{equation}
for which \eqref{commcoprodcomp} becomes the classical symmetric Leibniz rule.

Let us now pass to a noncommutative deformation of the algebra of functions on
$\NW_6$ via a quantisation map $\Omega:\CC^\infty(\mfn^\vee) \to
\overline{U(\mfn)^\C}$ corresponding to a specific $\star$-product $\star$
on $\CC^\infty(\mfn^\vee)$ (or equivalently a specific operator ordering in
$U(\mfn)$). This isomorphism can be used to induce an action of $U(\mfg)$ on the
algebra $\overline{U(\mfn)^\C}$ through
\begin{equation}
  \label{nablastar}
  \Omega(\nabla_\star)\triangleright\Omega(f):=
  \Omega(\nabla\triangleright f)
\end{equation}
which defines a set of quantised operators $\nabla_\star = \nabla+{\mathcal
  O}(\theta): \CC^\infty(\mfn^\vee)\to\CC^\infty(\mfn^\vee)$. However, the
bialgebra $U(\mfg)$ will no longer generate automorphisms with respect to the
noncommutative $\star$-product on $\CC^\infty(\mfn^\vee)$. It will only do so if
its coproduct can be deformed to a non-cocommutative one
$\Delta_\star=\Delta+O(\theta)$ such that the covariance condition
\begin{equation}
  \label{NCcoprodcomp}
  \nabla_\star\triangleright\mu_\star(f\otimes g)=\mu_\star\circ
  \Delta_\star(\nabla_\star)\triangleright(f\otimes g)
\end{equation}
is satisfied, where $\mu_\star(f\otimes g):=f\star g$. This deformation is
constructed by writing the $\star$-product $f\star g=\hat{\mathcal D}(f,g)$ in
terms of a bi-differential operator as in \eqref{fstargbidiff} or
\eqref{bidiffweyl} to define an invertible Abelian Drinfeld twist \cite{Resh1}
element $\hat{\mathcal F}_\star\in
\overline{U(\mfg)^\C}\otimes\overline{U(\mfg)^\C}$ through
\begin{equation}
  \label{Dtwistdef}
  f\star g=\mu\circ\hat{\mathcal F}_\star^{-1}\triangleright(f\otimes g)
\end{equation}
It obeys the cocycle condition
\begin{equation}
  \label{twistcocycle}
  (\hat{\mathcal F}_\star\otimes1) (\Delta\otimes1) \hat
  {\mathcal F}_\star=(1\otimes\hat{\mathcal F}_\star) 
  (\Delta\otimes1) \hat{\mathcal F}_\star
\end{equation}
and defines the twisted coproduct through
\begin{equation}
  \label{Deltastardef}
  \Delta_\star:=\hat{\mathcal F}_\star\circ\Delta\circ
  \hat{\mathcal F}_\star^{-1}
\end{equation}
where $(f\otimes g)\circ(f'\otimes g' ):=f f'\otimes g g'$. This new coproduct
obeys the requisite coassociativity condition $(\Delta_\star\otimes\1)
\circ\Delta_\star=(\1\otimes\Delta_\star) \circ\Delta_\star$. The important
property of the twist element $\hat{\mathcal F}_\star$ is that it modifies only
the coproduct on the bialgebra $U(\mfg)$, while leaving the original product
structure (inherited from the Lie algebra $\mfg=\mfn_{\rm L}\oplus\mfn_{\rm R}$)
unchanged.

As an example, let us illustrate how to compute the twisting of the quantised
translation generators by the noncommutative geometry of $\NW_6$. For this, we
introduce a Weyl system $(\mbbV,\comp,\weyl,\omega)$ corresponding to the chosen
$\star$-product $\star$. With the same notations as in the previous section, for
$a=1,\dots,5$ we may use \eqref{Weylcomprule}, \eqref{involdef} with
$\Pi=\Omega$, and \eqref{nablastar} with $\nabla=\d^a$ to compute
\begin{eqnarray}
  \label{partialstarderiv}
  \Omega\bigl(\d_\star^a\bigr)\triangleright\Omega\bigl(
 e^{ i\mk\cdot\mx}\bigr)\cdot\Omega\bigl( e^{ i\mk'\cdot\mx}
\bigr)&=&\Omega\bigl(\d_\star^a\bigr)\triangleright
 e^{\frac i2 \omega(\mk,\mk' ) \T} \cdot \Omega\bigl(
 e^{ i(\mk\comp\mk' )\cdot\mx}\bigr) \\\nn &=&
 i  e^{\frac i2 \omega(\mk,\mk' ) \T} \cdot \Omega
\bigl((\mk\comp\mk' )^a  e^{ i(\mk\comp\mk' )\cdot\mx}\bigr)
\\\nn &=&  i  \sum\limits_i  \Omega\bigl(d^a_{(1) i}(
- i\bfd_\star)\bigr)\triangleright\Omega\bigl( e^{ i\mk\cdot\mx}
\bigr)\\\nn && \qquad\qquad
\cdot \Omega\bigl(d^a_{(2) i}(- i\bfd_\star)\bigr)
\triangleright\Omega\bigl( e^{ i\mk'\cdot\mx}\bigr)
\end{eqnarray}
where we have assumed that the group composition law of the Weyl system has an
expansion of the form $(\mk\comp\mk' )^a := \sum_i d^a_{(1) i}(\mk) d^a_{(2)
  i}(\mk' )$. From the covariance condition \eqref{NCcoprodcomp} it then follows
that the twisted coproduct assumes a Sweedler form
\begin{equation}
  \label{Deltastarexpl}
  \Delta_\star\left(\d_\star^a\right)= i 
  \sum\limits_i  d^a_{(1) i}(- i\bfd_\star)\otimes
  d^a_{(2) i}(- i\bfd_\star)
\end{equation}
Analogously, if we assume that the group cocycle of the Weyl system admits an
expansion of the form $\omega(\mk,\mk' ) := \sum_i w^i_{(1)}(\mk) w^i_{(2)}(\mk'
)$, then a similar calculation gives the twisted coproduct of the quantised
plane wave time derivative as
\begin{equation}
  \label{Deltastartime}
  \Delta_\star\left(\d_+^\star\right)=\d_+^\star
  \otimes1+1\otimes\d_+^\star- \frac12 \sum\limits_i  
  w^i_{(1)}(- i\bfd_\star)\otimes w^i_{(2)}(- i\bfd_\star)
\end{equation}
Note that now the corresponding Leibniz rules \eqref{NCcoprodcomp} are no longer
the usual ones associated with the product $\star$ but are the deformed,
generically non-symmetric ones given by
\begin{eqnarray}
  \label{defLeibniz}
  \d^a_\star\triangleright(f\star g)&=& i  \sum\limits_i  
  \bigl(d^a_{(1) i}(- i\bfd_\star)\triangleright f\bigr) \star \bigl(
  d^a_{(2) i}(- i\bfd_\star)\triangleright g\bigr) \\\nn
\d_+^\star\triangleright(f\star g)&=&\left(\d_+^\star
\triangleright f\right)\star g+f\star\left(\d_+^\star
\triangleright g\right)\\\nn&&
- \frac12 \sum\limits_i  
\bigl(w^i_{(1)}(- i\bfd_\star)\triangleright f\bigr) \star 
\bigl(w^i_{(2)}(- i\bfd_\star)\triangleright g\bigr)
\end{eqnarray}
arising from the twisting of the coproduct. Thus these derivatives do {\it not}
define derivations of the noncommutative algebra of functions, but rather
implement the twisting of isometries of flat space appropriate to the plane wave
geometry \cite{PK1,CFS1,BOL1,Halliday:2005zt}.

In the language of quantum groups \cite{QG1}, the twisted isometry group of the
spacetime $\NW_6$ coincides with the quantum double of the cocommutative Hopf
algebra $U(\mfn)$. The antipode ${S}_\star:U(\mfg)\to U(\mfg)$ of the given
non-cocommutative Hopf algebra structure on the bialgebra $U(\mfg)$ gives the
dual action of the isometries of the noncommutative plane wave and provides the
analog of inversion of isometry group elements. This analogy is made precise by
computing ${S}_\star$ from the group inverses $\underline{\mk}$ of elements
$\mk\in\mbbV$ of the corresponding Weyl system. Symbolically, one has
${S}_\star(\bfd_\star)=\underline{\bfd_\star}$. In particular, if
$\underline{\mk}=-\mk$ (as in the case of our symmetric $\star$-products) then
${S}_\star(\d_\star^a)=-\d_\star^a$ and the action of the antipode is trivial.
In what follows we will only require the underlying bialgebra structure of
$U(\mfg)$.

The generic non-triviality of the twisted coproducts \eqref{Deltastarexpl} and
\eqref{Deltastartime} is consistent with and extends the fact that generic
translations are not classically isometries of the plane wave geometry, but
rather only appropriate twisted versions are
\cite{PK1,CFS1,BOL1,Halliday:2005zt}. Similar computations can also be carried
through for the remaining five isometry generators of $\mfg$ and correspond to
the right-acting counterparts of the derivatives above, giving the full action
of the noncommutative isometry group on $\NW_6$. We shall not display these
formulas here. In the next section we will explicitly construct the quantised
derivative operators $\d_\star^a$ and $\d_+^\star$ above. We now proceed to list
the coproducts corresponding to our three $\star$-products.

\subsection{Time Ordering}
\label{TOcoprod}
The Drinfeld twist $\hat{\mathcal F}_*$ for the time-ordered $\star$-product is
the inverse of the exponential operator appearing in \eqref{TOstargen}.
Following the general prescription given above, from the group composition law
\eqref{TOcomplaw} of the corresponding Weyl system we deduce the time-ordered
coproducts
\begin{eqnarray}
  \label{TOcoprods}
  \Delta_*\left(\d_-^*\right)&=&\d_-^*\otimes1+
1\otimes\d_-^* \\\nn
\Delta_*\left(\d^i_*\right)&=&\d^i_*\otimes1+
 e^{ i\theta \d_-^*}\otimes\d^i_* \\\nn
\Delta_*\left( \overline{\d}^{ i}_*\right)&=&
\overline{\d}^{ i}_*\otimes1+ e^{- i\theta \d_-^*}
\otimes\overline{\d}^{ i}_*
\end{eqnarray}
while from the group cocycle \eqref{TOcocycle} we obtain
\begin{equation}
  \label{TOcoprodtime}
  \Delta_*\left(\d_+^*\right)=\d_+^*\otimes1+
1\otimes\d_+^*+\theta  e^{- i\theta 
\d_-^*} \left(\bfd_*\right)^\top\otimes\overline{\bfd}_*-\theta 
 e^{ i\theta \d_-^*} 
\left( \overline{\bfd}_*\right)^\top\otimes\bfd_*
\end{equation}
The corresponding Leibniz rules read
\begin{eqnarray}
  \label{TOLeibniz}
  \d_-^*\triangleright(f*g)&=&\left(\d_-^*\triangleright
f\right)*g+f*\left(\d_-^*\triangleright g\right) \\\nn
\d_+^*\triangleright(f*g)&=&\left(\d_+^*\triangleright
f\right)*g+f*\left(\d_+^*\triangleright g\right)\\\nn &&
+ \theta \bigl( e^{- i\theta \d_-^*} (\bfd_*)^\top
\triangleright f\bigr) * \left( \overline{\bfd}_*
\triangleright g\right)-\theta \bigl( e^{ i\theta \d_-^*} 
(\overline{\bfd}_*)^\top
\triangleright f\bigr) * \left(\bfd_*\triangleright g\right)
\\\nn \d^i_*\triangleright(f*g)&=&\left(\d^i_*
\triangleright f\right)*g+\bigl( e^{ i\theta \d_-^*}
\triangleright f\bigr)*\left(\d^i_*\triangleright g\right)
\\\nn \overline{\d}^{ i}_*
\triangleright(f*g)&=&\left( \overline{\d}^{ i}_*
\triangleright f\right)*g+\bigl( e^{- i\theta \d_-^*}
\triangleright f\bigr)*\left( \overline{\d}^{ i}_*
\triangleright g\right)
\end{eqnarray}

\subsection{Symmetric Time Ordering}
\label{STOcoprod}
The Drinfeld twist $\hat{\mathcal F}_\bullet$ associated to the symmetric
time-ordered $\star$-product is given by the inverse of the exponential operator
in \eqref{TOsymstargen}. From the group composition law \eqref{TOsymcomplaw} of
the corresponding Weyl system we deduce the symmetric time-ordered coproducts
\begin{eqnarray}
  \label{STOcoprods}
  \Delta_\bullet\left(\d_-^\bullet\right)&=&\d_-^\bullet
  \otimes1+1\otimes\d_-^\bullet \\\nn
  \Delta_\bullet\left(\d^i_\bullet\right)&=&\d^i_\bullet\otimes
  e^{-\frac{ i\theta}2 \d_-^\bullet}+ e^{\frac{ i\theta}2 
    \d_-^\bullet}\otimes\d^i_\bullet \\\nn
  \Delta_\bullet\left( \overline{\d}^{ i}_\bullet\right)&=&
  \overline{\d}^{ i}_\bullet\otimes
  e^{\frac{ i\theta}2 \d_-^\bullet}+ e^{-\frac{ i\theta}2 
    \d_-^\bullet}\otimes\overline{\d}^{ i}_\bullet
\end{eqnarray}
while from the group cocycle \eqref{TOsymcocycle} we find
\begin{eqnarray}
  \label{STOcoprodtime}
  \Delta_\bullet\left(\d_+^\bullet\right)&=&\d_+^\bullet\otimes1+
  1\otimes\d_+^\bullet\\\nn &&
  + \theta  e^{-\frac{ i\theta}2 
    \d_-^\bullet} \left(\bfd_\bullet\right)^\top\otimes
  e^{-\frac{ i\theta}2 \d_-^\bullet} \overline{\bfd}_\bullet-
  \theta  e^{\frac{ i\theta}2 \d_-^\bullet}\left( 
    \overline{\bfd}_\bullet\right)^\top\otimes e^{\frac{ i\theta}2 
    \d_-^\bullet} \bfd_\bullet
\end{eqnarray}
The corresponding Leibniz rules are given by
\begin{eqnarray}
  \label{STOLeibniz}
  \d_-^\bullet\triangleright(f\bullet g)&=&
  \left(\d_-^\bullet\triangleright f\right)\bullet g+
f\bullet\left(\d_-^\bullet\triangleright g\right)
\\\nn \d_+^\bullet\triangleright(f\bullet g)&=&
\left(\d_+^\bullet\triangleright f\right)\bullet g+
f\bullet\left(\d_+^\bullet\triangleright g\right)
+\theta \bigl( e^{-\frac{ i\theta}2 \d_-^\bullet} 
(\bfd_\bullet)^\top\triangleright f\bigr) \bullet \bigl(
 e^{-\frac{ i\theta}2 \d_-^\bullet} 
\overline{\bfd}_\bullet\triangleright g\bigr) \\\nn &&
\qquad\qquad\qquad\qquad\qquad\qquad - 
\theta \bigl( e^{\frac{ i\theta}2 \d_-^\bullet} 
( \overline{\bfd}_\bullet)^\top\triangleright f\bigr) \bullet 
\bigl( e^{\frac{ i\theta}2 \d_-^\bullet} \bfd_\bullet
\triangleright g\bigr) \\\nn
\d^i_\bullet\triangleright(f\bullet g)&=&\left(\d^i_\bullet
\triangleright f\right)\bullet\bigl( e^{-\frac{ i\theta}2 
\d_-^\bullet}\triangleright g\bigr)+\bigl(
 e^{\frac{ i\theta}2 \d_-^\bullet}\triangleright f
\bigr)\bullet\left(\d^i_\bullet\triangleright g\right)
\\\nn \overline{\d}^{ i}_\bullet\triangleright(f\bullet g)
&=&\left( \overline{\d}^{ i}_\bullet
\triangleright f\right)\bullet\bigl( e^{\frac{ i\theta}2 
\d_-^\bullet}\triangleright g\bigr)+\bigl(
 e^{-\frac{ i\theta}2 \d_-^\bullet}\triangleright f
\bigr)\bullet\left( \overline{\d}^{ i}_\bullet\triangleright g\right)
\end{eqnarray}

\subsection{Weyl Ordering}
\label{WOcoprod}
Finally, for the Weyl-ordered $\star$-product \eqref{Weylstargen} we read off
the twist element $\hat{\mathcal F}_\star$ in the standard way, and use the
associated group composition law \eqref{Weylcomplaw} to write down the
coproducts
\begin{eqnarray}
  \label{Weylcoprods}
  \Delta_\star\left(\d_-^\star\right)&=&\d_-^\star\otimes1+
1\otimes\d_-^\star \\\nn
\Delta_\star\left(\d^i_\star\right)&=& \frac{\phi_{-\theta}\left( i
\d_-^\star\right) \d^i_\star\otimes1+ e^{ i\theta \d_-^\star}
\otimes\phi_{-\theta}\left( i\d_-^\star\right) \d^i_\star}
{\phi_{-\theta}\left( i\d_-^\star\otimes1+1\otimes i
\d_-^\star\right)}\\\nn
\Delta_\star\left( \overline{\d}^{ i}_\star\right)&=&
 \frac{\phi_{\theta}\left( i
\d_-^\star\right) \overline{\d}^{ i}_\star\otimes1+
 e^{- i\theta \d_-^\star}
\otimes\phi_{\theta}\left( i\d_-^\star\right) 
\overline{\d}^{ i}_\star}
{\phi_{\theta}\left( i\d_-^\star\otimes1+1\otimes i
\d_-^\star\right)}
\end{eqnarray}
The remaining coproduct may be determined from the cocycle
\eqref{Weylcocycle} as
\begin{eqnarray}
  \label{Weylcoprodtime}
  \Delta_\star\left(\d_+^\star\right)&=&\d_+^\star\otimes1+
1\otimes\d_+^\star \\\nn &&+ 2 i\theta \Bigl[
\phi_\theta\left( i\d_-^\star\right) \left(\bfd_\star
\right)^\top\otimes\phi_\theta\left( i\d_-^\star\right) 
\overline{\bfd}_\star-\phi_{-\theta}\left( i\d_-^\star
\right) \left( \overline{\bfd}_\star
\right)^\top\otimes\phi_{-\theta}\left( i\d_-^\star\right) 
\bfd_\star\Bigr.
\\\nn &&\quad
+ \bigl(\gamma_\theta( i\d_-^\star)\otimes1-
\gamma_\theta( i\d_-^\star\otimes1+
1\otimes i\d_-^\star)\bigr)\bigl(\phi_\theta( i\d_-^\star)
 \phi_{-\theta}( i\d_-^\star) \overline{\bfd}_\star
\cdot\bfd_\star\otimes1\bigr) \\\nn &&\quad
+ \bigl(1\otimes\gamma_\theta( i\d_-^\star)-
\gamma_\theta( i\d_-^\star\otimes1+
1\otimes i\d_-^\star)\bigr)\bigl(1\otimes
\phi_\theta( i\d_-^\star)
 \phi_{-\theta}( i\d_-^\star) \overline{\bfd}_\star
\cdot\bfd_\star\bigr) \\\nn &&\quad -\Bigl.\gamma_\theta\left(
 i\d_-^\star\otimes1+1\otimes i\d_-^\star\right)
\bigl( e^{- i\theta \d_-^\star} \phi_{-\theta}( i
\d_-^\star) (\bfd_\star)^\top\otimes\phi_\theta( i
\d_-^\star) \overline{\bfd}_\star\bigr.\\\nn &&
\quad\qquad\qquad\qquad\qquad\qquad\qquad+\bigl. e^{ i\theta 
\d_-^\star} \phi_{\theta}( i
\d_-^\star) ( \overline{\bfd}_\star)^\top\otimes
\phi_{-\theta}( i\d_-^\star) \bfd_\star\bigr)\Bigr]
\end{eqnarray}
In \eqref{Weylcoprods} and \eqref{Weylcoprodtime} the functionals of the
derivative operator $ i\d_-^\star\otimes1 +1\otimes i\d_-^\star$ are
understood as usual in terms of the power series expansions given in section
\ref{WOP}. This leads to the corresponding Leibniz rules.

\begin{eqnarray}
  \label{WeylLeibniz}
  \d_-^\star\triangleright(f\star g)&=&\left(\d_-^\star
    \triangleright f\right)\star g+f\star\left(\d_-^\star
\triangleright g\right) \\\nn
\d_+^\star\triangleright (f\star g)&=&\left(\d_+^\star
\triangleright f\right)\star g+f\star\left(\d_+^\star
\triangleright g\right) \\\nn
&&+ 2 i\theta\left\{
\Bigl( \frac{(1- e^{- i\theta \d_-^\star} 
(\bfd_\star)^\top}{ i\theta \d_-^\star} \triangleright f
\Bigr) \star \Bigl( \frac{(1- e^{- i\theta \d_-^\star} 
\overline{\bfd}_\star}{ i\theta \d_-^\star} \triangleright
g\Bigr)\right. \\\nn && \qquad\qquad
- \Bigl( \frac{(1- e^{ i\theta 
\d_-^\star} ( \overline{\bfd}_\star)^\top}
{ i\theta \d_-^\star} \triangleright f
\Bigr) \star \Bigl( \frac{(1- e^{ i\theta \d_-^\star} 
\bfd_\star}{ i\theta \d_-^\star} \triangleright
g\Bigr)\\\nn &&\qquad\qquad
+ \Bigl( \Bigl[\frac12+\frac{(1+ i\theta 
\d_-^\star)  e^{- i\theta \d_-^\star}-1}{( e^{- i\theta 
\d_-^\star}-1)^2}\Bigr] \frac{\sin^2(\frac\theta2 \d_-^\star) 
\overline{\bfd}_\star\cdot\bfd_\star}{(\theta \d_-^\star)^2} 
\triangleright f\Bigr) \star g\\\nn &&\qquad\qquad+ f \star 
\Bigl( \Bigl[\frac12+\frac{(1+ i\theta 
\d_-^\star)  e^{- i\theta \d_-^\star}-1}{( e^{- i\theta 
\d_-^\star}-1)^2}\Bigr] \frac{\sin^2(\frac\theta2 \d_-^\star) 
\overline{\bfd}_\star\cdot\bfd_\star}{(\theta \d_-^\star)^2} 
\triangleright g\Bigr) \\\nn && \hspace{-20pt}+ \sum_{n=1}^\infty 
\sum_{k=0}^n \frac{B_{n+1} (- i\theta)^{n-2}}{k! (n-k)!} 
\Biggl[\bigl((\d_-^\star)^{n-k-2} \sin^2( \frac\theta2  
\d_-^\star) \overline{\bfd}_\star\cdot\bfd_\star
\triangleright f\bigr) \star \bigl((\d_-^\star)^k
\triangleright g\bigr)\\\nn && \quad + 
\bigl((\d_-^\star)^{n-k}\triangleright
f\bigr) \star \bigl((\d_-^\star)^{k-2} \sin^2( \frac\theta2  
\d_-^\star) \overline{\bfd}_\star\cdot\bfd_\star
\triangleright g\bigr) \\\nn && \quad - 
\bigl(( e^{- i\theta \d_-^\star}-1) (\d_-^\star)^{n-k-1}
 (\bfd_\star)^\top\triangleright f\bigr) \star \bigl(
( e^{- i\theta \d_-^\star}-1) (\d_-^\star)^{k-1}
 \overline{\bfd}_\star\triangleright g\bigr) \\\nn &&
\quad -
\bigl(( e^{ i\theta \d_-^\star}-1) (\d_-^\star)^{n-k-1}
 ( \overline{\bfd}_\star)^\top\triangleright f\bigr) \star \bigl(
( e^{ i\theta \d_-^\star}-1) (\d_-^\star)^{k-1}
 \bfd_\star\triangleright g\bigr)\Biggr] \\\nn
\d^i_\star\triangleright(f\star g)&=&\sum_{n=0}^\infty 
\sum_{k=0}^n \frac{B_n ( i\theta)^{n-1}}{k! (n-k)!} \left[
\bigl(( e^{ i\theta \d_-^\star}-1) 
(\d_-^\star)^{n-k-1} \d^i_\star\triangleright f\bigr) \star 
\bigl((\d_-^\star)^k\triangleright g\bigr)\right.\\\nn &&
\qquad\qquad\qquad\quad
+\left.\bigl( e^{ i\theta \d_-^\star} (\d_-^\star)^{n-k}
\triangleright f\bigr) \star \bigl(( e^{ i\theta \d_-^\star}-1)
 (\d_-^\star)^{k-1} \d^i_\star\triangleright g\bigr)\right]
\\\nn\overline{\d}^{ i}_\star\triangleright(f\star g)&=&
\sum_{n=0}^\infty \sum_{k=0}^n \frac{B_n (- i\theta)^{n-1}}{k! (n-k)!}
 \left[\bigl(( e^{- i\theta \d_-^\star}-1) 
(\d_-^\star)^{n-k-1} \overline{\d}^{ i}_\star
\triangleright f\bigr) \star 
\bigl((\d_-^\star)^k\triangleright g\bigr)\right.\\\nn &&
\qquad\qquad\qquad
+\left.\bigl( e^{- i\theta \d_-^\star} (\d_-^\star)^{n-k}
\triangleright f\bigr) \star \bigl(( e^{- i\theta \d_-^\star}-1)
 (\d_-^\star)^{k-1} \overline{\d}^{ i}_\star
\triangleright g\bigr)\right]
\end{eqnarray}
Note that a common feature to all three deformations is that the coproduct
of the quantisation of the light-cone position translation generator
$\d_-$ coincides with the trivial one \eqref{trivialcoprod}, and
thereby yields the standard symmetric Leibniz rule with respect to the
pertinent $\star$-product. This owes to the fact that the action of
$\d_-$ on the spacetime $\NW_6$ corresponds to the commutative
action of the central Lie algebra generator $\T$, whose left and right actions
coincide. In the next section we shall see that the action of the
quantised translations in $x^-$ on $\CC^\infty(\mfn^\vee)$ coincides with
the standard commutative action \eqref{commtranslaction}. This is
consistent with the fact that all frames of reference for the
spacetime $\NW_6$ possess an $x^-$-translational symmetry, while
translational symmetries in the other coordinates depend crucially on
the frame and generally need to be twisted in order to generate an
isometry of $\NW_6$. Notice also that ordinary time translation
invariance is always broken by the time-dependent Neveu-Schwarz
background \eqref{NS2formBrink}.

\section{Derivatives}
\label{Derivatives}
In this section we will systematically construct a set of quantised derivative
operators $\d^a_\star$, $a=1,\dots,6$ satisfying the conditions of the
previous section. In general, there is no unique way to build up such
derivatives. To this end, we will impose some weak conditions, namely that the
quantised derivatives be deformations of ordinary derivatives,
$\d_\star^a=\d^a+O(\theta)$, and that they commute among themselves,
$[\d_\star^a,\d_\star^b]_\star=0$. The latter condition is
understood as a requirement for the iterated action of the derivatives on
functions $f\in\CC^\infty(\mfn^\vee)$,
$[\d_\star^a,\d_\star^b]_\star\triangleright f=0$ or equivalently
\begin{equation}
  \label{derivcommute}
  \d_\star^a\triangleright\bigl(\d_\star^b\triangleright f
  \bigr)=\d_\star^b\triangleright\bigl(\d_\star^a\triangleright
  f\bigr)
\end{equation}
For the former condition, the simplest consistent choice is to assume a linear
derivative deformation on the coordinate functions, $[\d_\star^a,
x_b]_\star = \delta^a_{ b}+ i\theta \rho^a_{ bc} \d^c_\star$, which is
understood as the requirement
\begin{equation}
  \label{dxreq}
  \left[\d_\star^a , x_b\right]_\star\triangleright f:=
  \d^a_\star\triangleright\left(x_b\star f\right)-
  x_b\star\left(\d_\star^a\triangleright f\right)=
  \delta^a_{ b} f+ i\theta \rho^a_{ bc} \d^c_\star
  \triangleright f
\end{equation}
A set of necessary conditions on the constant tensors $\rho^a_{ bc}\in\R$ may
be derived by demanding consistency of the derivatives with the original
$\star$-commutators of coordinates \eqref{xaxbstarcomm}. Applying the Jacobi
identity for the $\star$-commutators between $\d^a_\star$, $x_b$ and $x_c$
leads to the relations
\begin{eqnarray}
  \label{rhoCrels}
  \rho^a_{ bc}-\rho^a_{ cb}&=&C_{bc}^{  a} \\\nn
  \rho^a_{ bc} \rho^c_{ de}-\rho^a_{ dc} \rho^c_{ be}&=&
  C_{bd}^{  c} \rho^a_{ ce}
\end{eqnarray}

With these requirements we now seek to find quantised derivative operators
$\d_\star^a$ as functionals of ordinary derivatives $\d^a$ acting on
$\CC^\infty(\mfn^\vee)$ as in \eqref{commtranslaction}. However, there are
(uncountably) infinitely many solutions $\rho^a_{ bc}$ obeying \eqref{rhoCrels}
\cite{DMT1} with $C_{ab}^{ c}$ the structure constants of the Lie algebra $\mfn$
given by \eqref{NW4algdef}. We will choose the simplest intuitive one defined by
the $\star$-commutators
\begin{align}
  \nn
  \left[\d_-^\star , x^-\right]_\star&=1
  &\left[\d_+^\star , x^-\right]_\star&=0
  &\left[\d_\star^i , x^-\right]_\star&=- i\theta \d_\star^i
  &\left[ \overline{\d}_\star^{ i} , x^-\right]_\star&=
   i\theta \overline{\d}_\star^{ i}
  \\ \nn
  \left[\d_-^\star , x^+\right]_\star&=0
  &\left[\d_+^\star , x^+\right]_\star &=1
  &\left[\d_\star^i , x^+\right]_\star&=0
  &\left[ \overline{\d}_\star^{ i} , x^+\right]_\star&=0
  \\ \nn
  \left[\d_-^\star , z_i\right]_\star&=0
  &\left[\d_+^\star , z_i\right]_\star&=- i\theta 
  \overline{\d}^{ i}_\star
  &\left[\d_\star^i , z_j\right]_\star&=\delta^i_{ j}
  &\left[ \overline{\d}_\star^{ i} , z_j\right]_\star&=0
  \\\nn
  \left[\d_-^\star , \overline{z}_i\right]_\star&=0
  &\left[\d_+^\star , \overline{z}_i\right]_\star&=
   i\theta \d_\star^i
  &\left[\d_\star^i , \overline{z}_j\right]_\star&=0
  &\left[ \overline{\d}_\star^{ i} , \overline{z}_j\right]_\star&
  =\delta^i_{ j}
  \\
    \label{eq:rho:nw4}
\end{align}
whose ${\mathcal O}(\theta)$ parts mimick the structure of the Lie brackets
\eqref{NW4algdef}. All other admissible choices for $\rho^a_{ bc}$ can be mapped
into those given by \eqref{eq:rho:nw4} via non-linear redefinitions of the
derivative operators $\d^a_\star$ \cite{DMT1}. It is important to realise that
the quantised derivatives do not generally obey the classical Leibniz rule, i.e.
$\d_\star^a\triangleright(f g)\neq f (\d_\star^a\triangleright
g)+(\d_\star^a\triangleright f) g$ in general, but rather the generalised
Leibniz rules spelled out in the previous section in order to achieve
consistency for $\theta\neq0$. Let us now construct the three sets of
derivatives of interest to us here.

\subsection{Time Ordering}
\label{TOderiv}
For the time ordered case, we use \eqref{TOstargen} to compute the
$\ast$-products
\begin{eqnarray}
  \label{TOxfprods}
  x^-*f&=&\left(x^-- i\theta \mz\cdot\bfd+ i\theta 
\overline{\mz}\cdot\overline{\bfd} \right) f\\\nn
x^+*f&=&x^+ f \\\nn
z_i*f&=&\left(z_i- i\theta x^+ \overline{\d}^{ i}
\right)  f \\\nn
\overline{z}_i*f&=&\left( \overline{z}_i+ i\theta x^+ \d^i
\right) f
\end{eqnarray}
Substituting these into \eqref{dxreq} using \eqref{eq:rho:nw4} then shows that
the actions of the $*$-derivatives simply coincide with the canonical actions of
the translation generators on $\CC^\infty(\mfn^\vee)$, so that
\begin{equation}
  \label{TOderivs}
  \d_*^a\triangleright f=\d^af
\end{equation}
Thus the time-ordered noncommutative geometry of $\NW_6$ is invariant under {\it
  ordinary} translations of the spacetime in all coordinate directions, with the
generators obeying the twisted Leibniz rules \eqref{TOLeibniz}.

\subsection{Symmetric Time Ordering}
\label{STOderiv}
Next, consider the case of symmetric time ordering. From \eqref{TOsymstargen} we
compute the $\bullet$-products
\begin{eqnarray}
  \label{STOxfprods}
  x^-\bullet f&=&\left(x^-- \frac{ i\theta}2  
\mz\cdot\bfd+ \frac{ i\theta}2  \overline{\mz}
\cdot\overline{\bfd} \right) f \\\nn
x^+\bullet f&=&x^+ f \\\nn
z_i\bullet f&=& e^{\frac{ i\theta}2 \d_-} 
\left(z_i- i\theta x^+ \overline{\d}^{ i}
\right) f \\\nn
\overline{z}_i\bullet f&=& e^{-\frac{ i\theta}2 \d_-} 
\left( \overline{z}_i+ i\theta 
x^+ \d^i\right) f
\end{eqnarray}
Substituting \eqref{STOxfprods} into \eqref{dxreq} using \eqref{eq:rho:nw4}
along with the derivative rule
\begin{equation}
  \label{derivrule1}
  e^{i\theta\partial_-}x^-=(x^-+i\theta)e^{i\theta\partial_-}
\end{equation}
we find that the actions of the $\bullet$-derivatives on $\CC^\infty(\mfn^\vee)$
are generically non-trivial and given by
\begin{eqnarray}
  \label{STOderivs}
  \d_-^\bullet\triangleright f&=&\d_-f \\\nn
  \d_+^\bullet\triangleright f&=&\d_+f \\\nn
  \d^i_\bullet\triangleright f&=& e^{-\frac{ i\theta}2 \d_-} \d^if \\\nn
  \overline{\d}^{ i}_\bullet\triangleright f&=&
  e^{\frac{ i\theta}2 \d_-} 
  \overline{\d}^{ i}f
\end{eqnarray}
Only the transverse space derivatives are modified: a consequence of the
invariance of the Brinkman coordinate system under translations of the
light-cone coordinates $x^\pm$. Again the twisted Leibniz rules
\eqref{STOLeibniz} are straightforward to verify in this instance.

\subsection{Weyl Ordering}
\label{Weylderiv}
Finally, from the Weyl-ordered $\star$-product \eqref{Weylstargen} we compute
\begin{eqnarray}
  \nn
  x^-\star f&=&\left[x^-+\left(1-\frac1{\phi_{-\theta}( i\d_-)}
\right) \frac{\mz\cdot\bfd}{\d_-}+
\left(1-\frac1{\phi_{\theta}( i\d_-)}
\right) \frac{\overline{\mz}\cdot\overline{\bfd}}{\d_-}
\right.\\\nn &&\qquad-\left.2\theta x^+\left(\frac{2}
{\theta \d_-}-\cot\left( \frac\theta2 \d_- 
\right)\right) \frac{\overline{\bfd}
\cdot\bfd}{\d_-}\right] f \\\nn
x^+\star f&=&x^+ f \\\nn
z_i\star f&=&\left[\frac{z_i}{\phi_{-\theta}( i\d_-)}+
2x^+ \left(1-\frac1{\phi_{-\theta}( i\d_-)}\right) 
\frac{\overline{\d}^{ i}}{\d_-}\right] f
\\ \overline{z}_i\star f&=&
\left[\frac{\overline{z}_i}{\phi_{\theta}( i\d_-)}+
2x^+ \left(1-\frac1{\phi_{\theta}( i\d_-)}\right) 
\frac{\d^i}{\d_-}\right] f
  \label{Weylxfprods}
\end{eqnarray}
From \eqref{dxreq}, \eqref{eq:rho:nw4} and the derivative rule
\begin{equation}
  \label{derivrule2}
  \phi_\theta(i\partial_-)x^-=\frac{e^{i\theta\partial_-}
    -\phi_\theta(i\partial_-)}{i\partial_-}
  +x^-\phi_\theta(i\partial_-)
\end{equation}
it then follows that the actions of the $\star$-derivatives on
$\CC^\infty(\mfn^\vee)$ are given by
\begin{eqnarray}
  \label{Weylderivs}
  \d_-^\star\triangleright f&=&\d_-f \\\nn
  \d_+^\star\triangleright f&=&\left[\d_++2 
  \left(1-\frac{\sin(\theta \d_-)}{\theta \d_-}
  \right) \frac{\overline{\bfd}\cdot\bfd}{\d_-}
\right]f \\\nn \d_\star^i\triangleright f&=&
-\frac{1- e^{ i\theta \d_-}}{ i\theta \d_-} 
\d^if \\\nn \overline{\d}_\star^{ i}
\triangleright f&=&\frac{1- e^{- i\theta \d_-}}
{ i\theta \d_-} \overline{\d}^{ i}f
\end{eqnarray}
Thus in the completely symmetric noncommutative geometry of $\NW_6$ both the
light-cone and the transverse space of the plane wave are generically only
invariant under rather complicated twisted translations, obeying the involved
Leibniz rules \eqref{WeylLeibniz}.

\section{Integrals}
\label{Integrals}
The final ingredient required to construct noncommutative field theory action
functionals is a definition of integration. At the algebraic level, we define an
integral to be a trace on the algebra $\overline{U(\mfn)^\C}$, i.e. a map
$\ncintsmall:\overline{U(\mfn)^\C}\to\C$ which is linear,
\begin{equation}
  \label{ncintlin}
  \ncint\bigl(c_1 \Omega(f)+c_2 \Omega(g)\bigr)=
  c_1 \ncint\Omega(f)+c_2 \ncint\Omega(g)
\end{equation}
for all $f,g\in\CC^\infty(\mfn^\vee)$ and $c_1,c_2\in\C$, and which is cyclic,
\begin{equation}
  \label{ncintcyclic}
  \ncint\Omega(f)\cdot\Omega(g)=\ncint\Omega(g)\cdot\Omega(f)
\end{equation}
We define the integral in the $\star$-product formalism using the usual
definitions for the integration of commuting Schwartz functions in
$\CC^\infty(\R^6)$. Then the linearity property \eqref{ncintlin} is
automatically satisfied. To satisfy the cyclicity requirement
\eqref{ncintcyclic}, we introduce
\cite{CalWohl1,BehrSyk1,Agostini:2004cu,Dimitrijevic:2003wv,FelShoi1} a measure
$\kappa$ on $\R^6$ which deforms the flat space volume element $\dd\mbf x$ and
define
\begin{equation}
  \label{ncintdef}
  \ncint\Omega(f):=\int\limits_{\R^6} \dd\mbf x \kappa(\mbf x) f(\mbf x)
\end{equation}
The measure $\kappa$ is chosen in order to achieve the property
\eqref{ncintcyclic}, so that
\begin{equation}
  \label{mucyclic}
  \int\limits_{\R^6} \dd\mbf x \kappa(\mbf x) (f\star g)(\mbf x)=
  \int\limits_{\R^6} \dd\mbf x \kappa(\mbf x) (g\star f)(\mbf x)
\end{equation}
Such a measure always exists \cite{CalWohl1,Dimitrijevic:2003wv,FelShoi1} and
its inclusion in the present context is natural for the curved spacetime $\NW_6$
which we are considering here. It is important note that, for the
$\star$-products that we use, a measure which satisfies \eqref{mucyclic} gives
the integral the additional property
\begin{equation}
  \label{ncintaddprop}
  \int\limits_{\R^6} \dd\mbf x \kappa(\mbf x) (f\star g)(\mbf x)=
  \int\limits_{\R^6} \dd\mbf x \kappa(\mbf x) f(\mbf x) g(\mbf x)
\end{equation}
providing an explicit realisation of the Connes-Flato-Sternheimer conjecture
\cite{FelShoi1}.

Since the coordinate functions $x_a$ generate the noncommutative algebra, the
cyclicity constraint \eqref{mucyclic} is equivalent to the $\star$-commutator
condition
\begin{equation}
  \label{starcommcond}
  \int\limits_{\R^6} \dd\mbf x \kappa(\mbf x) \bigl[(x_a)^n , f(\mbf x)
  \bigr]_\star=0
\end{equation}
which must hold for arbitrary functions $f\in\CC^\infty(\R^6)$ (for which the
integral makes sense) and for all $n\in\N$, $a=1,\dots,6$. We may rewrite any
commutator of the form $\cb{\hat x^n}{\hat y}$ using the following identity
\begin{equation}
  \label{eq:nw:int:samstheory}
  \cb{\hat x^{n+1}}{\hat y} = \sum_{m=0}^n\binom nm
  \hat x^{n-m}\cb{\hat x}{\hat y}\hat x^m
\end{equation}
allowing us to expand \eqref{starcommcond} to the form
\begin{equation}
  \label{commcondexp}
  \int\limits_{\R^6} \dd\mbf x \kappa(\mbf x) \sum_{m=0}^n {\binom nm}
  (x_a)^{n-m}\star\bigl[x_a , f(\mbf x)\bigr]_\star\star (x_a)^m=0
\end{equation}
We may thus insert the explicit form of $[x_a,f]_\star$ for generic $f$ and use
the ordinary integration by parts property
\begin{equation}
  \label{intpartsfgh}
  \int\limits_{\R^6} \dd\mbf x f(\mbf x) g(\mbf x) \partial_a^nh(\mbf x)=
  (-1)^n \int\limits_{\R^6} \dd\mbf x
  \Bigl[f(\mbf x) \left(\partial_a^n g(\mbf x) \right)
  h(\mbf x)+\left( \partial_a^nf(\mbf x)\right) g(\mbf x) h(\mbf x)\Bigr]
\end{equation}
for Schwartz functions $f,g,h\in\CC^\infty(\R^6)$. This will lead to a number of
constraints on the measure $\kappa$.

The trace \eqref{ncintdef} can also be used to define an inner product
$(-,-):\CC^\infty(\mfn^\vee)\times\CC^\infty(\mfn^\vee)\to\C$ through
\begin{equation}
  \label{ncintinnprod}
  (f,g):=\int\limits_{\R^6} \dd\mbf x \kappa(\mbf x) \bigl( \overline{f}\star
  g\bigr)(\mbf x)
\end{equation}
Note that this is different from the inner product introduced in Section
\ref{NWPW}. When we come to deal with the variational principle in the next
section, we shall require that our $\star$-derivative operators
$\partial^a_\star$ be anti-Hermitean with respect to the inner product
\eqref{ncintinnprod}, i.e. $(f,\partial^a_\star\triangleright
g)=-(\partial^a_\star\triangleright f,g)$, or equivalently
\begin{equation}
  \label{ncintparts}
  \int\limits_{\R^6} \dd\mbf x \kappa(\mbf x) \bigl( \overline{f}\star
  \partial_\star^a
  \triangleright g\bigr)(\mbf x)=-\int\limits_{\R^6} \dd\mbf x \kappa(\mbf x) 
  \bigl( \overline{\partial_\star^a\triangleright f}\star g\bigr)(\mbf x)
\end{equation}
This allows for a generalised integration by parts property
\cite{Dimitrijevic:2003wv} for our noncommutative integral. As always, we will
now go through our list of $\star$-products to explore the properties of the
integral in each case. We will find that the measure $\kappa$ is not uniquely
determined by the above criteria and that there is a large flexibility in the
choices that can be made. We will also find that the derivatives of the previous
section must be modified by a $\kappa$-dependent shift in order to satisfy
\eqref{ncintparts}.

\subsection{Time Ordering}
\label{TOint}
Using \eqref{TOxfprods} along with the analogous $*$-products $f*x_a$
\begin{eqnarray}
  \label{eq:time:starx}
  f\ast x^- &=&x^-f\\ \nn
  f\ast x^+ &=&x^+f\\ \nn
  f\ast z_i&=&(z_i+i\theta x^+\overline\d^i)e^{-i\theta\d_-}f\\ \nn
  f\ast \overline{z}_i&=&(\overline{z}_i-i\theta x^+\d^i)e^{i\theta\d_-}f
\end{eqnarray}
we arrive at the $*$-commutators
\begin{eqnarray}
  \label{eq:time:comm}
  \cb{x^-}{f}_\ast&=&i \theta \left( \obfz\cdot\obfd
    -\bfz\cdot\bfd\right)f   \\ \nn
  \cb{x^+}{f}_\ast&=&0   \\ \nn
  \cb{z_i}{f}_\ast&=&z_i \left(1
    -e^{-i \theta \d_-}\right)f-i \theta x^+ \left(1
    +e^{-i \theta \d_-}\right)\od^{ i}f   \\
  \cb{ \oz_i}{f}_\ast&=&\oz_i \left(1
    -e^{i \theta \d_-}\right)f+i \theta x^+ \left(1
    +e^{i \theta \d_-}\right)\d^if \nn
\end{eqnarray}
When inserted into \eqref{commcondexp}, after integration by parts and
application of the derivative rule \eqref{derivrule1} these expressions imply
constraints on the corresponding measure $\kappa_\ast$. We highlight the steps
involved in calculating the constraint for $x^-$
\begin{eqnarray}
  \label{eq:time:comm:mu:detail:1}
  i\theta \int\limits_{\R^6} \dd\mbf x \kappa_\ast \sum_{m=0}^n {\binom nm}
  (x^-)^n \left( \obfz\cdot\obfd
    -\bfz\cdot\bfd\right)f &=&0 \\\nn
  \implies \int\limits_{\R^6} \dd\mbf x
  \kappa_\ast \left( \obfz\cdot\obfd
    -\bfz\cdot\bfd\right)f &=&0 \\\nn
  \cancel{\kappa_\ast}
  + \bfz\cdot\bfd\kappa_\ast&=&
  \cancel{\kappa_\ast}
  + \obfz\cdot\obfd\kappa_\ast
  \\ \nn \bfz\cdot\bfd\kappa_\ast&=&\obfz\cdot\obfd\kappa_\ast
\end{eqnarray}
Clearly $\kappa_\ast$ terms appear from both $\bfz\cdot\bfd$ and
$\obfz\cdot\obfd$ parts of \eqref{eq:time:comm}. As $\kappa$-Minkowski only
consists of a single set of $z_i$ (missing the complementary $\overline{z_i}$),
this cancellation does not occur \cite{Agostini:2004cu} and
\eqref{eq:time:comm:mu:detail:1} does not simplify so elegantly. Unfortunately,
this limits the generalisation outlined in this thesis and one must choose the
appropriate $\kappa(\mbf x)$ for calculations on $\kappa$-Minkowski. This is
true for all of our orderings.

The remaining constraints are
\begin{eqnarray}
  \label{eq:time:mu:all}
  \left(1+e^{i \theta \d_-}\right)\od^{ i}\kappa_\ast&=&0 \\\nn
  \left(1-e^{-i \theta \d_-}\right)\d^i\kappa_\ast&=&0
\end{eqnarray}
It is straightforward to see that the equations \eqref{eq:time:mu:all} imply
that the measure must be independent of both the light-cone position and
transverse coordinates, so that
\begin{equation}
  \label{eq:time:mu}
  \partial_-\kappa_\ast=\d^i\kappa_\ast=\od^{ i}\kappa_\ast=0 
\end{equation}
However, the derivative $\d_+^\ast$ in \eqref{TOderivs} does not
satisfy the anti-hermiticity requirement \eqref{ncintparts}. This can be
remedied by translating it by a logarithmic derivative of the measure
$\kappa_*$ and defining the modified $*$-derivative
\begin{equation}
  \label{eq:time:d}
 \widetilde\d^{ \ast}_+=\d_+ + \frac12 \d_+\ln\kappa_\ast 
\end{equation}
The remaining $*$-derivatives in \eqref{TOderivs} are unaltered. While this
redefinition has no adverse effects on the commutation relations
\eqref{eq:rho:nw4}, the action $\widetilde\d^{ \ast}_+\triangleright f$ contains
an additional linear term in $f$ even if the function $f$ is independent of the
time coordinate $x^+$.

It is also unfortunate that the $\kappa$-Minkowski and $\NW$ measure constraints
are mutually exclusive. It is interesting to note that the measure is much
simpler for $\NW$, an unexpected result.

\subsection{Symmetric Time Ordering}
\label{STOint}
Using \eqref{STOxfprods} along with the corresponding $\bullet$-products
$f\bullet x_a$
\begin{eqnarray}
  \label{eq:symtime:starx}
  f\bullet x^-&=& \left(x^-+i\frac \theta 2\bfz\cdot\bfd
    -i\frac \theta 2\obfz\cdot\obfd\right) f\\ \nn
  f\bullet x^+&=&x^+f\\ \nn
  f\bullet z_i&=&\left(z_i+i\theta x^+\od^i\right)
        e^{-i\frac \theta 2 \d_-}f\\ \nn
  f\bullet \oz_i&=&\left(\oz_i-i\theta x^+\d^i\right)
        e^{i\frac \theta 2 \d_-}f
\end{eqnarray}
we arrive at the $\bullet$-commutators
\begin{eqnarray}
  \label{eq:symtime:comm}
  \cb{x^-}{f}_\bullet&=&i \theta \left( 
    \obfz\cdot\obfd -\bfz\cdot\bfd\right)f   \\ \nn
  \cb{x^+}{f}_\bullet&=&0   \\ \nn
  \cb{z_i}{f}_\bullet&=& 2iz_i \sin\left( \frac\theta2 \partial_- \right)f-2i
  \theta x^+ \od^{ i}
  \cos\left( \frac\theta2 \partial_-\right)f   \\
  \cb{ \oz_i}{f}_\bullet&=&-2i \oz_i \sin\left( \frac\theta2
    \partial_-\right)f+2i \theta x^+\d^i\cos
  \left( \frac\theta2 \partial_-\right)f 
\end{eqnarray}
Substituting these into \eqref{commcondexp} and integrating by parts, we arrive
at constraints on the measure $\kappa_\bullet$ given by
\begin{eqnarray}
  \label{eq:symtime:mu:all}
  \bigl(1-\od^{ i}\bigr)\sin\left( \frac\theta2
    \partial_-\right)\kappa_\bullet&=&0   \\ \nn
  \bigl(1+\d^i\bigr)\sin\left( \frac\theta2
    \partial_-\right)\kappa_\bullet&=&0   \\
  \bfz\cdot\bfd\kappa_\bullet&=&\obfz\cdot\obfd\kappa_\bullet \nn
\end{eqnarray}
which can be reduced to the conditions
\begin{equation}
  \label{eq:symtime:mu:rest}
  \bfz\cdot\bfd\kappa_\bullet=\obfz\cdot\obfd\kappa_\bullet  
  \quad \d_-\kappa_\bullet=0 
\end{equation}
Now the derivative operators $\d_+^\bullet$, $\d^i_\bullet$ and $\od^{
  i}_\bullet$ all violate the requirement \eqref{ncintparts}. Introducing
translates of $\d^i_\bullet$ and $\od^{ i}_\bullet$ analogously to what we did
in \eqref{eq:time:d} is problematic. While such a shift does not alter the
canonical commutation relations between the coordinates and derivatives, i.e.
the algebraic properties of the differential operators, it does violate the
$\bullet$-commutator relationships \eqref{dxreq} and \eqref{eq:rho:nw4} for
generic functions $f$. Consistency between differential operator and function
commutators would only be possible in this case by demanding that multiplication
from the left follow a Leibniz-like rule for the translated part.

Thus in order to satisfy both sets of constraints, we are forced to further
require that the measure $\kappa_\bullet$ depend only on the plane wave time
coordinate $x^+$ so that \eqref{eq:symtime:mu:rest} truncates to
\begin{equation}
  \label{eq:symtime:mu}
  \d^i\kappa_\bullet=\od^{ i}\kappa_\bullet=\d_-\kappa_\bullet=0 
\end{equation}
The logarithmic translation of $\d_+^\bullet$ must still be applied in order to
ensure that the time derivative is anti-hermitean with respect to the
noncommutative inner product. This modifies its action to
\begin{equation}
  \label{eq:symtime:d}
  \widetilde\d^{ \bullet}_+=\d_++ \frac12 \d_+\ln\kappa_\bullet
\end{equation}
The actions of all other $\bullet$-derivatives are as in \eqref{STOderivs}.
Again this shifting has no adverse effects on \eqref{eq:rho:nw4}, but it carries
the same warning as in the time ordered case regarding extra linear terms from
the action $\widetilde\d^{ \bullet}_+\triangleright f$.

\subsection{Weyl Ordering}
\label{Weylint}
Finally, the Weyl ordered $\star$-products \eqref{Weylxfprods} along with the
corresponding $f\star x_a$ products
\begin{eqnarray}
  \label{eq:weyl:starx}
  f\star x^-&=&\left\{
    \left(1-\frac{1}{\phi_\theta(i\d_-)}\right)
        \frac{\bfz \cdot \bfd}{\d_-}
    +\left(1-\frac{1}{\phi_{-\theta}(i\d_-)}\right)
        \frac{\obfz \cdot \obfd}{\d_-}\right.\\ \nn
    &&+x^--2\theta x^+\left(\frac{2}{\theta\d_-}
        \left.-\cot\left(\frac\theta2\d_-\right)\right)
        \frac{\bfd\cdot\obfd}{\d_-}\right\}f\\ \nn
  f\star x^+&=&x^+f\\ \nn
  f\star z_i&=&\left\{\frac{z_i}{\phi_\theta(i\d_-)}
    +2x^+\left(1-\frac 1{\phi_\theta(i\d_-)}\right)
      \frac{\od^i}{\d_-}\right\}f\\ \nn
  f\star \oz_i&=&\left\{\frac{\oz_i}{\phi_{-\theta}(i\d_-)}
    +2x^+\left(1-\frac 1{\phi_{-\theta}(i\d_-)}\right)
      \frac{\d^i}{\d_-}\right\}f
\end{eqnarray}
lead to the $\star$-commutators
\begin{eqnarray}
  \label{eq:weyl:comm}
  \cb{x^-}{f}_\star&=&i \theta \left( \obfz\cdot\obfd
    -\bfz\cdot\bfd\right)f   \\ \nn
  \cb{x^+}{f}_\star&=&0   \\ \nn
  \cb{z_i}{f}_\star&=&i \theta \left(z_i \d_-
    -2x^+ \od^{ i}\right)f   \\
  \cb{ \oz_i}{f}_\star&=&i \theta \left(-\oz_i \d_-
    +2x^+ \d^i\right)f \nn
\end{eqnarray}
Substituting these commutation relations into \eqref{commcondexp}, integrating
by parts and using the derivative rule \eqref{derivrule1} along with
\eqref{derivrule2} leads to the corresponding measure constraints
\begin{eqnarray}
  \label{eq:weyl:mu:all}
  z_i \d_-\kappa_\star&=&2x^+ \od^{ i}\kappa_\star   \\ \nn
  \oz_i \d_-\kappa_\star&=&2x^+ \d^i\kappa_\star   \\
  \bfz\cdot\bfd\kappa_\star&=&\obfz\cdot\obfd\kappa_\star \nn
\end{eqnarray}
Again these differential equations imply that the measure
$\kappa_\star$ depends only on the plane wave time coordinate $x^+$ so
that
\begin{equation}
  \label{eq:weyl:mu}
  \d_-\kappa_\star=\d^i\kappa_\star=\od^{ i}\kappa_\star=0
\end{equation}
Translating the derivative operator $\d_+^\star$ as before in order to satisfy
\eqref{ncintparts} yields the modified derivative
\begin{equation}
  \label{eq:weyl:d}
  \widetilde\d^{ \star}_+=\d_++2 
  \left(1-\frac{\sin(\theta \partial_-)}{\theta \partial_-}
  \right) \frac{\overline{\bfd}\cdot\bfd}{\partial_-}+
  \frac12 \d_+\ln\kappa_\star
\end{equation}
with the remaining $\star$-derivatives in \eqref{Weylderivs} unchanged. Once
again this produces no major alteration to \eqref{eq:rho:nw4} but does yield
extra linear terms in the actions $\widetilde\d^{ \star}_+\triangleright f$.

%%% Local Variables: 
%%% mode: latex
%%% TeX-master: "main.tex"
%%% End: 
