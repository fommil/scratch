\chapter{Introduction}
\label{intro}
\section{Noncommutative Geometry}
The formulation of quantum mechanics by Dirac \cite{Dirac} sets the
commutativity of position operators by borrowing from classical mechanics, such
that position operators commute with each other. However, this choice was merely
convenience as there was no experimental evidence to question such a definition.
It was not long before esteemed authors wrote about the effect of introducing
noncommuting coordinates \cite{Snyder:1946qz}. The motivation for studying
noncommutative geometry was to hopefully find a fix for the infinities
appearing in quantum field theory, but was sidelined as renormalisation became
successful at achieving that goal.

Noncommutative geometry has seen recent interest since open string models have
been shown to possess a noncommutative geometry \cite{witten, SW1} where there
is a non-zero $B$-field. Should string theory turn out to be a viable
explanation of our universe, one would expect to be able to observe the
noncommutativity on a quantum mechanical level. The coordinates would possess
commutator brackets of the form
\begin{equation}
  \label{eq:i:com}
  [\hat x^i, \hat x^j]=i\Theta^{ij}(\hat x)
\end{equation}
where $\Theta$ is an antisymmetric matrix. However, no such noncommutativity has
been experimentally observed \cite{Hinchliffe:2002km}. The noncommutativity
between position and momenta gives rise to the Heisenberg Uncertainty Principle
and similarly \eqref{eq:i:com} implies an uncertainty on the coordinates
themselves
\begin{equation}
  \label{eq:i:planck}
  \Delta x^i  \Delta x^j \geq \frac 12|\Theta^{ij}|
\end{equation}
This means there is no longer an idea of ``a point'', but we instead have the
notion of ``Planck cells''. The most widely studied type of noncommutative
geometry is the canonical case where $\Theta$ is a constant antisymmetric
matrix.

Unfortunately a noncommutative geometry will lead to a breaking of Lorentz
invariance. Lorentz covariance of $\Theta$ means that different inertial
observers see different noncommutativity of coordinates due to different
projections of the noncommuting planes. This is perhaps one of the greatest
hurdles for theories which rely upon a noncommutative geometry, but there is
hope. Theories such as those on $\kappa$-Minkowski spacetime
\cite{Dimitrijevic:2003wv} preserve Lorentz symmetries as deformed quantum
symmetries (a bonus from having a quantum algebra) and twisted symmetries even
for constant $\Theta$ can realise Lorentz invariance by acting in a twisted way
\cite{CKNT1, CPT1}.

A spacetime which is of particular interest is the Nappi-Witten background
\cite{NW1} which possesses a quantum algebra, is four dimensional with Minkowski
signature and is an exactly solvable background for string theory. We shall look
at the properties of this spacetime in Section \ref{sec:i:nw} and proceed to
investigate it in detail throughout this thesis.

Instead of calculating physics on a noncommutative space itself, we may
equivalently stay in commutative space and replace pointwise multiplication by a
deformed $\star$-product. A $\star$-commutator bracket between the coordinate
functions produces the functional equivalent of \eqref{eq:i:com}
\begin{equation}
  \label{eq:i:starcom}
  [ x^i,  x^j]_\star = x^i\star x^j - x^j\star x^i = i \Theta^{ij}(x)
\end{equation}
Although we shall only return to the complicated $\star$-product in detail for
Chapter \ref{star}, a basic understanding shall be assumed in Chapter
\ref{liebranes} whereby the reader recognises that in general $f\star g \neq
g\star f$ for functions depending upon the coordinates and that the
$\star$-product is not unique for a given algebra.

\section{pp-Waves and the AdS/CFT Correspondence}
A pp-wave spacetime is any Lorentzian manifold whose metric can be described in
Brinkman coordinates \cite{Brink1} as
\begin{equation}
  \label{eq:brinkman}
  G = 2dx^+dx^- + |d\mz|^2 + H(x^+, x^-, \mz)\left( d x^+\right)^2
\end{equation}
where $H(x^+, x^-, \mz)$ is any smooth function. $x^+$ is a time-like
coordinate\footnote{Please note that this convention has not been standardised
  in the literature; some authors label the time-like coordinate by $x^-$.},
$x^-$ is null and the $\mz$ are euclidean coordinates. The term ``pp'' modestly
stands for \textit{plane-fronted waves with parallel propagation}. As we shall
see in Chapter \ref{pglimits}, Penrose \cite{Penrose1} observed that near a null
geodesic, every Lorentzian spacetime looks like a plane wave. pp-waves are an
important family of exact solutions to Einstein's field equations and exhibit
the characteristic effect of a gravitational wave on light.

The AdS/CFT correspondence \cite{Aharony:1999ti} is the equivalence between a
string or supergravity theory defined on anti de Sitter space and a conformal
field theory defined on its conformal boundary, with dimension one lower. It is
the most successfully tested realisation of the holographic principle
\cite{Bousso:2002ju}.

The dynamics of strings in the backgrounds of pp-waves has been of interest
recently for a variety of reasons. They provide explicit realisations of string
theory in time-dependent backgrounds which is necessary for applications of
string cosmology. They also provide scenarios in which the AdS/CFT
correspondence may be tested beyond the supergravity approximation by taking the
Penrose limit of an $\AdS_m\times\S^n$ background \cite{BFP1} and the BMN limit
of the dual superconformal field theory \cite{BMN1}. The property of these
backgrounds that make them appealing in these contexts is that string dynamics
on them are solvable in some instances, even in the presence of non-trivial
$B$-fields \cite{BMN1,BOLPT1,Met1,PRT1,RT1}. The spectrum of the theory can be
studied in light-cone gauge wherein the two-dimensional $\sigma$-models become
free, while scattering amplitudes can be analysed using light-cone string field
theory.

D-branes (the D standing for Dirichlet) are membrane-like structures which are
considered to be as physically fundamental to string theory as the strings
themselves; they are the surfaces to which the strings are attached. If it were
not for D-branes, energy could flow along a string, slip off the endpoint and
vanish. Because the endpoints of open strings cannot detach from the D-branes to
which they are affixed, the D-branes determine the boundary conditions for the
string's equation of motion (either Neumann or Dirichlet), thereby ensuring
conservation of energy. D-branes are typically classified by their spatial
worldvolume dimension, which is indicated by a number written after the D. For
example, A D0-brane is a ``D-particle'', a D1-brane is a line (sometimes called
a ``D-string'') and a D2-brane is a plane. A ``D-instanton'' is a fixed point in
all space-time coordinates,

When D-branes are added to such closed string backgrounds, in some cases
decoupling limits exist in which one can freeze out massive open string modes
and closed string excitations. The low-energy effective theory governing the
dynamics of open strings living on the branes is non-gravitational and can be
reformulated as a field theory. The typical result is a noncommutative gauge
theory with a spacetime dependent noncommutativity parameter
\cite{CLO1,DRRS1,DN1,HS1,HT1,LNR1}. The role of spacetime dependence in these
worldvolume field theories leads to interesting violations of energy-momentum
conservation \cite{BG1,RS1}, and their potential time-dependence is especially
important for cosmological applications. In some instances the decoupled open
strings also have a dual description in terms of a gravitational theory via the
AdS/CFT correspondence \cite{HS1}. This suggests that the holographic
description of cosmological spacetimes may be described by non-local field
theories.

The general construction and analysis of noncommutative gauge theories on curved
spacetimes is one of the most important outstanding problems in the applications
of noncommutative geometry to string theory. These non-local field theories
arise naturally as certain decoupling limits of open string dynamics on D-branes
in curved superstring backgrounds in the presence of a non-constant background
Neveu-Schwarz $B$-field. On a generic Poisson manifold $M$, they are formulated
using the Kontesevich star-product \cite{Kont1} which is linked to a topological
string theory known as the Poisson sigma-model \cite{CattFel1}. Under suitable
conditions, the quantisation of D-branes in the Poisson sigma-model which wrap
coisotropic submanifolds of $M$, i.e. worldvolumes defined by first-class
constraints, may be consistently carried out and related to the deformation
quantisation in the induced Poisson bracket \cite{CattFel2}. Branes defined by
second-class constraints may also be treated by quantising Dirac brackets on the
worldvolumes \cite{CFal1}.

However, in other concrete string theory settings, most studies of
noncommutative gauge theories on curved D-branes have been carried out only
within the context of the AdS/CFT correspondence by constructing the branes as
solutions in the dual supergravity description of the gauge theory (see for
example \cite{Cai1,CLO1,HS1,HashTh1,ASY1}). It is important to understand
how to describe the classical solutions and quantisation of these models
directly at the field theoretic level in order to better understand to what
extent the noncommutative field theories capture the non-local aspects of string
theory and quantum gravity, and also to be able to extend the descriptions to
more general situations which are not covered by the AdS/CFT correspondence.

Open string dynamics on the $\NW$ background are particularly interesting
because it has the potential to display a time-dependent noncommutative geometry
\cite{DN1,HS1}, and hence the noncommutative field theories built on $\NW_6$ can
serve as interesting toy models for string cosmology which can be treated for
the most part as ordinary field theories. However, this point is rather subtle
for the present geometry \cite{DN1,HT1}. A particular gauge choice which leads
to a time-dependent noncommutativity parameter breaks conformal invariance of
the worldsheet sigma-model, i.e. it does not satisfy the Born-Infeld field
equations, while a conformally invariant background yields a non-constant but
time-independent noncommutativity. In this thesis we partially clarify this
issue.

\section{Nappi-Witten Spacetime}
\label{sec:i:nw}

In this thesis, which is based on the publications \cite{Halliday:2005zt,
  Halliday:2006qc}, we will study the noncommutative gauge theories that reside
on some D-branes in the four-dimensional Nappi-Witten gravitational wave
\cite{NW1} and its six-dimensional generalisation \cite{KM1}. We will refer to
both of these pp-waves as Nappi-Witten spacetimes and denote them respectively
by $\NW_4$ and $\NW_6$. In the full superstring setting the backgrounds we study
are $\NW_4\times\Torus^6$ and $\NW_6\times\Torus^4$, although we shall not write
the toroidal factors explicitly in what follows. We will only consider the
noncommutative deformations of the bosonic parts of these string theories, and
hence only a non-trivial NS background.

The interest in this particular class of pp-waves is that string theory in these
backgrounds can be solved completely and in a fully covariant way
\cite{BAKZ1,CFS1,DAK1,FHHP1,KK1,KKL1,RT2}. They describe homogeneous
gravitational waves (Hpp-waves) and represent the ``minimal'' deformation of
flat spacetime by $H$-flux ($H=dB$, the flux associated to the NS field). They
may be formulated as WZW models based on a twisted Heisenberg group, for which
the wave is an exact solution of the worldsheet $\sigma$-model \cite{KM1,NW1}.

The $\NW_6$ spacetime already captures the generic features of
higher-dimensional Hpp-waves. It can be regarded as the Penrose-G\"uven limit of
the background $\AdS_3\times$ $\S^3\times\Torus^4$ \cite{BFP1,BFHP1} supported
by an NS--NS three-form flux, which describes the near horizon geometry of an
NS5/F1 bound state \cite{GKS1}. The dual superconformal field theory is believed
to be the nonlinear $\sigma$-model with target space the symmetric product
orbifold ${\rm Sym}^N(\Torus^4)$. Similarly, the plane wave metric of $\NW_4$
arises from the Penrose limit of $\AdS_2\times\S^2$ \cite{BFP1,BFHP1}.

Both $\NW_4$ and the Penrose limit of $\AdS_3\times\S^3$ are examples of
non-dilatonic, pp-wave solutions of six-dimensional supergravity \cite{SS-J1}.
However, as we later discuss in detail, the Penrose-G\"uven limit of
$\AdS_2\times\S^2$ does {\it not} induce the full NS-supported geometry of the
$\NW_4$ spacetime. Therefore, contrary to some claims \cite{DeKa1,SF1}, the
four-dimensional Nappi-Witten spacetime cannot be studied as the Penrose-G\"uven
limit of $\AdS_2\times\S^2$. Instead, it arises as a Penrose-G\"uven limit of
the near horizon geometry of NS5-branes \cite{GO1}, on which string theory is
dual to little string theory. This feature can be understood by regarding
$\AdS_2\times\S^2$ as the worldvolume of a symmetric D-brane in the
$\AdS_3\times\S^3$ spacetime, while $\NW_4$ may only be realised as the
worldvolume of a non-symmetric D-brane in $\NW_6$.

We shall find that the most natural plane wave limits of embedded
$\AdS_2\times\S^2$ submanifolds of $\AdS_3\times\S^3$ correspond to two classes
of symmetric D-branes in $\NW_6$. The first one is a {\it flat} euclidean
D3-brane in a constant magnetic field, which carries a noncommutative
worldvolume field theory with constant noncommutativity parameter determined by
the constant time slices of the plane wave background. The second one is a
Lorentzian D3-brane isometric to $\NW_4$ with vanishing NS fields but with a
null worldvolume electric field, which is described in the decoupling limit by a
non-gravitational theory of noncommutative open strings, rather than by a
noncommutative field theory. It is tempting to speculate that the full $\NW_4$
deformation of this noncommutative open string theory describes the dynamics of
the dual little string theory.

This problem is not peculiar to the class of plane wave geometries that we
study, and it leads us into a detailed investigation of how D-branes behave
under the Penrose-G\"uven limit of a spacetime. Similar analyses in some
specific contexts are considered in \cite{DK2,SZ1,SF1}. We formulate and solve
this problem in some generality, and then apply it to our specific backgrounds
of interest. With this motivation at hand, we then proceed to reanalyse the
classification of the symmetric D-branes of $\NW_6$, elaborating on the analysis
initiated in \cite{FS1,SF1} and extending it to a detailed study of the
worldvolume supergravity fields supported by each of these branes.

We also clarify some points which were missed in the analysis of \cite{SF1}. In
each instance we identify the $\AdS_3\times\S^3$ origin of the brane in
question, and quantise its worldvolume geometry using standard techniques and
the representation theory of the twisted Heisenberg group
\cite{BAKZ1,CFS1,KK1,Streater1}. We will find that most of these branes support
{\it local} worldvolume effective field theories, because on most of them the
pertinent supergravity form fields are trivial. In fact, we find that all
symmetric D-branes in $\NW_6$ (both twisted and untwisted) have vanishing NS--NS
three-form flux, and only the two classes of branes mentioned above support a
non-vanishing gauge-invariant two-form field. The overall consistency of these
results, along with their agreement with the exact boundary conformal field
theory description of Cardy branes in $\NW_4$ \cite{DK2}, provides an important
check that the standard techniques for quantisation of worldvolume geometries in
compact group manifolds (see \cite{Schom1} for a review) extend to these classes
of non-compact (and non-semisimple) Lie groups.

Somewhat surprisingly, even the spacetime filling symmetric D5-brane in $\NW_6$
has trivial supergravity form fields. Motivated by this fact, we systematically
construct the noncommutative geometry underlying the non-local field theory
living on a non-symmetric D5-brane wrapping $\NW_6$. The resulting
noncommutativity is non-constant, but independent of the plane wave time
coordinate. This agrees with the recent analysis in \cite{HT1} of the
Dolan-Nappi model \cite{DN1} which describes a time-dependent noncommutative
geometry on the worldvolume of a D3-brane wrapping $\NW_4$. However, the
background used in \cite{DN1} is not conformally invariant and hence not a
closed string background. Correctly reinstating conformal invariance \cite{HT1}
gives a spatially dependent but time-independent noncommutativity parameter. We
elaborate on this noncommutativity somewhat and show that it may be regarded as
arising from a formal quantisation of the twisted Heisenberg algebra.

In Section \ref{NWPW} we review the definition and geometrical properties of the
four-dimensional Nappi-Witten spacetime. We show that its natural pp-wave
isometry group is isomorphic to the six-dimensional twisted Heisenberg group,
paving the way to an analysis of the isometric embeddings
$\NW_4 \hookrightarrow \NW_6$. We also emphasise the time-independent harmonic
oscillator character of point-particle dynamics in these backgrounds, as it
helps to clarify the nature of the noncommutative worldvolume field theories
constructed later on. In Section \ref{IsomEmb} we study the interplay between
isometric embeddings and Penrose-G\"uven limits of branes, first in generality
and then to the particular instances of Nappi-Witten spacetimes. From this
analysis it becomes clear that both the $\NW_4$ and $\NW_6$ gravitational waves
are necessarily wrapped by non-symmetric D-branes. In Section \ref{NCBranes} we
begin our analysis of the symmetric branes in $\NW_6$, beginning with those
described by conjugacy classes of the twisted Heisenberg group.

We identify classes of null branes (with degenerate worldvolume metrics), and
show that their quantised geometries are commutative but generically differ from
those of the classical conjugacy classes due to a unitary rotational symmetry of
the background. We also find a class of euclidean D3-branes and show, directly
from the representation theory of the twisted Heisenberg group, that their
worldvolumes carry a Moyal-type noncommutativity akin to that induced on branes
in constant magnetic fields \cite{DNek1,SW1,Sz1,Sz2}. This sort of
noncommutativity is natural from the point of view of the time-independent
harmonic oscillator dynamics. In Section \ref{TwistedNCBranes} we analyse
symmetric branes in $\NW_6$ which are described by twisted conjugacy classes. We
show, again through explicit quantisation via representation theory and analysis
of the worldvolume supergravity fields, that the low-energy effective field
theories on {\it all} twisted D-branes are local.

\subsection{Definitions}
\label{NWPW}
In this section we will define and analyse the geometry of the Nappi-Witten
spacetime $\NW_4$ \cite{NW1}. It is a four-dimensional homogeneous spacetime of
Minkowski signature which defines a monochromatic plane wave. It is further
equipped with a supergravity NS $B$-field of constant flux, which in the
presence of D-branes is responsible for the spacetime noncommutativity of the
pp-wave. We will emphasise the simple, time-independent harmonic oscillator form
of the dynamics in this background, as it will play a crucial role in subsequent
sections.

The spacetime $\NW_4$ is defined as the group manifold of the Nappi-Witten
group, the universal central extension of the two-dimensional euclidean group
${\rm ISO}(2)={\rm SO}(2)\ltimes\R^2$. The corresponding simply connected
group $\mathcal N_4$ is homeomorphic to four-dimensional Minkowski space
$\E^{1,3}$. Its non-semisimple Lie algebra $\mathfrak n_4$ is generated by
elements $\P^\pm$, $\J$, $\T$ obeying the commutation relations
\begin{eqnarray}
  \label{NW4algdef}
  \left[\P^+ , \P^-\right]&=&2 i \T \nn\\
  \left[\J , \P^\pm\right]&=&\pm i \P^\pm \nn\\
  \left[\T , \J\right]&=&\left[\T , \P^\pm\right] = 0
\end{eqnarray}
This is just the three-dimensional Heisenberg algebra extended by an outer
automorphism which rotates the noncommuting coordinates. The twisted Heisenberg
algebra may be regarded as defining the harmonic oscillator algebra of a
particle moving in one-dimension, with the additional generator $\J$ playing the
role of the number operator (or equivalently the oscillator hamiltonian). It is
a solvable algebra whose properties are much more tractable than, for instance,
those of the ${\rm su}(2)$ or ${\rm sl}(2,\R)$ Lie algebras which are at the
opposite extreme.

The centre of the universal enveloping algebra $U(\mathfrak n_4)$ contains the
central element $\T$ of the Lie algebra $\mathfrak{n}_4$ and also the quadratic
Casimir element
\begin{equation}
  \label{NW4Casimir}
  \Casimir_4=2 \J \T+ \frac12 \left(\P^+ \P^-+\P^- \P^+\right)
\end{equation}
The most general invariant, non-degenerate symmetric bilinear form $\langle
\cdot , \cdot \rangle:\mathfrak{n}_4\times\mathfrak{n}_4\to\R$ is defined by
\cite{NW1}
\begin{eqnarray}
  \label{NW4innerprod}
  \left\langle\P^+ , \P^-\right\rangle&=&2 
  \left\langle\J , \T\right\rangle = 2 \nn\\
  \left\langle\J , \J\right\rangle&=&b \nn\\
  \left\langle\P^\pm , \P^\pm\right\rangle&=&
  \left\langle\T , \T\right\rangle = 0 \nn\\
  \left\langle\J , \P^\pm\right\rangle&=&\left\langle\T , 
    \P^\pm\right\rangle = 0
\end{eqnarray}
for any $b\in\R$. This inner product has Minkowski signature (when $b=0$), so
that the group manifold of $\mathcal N_4$ possesses a homogeneous, bi-invariant
Lorentzian metric defined by the pairing of the Cartan-Maurer left-invariant,
$\mathfrak n_4$-valued one-forms $g^{-1} d g$ for $g\in\mathcal N_4$ as
\begin{equation}
  \label{NW4CM}
  d s_4^2=\left\langle g^{-1}  d g , g^{-1}  d g\right\rangle
\end{equation}
A generic group element $g\in\mathcal N_4$ may be parametrised as
\begin{equation}
  \label{NW4coords}
  g(u,v,a,\overline{a} )=e^{a \P^++\overline{a} \P^-} 
  e^{\theta u \J} e^{\theta^{-1} v \T}
\end{equation}
where $u,v\in\R$, $a\in\C$, and the parameter $\theta\in\R, \theta > 0$ controls
the strength of the NS $B$-field background. In these global coordinates, the
Cartan-Maurer one-form is given by
\begin{equation}
  \label{NW4CMform}
  g^{-1}  d g=e^{- i \theta u}  d a \P^++e^{ i \theta u} 
  d\overline{a} \P^-+\theta  d u \J+\left(\theta^{-1}  d v+ i 
    a  d\overline{a}- i \overline{a}  d a\right) \T
\end{equation}
so that the metric \eqref{NW4CM} reads
\begin{equation}
  \label{NW4metricNW}
  d s_4^2=2  d u  d v+| d a|^2+2 i \theta \left(a 
    d\overline{a}-\overline{a}  d a\right)  d u+b \theta^2  d u^2
\end{equation}

The metric \eqref{NW4metricNW} assumes the standard form of the plane wave
metric for a conformally flat, indecomposable Cahen-Wallach Lorentzian symmetric
spacetime $\CW_4$ in four dimensions \cite{CW1} upon introduction of Brinkman
harmonic coordinates $(x^+,x^-,z)$ \cite{Brink1} defined by rotating the
transverse plane at a Larmor frequency as $u=x^+$, $v=x^-$ and $a=e^{\frac{ i
    \theta}2 x^+} z$. In these coordinates the metric assumes the stationary form
\begin{equation}
  \label{NW4metricBrink}
  d s_4^2=2  d x^+  d x^-+| d z|^2+\theta^2 
  \left(b-\frac14 |z|^2\right) 
  \left( d x^+\right)^2
\end{equation}
revealing the pp-wave nature of the geometry for $b=0$. The physical meaning of
the arbitrary parameter $b$ will be elucidated below. It may be set to zero by
exploiting the translational symmetry of the geometry in $x^-$ to shift $x^-\to
x^--\frac{\theta^2 b}2 x^+$, which corresponds to a Lie algebra automorphism of
$\mathfrak n_4$. Note that on the null planes of constant $u=x^+$, the geometry
becomes that of flat two-dimensional euclidean space $\E^2$. This is the
geometry appropriate to the Heisenberg subgroup of $\mathcal{N}_4$, where the
effects of the twisting generator $\J$ are turned off.

Thus far, the Nappi-Witten spacetime has been described geometrically as a
four-dimensional Cahen-Wallach space $\CW_4$. The spacetime $\NW_4$ is further
supported by a Neveu-Schwarz two-form field $B_4$ of constant field strength
\begin{equation}
  \label{NS3formBrink}
  H_4=-\frac13 \bigl\langle g^{-1}  d g , 
  \left[g^{-1}  d g , g^{-1}  d g\right]\bigl\rangle
   = 2 i \theta  d x^+\wedge d z\wedge d\overline{z} =  d B_4
\end{equation}
where
\begin{equation}
  \label{NS2formBrink}
  B_4=-\frac12 \bigl\langle g^{-1}  d g , 
  \frac{\1+{\rm Ad}_g}{\1-{\rm Ad}_g} g^{-1}  d g\bigl\rangle = 
  2 i \theta x^+  d z\wedge d\overline{z}
\end{equation}
is defined to be non-zero only on those vector fields lying in the range of the
operator $\1-{\rm Ad}_g$ on $T_g\mathcal{N}_4$, i.e. on vectors tangent to the
conjugacy class containing $g\in\mathcal{N}_4$. The corresponding contracted
two-form $H_4^2$ compensates exactly the constant Riemann curvature of the
metric \eqref{NW4metricBrink}, so that $\NW_4$ provides a viable supergravity
background. In fact, in this case the cancellation is exact at the level of the
full string equations of motion, so that the plane wave is an exact background
of string theory \cite{NW1}. It is the presence of this $B$-field that induces
noncommutativity of the string background in the presence of D-branes.

\subsection{Isometries}
\label{Isoms}
The realisation of the geometry of $\NW_4$ as a standard plane wave of
Cahen-Wallach type enables us to study its isometry group using the standard
classification \cite{BOL1}. Writing $\partial_\pm := \partial/\partial x^\pm$,
the metric \eqref{NW4metricBrink} has the obvious null Killing vector
\begin{equation}
  \label{ZKilling}
  T=\theta \partial_-
\end{equation}
generating translations in $x^-$ and characterising a pp-wave, and also the null
Killing vector
\begin{equation}
  \label{HKilling}
  J=\theta^{-1} \partial_+
\end{equation}
generating translations in $x^+$. An analysis of the Killing equations
\cite{BOL1} shows that there are also four extra Killing vectors $P^{(k)}$,
$P^{\prime (k)}$, $k=1,2$ which generate twisted translations in the transverse
plane $z\in\C$ to the motion of the plane wave. Denoting
$\partial:=\partial/\partial z$, they are given in the form
\begin{eqnarray}
  \label{XkXprimegen}
  P^{(k)}&=&c^{(k)}(x^+) \partial+\overline{c}^{ (k)}(x^+) 
  \overline{\partial}-\theta^{-1} \left(\dot c^{(k)}(x^+) \overline{z}+
    \dot{\overline{c}}^{ (k)}(x^+) z\right) \partial_- \nn\\
  P^{\prime (k)}&=&c^{\prime (k)}(x^+) \partial+\overline{c}^{ 
    \prime (k)}(x^+) 
  \overline{\partial}-\theta^{-1} \left(\dot c^{\prime (k)}(x^+) \overline{z}+
    \dot{\overline{c}}^{ \prime (k)}(x^+) z\right) \partial_-  
\end{eqnarray}
where the dots denote differentiation with respect to the light-cone time
coordinate $u=x^+$, and the complex-valued coefficient functions in
\eqref{XkXprimegen} solve the harmonic oscillator equation of motion
\begin{equation}
  \label{HODE}
  \dot c(x^+)=-\frac{\theta^2}4 c(x^+)  
\end{equation}
The four linearly independent solutions of \eqref{HODE} are characterised by
their initial conditions on the null surface $x^+=0$ as
\begin{eqnarray}
  \label{cinitialconds}
  c^{(k)}(0) = \delta_{k1}+ i \delta_{k2} && \dot c^{(k)}(0) = 0\nn\\
  c^{\prime (k)}(0) = 0 && \dot c^{\prime (k)}(0) =
  \theta \left(\delta_{k1}+ i \delta_{k2}\right)  
\end{eqnarray}

The solutions of \eqref{HODE} and \eqref{cinitialconds} are given by
\begin{eqnarray}
  \label{cexplsoln}
  c^{(1)}(x^+) = \cos\frac{\theta x^+}2 &&
  c^{(2)}(x^+) =  i \cos\frac{\theta x^+}2   \nn\\
  c^{\prime (1)}(x^+) = 2\sin\frac{\theta x^+}2 &&
  c^{\prime (2)}(x^+) = 2 i \sin\frac{\theta x^+}2  
\end{eqnarray}
An interesting feature of these functions is that they generate the Rosen form
\cite{Rosen1} of the plane wave metric \eqref{NW4metricBrink}. It is defined by
the transformation to local coordinates $(u,v,y^1,y^2)$ given by
\begin{eqnarray}
  \label{Rosen}
  u&=&x^+   \nn\\v&=&x^--\frac\theta4 (z+\overline{z} )^2 
  \tan\frac{\theta x^+}2-\frac\theta4 (z-\overline{z} )^2 
  \cot\frac{\theta x^+}2    \nn\\y^1&=&\frac12  
  \left((z+\overline{z} )\sec\frac{\theta x^+}2 + i (z-\overline{z} )
    \csc\frac{\theta x^+}2 \right)   \nn\\y^2&=&\frac12  
  \left((z+\overline{z} )\sec\frac{\theta x^+}2 - i (z-\overline{z} )
    \csc\frac{\theta x^+}2 \right)  
\end{eqnarray}
under which the metric becomes
\begin{equation}
  \label{NW4metricRosen}
  d s_4^2=2  d u  d v+C_{ij}(u)  d y^i  d y^j+b \theta^2  d u^2
\end{equation}
where
\begin{equation}
  \label{Cumatrix}
  C(u)=\bigl(C_{ij}(u)\bigr)=
  \begin{pmatrix}
    1&\cos\theta u\\\cos\theta u&1
  \end{pmatrix}  
\end{equation}
This form of the metric is degenerate at the conjugate points where $\cos\theta
u=\pm 1$. The harmonic oscillator solutions \eqref{cexplsoln} then generate an
orthonormal frame for the transverse plane metric \eqref{Cumatrix},
\begin{equation}
  \label{CQrel}
  C(u)=E(u) E^\top(u)  
\end{equation}
with
\begin{eqnarray}
  \label{Qvielbein}
  E=\frac12
  \begin{pmatrix}c^{(1)}+\frac12  c^{\prime (2)}+
    \overline{c}^{ (1)}+\frac12  \overline{c}^{ \prime
      (2)}& &-\left(c^{(2)}-\frac12  c^{\prime (1)}+
      \overline{c}^{ (2)}-\frac12  \overline{c}^{ \prime
        (1)}\right)\\c^{(1)}+\frac12  c^{\prime (2)}+
    \overline{c}^{ (1)}+\frac12  \overline{c}^{ \prime
      (2)}& &c^{(2)}-\frac12  c^{\prime (1)}+
    \overline{c}^{ (2)}-\frac12  \overline{c}^{ \prime(1)}
  \end{pmatrix}
\end{eqnarray}
satisfying the symmetry condition
\begin{equation}
  \label{Esymcond}
  \dot E(u) E^\top(u)=E(u) \dot E^\top(u)
\end{equation}
Note that in contrast to the Brinkman coordinate system, in the Rosen form
\eqref{NW4metricRosen} two extra commuting translational symmetries in the
transverse plane $(y^1,y^2)$ are manifest, while time translation symmetry is
lost.

By defining $P^\pm:=P^{\prime (1)}\pm i P^{(1)}$ and $ Q^{\pm}:=P^{\prime
  (2)}\pm i P^{(2)}$, the six Killing vectors generated by the basic
Cahen-Wallach structure of the plane wave may be summarised as
\begin{eqnarray}
  \label{6CWKilling}
  T&=&\theta \partial_-   \\
  J&=&\theta^{-1} \partial_+   \nn\\
  P^\pm&=&\left(\sin\frac{\theta x^+}2 \pm i e^{\mp 
      \frac{ i \theta}2 x^+}\right)
  \left(\partial+\overline{\partial} \right)-\theta e^{\mp 
    \frac{ i \theta}2 x^+} \left(
    z+\overline{z} \right) \partial_-   \nn\\
  Q^{\pm}&=&\left( i \sin\frac{\theta x^+}2 \mp
    e^{\mp \frac{ i \theta}2 x^+}\right)
  \left(\partial-\overline{\partial} \right)+ i \theta e^{\mp 
    \frac{ i \theta}2 x^+} \left(
    z-\overline{z} \right) \partial_-  \nn
\end{eqnarray}
Together, they generate the harmonic oscillator algebra
$\mathfrak{n}_6$ of a particle moving in {\it two} dimensions,
\begin{eqnarray}
  \label{NW4isomalg}
  \left[P^\alpha ,  Q^{\beta}\right]&=&0 \qquad\qquad\alpha,\beta=\pm   \\
  \left[T ,P^\pm\right]&=&\left[T, Q^{\pm}\right] = \left[T , J\right] = 0\nn\\
  \left[P^+ , P^-\right]&=&\left[Q^{+}, Q^{-}\right] = 2iT\nn\\
  \left[J , P^\pm\right]&=&\pm iP^\pm\nn\\
  \left[J ,  Q^{\pm}\right]&=&\pm i  Q^{\pm} \nn
\end{eqnarray}
This isometry algebra acts transitively on the null planes of constant
  $x^+$ and it generates a central extension ${\mathcal
  N}_6$ of the subgroup
\begin{equation}
  \label{S5subgp}
  \mathcal{S}_5={\rm SO}(2)\ltimes\R^4
\end{equation}
of the four-dimensional euclidean group ${\rm ISO}(4)={\rm SO}(4)\ltimes\R^4$,
where ${\rm SO}(2)$ is the diagonal subgroup of ${\rm SO}(2)\times{\rm
  SO}(2)\subset{\rm SO}(4)$. It is defined by extending the commutation
relations \eqref{NW4algdef} by generators $\Q^{\pm}$ obeying relations as in
\eqref{NW4isomalg}. The quadratic Casimir element $\Casimir_6\in
U(\mathfrak{n}_6)$ and inner product on $\mathfrak{n}_6$ are defined in the
obvious way by extending \eqref{NW4Casimir} and \eqref{NW4innerprod}
symmetrically under $\P^\pm\leftrightarrow\Q^\pm$.

Following the analysis of the previous section, one can show that the
group manifold of ${\mathcal N}_6$ is a six-dimensional Cahen-Wallach
space $\CW_6$, with Brinkman metric
\begin{equation}
  \label{NW6metricBrink}
  d s_6^2=2  d x^+  d x^-+| d\mz|^2+\theta^2\left(b-\frac14
    |\mz|^2\right)\left( d x^+\right)^2
\end{equation}
where $\mz^\top=(z,w)\in\C^2$, which carries a constant Neveu-Schwarz
three-form flux
\begin{eqnarray}
  \nn H_6&=&-2 i \theta  d x^+\wedge d\overline{\mz}^{ \top}
  \wedge d\mz= d B_6\\\label{NW6Bfield}
  B_6&=&-2 i \theta x^+  d\overline{\mz}^{ \top}\wedge d\mz  
\end{eqnarray}
It thereby defines a six-dimensional version $\NW_6$ of the Nappi-Witten pp-wave
\cite{KM1}. This observation will be exploited in the ensuing sections to view
the Nappi-Witten wave as an isometrically embedded D-submanifold ${\rm
  NW}_4\hookrightarrow{\rm NW}_6$. In this setting, it corresponds to a
symmetry-breaking D3-brane in a non-zero $H$-flux.

However, for the Nappi-Witten wave this is not the end of the story. Because of
the bi-invariance of the metric \eqref{NW4CM}, the actual isometry group is the
direct product $\mathcal N_4\times\overline{\mathcal N}_4$ acting by left and
right multiplication on the group $\mathcal N_4$ itself. Since the left and
right actions of the central generator $\T$ coincide, the isometry group is
seven-dimensional. In the present basis, the missing generator from the list
\eqref{6CWKilling} is the left-moving copy $\overline{J}$ of the oscillator
hamiltonian with
\begin{equation}
  \label{JPpm}
  \left[ \overline{J} , P^\pm\right]=\mp i 
  P^\pm+ Q^{\pm}      \left[ \overline{J} , 
    Q^{\pm}\right]=\mp i  Q^{\pm}-P^\pm  
\end{equation}
and it is straightforward to compute that it is given by
\begin{equation}
  \label{extraHKilling}
  \overline{J}=-\theta^{-1} \partial_+- i \left(z \partial-\overline{z} 
    \overline{\partial} \right)  
\end{equation}
The vector field $J+\overline{J}$ generates rigid rotations in the transverse
plane.

\subsection{Commutative Field Theory}
\label{Dynamics}
Standard covariant quantisation of a massless relativistic scalar particle in
$\NW_4$ leads to the Klein-Gordon equation in the curved background,
\begin{equation}
  \label{KGeqn}
  \Box_4\phi = 0  
\end{equation}
where
\begin{equation}
  \label{BoxNW4def}
  \Box_4=2 \partial_+ \partial_--\theta^2 
  \left(b-\frac14  |z|^2\right) \partial_-^2+|\partial|^2
\end{equation}
is the laplacian corresponding to the Brinkman metric \eqref{NW4metricBrink}. It
coincides with the Casimir \eqref{NW4Casimir} expressed in terms of left or
right isometry generators \eqref{6CWKilling}, \eqref{extraHKilling}. The
dependence on the light-cone coordinates $x^\pm$ drops out of the Klein-Gordon
equation because of the isometries generated by the Killing vectors
\eqref{ZKilling} and \eqref{HKilling}.

By using a Fourier transformation of the covariant Klein-Gordon field
$\phi$ along the $x^-$ direction,
\begin{equation}
  \label{KGphiFT}
  \phi\left(x^+,x^-,z,\overline{z} \right)=\int\limits_{-\infty}^\infty
  d p^+ \psi\left(x^+,z,\overline{z};p^+\right) e^{ i  p^+x^-}  
\end{equation}
we may write \eqref{KGeqn} equivalently as
\begin{equation}
  \label{KGmomsp}
  \left[|\partial|^2+2 i  p^+ \partial_++\left(b-
      \frac14  |z|^2\right) \left(\theta p^+\right)^2\right]\psi\left(x^+,z,
    \overline{z};p^+\right)=0  
\end{equation}
Introducing the time parameter $\tau$ through
\begin{equation}
  \label{utaudef}
  u=x^+=p^+ \tau  
\end{equation}
the differential equation \eqref{KGmomsp}) becomes the
Schr\"odinger wave equation
\begin{equation}
  \label{SchHO}
  i  \frac{\partial\psi\left(\tau,z,\overline{z};p^+\right)}{\partial\tau}=
  \left[-\frac12  |\partial|^2+\frac12  \left(
      \frac{\theta p^+}2 \right)^2 
    |z|^2-\frac b2  \left(\theta p^+\right)^2\right]\psi\left(\tau,z,
    \overline{z};p^+\right)
\end{equation}
for the non-relativistic two-dimensional harmonic oscillator with a time
independent frequency given by the light-cone momentum and $H$-flux as
$\omega=|\theta p^+|/2$. The only role of the arbitrary parameter $b$ is to shift
the zero-point energy of the harmonic oscillator, and it thereby carries no
physical significance.

Let us remark that the same hamiltonian that appears in the Schr\"odinger
equation \eqref{SchHO} could also have been derived in light-cone gauge in the
plane wave metric \eqref{NW4metricBrink} starting from the massless relativistic
particle Lagrangian
\begin{equation}
  \label{masslessLag}
  L=\dot x^+ \dot x^-+\frac{\theta^2}2  \left(b-\frac14  
    |z|^2\right) \left(\dot x^+\right)^2+\frac12  \left|\dot z\right|^2
\end{equation}
describing free geodesic motion in the Nappi-Witten spacetime. In the light-cone
gauge, the light-cone momentum is $p^+=p_-=\partial L/\partial\dot x^-=\dot
x^+=1$, while the hamiltonian is $J=p_+=\partial L/\partial\dot x^+$. Imposing
the mass-shell constraint $L=0$ at $\dot x^+=1$ gives the equation of motion for
$x^-$, which when substituted into $J$ yields exactly the hamiltonian appearing
on the right-hand side of \eqref{SchHO} with transverse momentum $p^{
}_\perp=\dot z=- i \partial$.

The time-independence of the effective dynamics here follows from homogeneity of
the plane wave geometry, which prevents dispersion along the light-cone time
direction. These calculations give the quantisation of a particle in $\NW_4$
only in the commutative geometry limit, i.e. in the spacetime $\CW_4$, because
they do not incorporate the supergravity $B$-field supported by the Nappi-Witten
spacetime. In the following we will describe how to incorporate the deformation
of $\CW_4$ caused by the non-trivial NS-sector. Henceforth we will drop the
zero-point energy and set $b=0$.

\subsection{Isometric Embeddings of Branes}
\label{IsomEmb}
A remarkable feature of the Nappi-Witten spacetime is the extent to which it
shares common features with many of the more ``standard'' curved spaces. It is
formally similar to the spacetimes built on the ${\rm SL}(2,\R)$ and ${\rm
  SU}(2)$ group manifolds, but in many ways is much simpler. As a twisted
Heisenberg group, it lies somewhere in between these curved spaces and the flat
space based on the usual Heisenberg algebra. One way to see this feature at a
quantitative level is by examining Penrose-G\"uven limits involving the
(universal covers of the) ${\rm SL}(2,\R)$ and ${\rm SU}(2)$ group manifolds
which produce the spacetime $\NW_4$. This will provide an aid in understanding
various physical properties which arise in later constructions.

In looking for D-submanifolds, we are primarily interested in D-embeddings which
are NS-supported and thereby carry a noncommutative geometry. As we will
discuss, this involves certain important subtleties that must be carefully taken
into account. As the Nappi-Witten spacetime can be viewed as a Cahen-Wallach
space, i.e. as a plane wave, its geometry will arise as Penrose limits of other
metrics. This opens up the possibility of extracting features of $\NW_4$ by
mapping them directly from properties of simpler, better studied noncommutative
spaces. In this section we will begin with a thorough general analysis of the
interplay between Penrose-G\"uven limits and isometric embeddings of Lorentzian
manifolds, and derive simple criteria for the limit and embedding to commute.
Then we apply these results to derive the possible limits that can be used to
describe the NS-supported D-embeddings of $\NW_4$.

\section{Outline of Thesis}
The outline of this thesis is as follows. In Chapter \ref{pglimits} we introduce
the concept of the Penrose-G\"{u}ven limit and impose a constraint for isometric
embedding diagrams. We then examine the Lie branes of $\NW_6$ in Chapter
\ref{liebranes}, highlighting the implied noncommutative geometries. We close
the Chapter with the discovery of a time dependent, noncommutative geometry on
$\NW_6$ which exhibits the same algebra as $\mathfrak{n}_6$.

The discovery of such a spacetime is motivation for studying three classes of
Nappi-Witten $\star$-products in Chapter \ref{star}. We construct the necessary
tools for investigating the associated free scalar field theories and Chapter
\ref{fieldtheory} concludes by investigating the field theories of our three
orderings and the regularly embedded D-branes in $\NW_6$.

%%% Local Variables: 
%%% mode: latex
%%% TeX-master: "main.tex"
%%% End: 
