\chapter{Scalar Field Theory}
\label{fieldtheory}

We are now ready to apply the detailed constructions of the preceding sections
to the analysis of noncommutative field theories on the plane wave $\NW_6$,
regarded as the worldvolume of a non-symmetric D5-brane \cite{KNSanjay1}. In
this paper we will only study the simplest example of free scalar fields,
leaving the detailed analysis of interacting field theories and higher spin
(fermionic and gauge) fields for future work. The analysis of this section will
set the stage for more detailed studies of noncommutative field theories in
these settings, and will illustrate some of the generic features that one can
expect.

Given a real scalar field $\Phi\in\CC^\infty(\mfn^\vee)$ of mass $m$, we define
an action functional using the integral \eqref{ncintdef} by
\begin{equation}
  \label{Svarphidef}
  S[\Phi]=\int\limits_{\R^6} \dd\mbf x \kappa(\mbf x) \left[
    \frac12  \eta_{ab} \bigl( \widetilde
    {\partial}^{ a}_{\star}\triangleright\Phi\bigr)\star
    \bigl( \widetilde{\partial}^{ b}_{\star}\triangleright\Phi\bigr)+
    \frac12  m^2 \Phi\star\Phi\right]
\end{equation}
where $\eta_{ab}$ is the invariant Minkowski tensor induced by the inner product
\eqref{NW4innerprod} with the non-vanishing components $\eta_{\pm \mp}=1$ and
$\eta_{z_i \overline{z}_j}=\frac12 \delta_{ij}$. The tildes on the derivatives
in \eqref{Svarphidef} indicate that the time component must be appropriately
shifted as described in the previous section. Using the property
\eqref{ncintaddprop} we may simplify the action to the form
\begin{equation}
  \label{Svarphisimpl}
  S[\Phi]=\int\limits_{\R^6} \dd\mbf x \kappa(\mbf x) \left[
    \frac12  \eta_{ab} \bigl( \widetilde
    {\partial}^{ a}_{\star}\triangleright\Phi\bigr)
    \bigl( \widetilde{\partial}^{ b}_{\star}\triangleright\Phi\bigr)+
    \frac12  m^2 \Phi^2\right]
\end{equation}

By using the integration by parts property \eqref{ncintparts} on Schwartz fields
$\Phi$, we may easily compute the first order variation of the action
\eqref{Svarphisimpl} to be
\begin{equation}
  \label{actionvary1}
  \frac{\delta S[\Phi]}{\delta\Phi} \delta\Phi:=
  S[\Phi+\delta\Phi]-S[\Phi]=-\int\limits_{\R^6} \dd\mbf x 
  \kappa(\mbf x) \left[\eta_{ab} \overline{\widetilde
      {\partial}_\star^{ a}}\triangleright\bigl( \widetilde
    {\partial}_\star^{ b}\triangleright\Phi\bigr)-m^2 \Phi^2
  \right] \delta\Phi
\end{equation}
Applying the variational principle $\frac{\delta S[\Phi]}{\delta\Phi}=0$
to \eqref{actionvary1} thereby leads to the noncommutative Klein-Gordon field
equation
\begin{equation}
  \label{NCeom}
  \Box^\star\triangleright\Phi-m^2 \Phi=0
\end{equation}
where
\begin{equation}
  \label{Boxstardef}
  \Box^\star\triangleright\Phi:=2 \partial_+\triangleright
  \partial_-\Phi+\bfd^\top\triangleright
  \overline{\bfd}\triangleright\Phi+ \frac12  
  \partial_+\ln\kappa \partial_-\Phi
\end{equation}
and we have used $\partial_-\kappa=0$. The second order $\star$-differential
operator $\Box^\star$ should be regarded as a deformation of the covariant
Laplace operator $\Box_0^\star$ corresponding to the commutative plane wave
geometry of $\NW_6$. This Laplacian coincides with the quadratic Casimir element
\begin{equation}
  \label{quadCasimir}
  {\sf C}_6:=\theta^{-2} \eta^{ab} \X_a \X_b
  =2 \J \T+ \frac12 \sum\limits_{i=1,2}  \bigl(
  \P_+^i \P_-^i+\P_-^i \P_+^i\bigr)
\end{equation}
of the universal enveloping algebra $U(\mfn)$, expressed in terms of left or
right isometry generators for the action of the isometry group $\mathcal{N}_{\rm
  L}\times\mathcal{N}_{\rm R}$ on $\NW_6$ \cite{PK1,CFS1,Halliday:2005zt}.

However, in the manner which we have constructed things, this is not the case.
Recall that the approximation in which our quantisation of the geometry of
$\NW_6$ holds is the small time limit $x^+\to0$ in which the plane wave
approaches flat six-dimensional Minkowski space $\E^{1,5}$. To incorporate the
effects of the curved geometry of $\NW_6$ into our formalism, we have to replace
the derivative operators $\widetilde{\partial}_\star^{ a}$ appearing in
\eqref{Svarphidef} with appropriate curved space analogs $\delta_\star^a$
\cite{BehrSyk1,HoMiao1}.

Recall that the derivative operators $\partial_\star^{a}$ are {\it not}
derivations of the $\star$-product $\star$, but instead obey the deformed Leibniz
rules \eqref{defLeibniz}. The deformation arose from twisting the co-action of
the bialgebra $U(\mfg)$ so that it generated automorphisms of the noncommutative
algebra of functions, i.e. isometries of the noncommutative plane wave. The
basic idea is to now ``absorb'' these twistings into derivations
$\delta_\star^a$ obeying the usual Leibniz rule
\begin{equation}
  \label{deltaLeibniz}
  \delta_\star^a\triangleright(f\star g)=\left(\delta_\star^a
    \triangleright f\right)\star g+f\star\left(\delta_\star^a
    \triangleright g\right)
\end{equation}
These derivations generically act on $\CC^\infty(\mfn^\vee)$ as noncommutative
$\star$-polydifferential operators
\begin{equation}
  \label{polydiffops}
  \delta_\star^a\triangleright f=\sum_{n=1}^\infty \xi_a^{a_1\cdots
    a_n}\star\left(\partial_{a_1}^\star\triangleright\cdots\triangleright
    \partial_{a_n}^\star \triangleright f\right)
\end{equation}
with $\xi_a^{a_1\cdots a_n}\in\CC^\infty(\mfn^\vee)$. Unlike the derivatives
$\partial_\star^a$, these derivations will no longer $\star$-commute among each
other. There is a one-to-one correspondence \cite{Kont1} between such
derivations $\delta_a^\star$ and Poisson vector fields $E^a=E^a_b \partial^b$ on
$\mfn^\vee$ obeying
\begin{equation}
  \label{Poissonvecfields}
  E^a\circ\Theta(f,g)=\Theta(E^af,g)+\Theta(f,E^ag)
\end{equation}
for all $f,g\in\CC^\infty(\mfn^\vee)$. To leading order one has
$\delta_\star^a\triangleright f=E^a_b\star(\partial_\star^b\triangleright
f)+O(\theta)$. By identifying the Lie algebra $\mfn$ with the tangent space to
$\NW_6$, at this order the vector fields $E^a$ can be thought of as defining a
natural local frame with flat metric $\eta_{ab}$ and a curved metric tensor
$G^\star_{ab}=\frac12 \eta_{cd} (E^c_a\star E^d_b+E^d_a\star E^c_b)$ on the
noncommutative space $\NW_6$. However, for our $\star$-products there are always
higher order terms in \eqref{polydiffops} which spoil this interpretation. The
noncommutative frame fields $\delta_\star^a$ describe the {\it quantum} geometry
of the plane wave $\NW_6$. In particular, the metric tensor $G^\star$ will in
general differ from the classical open string metric $G_{\rm open}$. While the
operators $\delta_\star^a$ always exist as a consequence of the Kontsevich
formality map \cite{Kont1,BehrSyk1}, computing them explicitly is a highly
difficult problem. We will see some explicit examples below, as we now begin to
tour through our three $\star$-products. Throughout we shall take the natural
choice of measure $\kappa=\sqrt{|\det G|}=\frac12$, the constant Riemannian
volume density of the $\NW_6$ plane wave geometry.

\section{Time Ordering}
\label{ScalarTO}
In the case of time ordering, we use \eqref{TOderivs} to compute
\begin{equation}
  \label{TOBoxeq}
  \Box^*\triangleright\Phi=\left(2 \partial_+ \partial_-+
    \overline{\bfd}\cdot\bfd\right)\Phi
\end{equation}
and thus the equation of motion coincides with that of a free scalar particle on
flat Minkowski space $\E^{1,5}$ (Deviations from flat spacetime can only come
about here by choosing a time-dependent measure $\kappa_*$). This illustrates
the point made above that the treatment of this thesis tackles only the
semi-classical flat space limit of the spacetime $\NW_6$. The appropriate curved
geometry for this ordering corresponds to the global coordinate system
\eqref{NW4metricNW} in which the classical Laplace operator is given by
\begin{equation}
  \label{TOBox0}
  \Box_0^*=2 \partial_+ \partial_-+
  \left|\bfd+ \frac i 2  \theta \overline{\mz} 
    \partial_-\right|^2
\end{equation}
so that the free wave equation $(\Box_0^*-m^2)\Phi=0$ is equivalent to the
Schr\"odinger equation for a particle of charge $p^+$ (the momentum along the
$x^-$ direction) in a constant magnetic field of strength $\theta$. A global
pseudo-orthonormal frame is provided by the commutative vector fields
\begin{eqnarray}
  \label{TOorthoframe}
  E_-^*&=&\partial_-   \nn\\ E_+^*&=&\partial_+- i \theta \left(
    \mz\cdot\bfd-\overline{\mz}\cdot\overline{\bfd} \right)
  \\ E^{i}_*&=&\partial^i   \nn\\
  \overline{E}_{*}^{ i}&=&\overline{\partial}^{ i} \nn
\end{eqnarray}

Determining the derivations $\delta_*^a$ corresponding to the commuting frame
\eqref{TOorthoframe} on the quantum space is in general rather difficult.
Evidently, from the coproduct structure \eqref{TOLeibniz} the action along the
light-cone position is given by
\begin{equation}
  \label{TOdeltaminus}
  \delta_-^*\triangleright f=\partial_-f
\end{equation}
This is simply a consequence of the fact that translations along $x^-$ generate
an automorphism of the noncommutative algebra of functions, i.e. an isometry of
the noncommutative geometry. From the Hopf algebra coproduct \eqref{TOcoprods}
we have
\begin{equation}
  \label{TOcoprodglobal}
  \Delta_*\bigl( e^{ i \theta \partial_-}\bigr)=
   e^{ i \theta \partial_-}\otimes e^{ i \theta \partial_-}
\end{equation}
and consequently
\begin{equation}
  \label{xmautoNC}
   e^{ i \theta \partial_-}\triangleright(f*g)=\bigl(
   e^{ i \theta \partial_-}\triangleright f\bigr)*\bigl(
   e^{ i \theta \partial_-}\triangleright g\bigr)  
\end{equation}

On the other hand, the remaining isometries involve intricate twistings between
the light-cone and transverse space directions. For example, let us demonstrate
how to unravel the coproduct rule for $\partial_+^*$ in \eqref{TOLeibniz} into
the desired symmetric Leibniz rule \eqref{deltaLeibniz} for $\delta_+^*$. This
can be achieved by exploiting the $*$-product identities
\begin{eqnarray}
  \label{TOstarprodcomm}
  z_i*f&=&\bigl( e^{ i \theta \partial_-}f\bigr)*z_i-2 i \theta 
  x^+ \overline{\partial}^{ i}f\\
  \overline{z}_i*f&=&\bigl( e^{- i \theta \partial_-}f\bigr)*
  \overline{z}_i+2 i \theta x^+ \partial^if \nn
\end{eqnarray}
along with the commutativity properties
$[\partial_-^*,z_i]_*=[\partial_-^*,\overline{z}_i]_*=0$ for $i=1,2$ and for
arbitrary functions $f$. Using in addition the modified Leibniz rules
\eqref{TOLeibniz} along with the $*$-multiplication properties \eqref{TOxfprods}
we thereby find
\begin{equation}
  \label{TOdeltap}
  \delta_+\triangleright f=\left[x^+ \partial_++ \frac1{2 i }  
    \left(\mz\cdot\overline{\bfd}+\overline{\mz}\cdot\bfd\right)
  \right]f  
\end{equation}
This action mimics the form of the classical frame field $E_+^*$ in
\eqref{TOorthoframe}.

Finally, for the transversal isometries, one can attempt to seek functions
$g^i\in\CC^\infty(\mfn^\vee)$ such that $g^i*f=( e^{- i
  \theta \partial_-}f)*g_i$ in order to absorb the light-cone translation in the
Leibniz rule for $\partial_*^i$ in \eqref{TOLeibniz}. This would mean that the
$x^-$ translations are generated by {\it inner} automorphisms of the
noncommutative algebra. If such functions exist, then the corresponding
derivations are given by $\delta^i_*\triangleright f=g^i*\partial_*^if$ (no sum
over $i$) and similarly for $\overline{\delta}^{ i}_*$. However, it is doubtful
that such inner derivations exist and the transverse space frame fields are more
likely to be given by higher-order $*$-polyvector fields. For example, using
similar steps to those which led to \eqref{TOdeltap}, one can show that the
actions
\begin{eqnarray}
  \label{TOdeltatransv}
  \delta_*\triangleright f&:=&\left( \overline{\mz}\cdot\bfd+2 i 
    x^+ \partial_+- i \theta x^+ \overline{\bfd}\cdot\bfd\right)f\\
  \overline{\delta}_*\triangleright f&:=&\left({\mz}\cdot
    \overline{\bfd}-2 i 
    x^+ \partial_++ i \theta x^+ \overline{\bfd}\cdot\bfd\right)f\nn
\end{eqnarray}
define derivations of the $*$-product on $\NW_6$, and hence naturally
determine elements of a noncommutative transverse frame.

The action of the corresponding noncommutative Laplacian $\eta_{ab}
\delta_*^a\triangleright(\delta_*^b\triangleright\Phi)$ deforms the harmonic
oscillator dynamics generated by \eqref{TOBox0} by non-local higher spatial
derivative terms. These extra terms will have significant ramifications at large
energies for motion in the transverse space. This could have profound physical
effects in the interacting noncommutative quantum field theory. In particular,
it may alter the UV/IR mixing story \cite{MVRS1} in an interesting way. For
time-dependent noncommutativity with standard tree-level propagators, UV/IR
mixing becomes intertwined with violations of energy conservation in an
intriguing way \cite{BG1,RS1}, and it would be interesting to see how our
modified free field propagators affect this analysis. It would also be
interesting to see if and how these modifications are related to the generic
connection between wave propagation on homogeneous plane waves and the
Lewis-Riesenfeld theory of time-dependent harmonic oscillators \cite{BOL1}.

\section{Symmetric Time Ordering}
\label{ScalarSTO}
The analysis in the case of symmetric time ordering is very similar to that just
performed, so we will be very brief and only highlight the essential changes.
From \eqref{STOderivs} we find once again that the Laplacian \eqref{Boxstardef}
coincides with the flat space wave operator
\begin{equation}
  \label{STOBoxeq}
  \Box^\bullet\triangleright\Phi=\left(2 \partial_+ \partial_-+
    \overline{\bfd}\cdot\bfd\right)\Phi  
\end{equation}
The relevant coordinate system in this case is given by the Brinkman metric
\eqref{NW4metricBrink} for which the classical Laplace operator reads
\begin{equation}
  \label{STOBox0}
  \Box_0^\bullet=2 \partial_+ \partial_-+\overline{\bfd}\cdot
  \bfd- \frac14  \theta^2 |\mz|^2 \partial_-^2  
\end{equation}
A global pseudo-orthonormal frame in this case is provided by the vector fields
\begin{eqnarray}
  \label{STOorthoframe}
  E_-^\bullet&=&\partial_-   \nn\\ E_+^\bullet&=&\partial_++
  \frac18  \theta^2 |\mz|^2 \partial_-\\ E^{i}_\bullet&=&\partial^i   \nn\\
  \overline{E}_{\bullet}^{ i}&=&\overline{\partial}^{i} \nn
\end{eqnarray}
The corresponding twisted derivations $\delta^a_\bullet$, which symmetrise the
Leibniz rules \eqref{STOLeibniz} can be constructed analogously to those of the
time ordering case in Section \ref{ScalarTO} above.

\section{Weyl Ordering}
\label{ScalarWeyl}
Finally, the case of Weyl ordering is particularly interesting because the
effects of curvature are present even in the flat space limit. Using
\eqref{Weylderivs} we find the Laplacian
\begin{equation}
  \label{WeylBoxeq}
  \Box^\star\triangleright\Phi=\left(2 \partial_+ \partial_-
    +2 \left[2 \left(1-\frac{\sin(\theta \partial_-)}
        {\theta \partial_-}\right)+\frac{1-\cos(\theta \partial_-)}
      {\theta^2 \partial_-^2}\right] \overline{\bfd}\cdot\bfd
  \right)\Phi
\end{equation}
which coincides with the flat space Laplacian only at $\theta=0$. To
second order in the deformation parameter $\theta$, the equation of
motion \eqref{NCeom} thereby yields a second order correction to the
usual flat space Klein-Gordan equation given by
\begin{equation}
  \label{KGeqcorr}
  \left[\left(2 \partial_+ \partial_-+\overline{\bfd}\cdot\bfd-m^2
    \right)+ \frac7{12}  \theta^2 \partial_-^2 \overline{\bfd}\cdot
    \bfd+O\left(\theta^4\right)\right]\Phi=0  
\end{equation}
Again we find that only the transverse space motion is altered by
noncommutativity, but this time through a non-local dependence on the light-cone
momentum $p^+$ yielding a drastic modification of the dispersion relation for
free wave propagation in the noncommutative spacetime. This dependence is
natural. The classical mass-shell condition for motion in the curved background
is $2 p^+ p^-+|4 \theta p^+ \mbf\lambda|^2=m^2$, where $\mbf\lambda\in\C^2$
represents the position and radius of the circular trajectories in the
background magnetic field \cite{CFS1}. Thus the quantity $4 \theta p^+
\mbf\lambda$ can be interpreted as the momentum for motion in the transverse
space. The operator \eqref{WeylBoxeq} incorporates the appropriate
noncommutative deformation of this motion. It illustrates the point that the
fundamental quanta governing the interactions in the present class of
noncommutative quantum field theories are not likely to involve the
particle-like dipoles of the flat space cases, but more likely string-like
objects owing to the nonvanishing $H$-flux in \eqref{NS2formBrink}. These open
string quanta become polarised as dipoles experiencing a net force due to their
couplings to the non-uniform $B$-field. It is tempting to speculate that, in
contrast to the other orderings, the Weyl ordering naturally incorporates the
new vacua corresponding to long string configurations which are due entirely to
the time-dependent nature of the background Neveu-Schwarz field \cite{BAKZ1}.

While the Weyl ordered $\star$-product is natural from an algebraic point of
view, it does not correspond to a natural coordinate system for the plane wave
$\NW_6$ due to the complicated form of the group product rule
\eqref{Weylgpprodexpl} in this case. In particular, the frame fields in this
instance will be quite complicated. Computing the corresponding twisted
derivations $\delta^a_\star$ directly would again be extremely cumbersome, but
luckily we can exploit the equivalence between the $\star$-products $\star$ and
$*$ derived in Section \ref{WOP}. Given the derivations $\delta^a_*$ constructed
in Section \ref{ScalarTO} above, we may use the differential operator
\eqref{Gdiffopexpl} which implements the equivalence \eqref{WeylTOrel} to define
\begin{equation}
  \label{Weyldelta}
  \delta^a_\star\triangleright f:=\mathcal{G}^{ }_\Omega\circ
  \delta^a_*\triangleright\bigl(\mathcal{G}_\Omega^{-1}(f)\bigr)  
\end{equation}
These noncommutative frame fields will lead to the appropriate curved space
extension of the Laplace operator in \eqref{WeylBoxeq}.

\section{Worldvolume Field Theories}
\label{D3Branes}
In this final section we will describe how to build noncommutative field
theories on regularly embedded worldvolumes of D-branes in the spacetime $\NW_6$
using the formalism described above. We shall describe the general technique on
a representative example by comparing the noncommutative field theory on $\NW_6$
to that of the noncommutative D3-branes which was constructed in Chapter
\ref{liebranes}. We shall do so in a general fashion which illustrates how the
construction extends to generic D-branes. This will provide further perspective
on the natures of the different quantisations we have used throughout, and also
illustrate the overall consistency of our results. As we will now demonstrate,
we can view the noncommutative geometry of $\NW_6$, in the manner constructed
above, as a collection of all euclidean noncommutative D3-branes taken together.
This is done by restricting the geometry to obtain the usual quantisation of
coadjoint orbits in $\mfn^\vee$ (as opposed to all of $\mfn^\vee$ as described above).
This restriction defines an alternative and more geometrical approach to the
quantisation of these branes which does not rely upon working with
representations of the Lie group $\mathcal{N}$, and which is more adapted to the
flat space limit $\theta\to0$.

This procedure can be thought of as somewhat opposite to the philosophy of
\cite{Halliday:2005zt}, which quantised the geometry of a non-symmetric D5-brane
wrapping $\NW_6$ \cite{KNSanjay1} by viewing it as a noncommutative foliation by
these euclidean D3-branes. Here the quantisation of the spacetime-filling brane
in $\NW_6$ has been carried out independently leading to a much simpler
noncommutative geometry which correctly induces the anticipated worldvolume
field theories on the $\E^4$ submanifolds of $\NW_6$.

The euclidean D3-branes of interest wrap the non-degenerate conjugacy classes of
the group $\mathcal{N}$ and are coordinatised by the transverse space
$\mz\in\C^2\cong\E^4$ \cite{SF1}. They are defined by the spacelike
hyperplanes of constant time in $\NW_6$ given by the transversal intersections
of the null hypersurfaces
\begin{eqnarray}
  \label{D3subsps}
  x^+&=&{\rm constant}\\
  x^-+ \frac14  \theta |\mz|^2 \cot\left( \frac12  
    \theta x^+\right)&=&{\rm constant}\nn
\end{eqnarray}
independently of the chosen coordinate frame. This describes the brane
worldvolume as a wavefront expanding in a sphere in the transverse space. In the
semi-classical flat space limit $\theta\to0$, the second constraint in
\eqref{D3subsps} to leading order becomes
\begin{equation}
  \label{Cdefconst}
  C:=2 x^+ x^-+|\mz|^2={\rm constant}
\end{equation}
The function $C$ on $\mfn^\vee$ corresponds to the Casimir element
\eqref{quadCasimir} and the constraint \eqref{Cdefconst} is analogous to the
requirement that Casimir operators act as scalars in irreducible
representations. Similarly, the constraint on the time coordinate $x^+$ in
\eqref{D3subsps} is analogous to the requirement that the central element $\T$
act as a scalar operator in any irreducible representation of $\mathcal{N}$.

Let $\pi:\NW_6\to\E^4$ be the projection of the six-dimensional plane wave
onto the worldvolume of the symmetric D3-branes. Let
$\pi^\sharp:\CC^\infty(\E^4)\to\CC^\infty(\NW_6)$ be the induced algebra
morphism defined by pull-back $\pi^\sharp(f)=f\circ\pi$. To consistently reduce
the noncommutative geometry from all of $\NW_6$ to its conjugacy classes, we
need to ensure that the candidate $\star$-product on $\mfn^\vee$ respects the Casimir
property of the functions $x^+$ and $C$, i.e. that $x^+$ and $C$ $\star$-commute
with every function $f\in\CC^\infty(\mfn^\vee)$. Only in that case can the
$\star$-product be consistently restricted from all of $\NW_6$ to a $\star$-product
$\star_{x^+}$ on the conjugacy classes $\E^4$ defined by
\begin{equation}
  \label{starxpdef}
  f\star_{x^+}g:=\pi^\sharp(f)\star\pi^\sharp(g)  
\end{equation}
Then one has the compatibility condition
\begin{equation}
  \label{compcondWeyl}
  \iota^\sharp(f\star g)=\iota^\sharp(f)\star_{x^+}\iota^\sharp(g)
\end{equation}
where $\iota^\sharp:\CC^\infty(\NW_6)\to\CC^\infty(\E^4)$ is the pull-back
induced by the inclusion map $\iota:\E^4\hookrightarrow\NW_6$. In this case one
has an isomorphism $\CC^\infty(\E^4)\cong\CC^\infty(\NW_6)/\mathcal{J}$ of
associative noncommutative algebras \cite{Waldmann1}, where $\mathcal{J}$ is the
two-sided ideal of $\CC^\infty(\NW_6)$ generated by the Casimir constraints
$(x^+-{\rm constant})$ and $(C-{\rm constant})$. This procedure is essentially a
noncommutative version of the Poisson reduction, with the Poisson ideal
$\mathcal{J}$ implementing the geometric requirement that the Seiberg-Witten
bi-vector $\Theta$ be tangent to the conjugacy classes.

From the $\star$-commutators \eqref{eq:time:comm}, \eqref{eq:symtime:comm} and
\eqref{eq:weyl:comm} we see that $[x^+,f]_\star=0$ for all three of our
$\star$-products. However, the condition $[C,f]_\star=0$ is {\it not} satisfied.
Although classically one has the Poisson commutation $\Theta(C,f)=0$, one can
only consistently restrict the $\star$-products by first defining an appropriate
projection of the algebra of functions on $\mfn^\vee$ onto the
$\star$-subalgebra $\mathcal{C}$ of functions which $\star$-commute with the
Casimir function $C$. One easily computes that $\mathcal{C}$ naturally consists
of functions $f$ which are independent of the light-cone position, i.e.
$\partial_-f=0$. Then the projector $\iota^\sharp$ above may be applied to the
subalgebra $\mathcal{C}$ on which it obeys the requisite compatibility condition
\eqref{compcondWeyl}. The general conditions for reduction of Kontsevich
star-products to D-submanifolds of Poisson manifolds are described
in \cite{CattFel2,CFal1}.

With these projections implicitly understood, one straightforwardly finds that
all three $\star$-products \eqref{TOstargen}, \eqref{TOsymstargen} and
\eqref{Weylstargen} restrict to
\begin{equation}
  \label{fstargrestrict}
  f\star_{x^+}g=\mu\circ\exp\left[ i \theta x^+ \left(
      \bfd^\top\otimes\overline{\bfd}-
      {\overline{\bfd}}^{ \top}\otimes\bfd\right)\right]f\otimes g
\end{equation}
for functions $f,g\in\CC^\infty(\E^4)$. This is just the Moyal product, with
noncommutativity parameter $\theta x^+$, on the noncommutative euclidean
D3-branes. It is cohomologically equivalent to the Voros product
\eqref{D3starfinal} which arises from quantising the conjugacy classes through
endomorphism algebras of irreducible representations of the twisted Heisenberg
algebra $\mfn$, with a normal or Wick ordering prescription for the generators
$\P_\pm^i$. In this case, the noncommutative euclidean space arises from a
projection of $U(\mfn)$ in the discrete representation $V^{p^+,p^-}$ whose
second Casimir invariant \eqref{quadCasimir} is given in terms of light-cone
momenta as ${\sf C}=-2 p^+ (p^-+\theta)$ and with $\T=\theta p^+$. In this
approach the noncommutativity parameter is naturally the {\it inverse} of the
effective magnetic field $p^+ \theta$. On the other hand, the present analysis
is a more geometrical approach to the quantisation of symmetric D3-branes in
$\NW_6$ which deforms the euclidean worldvolume geometry by a time parameter
$\theta x^+$ without resorting to endomorphism algebras. The relationship
between the two sets of parameters is given by $x^+=p^+ \tau$, where $\tau$ is
the proper time coordinate for geodesic motion in the pp-wave geometry of
$\NW_6$.

In contrast to the coadjoint orbit quantisation \cite{Halliday:2005zt}, the
noncommutativity found here matches exactly that predicted from string theory in
the semi-classical limit \cite{DAK1}, which asserts that the Seiberg-Witten
bi-vector on the D3-branes is given by $\Theta_{x^+}=\frac i 2 \sin(\theta x^+)$
$\bfd^\top\wedge \overline{\bfd}$. Note that the present analysis also covers as
a special case the degenerate cylindrical null branes located at time $x^+=0$
\cite{SF1}, for which \eqref{fstargrestrict} becomes the ordinary pointwise
product $f\star_0g=f\,g$ of worldvolume fields and as expected these branes
support a {\it commutative} worldvolume geometry. In contrast, the commutative
null branes correspond to the class of continuous representations of the twisted
Heisenberg algebra having quantum number $p^+=0$ which must be dealt with
separately \cite{Halliday:2005zt}.

It is elementary to check that the rest of the geometrical constructs of this
thesis reduce to the standard ones appropriate for a Moyal space. By defining
\begin{equation}
  \label{derivxpdef}
  \partial_{\star_{x^+}}^a\triangleright
  f:=\iota^\sharp\circ\partial_\star^a\triangleright\bigl(\pi^\sharp(f)\bigr)
\end{equation}
one finds that the actions of the derivatives constructed in Section
\ref{Derivatives} all reduce to the standard ones of flat noncommutative
euclidean space, i.e. $\partial_{\star_{x^+}}^i\triangleright f=\partial^if$,
$\overline{\partial}^{ i}_{\star_{x^+}}\triangleright f=\overline{\partial}^{
  i}f$ for $f\in\CC^\infty(\E^4)$. From Section \ref{Coprod} one recovers the
standard Hopf algebra of these derivatives with trivial coproducts
$\Delta_{\star_{x^+}}$ defined by
\begin{equation}
  \label{Deltaxpdef}
  \Delta_{\star_{x^+}}(\nabla_{\star_{x^+}})\triangleright
  (f\otimes g):=\bigl(\iota^\sharp\otimes\iota^\sharp\bigr)\circ
  \Delta_\star(\nabla_\star)
  \triangleright\bigl(\pi^\sharp(f)\otimes\pi^\sharp(g)\bigr)  
\end{equation}
and hence the symmetric Leibniz rules appropriate to the translational symmetry
of field theory on Moyal space. Consistent with the restriction to the conjugacy
classes, one also has $\partial_\pm^{\star_{x^+}}\triangleright f=0$.

However, from \eqref{TOcoprodtime}, \eqref{STOcoprodtime} and
\eqref{Weylcoprodtime} one finds a non-vanishing co-action of time translations
given by
\begin{equation}
  \label{Moyalcoprodtime}
  \Delta_{\star_{x^+}}\bigl(\partial_+^{\star_{x^+}}\bigr)=
  \theta \bigl(\bfd_{\star_{x^+}}^\top\otimes
  \overline{\bfd}_{\star_{x^+}}-
  \overline{\bfd}_{\star_{x^+}}^\top\otimes\bfd_{\star_{x^+}}\bigr)  
\end{equation}
This formula is very natural. The isometries of $\NW_6$ in $\mfg=\mfn_{\rm
  L}\oplus\mfn_{\rm R}$ corresponding to the number operator $\J$ of the twisted
Heisenberg algebra are generated by the vector fields $J_{\rm
  L}=\theta^{-1} \partial_+$ and $J_{\rm R}=-\theta^{-1} \partial_+- i
(\mz\cdot\bfd- \overline{\mz}\cdot\overline{\bfd} )=\theta^{-1} E_+^*$ (in
Brinkman coordinates). The vector field $J_{\rm L}+J_{\rm R}$ generates rigid
rotations in the transverse space. Restricted to the D3-brane worldvolume, the
time translation isometries thus truncate to rotations of $\E^4$ in ${\rm
  so}(4)$. The coproduct \eqref{Moyalcoprodtime} gives the standard twisted
co-action of rotations for the Moyal algebra which define quantum rotational
symmetries of noncommutative euclidean space \cite{CPT1,CKNT1,Wess1}. This
discussion also drives home the point made earlier that our derivative operators
$\partial_\star^a$ indeed do generate, through their twisted co-actions (Leibniz
rules), quantum isometries of the full noncommutative plane wave.

Finally, a trace on $\CC^\infty(\E^4)$ is induced from \eqref{ncintdef} by
restricting the integral to the submanifold $\iota:\E^4\hookrightarrow\NW_6$ and
using the induced measure $\iota^\sharp(\kappa)$. For the measures constructed
in Section \ref{Integrals}, $\iota^\sharp(\kappa)$ is always a constant function
on $\E^4$ and hence the integration measures all restrict to the constant volume
form of $\E^4$. Thus noncommutative field theories on the spacetime $\NW_6$
consistently truncate to the anticipated worldvolume field theories on
noncommutative euclidean D3-branes in $\NW_6$, together with the correct twisted
implementation for the action of classical worldvolume symmetries. The advantage
of the present point of view is that many of the novel features of these
canonical Moyal space field theories naturally originate from the pp-wave
noncommutative geometry when the Moyal space is regarded as a regularly embedded
coadjoint orbit in $\mfn^\vee$, as described above. Furthermore, the method
detailed in this thesis allows a more systematic construction of the deformed
worldvolume field theories of {\it generic} D-branes in $\NW_6$ in the
semi-classical regime, and not just the symmetric branes analysed here. For
instance, the analysis can in principle be applied to describe the dynamics of
symmetry-breaking D-branes which localise along products of twisted conjugacy
classes in the Lie group $\mathcal{N}$ \cite{Quella1}. However, these branes
have yet to be classified in the case of the gravitational wave $\NW_6$.

%%% Local Variables: 
%%% mode: latex
%%% TeX-master: "main.tex"
%%% End: 
