\chapter{Conclusions}
\label{conclusion}

In this thesis, we have studied symmetric and non-symmetric (twisted) D-branes
of the six dimensional Nappi-Witten spacetime and discovered that the majority
of the noncommutative geometries are simpler than one would have expected; the
symmetric branes all observe a flat space with zero NS flux, allowing standard
open string models to be used to calculate the noncommutativity parameter. We
found a class of commutative null branes and a class of noncommutative Euclidean
D3-branes, analogous to branes in a constant magnetic field.

Central to the discoveries of the noncommutative nature of branes in $\NW_6$ was
the ability to perform Penrose-G\"uven limits on embedded spaces and keep track
of the supergravity fields in isometric embedding diagrams. To achieve this, we
produced formal coordinate free restrictions that one must observe in order to
be guaranteed a commuting diagram.

The most interesting non-symmetric branes were found to be Hpp-waves
\textit{without} the characteristic Nappi-Witten NS flux, and hence classified
as Cahen-Wallach spacetimes, not Nappi-Witten spacetimes as previously thought.
One of these non-symmetric branes was found to possess a complicated worldvolume
flux that corresponds to an electric field and hence does not exhibit a well
defined decoupling of the massive string states. However, by choosing an
appropriate gauge for the $\NW_6$ $B$ field, we have been able to discover a
spatially varying noncommutativity, analogous to that of the Dolan-Nappi model.
This gauge choice results in non-vanishing Poisson brackets that reproduce the
Nappi-Witten Lie algebra in the small time limit.

Inspired by this noncommutative geometry that reproduces the Nappi-Witten Lie
algebra, we proceeded to obtain closed and explicit forms of the
$\star$-products for several physically important orderings; corresponding to
the global and Brinkman coordinatisations of the spacetime and the Weyl
symmetric ordering. By using the formalism of generalised Weyl systems, we were
able to calculate the Hopf algebra of twisted isometries.

By placing linear constraints on the $\star$-derivatives, we restricted our
analysis to the flat space limit; the trade off being that we were able to
obtain Leibniz rules for the derivatives and therefore proceed in a systematic
manner. After formalising the rules for integration we advanced to the scalar
field theory, confirming that our analysis is consistent with a flat space
limit. We document the pseudo-orthonormal frames and twisted derivatives that
deform the commutative Laplacian, finding that only transverse space motion is
effected by the commutativity.

Restricting the algebraic analysis of the Nappi-Witten Lie algebra to embedded
worldvolumes, we confirmed the previous analysis of the noncommutative
geometries and developed a method that allows a more systematic construction of
the deformed worldvolume field theories of generic D-branes in $\NW_6$ in the
semi-classical regime.

All techniques throughout this thesis have been presented in such a manner that
they may be applied to a broad range of homogeneous pp-waves supported by a
constant Neveu-Schwarz flux.

Further research in this area could involve the study of an interacting $\Phi^4$
theory. In canonical noncommutative field theory, we calculate products such as
$e^{iq_ix^i}\star e^{iq_i'x^i}$ at each vertex in a Feynman diagram. We now know
that such an exponential product will be spacetime-dependent, meaning that
planar Feynman diagrams will be granted a phase shift analogous to those
observed in non-planar quantum field theories (such that result in UV/IR
mixing). We must also face the possibility of violations in energy and momentum
conservation.

%%% Local Variables: 
%%% mode: latex
%%% TeX-master: "main.tex"
%%% End: 
