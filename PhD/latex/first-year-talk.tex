\documentclass[14pt, a4paper, titlepage]{slides}
\usepackage{amsmath,amsfonts,amssymb,amsthm,amstext,amscd,array,bbold}
\usepackage[latin1]{inputenc}
\DeclareMathOperator{\AdS}{AdS}
\DeclareMathOperator{\Sphere}{S}
\DeclareMathOperator{\NW}{NW}
\DeclareMathOperator{\CW}{CW}
\let\S\Sphere
% Commutator Bracket (in math mode)
\newcommand{\cb}[2]{\left[ {#1} , {#2} \right]}
% Left Bracket (in math mode)
\newcommand{\lb}{\left(}
% Right Bracket (in math mode)
  \newcommand{\rb}{\right)}

\begin{document}
\title{End of First Year Report: Talk}
\author{Sam Halliday}
\maketitle

\textbf{Time Dependent Harmonic Oscillator}

A relatively unknown technique was outlined in 1969\footnote{H. R. Lewis, W.  B.
  Riesenfeld, {\it Journ. Math. Phys.} {\bf 10} 8 (1969) 1458} for calculating
exact solutions to time dependent problems in quantum mechanics. As an
application of the technique, a harmonic oscillator of the type
\begin{equation*}
  H=p^2+\Omega(t)^2q^2
\end{equation*}
was studied. As this is not the most favoured form of the harmonic oscillator,
we attempted to apply the same technique to a harmonic oscillator of the form
\begin{equation*}
  H=\Omega(t)\left(p^2+q^2\right)
\end{equation*}
The technique requires an invariant quantity $I$ to exist, which we calculated
as
\begin{equation*}
  I=\lambda\left(Q^2-P^2\right)  
\end{equation*}
where $Q$ and $P$ are canonically conjugate operators. The states (to within a
gauge) of $I$ are $e^{i\Theta(t)}|\phi\rangle$ where $\Theta$ is defined by
\begin{equation*}
  \hbar \frac{d\Theta}{dt}  =\langle\phi|i\hbar \frac{\partial}{\partial t}
  -H|\phi\rangle
\end{equation*}
by the technique we can get the states of the original problem by
\begin{equation*}
  \Phi=\int \frac{e^{i\left(\theta + \mu \ln{U}\right)}}{\sqrt{2\pi U}}d\mu
\end{equation*}
where $U=Q-P$ and $\mu$ is related to the eigenvalues of $I$. Unfortunately
this requires more than just the solution of the inverted harmonic oscillator,
but also the solution of $\langle\phi|i\hbar\frac{\partial}{\partial t}
-H|\phi\rangle$. We reduced the solution to
\begin{eqnarray*}
  \langle \phi''|i\hbar \frac{\partial }{\partial t} - H|\phi' \rangle &=& \frac{1}{2\pi}
  \int^\infty_{-\infty} e^{i\left(\mu' - \mu'' \right) x}dx \\
  &&\times\left\{2B \left( i\mu' - \frac{1}{2}\right)^2\right.\nonumber\\
  &&+ B \left( i\mu'-\frac{1}{2}\right) i \mu' e^{-2x} \nonumber\\
  &&- Ae^{2x} - \frac{3B}{2} \left( i\mu'-\frac{1}{2}\right) e^{-\frac{x}{2}}\nonumber\\
  &&\left.- \dot{\mu}' \hbar x \right\}\nonumber
\end{eqnarray*}
where $x=\ln{U}$ and $A$, $B$ can be thought of as frequencies. All the
parameters are time dependent and there are three terms which clearly cause the
solution to blow up to infinity, hence we found no solution to the problem we
faced by using this technique.

\textbf{Light Cone Quantisation of a Free Particle on $\NW_4$}

The $\NW_4$ spacetime can be thought of as a plane wave
\begin{equation*}
  ds^2 = 2dx^-dx^+ + A_{ij}z^iz^j(dx^-)^2 + d\vec{z}^2
\end{equation*}
Using only the metric of a plane wave, a technique is known\footnote{M. Blau,
  M. O'Loughlin, {\it hep-th/0212135}} for finding the Killing vectors and light
cone quantisation of a spacetime. Apart from the obvious Killing vectors
\begin{eqnarray*}
  Z &=& \partial_-\\
  X &=& \partial_+
\end{eqnarray*}
the technique gave four extra
\begin{eqnarray*}
  &&X^{(1)}=\cos{\lb\frac{x^+}{2}\rb}\partial_1+
  \sin{\lb\frac{x^+}{2}\rb}z^1\partial_- \nonumber \\
  &&X^{(2)}=\cos{\lb\frac{x^+}{2}\rb}\partial_2+
  \sin{\lb\frac{x^+}{2}\rb}z^2\partial_- \nonumber \\
  &&X^{\prime(1)}=2\sin{\lb\frac{x^+}{2}\rb}\partial_1-
  \cos{\lb\frac{x^+}{2}\rb}z^1\partial_- \nonumber \\
  &&X^{\prime(2)}=2\sin{\lb\frac{x^+}{2}\rb}\partial_2-
  \cos{\lb\frac{x^+}{2}\rb}z^2\partial_-
\end{eqnarray*}
exhibiting the harmonic oscillator algebra
\begin{eqnarray*}
  \cb{X^{(k)}}{X^{(l)}}&=&\cb{X^{\prime(k)}}{X^{\prime(l)}}=0 \nonumber \\
  \cb{X^{(k)}}{Z}&=&\cb{X^{\prime(k)}}{Z}=0 \nonumber \\
  \cb{X^{(k)}}{X^{\prime(l)}}&=&-\delta^{kl}Z \nonumber \\
  \cb{X}{X^{(k)}}&=&-\delta^k_l X^{\prime(l)} \nonumber \\
  \cb{X}{X^{\prime(k)}}&=&X^{(k)} \nonumber \\
  \cb{X}{Z}&=&0
\end{eqnarray*}
The Lagrangian for this metric is
\begin{equation*}
  L=\dot{x}^-\dot{x}^+
  +\frac{1}{2}\lb b-\frac{\vec{z}^2}{4}\rb\lb\dot{x}^-\rb^2 +\frac{\vec{z}^2}{2}
\end{equation*}
where the light-cone momentum
\begin{equation*}
  P_+=\frac{\partial L}{\partial \dot{x}^+}=\dot{x}^-=1
\end{equation*}
is conserved. Imposing the constraint $L=0$ for a massless particle, we find the
conserved quantity
\begin{equation*}
  \left. \frac{\partial L}{\partial \dot{x}^-} \right|_{\dot{x}^-=1}
  =\frac{b}{2}-\frac{1}{2}\lb \vec{p}^2 + \frac{\vec{z}^2}{4} \rb
\end{equation*}
which is a trivial time-independent harmonic oscillator with an additional
ground state energy. In the point-particle case this conserved quantity is
simply the Hamiltonian of the system.

We confirmed this result by using a direct Klein-Gordon approach
\begin{equation*}
  \lb g^{\mu\nu}\partial_\mu\partial_\nu\rb \phi =
  \left\{ 2\partial_-\partial_+ +\lb
    \frac{\vec{z}^2}{4}-b\rb\partial_-^2+\partial_{\vec{z}}^2\right\}\phi
\end{equation*}
giving (after a change of representation and parameter substitution)
\begin{equation*}
  \label{eq:NW:KG:KGSHO}
  -i\frac{\partial \psi\lb \tau,p_-,\vec{z}\rb}{\partial \tau}=
  \left\{ \frac{b}{2} - \frac{1}{2}\lb
    \frac{\vec{z}^2p_-^2}{4} - \partial_{\vec{z}}^2 \rb\right\}
  \psi\lb \tau,p_-,\vec{z}\rb
\end{equation*}
These calculations can be thought of as the light cone quantisation of $\NW_4$
in the commutative geometry limit ($\Theta$, $B=0$).

\textbf{Star-Product Contractions}

A group contraction is a way of getting to a different algebra by some rescaling
of the generators. For example $Sl(2,\mathbb{R})\rightarrow h(1)$.
\begin{eqnarray*}
  \cb{x_1}{x_2} &=& -x_3\\
  \cb{x_1}{x_3} &=& x_2\\
  \cb{x_2}{x_3} &=& x_1
\end{eqnarray*}
and defining
\begin{eqnarray*}
  P_1&=&\Omega x_1\\
  P_2&=&\Omega x_2\\
  T&=&-\Omega^2x_3
\end{eqnarray*}
with $\Omega\rightarrow 0$ we get
\begin{eqnarray*}
  \cb{P_1}{P_2}&=&T\\
  \cb{P_1}{T}&=&\cb{P_2}{T}=0
\end{eqnarray*}
which is $h(1)$. Similarly for the $\star$-product
\begin{eqnarray*}
  x_j\star f&=&\left\{x_j-\frac{i\theta}{2}g_{jk}c_{klm}x_l\partial_m\right.\\
  &&-\frac{\theta^2}{8}\left[\left(1+x_l\partial_l\right)g_{jk}\partial_k -
    \frac{x_j}{2}g_{lm}\partial_l\partial_m\right]\left.\right\}f
\end{eqnarray*}
gives
\begin{eqnarray*}
  T\star f &=& Tf\\
  P_1\star f &=& \left\{P_1+\frac{i\theta}{2}T\partial_2\right\}f\\
  P_2\star f &=& \left\{P_2-\frac{i\theta}{2}T\partial_1\right\}f
\end{eqnarray*}
which are the correct star products for $h(1)$. We wished to perform a similar
contraction on $\mathrm{SU}(2)\oplus \mathrm{Sl}(2,\mathbb{R})$ to get the
double extension of the Poincar\'{e} algebra (which is the group manifold of
$\NW_6$, a six-dimensional analogy of $\NW_4$). Unfortunately we got unexpected
infinities, showing that $\star$-products do not always contract well. As a
simpler example of the infinities, let us look at $SU(2)\rightarrow E(2)$
\begin{eqnarray*}
  \cb{x_1}{x_2}&=&x_3\\
  \cb{x_1}{x_3}&=&-x_2\\
  \cb{x_2}{x_3}&=&x_1
\end{eqnarray*}
and define
\begin{eqnarray*}
  P_1&=&\Omega x_1\\
  P_2&=&\Omega x_2\\
  J&=&x_3
\end{eqnarray*}
with $\Omega\rightarrow 0$. We get
\begin{eqnarray*}
  \cb{P_1}{P_2}&=&0\\
  \cb{J}{P_1}&=&P_2\\
  \cb{J}{P_2}&=&-P_1
\end{eqnarray*}
which is $E(2)$. However, on $SU(2)$, the $\star$-products are
\begin{eqnarray*}
  x_i\star f&=&\left\{x_i +\frac{\delta_{ij}x_0}{2}\partial_j +
    \frac{i\epsilon_{ijk}}{2}x_k\partial_j\right\}f\\
  x_0\star f&=& \left\{x_0 + \frac{x^k}{2}\partial_k\right\}f
\end{eqnarray*}
which contract to
\begin{eqnarray*}
  P_1\star f&=& \left\{ P_1 -\frac{i}{2}P_2\partial_J\right\}f\\
  P_2\star f&=& \left\{ P_2 +\frac{i}{2}P_1\partial_J\right\}f\\
  J\star f &=&  \left\{ J + x_0\partial_J+\frac{i}{2}P_2\partial_1-\frac{i}{2}P_1\partial_2\right\}f
\end{eqnarray*}
but $x_0\rightarrow 0$, meaning that although the Lie algebra is still
preserved, point-wise multiplication is infinite.

\textbf{$\NW_4$ as a Penrose Limit}

Group contractions are closely related to Penrose limits. A Penrose limit is
essentially a way of getting a Plane wave from any Lorentz spacetime. Since it
is often more convenient to perform calculations on well studied spaces and take
the limit to a plane wave, they have become very popular recently; especially
since G\"{u}ven extended the Penrose limit to supergravity backgrounds. As
$\NW_4$ is itself a plane wave, it allows the possibility of doing calculations
on a simpler spacetime and map the results directly. A particular Penrose limit
of $\NW_4$ has been known for some time\footnote{M. Blau, J. Figueroa-O'Farrill,
  G. Papadopoulos, \textit{Class.  Quant. Grav} \textbf{19} (2002) 4753,
  \textit{hep-th/0202111}}, which is the limit of $\AdS_2\times\S^2$. It has
also been shown\footnote{S. Stanciu, J. Figueroa-O'Farrill, ``Penrose limits of Lie
  Branes and a Nappi-Witten braneworld'', \textit{JHEP} \textbf{0306} (2003)
  025, \textit{hep-th/0303212}} that $\NW_4$ can be achieved as a Penrose limit
of $\AdS_3\times\S^3$ and a projection, via a six dimensional analogy of the
Nappi-Witten space.
\begin{equation*}
  \begin{CD}
    @.\\
    \AdS_3 \times \S^3             @>\text{PGL}>> \NW_6\\
    \text{$\imath$}@AAA @AAA\text{$\widetilde{\imath}$}\\
    \AdS_2 \times \S^2             @>\text{PGL}>> \NW_4\\
    @.
  \end{CD}
\end{equation*}
where the top Penrose-G\"{u}ven limit is similar to the $\mathrm{SU}(2)\oplus
\mathrm{Sl}(2,\mathbb{R})$ group contraction we previously studied. We found
that this diagram does not actually commute, and in fact that $\NW_4$ is not
even a PGL of $\AdS_2\times\S^2$, but rather $\CW_4$ which is like the
commutative geometry limit of $\NW_4$. In finding why this diagram did not
commute, we generalised isometric embedding diagrams with PGLs and found
conditions to ensure that the metric and supergravity fields commute. The
conditions are written in my report and I will now show this diagram as an
explicit example

On $\AdS_3\times\S^3$ we may write the metric, NS-NS $B$ field and strength $H$
in conventional coordinates as
\begin{eqnarray*}
  G &=& - d\tau^2 + \sin^2{\tau}\left( \frac{dr^2}{1+r^2} +
    \left(1+r^2\right)d\beta^2\right) \\ \nonumber
  &&+d\phi^2+\sin^2{\phi}\left(\frac{d\chi^2}{1-\chi^2}+
    \left(1-\chi^2\right)d\theta^2\right)\\
  -H/2 &=& \cos^2{\tau} d\tau\wedge dr\wedge d\beta + \sin^2{\phi}d\phi\wedge
  d\chi\wedge d\theta\\
  -2B &=& \left(\sin{2\tau}+2\tau\right)dr\wedge d\beta\\
  &&+\left(\sin{2\phi}-2\phi\right)d\chi\wedge d\theta
\end{eqnarray*}
where $\AdS_3$ has coordinates $(\tau,r,\beta)$ and $\S^3$ has
$(\phi,\chi,\theta)$ so that $D=6$. Making a change of coordinates
\begin{eqnarray*}
  2\tau=U-V &\qquad& 2\phi=U+V \\ \nonumber
  r=Y^1\quad\beta=Y^2 &\qquad& \chi=Y^3\quad\theta=Y^4
\end{eqnarray*}
we can now express the metric and fields along the geodesic $\gamma$ as
\begin{eqnarray*}
  C(u)_{ii} &=& \sin^2{\frac{U}{2}} \\ \nonumber
  b(u)_{12}&=&-\frac{\sin{U}+U}{2}\\ \nonumber
  b(u)_{34}&=&\frac{U-\sin{U}}{2}
\end{eqnarray*}
The projection $\imath^*$ consists of setting $Y^q\rightarrow 0$ where $q=1,3$
and $\widetilde{\imath}$ as $y^r\rightarrow 0$ where $r=1,2$. We can now observe
the sets $S=\{2,4\}$, $T=\{3,4\}$ and we see clearly that there is no such
pairing between $S$ and $T$ such that the potential or field strength conditions
are satisfied (equation (20) in the report), as it will result in requiring
$b(u)_{12}=b(u)_{1j}$ where $j$ is either $3$ or $4$. We can however see that
equation (19) will always be satisfied since all the components of $C(u)_{ii}$
are identical. So we expect the metric but not the fields to commute in such a
diagram.

We can confirm these predictions by following each path explicitly\footnote{It
  is easier to work in Brinkman coordinates $(x^-,x^+,z^i)$ at this stage, we
  have made the transformation from Rosen coordinates by
  \begin{equation}
    \label{eq:ex:jose:brinkman}
    u=2x^- \qquad v=x^+ + \frac{\cot{x^-}}{2}\vec{z}^2 \qquad y^i=\frac{z^i}{\sin{x^-}}
  \end{equation}}. Firstly, the path
$\AdS_3\times\S^3\rightarrow\NW_6\rightarrow\NW_4$ yields
\begin{eqnarray*}
  \widetilde{G}&=&2dx^-dx^+ +\frac{\vec{z}^2}{4}\left(d
    x^-\right)^2 + \left(d\vec{z}\right)^2 \\
  \widetilde{H}&=&-dx^-\wedge dz^1\wedge dz^2 \\
  \widetilde{B}&=&-x^- dz^1\wedge dz^2
\end{eqnarray*}
which is in fact, the Nappi-Witten WZW model. The alternative path
$\AdS_3\times\S^3\rightarrow\AdS_2\times\S^2\rightarrow\NW_4$ yields
\begin{eqnarray*}
  \widetilde{G}&=&2dx^-dx^+ +\frac{\vec{z}^2}{4}\left(d
    x^-\right)^2 + \left(d\vec{z}\right)^2 \\
  \widetilde{H}&=&d\widetilde{B}=0
\end{eqnarray*}
confirming the prediction that the metric commutes, but the fields do not.

We can now see why the diagram breaks down and we may attempt to remedy it. The
essential problem is in choosing a suitable $\imath^*$ projection where we may
pair up the elements of $S$ and $T$. The easiest and most obvious way to fix
this is to make the projection $\widetilde{\imath}^*$ equivalent to $\imath^*$,
i.e. so that we choose $\imath^*: Y^l\rightarrow 0$ where $l=1,2$. Since the
sets $S=\{3,4\}$ and $T=\{3,4\}$ are now identical, the commutative conditions
are trivially satisfied. This has however changed the diagram somewhat into the
following
\begin{equation*}
  \begin{CD}
    @.\\
    \AdS_3 \times \S^3             @>\text{PGL}>> \NW_6\\
    \text{$\imath$}@AAA @AAA\text{$\widetilde{\imath}$}\\
    \mathbb{R}^1\times\S^3         @>\text{PGL}>> \NW_4\\
    @.
  \end{CD}
\end{equation*}
The final space of each path is the $\NW_4$ model and both metric and fields
commute in the diagram. It is interesting to note that the PGL on the bottom of
the diagram has also been studied\footnote{G. D'Appollonio, E. Kiritsis,
  ``String interactions in gravitational wave backgrounds'',
  \textit{hep-th/0305081}}, albeit with slight technical differences from the
PGL used here.

We could also have fixed the diagram by choosing the projection
$\widetilde{\imath}^*: y^l\rightarrow 0$ where $l=1,3$ so that it is equivalent
to $\imath$ and that the diagram will again trivially commute. Again this
involves a change in the diagram
\begin{equation*}
  \begin{CD}
    @.\\
    \AdS_3 \times \S^3             @>\text{PGL}>> \NW_6\\
    \text{$\imath$}@AAA @AAA\text{$\widetilde{\imath}$}\\
    \AdS_2 \times \S^2             @>\text{PGL}>> \CW_4\\
    @.
  \end{CD}
\end{equation*}
where $\CW_N$ is used to denote a Cahen-Wallach space in $N$ dimensions.
\end{document}
