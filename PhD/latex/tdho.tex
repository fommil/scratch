\documentclass[11pt, a4paper, titlepage]{article}
\usepackage{epsfig}
%\usepackage{color}
\usepackage{amssymb}
\usepackage{amsmath}
%\setlength\arraycolsep{2pt}
%\setlength\tabcolsep{5pt}

% These are new commands i made at the end of the type up, so i haven't used them much
% Partial Derivative
\newcommand{\pd}{\partial}
% Left Bracket
\newcommand{\lb}{\left(}
% Right Bracket
\newcommand{\rb}{\right)}
% Commutator Bracket
\newcommand{\cb}[2]{\left[ #1 , #2 \right]}
% Anti Commutator Bracket
\newcommand{\acb}[2]{\left[ #1 , #2 \right]_+}
% EQuatioN reference
\newcommand{\eqn}[1]{(\ref{#1})}

\begin{document}
\section{The Time Dependant Harmonic Oscillator}
Using the technique as outlined in \cite{ref_td}, a solution for the
harmonic oscillator of type
\begin{equation}
  \label{eqn_tdhoH}
  H(t) = \omega(t) \lb p^2 + q^2 \rb
\end{equation}
is investigated. Where $\omega(t)$ is a real function of time and the
$q$'s and $p$'s are canonically conjugate time independent
operators. The equations of motion are
\begin{eqnarray}
  \label{eqn_tdhoqeom}
  \frac{d q}{dt} &= \frac{1}{i \hbar} [q, H(t)] =& 2 \omega(t) p \\
  \label{eqn_tdhopeom}
  \frac{d p}{dt} &= \frac{1}{i \hbar} [p, H(t)] =& - 2 \omega(t) q
\end{eqnarray}
As required by the method, we postulate an invariant $I(t)$, which we
choose freely to be of the form
\begin{equation}
  \label{eqn_tdhoI}
   I(t) = \alpha(t) q^2 + \beta(t) p^2 + \gamma(t) [q, p]_+
\end{equation}
where $\alpha(t)$, $\beta(t)$, $\gamma(t)$ are real functions of time. By imposing the condition
\begin{equation}
  \label{eqn_tdhoIdot}
  \frac{dI(t)}{dt} = \frac{\pd I(t)}{\pd t} + \frac{1}{i \hbar} [I(t), H(t)] = 0
\end{equation}
henceforth dropping `$(t)$' from the notation and using dots to denote
partial derivatives with respect to time, we obtain
\begin{equation}
  \label{eqn_tdhoIdotcondition}
   \frac{dI(t)}{dt} = (\dot{\alpha} - \omega \gamma)q^2 + (\dot{\beta} + \omega \gamma)p^2 + (\dot{\gamma} + 2 \omega \alpha -2 \omega \beta)[q, p]_+
\end{equation}
which implies that
\begin{eqnarray}
  \label{eqn_cnd1}
  \dot{\alpha} &=& \omega \gamma \\
  \label{eqn_cnd2}
  \dot{\beta} &=&  - \omega \gamma \\
  \label{eqn_cnd3}
  \dot{\gamma} &=& 2 \omega \beta - 2 \omega \alpha
\end{eqnarray}
It is clear that $\dot{\alpha} = - \dot{\beta}$, and we are free to set
\begin{eqnarray}
  \label{eqn_cnd4}
   \alpha &=& - \beta \\
  \label{eqn_cnd5}
   \therefore \dot{\gamma} &=& - 4 \omega \alpha
\end{eqnarray}
This allows us to re-write (\ref{eqn_tdhoI}) in terms of $\alpha$
\begin{equation}
  \label{eqn_tdhoIa}
  I = \alpha \left( q^2 - p^2 \right) + \frac{\dot{\alpha}}{\omega} [q, p]_+
\end{equation}
We now require a constraint upon $\alpha$, which is easily obtained by
differentiating $\gamma$ in (\ref{eqn_cnd1}) and inserting into
(\ref{eqn_cnd5}), giving
\begin{equation}
  \label{eqn_cnsrnta}
  \ddot{\alpha} - \frac{\dot{\omega}}{\omega}\dot{\alpha} + 4\omega^2 \alpha = 0
\end{equation}
which has the trivial solution when $\omega$ is constant
\begin{equation}
  \label{eqn_cnsrntcnsta}
   \alpha = \sin (2\omega t + \theta)
\end{equation}
where $\theta$ is an arbitrary phase factor.

The form of $I$ in (\ref{eqn_tdhoIa}) is awkward to work with, and may
be re-written as a sum of two squares
\begin{equation}
  \label{eqn_tdhoIa2}
  I = \alpha
  \left\{
    \left(
      1 + \frac{\dot{\alpha}^2}{\alpha^2 \omega^2}
    \right) q^2 -
    \left(
      p - \frac{\dot{\alpha}}{\alpha \omega} q
    \right)^2
  \right\}
\end{equation}
allowing the substitutions
\begin{eqnarray}
  \label{eqn_chi}
   \chi^2 &=& 1 + \frac{\dot{\alpha}^2}{\alpha^2 \omega^2} \\
  \label{eqn_Q}
    Q &=& \sqrt{\frac{\chi}{\hbar}} q \\
  \label{eqn_P}
    P &=& \frac{1}{\sqrt{\chi \hbar}} \left( p - \frac{\dot{\alpha}}{\alpha \omega} q \right) \\
  \label{eqn_lambda}
    \lambda &=& 2\hbar\chi\alpha
\end{eqnarray}
to give
\begin{equation}
  \label{eqn_I}
  I = \frac{\lambda}{2} \left\{ Q^2 - P^2 \right\}
\end{equation}
where $Q$ and $P$ are canonically conjugate time dependant operators
\begin{equation}
  \label{eqn_QPcc}
   [Q, P] = i
\end{equation}
This is clearly the inverted harmonic oscillator, which is similar to
the `upside down' harmonic oscillator studied in \cite{ref_udho}. (In
that scenario, the minus is before the $q^2$, not the $p^2$, and thus
requiring a separate representation) In order to remove the minus
sign, we may introduce the canonically conjugate operators $U$ and
$V$, defined as
 \begin{eqnarray}
  \label{eqn_U}
    U &=& \frac {1}{\sqrt{2}} \left( Q - P \right) \\
  \label{eqn_V}
    V &=& \frac {1}{\sqrt{2}} \left( Q + P \right)
\end{eqnarray}
where
\begin{equation}
  \label{eqn_uvcc}
   [U, V] = i
\end{equation}
allowing a re-write of (\ref{eqn_I}) to give
\begin{equation}
  \label{eqn_Kuv}
   I = \frac{\lambda}{2} \left\{U V + V U \right\}
\end{equation}
The eigenfunctions $\phi$ are solutions to
\begin{equation}
   \label{eqn_schrodinger}
    \frac{\lambda}{2}(UV + VU)\phi = E \phi
\end{equation}
For $U$, introducing the parameter $\mu=\frac{E}{\lambda}$
\begin{equation}
  \label{eqn_uvev}
    U \frac{d \phi(U)}{d U} = (i \mu - \frac{1}{2}) \phi(U)
\end{equation}
the formal (normalised) solutions are \cite{ref_uvvu}
\begin{equation}
  \label{eqn_phisol}
    \phi(U) = \frac{1}{\sqrt{2 \pi}} U^{i\mu - \frac{1}{2}}
\end{equation}
Solutions (\ref{eqn_phisol}) are only defined for positive values of
$U$. We may find that the states are normalised and orthogonal. The
same can be said for the $\phi(V)$ solutions. 
\begin{equation}
  \label{eqn_phino}
    \langle \phi' | \phi'' \rangle = \frac{1}{2\pi}\int^\infty_0 U^{i(\mu'' - \mu') - 1} d U
\end{equation}
substituting $x = \ln U$
\begin{equation}
  \label{eqn_phinox}
    \langle \phi' | \phi'' \rangle = \frac{1}{2\pi}\int^\infty_{-\infty} e^{i(\mu'' - \mu')x} dx = \delta(\mu''-\mu')
\end{equation}
In order to make use of \cite{ref_td} we require the gauge transformed
states $e^{i \Theta_\phi(t)} | \phi \rangle$ where $\Theta_\phi(t)$ is defined by
\begin{equation}
  \label{eqn_alpha}
    \hbar \frac{d \Theta_\phi}{d t} = \langle \phi|i\hbar \frac{\partial }{\partial t} - H|\phi \rangle
\end{equation}
Firstly rewriting (\ref{eqn_tdhoH}) in terms of the operators $Q$
(\ref{eqn_Q}) and $P$ (\ref{eqn_P}), we get 
\begin{eqnarray}
  \label{eqn_HQP1}
        H &=& \left(\frac{w\hbar}{\chi}+\frac{\hbar \dot{\alpha}^2}{\chi\alpha^2\omega}\right)Q^2
          + \chi\omega\hbar P^2
          + \frac{\hbar\dot{\alpha}}{\alpha}[P,Q]_+ \\
  \label{eqn_HQP2}
          &=& \chi\omega\hbar \left( Q^2+P^2\right) + \frac{\hbar\dot{\alpha}}{\alpha}[P,Q]_+
\end{eqnarray}
Equations \eqn{eqn_U} and \eqn{eqn_V} can be rearranged to give
\begin{eqnarray}
  \label{eqn_U2}
    Q^2 + P^2 &=& 2U^2 + \acb Q P \\
  \label{eqn_V2}
    Q^2 + P^2 &=& 2V^2 - \acb Q P \\
   \label{eqn_UVac}
    \acb P Q  &=& \frac{1}{2}\lb V^2 - U^2\rb
\end{eqnarray}
Allowing $H$ to be written in two ways
\begin{eqnarray}
  \label{eqn_HU2}
    H &=& 2\chi\omega\hbar U^2 + \lb 2\chi\omega\hbar + \frac{\hbar\dot{\alpha}}{\alpha}\rb \acb P Q \\
  \label{eqn_HV2}
      &=& 2\chi\omega\hbar V^2 - \lb 2\chi\omega\hbar - \frac{\hbar\dot{\alpha}}{\alpha}\rb \acb P Q
\end{eqnarray}
Adding these linearly to obtain $2H$ and then halving, we get
\begin{equation}
   \label{eqn_HUV1}
    H = \chi\omega\hbar \lb U^2 + V^2 \rb + \frac{\hbar\dot{\alpha}}{\alpha}\acb P Q
\end{equation}
Using \eqn{eqn_UVac}, $H$ can be written as desired in terms of $U$
and $V$. 
\begin{equation}
   \label{eqn_HUV}
    H = A U^2 + B V^2
\end{equation}
where
\begin{equation}
   \label{eqn_AandB}
    A = \chi\omega\hbar - \frac{\hbar\dot{\alpha}}{2\alpha}
    \qquad
    B = \chi\omega\hbar + \frac{\hbar\dot{\alpha}}{2\alpha}
\end{equation}
We find
\begin{eqnarray}
  \label{eqn_phiHphi}
   \langle \phi''|H|\phi' \rangle &=&
   \frac{1}{2\pi}\int^\infty_{-\infty}
   e^{i\lb\mu' - \mu'' \rb x} \\
   &\times&
   \left\{
     Ae^{2x} - Be^x\lb i\mu'-\frac{1}{2}\rb\lb i\mu'e^{-3x}-\frac{3}{2}e^{-\frac{3}{2}x}\rb
   \right\}
   dx \nonumber
\end{eqnarray}
Noting that we may obtain
\begin{equation}
   \label{eqn_partialU}
    \frac{\pd U}{\pd t} = \frac{2BU}{\hbar}\lb \mu+\frac{1}{2}i\rb
\end{equation}
from the differential of $\hat{u}\phi=u\phi$. Thus we calculate
\begin{equation}
  \label{eqn_ddt}
   \langle \phi''|i\hbar \frac{\pd }{\pd t}|\phi' \rangle = \frac{i\hbar}{2\pi} \int^\infty_{-\infty}
   e^{i\lb\mu' - \mu'' \rb x}
   \left\{
     2B \lb i\mu' - \frac{1}{2}\rb^2 - \dot{\mu}' \hbar x
   \right\}
   dx
\end{equation}
and may now write
\begin{eqnarray}
  \label{eqn_phiddtHphi}
   \langle \phi''|i\hbar \frac{\partial }{\partial t} - H|\phi' \rangle &=& \frac{1}{2\pi}
   \int^\infty_{-\infty} e^{i\lb\mu' - \mu'' \rb x}dx \\
     &&2B \lb i\mu' - \frac{1}{2}\rb^2 + B \lb i\mu'-\frac{1}{2}\rb i \mu' e^{-2x} \nonumber\\
     &-& Ae^{2x} - \frac{3B}{2} \lb i\mu'-\frac{1}{2}\rb e^{-\frac{x}{2}} - \dot{\mu}' \hbar x \nonumber
\end{eqnarray}
If this were solvable, the states of the Schr\"{o}dinger equation (for
the original harmonic oscillator) could be written 
\begin{equation}
  \label{eqn_scheqn}
          i\hbar \frac{\pd \Phi(t)}{\pd t} = H(t) \Phi(t)
\end{equation}
where
\begin{equation}
  \label{eqn_schsol}
          \Phi = \int \frac{e^{i \lb \Theta +\mu \ln U \rb}}{\sqrt{2 \pi U}} d\mu
\end{equation}
Note: the quantities $A$, $B$, $\omega$, $\alpha$, $\dot{\alpha}$,
$\chi$, $\mu$, $\dot{\mu}$ and $U$ are all time dependant.

\begin{thebibliography}{100}
        \bibitem{ref_td} H. R. Lewis, W. B. Riesenfeld, {\it
        Journ. Math. Phys.} {\bf 10} 8 (1969) 1458
        \bibitem{ref_udho} M. Castagnino, R. Diener, L. Lara,
        G. Puccini, {\it quant-ph/0006011}
        \bibitem{ref_uvvu} C. G. Bollini, L. E. Oxman, {\it
        Phys. Rev.} {\bf A47} (1993) 2339
\end{thebibliography}
Typeset in $\LaTeXe$ using Emacs on GNU/Linux.
\end{document}
