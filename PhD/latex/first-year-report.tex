\documentclass[11pt, a4paper, titlepage]{article}
\usepackage{amsmath,amsfonts,amssymb,amsthm,amstext,amscd,array,bbold}
\usepackage[latin1]{inputenc}
\DeclareMathOperator{\AdS}{AdS}
\DeclareMathOperator{\Sphere}{S}
\DeclareMathOperator{\NW}{NW}
\DeclareMathOperator{\CW}{CW}
\let\S\Sphere

\begin{document}
\title{End of First Year Report}
\author{Sam Halliday}
\maketitle
\section{General Outline}
The original target of my thesis was to study an exact quantum (field) theory on
a time dependent non-commutative background. The stages which I have done so far
can be summarised as follows:
\begin{itemize}
\item Study exactly solvable techniques for doing time dependent quantum
  mechanics and generalise to the type of system we expect
\item Study time dependent QM/QFT models which incorporate a (constant)
  non-commutative background
\item Find a suitable background (Nappi-Witten) which generates a time dependent
  non-commutative geometry. Find the commutative geometry limit QFT
\item Attempt to find the non-commutativity parameter ($\Theta$) and the star
  products exactly, as previous calculations have only been approximate
\end{itemize}
However, step 4 has been a lot harder than expected and has led to some
interesting research in unexpected areas; Penrose-G\"{u}ven limits and
commutative isometric embedding diagrams in particular.

\section{The Nappi-Witten WZW Model ($NW_4$)}
The Nappi-Witten WZW model \cite{nw} is a very interesting background for
several reasons:
\begin{itemize}
\item Algebraically it is a central extension of the Poincar\'{e} algebra, with
  $T$ being the central generator
  \begin{equation}
    \label{eq:nw:poincare}
    \left[J,P_i\right]=\epsilon_{ij}P_j \qquad
    \left[P_i,P_j\right]=\epsilon_{ij}T
  \end{equation}
\item It is a 4-dimensional Lorentz signature background, just like the
  universe we see around us
\item Expressible as a plane wave
\item Exact solution to string equations of motion
\end{itemize}
In the NS-NS sector, we can write the metric and field strength $H$ in Brinkman
coordinates $(x^-,x^+,z^i)$ as
\begin{eqnarray}
  \label{eq:nw:G}
  G&=&2dx^-dx^+ +\frac{\vec{z}^2}{4}\left(d
    x^-\right)^2 + \left(d\vec{z}\right)^2 \\
  \label{eq:nw:H}
  H&=&dB=-dx^-\wedge dz^1\wedge dz^2
\end{eqnarray}

\section{So what is $\Theta$ and Why is it so Complicated in this Case?}
The simple answer is that in QM, $\Theta$ is the matrix defining the
non-commutativity of the coordinates
\begin{equation}
  \label{eq:theta}
  \left[q^i,q^j\right]=i\Theta^{ij}
\end{equation}
Seiberg and Witten \cite{stncg} showed how to derive the value of $\Theta$ from
string theory in a very easy manner. For $H=0$ (this includes most flat spaces)
the calculation to obtain $\Theta$ is trivial
\begin{equation}
  \label{eq:constantB}
  \Theta=B^{-1}
\end{equation}
If $H\neq 0$, this equation breaks down. We can see it breaking explicitly for
$\NW_4$ by comparing with the approximate result given by Dolan and Nappi
\cite{dn}. Using \eqref{eq:nw:H} and \eqref{eq:theta} we would get
$\Theta^{12}=-\frac{1}{x^-}$, but Dolan and Nappi's calculation hints that for
small $x^-$, we would expect to see something more like $\Theta^{12}=-x^-$.

Several authors have attempted to generalise the calculation of $\Theta$ to
$H\neq 0$ backgrounds, such as \cite{ho} and \cite{lbcb}, the former using an
approximation which results in non-commuting momentum (at best), the latter
approach failing for non-compact spaces. It is also possible to get $\Theta$
from the Moyal star product of a space, which is a deformation of pointwise
multiplication of functions. We hoped to get the star product of a doubly
extended Poincar\'{e} algebra as a group contraction of the star products on
$\mathrm{SU}(2)\oplus \mathrm{Sl}(2,\mathbb{R})$ and use this to get the star
product on the centrally extended Poincar\'{e} algebra, but we found curious
infinities when performing the limit, hinting us to try something else.

The particular group contraction which we were using is also closely related to
the Penrose limit of a spacetime, as the target space in this case is a plane
wave. The Penrose limit with $\NW_4$ as the target space was recently studied
\cite{pllbnwb} where it was shown that 2 such paths existed from the well
studied $\AdS_3\times\S^3$ space
\begin{equation}
  \label{eq:jose}
  \begin{CD}
    @.\\
    \AdS_3 \times \S^3    @>\text{PGL}>>          \NW_6\\
    \text{$\imath$}@AAA @AAA\text{$\widetilde{\imath}$}\\
    \AdS_2 \times \S^2    @>\text{PGL}>>          \NW_4\\
    @.
  \end{CD}
\end{equation}
where the vertical lines are embeddings and the horizontal lines
Penrose-G\"{u}ven limits. $\NW_6$ is a six dimensional analogy of the $\NW_4$
spacetime. Luckily $H=0$ on $\AdS_2\times\S^2$, which allows for standard
techniques on that space; the intention was to use the Penrose limit directly on
$\Theta$ and get the non-commutativity on $\NW_4$. Unfortunately, although it
was claimed that this diagram commuted, it does not for supergravity fields such
as $B$ and $H$, thus $\Theta$ does not map as expected. We then spent some time
trying to find out why the diagram was broken and this led to generalising
commutative isometric embedding diagrams with Penrose-G\"{u}ven limits.

\section{Penrose-G\"{u}ven Limits}
In \cite{pl}, Penrose showed that any Lorentzian spacetime has a limiting
spacetime which is a plane wave. This limit can be thought of as a ``first order
approximation'' along a null geodesic. The limiting spacetime depends on the
choice of null geodesic, and hence any spacetime can have more than one Penrose
limit. More recently, G\"{u}ven extended Penrose's argument to show that any
solution of a supergravity theory has plane wave limits which are also
solutions.

Let $(M,G)$ be a $D$-dimensional Lorentzian spacetime. By \cite{penrose} we can
always introduce local coordinates $(U,V,Y^i)$ in the neighbourhood of a portion
of a null geodesic $\gamma$ which contains no conjugate points. Such a metric
takes the form
\begin{equation}
  \label{eq:bpg:initial:G}
  G = \left(dU + \alpha dV + \beta_i dY^i \right)dV + C_{ij} dY^i dY^j
\end{equation}
where $\alpha$, $\beta_i$ and $C_{ij}$ are functions of the coordinates and
$C_{ij}$ is a symmetric positive-definite matrix, $i,j=1,2,\ldots,D-2$. The
coordinate system breaks down when $\det{C}=0$, signalling the existence of a
conjugate point. This coordinate system has the advantage that a null geodesic
congruence is singled out by $V,Y^i\rightarrow$ constant, with $U$ being an
affine parameter along these geodesics. The geodesic $\gamma$ is the one for
which $V=Y^i=0$.

In supergravity theories there are also general $p$-form potentials $A$ with
$(p+1)$-form field strengths $F=dA$. The potentials are defined up to gauge
transformations $A\mapsto A+d\Lambda$ in such a way that $F$ is gauge invariant.
G\"{u}ven extended the Penrose limit to supergravity theories by noting that in
order to ensure well defined potentials in the target space, a gauge must be
chosen for the potentials such that
\begin{equation}
  \label{eq:bpg:guven}
  A_{U(i_1\ldots i_{p-1})}=0
\end{equation}
By imposing this condition on $A$, we can now write general potential and
associated field strengths on $M$
\begin{eqnarray}
  \label{eq:bpg:initial:A}
  A &=& a_{i_1 \ldots i_p} dV\wedge dY^{i_1}\wedge\ldots\wedge
  dY^{i_{p-1}} \\ \nonumber
  &&+b_{i_1\ldots i_p} dY^{i_1}\wedge\ldots\wedge dY^{i_p}\\ \nonumber
  &&+c_{i_1\ldots i_p} dU\wedge dV\wedge dY^{i_1}\wedge\ldots\wedge
  dY^{i_{p-2}} \\
  \label{eq:bpg:initial:F}
  F &=& \left( \frac{\partial b_{i_1 \ldots i_{p+1}}}{\partial U}
  \right) dU\wedge dY^{i_1}\wedge\ldots\wedge dY^{i_{p}} \\ \nonumber
  &&+ d_{i_1 \ldots i_{p+1}} dY^{i_1}\wedge\ldots\wedge dY^{i_{p+1}} \\ \nonumber
  &&+ e_{i_1 \ldots i_{p+1}} dU\wedge dV\wedge dY^{i_1}\wedge\ldots\wedge
  dY^{i_{p-1}} \\ \nonumber
  &&+ f_{i_1 \ldots i_{p+1}} dV\wedge dY^{i_1}\wedge\ldots\wedge dY^{i_{p}}  
\end{eqnarray}
Where $a$, $b$, $c$, $d$, $e$ and $f$ are functions of the coordinates. The
Penrose limit is in practise the rescaling of the coordinates
\begin{equation}
  \label{eq:bpg:rescale}
  U\rightarrow u\qquad V\rightarrow\Omega^2v\qquad Y^i\rightarrow\Omega y^i
\end{equation}
where $\Omega\rightarrow 0$. Let us define the new fields on the target space as
\begin{eqnarray}
  \label{eq:bpg:tildefields}
  \widetilde{G}=\lim_{\Omega\rightarrow 0} \Omega^{-2}G(u,\Omega^2v,\Omega y^i) \\ \nonumber
  \widetilde{A}=\lim_{\Omega\rightarrow 0} \Omega^{-p}A(u,\Omega^2v,\Omega y^i) \\ \nonumber
  \widetilde{F}=\lim_{\Omega\rightarrow 0} \Omega^{-p}F(u,\Omega^2v,\Omega y^i)
\end{eqnarray}
these new fields are related to the original ones by a rescaling and (in the
case of the potentials) possibly a gauge transformation. We obtain the metric
and fields in Rosen \cite{rosen} coordinates $(u,v,y^i)$
\begin{eqnarray}
  \label{eq:bpg:target:G}
  \widetilde{G}&=& dudv + C(u)_{ij}dy^i dy^j \\
  \label{eq:bpg:target:A}
  \widetilde{A}&=& b(u)_{i_1\ldots i_p}dy^{i_1}
  \wedge\ldots\wedge dy^{i_p} \\ \nonumber
  &&+c(u)_{i_1\ldots i_p} du\wedge
  dv\wedge dy^{i_1}\wedge\ldots\wedge dy^{i_{p-2}}\\
  \label{eq:bpg:target:F}
  \widetilde{F}&=&\frac{\partial b(u)_{i_1\ldots i_{p+1}}}{\partial u}
  du\wedge dy^{i_1}\wedge\ldots\wedge dy^{i_p}
\end{eqnarray}
Due to \eqref{eq:bpg:rescale}, we can see that $C(u)_{ij}=C(U,0,0)_{ij}$ which
is just the value of $C$ along the geodesic $\gamma$. This is also true of
$b(u)$ and $c(u)$.

\subsection{Isometric Embedding Diagrams}
If we wish to generate a commutative isometric embedding diagram, whereby the
embeddings of $N$'s into the $M$'s are denoted by vertical arrows and
Penrose-G\"{u}ven limits (PGL) by horizontal arrows, we would have the following
picture:
\begin{equation}
  \label{eq:bpg:ie:generic}
  \begin{CD}
    @.\\
    M @>\text{PGL}>>                      \widetilde{M}\\
    \text{$\imath$}@AAA @AAA\text{$\widetilde{\imath}$}\\
    N @>\text{PGL}>>                      \widetilde{N}\\
    @.
  \end{CD}
\end{equation}
In order to ensure that the metric and fields commute in such a diagram (i.e.
that $M\rightarrow \widetilde{M} \rightarrow \widetilde{N}$ yields the same
result as $M\rightarrow N\rightarrow\widetilde{N}$), we must place some
restrictions on the kind of projections $\imath^*$, $\widetilde{\imath}^*$ we
can use. If we choose to set $m$ coordinates to zero\footnote{We require the
  coordinates to go to zero in order that the PGL is along the same null
  geodesic in each case, but requiring the coordinates to go constant rather
  than zero is often sufficient.} in each projection, then we can define
\begin{eqnarray}
  \label{eq:bpg:ie:i1}
  \imath^*&:& Y^{q_1},\ldots,Y^{q_m} \rightarrow 0\\
  \label{eq:bpg:ie:i2}
  \widetilde{\imath}^*&:& y^{r_1},\ldots,y^{r_m} \rightarrow 0
\end{eqnarray}
If $I=\{1,2,\ldots,D-2\}$ is the set of all indices and the sets of indices of
the coordinates set to zero are $Q=\{q_1,\ldots,q_m\}$ and
$R=\{r_1,\ldots,r_m\}$, then $Q,R\subset I$. We can now define two new sets of
indices, which are the indices of the coordinates which are not set to zero;
$S=I\backslash Q=\{s_1,\ldots,s_{D-2-m}\}$ and $T=I\backslash
R=\{t_1,\ldots,t_{D-2-m}\}$.

The following condition is required for the metric to be commutative in such a
diagram
\begin{equation}
  \label{eq:bpg:ie:metriccond}
  C(U,0,0)_{is} = C(U,0,0)_{it}
\end{equation}
where every $s$ is paired with a $t$ such that all the $t$'s are used. A similar
condition exists for field strengths
\begin{equation}
  \label{eq:bpg:ie:fieldcond}
  b(U,0,0)_{(i_1\ldots i_{p+1})s} = b(U,0,0)_{(i_1\ldots i_{p+1})t}
\end{equation}
For potentials we require the condition \eqref{eq:bpg:ie:fieldcond} (with
indexes up to $i_p$ rather than $i_{p+1}$) and also a second condition
\begin{equation}
  \label{eq:bpg:ie:potcond}
  c(U,0,0)_{(i_1\ldots i_{p})s} = c(U,0,0)_{(i_1\ldots i_{p})t}
\end{equation}
It is the particular pairing of $S$ and $T$ which we must choose carefully in
order to satisfy \eqref{eq:bpg:ie:metriccond}, \eqref{eq:bpg:ie:fieldcond} and
\eqref{eq:bpg:ie:potcond}. It may even be possible that no such choice exists,
in which case such a commutative diagram cannot be constructed. If we wish the
metric and fields to both commute simultaneously, we require the pairing of $S$
and $T$ to be the same in all the conditions.

These conditions are essentially just a trivial statement that $\imath^*$ and
$\widetilde{\imath}^*$ must be equivalent along the null geodesic. We found that
in the case of \eqref{eq:jose}, $\imath^*\neq\widetilde{\imath}^*$.

\section{Future Work}
The generalisation of the PGL's under isometric embeddings led to a solution for
the diagram we originally studied, in the form of
\begin{equation}
  \label{eq:bpg:ex:fixed1}
  \begin{CD}
    @.\\
    \AdS_3 \times \S^3             @>\text{PGL}>> \NW_6\\
    \text{$\imath$}@AAA @AAA\text{$\widetilde{\imath}$}\\
    \mathbb{R}^1\times\S^3         @>\text{PGL}>> \NW_4\\
    @.
  \end{CD}
\end{equation}
which is also quite interesting as the PGL on the bottom line has been recently
studied \cite{kiritsis}, and the vertex operators calculated explicitly. A
technique has already been outlined in \cite{lbcb} for calculating $\Theta$ on a
curved background (without the problems associated with it being a non-compact
space), if the vertex operators of a space are known. It is hoped that the
technique holds for $\NW_4$.

\bibliographystyle{unsrt}
\begin{thebibliography}{100}
\bibitem{nw} C. R. Nappi, E. Witten, ``A WZW model based on a non-semi-simple
  group'', \textit{Phys. Rev. Lett.} \textbf{71} (1993) 3751,
  \textit{hep-th/9310112}
\bibitem{stncg} N. Seiberg, E. Witten, ``String Theory and Noncommutative
  Geometry'', \textit{JHEP} \textbf{9909} (1999) 032, \textit{hep-th/9908142}
\bibitem{dn} L. Dolan, C. R. Nappi, ``Noncommutativity in a Time-Dependent
  Background'', \textit{Phys. Lett.} \textbf{B551} (2003) 369-377,
  \textit{hep-th/0210030}
\bibitem{ho} Pei-Ming Ho, Yu-Ting Yeh, ``Noncommutative D-Brane in
  Non-Constant NS-NS B Field Background'', \textit{Phys. Rev. Lett.}
  \textbf{85} (2000) 5523, \textit{hep-th/0005159}
\bibitem{lbcb} V. Schomerus, ``Lectures on Branes in Curved Backgrounds'',
  \textit{Class. Quant. Grav.}, \textbf{19} (2002) 5781, \textit{hep-th/0209241}
\bibitem{pllbnwb} S. Stanciu, J. Figueroa-O'Farrill, ``Penrose limits of Lie
  Branes and a Nappi-Witten braneworld'', \textit{JHEP} \textbf{0306} (2003)
  025, \textit{hep-th/0303212}
\bibitem{pl} R. Penrose, ``Any Space-Time has a Plane Wave as a Limit'',
  \textit{Differential Geometry and Relativity}, Reidel, Dordrecht (1976)
  271-275
\bibitem{penrose} D. Lerner, R. Penrose, ``Technique of Differential Topology in
  Relativity'', \textit{SIAM}, Philadelphia (1971)
\bibitem{guven} R. G\"{u}ven, ``Plane Wave Limits and T-Duality'', \textit{Phys.
    Lett.} \textbf{B482} (2000) 255-263, \textit{hep-th/0005061}
\bibitem{rosen} N. Rosen, \textit{Phys. Z. Sowjetunion} \textbf{12} (1937) 366
\bibitem{kiritsis} G. D'Appollonio, E. Kiritsis, ``String interactions in
  gravitational wave backgrounds'', \textit{hep-th/0305081}
\end{thebibliography}
\end{document}
