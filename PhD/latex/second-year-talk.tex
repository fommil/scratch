\documentclass[a4,12pt,titlepage]{seminar}
\usepackage{amsmath,amsfonts,amssymb,amsthm,amstext,amscd,array,bbold}
\setlength{\parindent}{0pt}
\setlength\arraycolsep{2pt}

\input{seminar.bug}
\usepackage{epsf,semhelv}

%% Local defines
\def\P{{\sf P}} \def\K{{\sf K}} \def\J{{\sf J}} \def\T{{\sf FIXME}}
\def\d{\partial}
\DeclareMathOperator{\AdS}{AdS}                         %% \AdS = Anti deSitter
\DeclareMathOperator{\Sphere}{S}                        %% \S = Sphere
\let\S\Sphere                                           %% (workaround to allow \S use)
\DeclareMathOperator{\NW}{NW}                           %% \NW = Nappi-Witten Space
\DeclareMathOperator{\CW}{CW}                           %% \NW = Cahen-Wallach Space
\DeclareMathOperator{\weyl}{{\cal W}}                   %% \weyl = Weyl operator
\DeclareMathOperator{\real}{{\mathbb R}}                %% \real = Real numbers
\newcommand{\comp}{\boxplus}                            %% \comp = composition
\newcommand{\compa}{\mkern+4mu\square\mkern-17.4mu{\mbox{
    \protect\raisebox{0.2ex}{$\ast$}}\mkern+5mu}}       %% \compa = composition style a
\newcommand{\compb}{\mkern+4mu\square\mkern-17.4mu{\mbox{
    \protect\raisebox{0.2ex}{$\bullet$}}\mkern+5mu}}    %% \compb = composition style b
\newcommand{\compc}{\mkern+4mu\square\mkern-17.4mu{\mbox{
    \protect\raisebox{0.2ex}{$\star$}}\mkern+5mu}}      %% \compc = composition style c
\newcommand{\cb}[2]{\left[#1,#2\right]}                 %% \cb{a}{b} = [ a, b ]
\newcommand{\scb}[2]{\left[#1,#2\right]_\star}          %% \scb{a}{b} = [ a, b ]_star
\newcommand{\NO}{\mbox{$\substack{\circ\\\circ}$}}      %% Normal ordering
\newcommand{\NOa}{\mbox{$\substack{\ast\\\ast}$}}       %% Normal ordering alt A
\newcommand{\NOb}{\mbox{$\substack{\bullet\\\bullet}$}} %% Normal ordering alt B

\begin{document}
\title{End of Second Year Report}
\author{Sam Halliday}
\maketitle

\section{Commutative D-Branes}

In SF03\footnote{S. Stanciu, J. Figueroa-O'Farrill, \textit{JHEP} \textbf{0306} (2003)
  025, \textit{hep-th/0303212}} the following diagram was postulated
\begin{equation*}
  \begin{CD}
    @.\\
    \AdS_3 \times \S^3             @>\text{PGL}>> \NW_6\\
    \text{$\imath$}@AAA @AAA\text{$\widetilde{\imath}$}\\
    \AdS_2 \times \S^2             @>\text{PGL}>> \NW_4\\
    @.
  \end{CD}
\end{equation*}
We have already assessed that this is incorrect, and that the embedded space is
$\CW_4$, not $\NW_4$. In the same paper, all the twisted conjugacy classes of
$\NW_6$ are investigated
\begin{equation*}
  C^r(g_0)=\{\omega(g)g_0g^{-1}|g\in {\cal N}\}
\end{equation*}
several are postulated to be $\NW_4$ and one a space-filling $\NW_6$, we find
that even these twisted conjugacy classes are Cahen-Wallach spaces, not
Nappi-Witten.

We have also investigated the geometry of these twisted conjugacy classes and
found only commutative geometries. The SUGRA fields are all zero, hinting from
standard techniques that the geometry is commutative. A technique outlined in
AFQS02\footnote{A.~Y.~Alekseev, S.~Fredenhagen, T.~Quella and V.~Schomerus,
  Nucl.\ Phys.\ B {\bf 646} (2002) 127 [arXiv:hep-th/0205123]} allows us to
further confirm our conclusions, which is easily achieved by showing that the
group $G^\omega$ is Euclidean.
\begin{equation*}
  G^\omega:=\{g\in G|\omega(g)=g\}
\end{equation*}

Noting that $\AdS_2 \times \S^2$ is a \textit{symmetric brane} (as is $\CW_4$),
whereas $\NW_4$ is a \textit{non-symmetric brane}, this has led us to the
postulate that these Penrose-G\"{u}ven Isometric Embedding diagrams preserve the
maximal symmetry of the embedded spaces.

\clearpage
\section{$\star$-Products}

The Weyl operator ${\hat\Omega}[f]$ can be thought of as a one-to-one map from local
coordinates $x^i$ (an algebra of fields on $\real^D$) to Hermitian operators
$\hat{x}^i$ (on an appropriate Hilbert space) exhibiting the algebra
\begin{equation*}
  \cb{\hat{x}^i}{\hat{x}^j}=i\theta^{ij}
\end{equation*}
we can define $\star$-commutator brackets for the functions
\begin{equation*}
  \scb{x^i}{x^j}=x^i\star x^j-x^j\star x^i
\end{equation*}
which allows us to replicate the operator algebra
\begin{equation*}
  \scb{x^i}{x^j}=i\theta^{ij}
\end{equation*}

The $\star$-product is a deformation of pointwise multiplication of functions on
$\real^D$ to an associative but non-commutative algebra
\begin{eqnarray*}
  {\hat\Omega}[f]{\hat\Omega}[g]&=&{\hat\Omega}[f\star g]\\
  f\star g &=& \iint
  \frac{\widetilde{f}(k)\widetilde{g}(k')}{(2\pi)^{2D}}{\hat\Omega}^{-1}[
  e^{ik_i\hat{x}^i}e^{ik'_i\hat{x}^i}]d^Dkd^Dk'
\end{eqnarray*}
For $\theta=0$, the $\star$-product reduces to the ordinary product of functions
and if $\theta$ is constant we get the Groenewold-Moyal $\star$-product
\begin{eqnarray*}
  f(x)\star g(x)&=&f(x)\exp\left(\frac{i}{2}\overleftarrow\partial_i
    \theta^{ij}\overrightarrow\partial_j\right)g(x)\\ \nonumber
  &=& f(x)g(x)+
  \sum_{n=1}^\infty
  \frac{i^n}{2^nn!}\theta^{i_1j_1}\ldots\theta^{i_nj_n}
  \partial_{i_1}\ldots\partial_{i_n}f(x)\partial_{j_1}\ldots\partial_{j_n}g(x)
\end{eqnarray*}
There is a standard form for Lie Algebras
\begin{equation*}
  f\star g(z)=f(z)\exp\left(-x^i\left[F_i-k_i-k_i'\right]\right)g(z)
\end{equation*}
where $F_i$ arises from the BCH formula
\begin{eqnarray*}
  (n+1)\mathrm{C}_{n+1}(X:Y)&&=\frac 12 \cb{X-Y}{\mathrm{C}_n(X:Y)}
  \\ \nonumber + \sum_{p \ge 1, 2p \le n} K_{2p}\!\!\!\!\!
  \sum_{\substack{k_1,\ldots,k_{2p}> 0 \\ k_1+\ldots+k_{2p}=n }}&&
  \!\!\!\!\!\!\!\!\!\!\cb{\mathrm{C}_{k_1}(X:Y)}
  {\left[\ldots,\cb{\mathrm{C}_{k_{2p}}(X:Y)}{X+Y}\ldots\right]}
\end{eqnarray*}

\clearpage
\section{$\NW_4$ $\star$-Products}

The $\NW_4$ algebra can be rewritten like so
\begin{eqnarray*}
  \cb{\hat{\P}^+}{\hat{\P}^-} &=& -2 \hat{\K} \\ \nonumber
  \cb{\hat{\J}}{\hat{\P}^\pm} &=& \mp\hat{\P}^\pm
% \\ \cb{\hat{\K}}{\Diamond} &=& 0
\end{eqnarray*}
which is an extension of the $\kappa$-Minkowski algebra. Using the
parameterisation (where $u$ and $v$ are real numbers, $w$ is a complex with
conjugate $\bar w$)
\begin{eqnarray*}
  R(u) &=& e^{i u\hat{\J}} \\
  T(w) &=& e^{i\bar w\hat\P^++iw\hat\P^-} \\
  Z(v) &=& e^{iv\hat\K}
\end{eqnarray*}
$R(u)$ is the rotation by an angle $u$ in the complex plane, $T(w)$ is the
translation by $w$ and $Z(v)$ is a one-parameter subgroup.

The group multiplication law tells us
\begin{eqnarray*}
  R(u_1)R(u_2)&=&R(u_1+u_2) \\
  R(u)T(w)&=&T(we^{iu})R(u) \\
  T(w_1)T(w_2)&=&T(w_1+w_2)Z(iw\bar w'-i\bar w w')
\end{eqnarray*}
which is invaluable when attempting to calculate exponential products such as
\begin{eqnarray*}
  e^{iu\hat\J}e^{i\bar w\hat\P^++iw\hat\P^-}&=&e^{i\bar w e^{-i
      u}\hat\P^+ +iwe^{i u}\hat\P^-}e^{iu\hat\J} \\
  e^{i\bar w\hat\P^++iw\hat\P^-}e^{iu\hat\J}&=&e^{iu\hat\J}e^{i\bar w e^{i
      u}\hat\P^+ +iwe^{-i u}\hat\P^-} \\
  e^{i\bar w\hat\P^++iw\P^-}e^{i\bar w'\hat\P^++iw'\hat\P^-}&=&
  e^{i(\bar w+\bar w')\hat\P^++i(w+w')\hat\P^-}e^{\left(\bar ww'-w\bar
      w'\right)\hat\K}
\end{eqnarray*}

\clearpage
\section{Orderings}

There is still an element of freedom in the $\star$-product with regards to the
ordering choice in the exponential, this is the reason why there are,
technically speaking, infinitely many $\star$-products associated with any
algebra. It is therefore important that we choose important orderings. We
investigate \textit{time ordering}
\begin{equation*}
  \NOa e^{ik_i\hat x^i}\NOa=e^{i\bar w\hat\P^++iw\hat\P^-}e^{iu\hat\J}e^{iv\hat\K}
\end{equation*}
which is the simplest ordering and generates the metric in Nappi-Witten
coordinates.
\textit{symmetric time ordering}
\begin{equation*}
  \NOb e^{ik_i\hat x^i}\NOb=e^{i\frac u2\hat\J}e^{i\bar w\hat\P^++iw\hat\P^-}
  e^{i\frac u2\hat\J}e^{iv\hat\K}
\end{equation*}
which generates the metric in plane wave coordinates
and \textit{Weyl ordering}
\begin{equation*}
  \NO e^{ik_i\hat x^i}\NO=e^{i\bar{w}\hat \P^++iw\hat \P^-+iu\hat \J+iv\hat\K}
\end{equation*}
which is considered the most obvious ordering, despite having the least physical
interpretation. Luckily it is possible to use the group multiplication law in
order to find the exponential product for the first two orderings, which allows
us to obtain the star products
\begin{eqnarray*}
  f\ast g&=&\frac{1}{(2\pi)^{8}}\iint e^{F_i(k,k')x^i}d^Dkd^Dk'\\ \nonumber
  F_1&=& i(\bar w+\bar w'e^{-i u})\\ \nonumber
  F_2&=& i(w + w'e^{i u})\\ \nonumber
  F_3&=& i(u+u')\\ \nonumber
  F_4&=& i(v+v')+\bar ww'e^{i u}-w\bar w'e^{-i u}
\end{eqnarray*}
and
\begin{eqnarray*}
  f\bullet g&=&\frac{1}{(2\pi)^{8}}\iint e^{F_i(k,k')x^i}d^Dkd^Dk'\\ \nonumber
  F_1&=&i\bar we^{\frac i2 u'}+i\bar w'e^{-\frac i2 u}\\ \nonumber
  F_2&=&iwe^{-\frac i2 u'}+iw'e^{\frac i2 u}\\ \nonumber
  F_3&=&i(u+u')\\ \nonumber
  F_4&=&i(v+v')+\bar ww'e^{\frac i2(u+u')}-w\bar w'e^{-\frac
    i2(u+u')}
\end{eqnarray*}
but for the Weyl ordering case, it was necessary to calculate several terms
explicitly using the BCH formula arriving at
\begin{eqnarray*}
  f\star g&=&\frac{1}{(2\pi)^{8}}\iint
  e^{F_i(k,k')x^i}d^Dkd^Dk'\\ \nonumber
  F_1&=&i\frac{\bar{w}\Phi(-iu)+\bar{w}'\Phi(-iu')e^{-i u}}{\Phi(-iu-iu')}\\
  \nonumber
  F_2&=&i\frac{w\Phi(iu)+w'\Phi(iu')e^{i u}}{\Phi(iu-iu')}\\
  \nonumber
  F_3&=&i(u+u')\\ \nonumber
  F_4&=&2\bar{w}w\Phi(iu)\Phi(-iu)\Upsilon(iu)
  +2\bar{w}'w'\Phi(iu')\Phi(-iu')\Upsilon(iu')\\ \nonumber
  &&-w\bar{w}'\Phi(iu)\Phi(-iu')e^{-i u}
  +\bar{w}w'\Phi(-iu)\Phi(iu')e^{i u}\\ \nonumber
  &&-2\left(\bar{w}\Phi(-iu)+\bar{w}'\Phi(-iu')e^{-i u}\right)\\ \nonumber
  &&\quad\!\!\!\times\left(w\Phi(iu)+w'\Phi(iu')e^{i
      u}\right)\Upsilon(iu+iu')+i(v+v')\\\\
  \Phi(a)&=&\frac{e^{ a}-1}a
  \qquad
  \Upsilon(a)=\frac 12 +\frac{e^{ a}-1- a
    e^{ a}}
  {\left(e^{ a}-1\right)^2}
\end{eqnarray*}

\clearpage
\section{Generalised Weyl Systems}

A \textit{generalised Weyl system} is a map in the set of operators $\weyl$ with
composition rule
\begin{equation*}
  \weyl(k)\weyl(k')=e^{\frac i2\omega(k,k')}\weyl(k\comp k')
\end{equation*}
$\comp$ forms a group on $\real^n$ satisfying
\begin{eqnarray*}
  (k\comp k')\comp k'' &=& k\comp(k'\comp k'') \\
  k\comp \underline k &=& 0
\end{eqnarray*}
where $\underline k$ is the inverse of $k$ and $0$ is the neutral element. The
original motivation for Weyl systems was to avoid the presence of unbounded
operators in the Q.M. formalism. Standard Weyl systems correspond to the
generalised Weyl system when $\comp$ is the normal addition on $\real^n$ but
does not hold for constant $\theta$. The generalised Weyl systems follow
trivially from the $\star$-products.

\clearpage
\section{Future Work}

We intend to submit the Penrose-G\"{u}ven Isometric Embedding diagram work
(which includes my 1st year work) in November. The $\star$-product and
Generalised Weyl System work will be expanded by calculating differentiability
and integrability conditions allowing for field theories to be pursued. This
work should not take long and we intend to submit by March.

At that stage we will decide upon either creating a field theory with this
algebra, or generalising the work people have done on $\kappa$-Minkowski.

It is clear that we will not find the originally sought after Time-Dependent
Non-Commutative Geometry which Dolan-Nappi\footnote{L.~Dolan and C.R.~Nappi,
  Phys. Lett. {\bf B551} (2003) 369--377 [{\tt hep-th/0210030}].} found on this
space, so i need a new title.
\end{document}
