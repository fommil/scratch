\documentclass[11pt, a4paper]{article}
\usepackage{amsmath,amsfonts,amssymb,bbold}
\usepackage[latin1]{inputenc}
\setlength{\parindent}{0pt}
\setlength\arraycolsep{2pt}
\renewcommand{\thefootnote}{\fnsymbol{footnote}}

% these break the label viewing in preview-latex
%\usepackage[notref, notcite]{shoasaqwkeys}
%\usepackage{showlabels}

%% Local defines
\def\P{{\sf P}} \def\K{{\sf K}} \def\J{{\sf J}} \def\T{{\sf FIXME}}
\def\d{\partial}
\DeclareMathOperator{\AdS}{AdS}
\DeclareMathOperator{\Sphere}{S}
\let\S\Sphere
\DeclareMathOperator{\NW}{NW}
\DeclareMathOperator{\CW}{CW}
\DeclareMathOperator{\weyl}{{\cal W}}
\DeclareMathOperator{\real}{{\mathbb R}}
\DeclareMathOperator{\im}{Im}
\DeclareMathOperator{\re}{Re}
\newcommand{\SU}{\mathrm{SU}}
\newcommand{\1}{\mathbb{1}}

\begin{document}
%\tableofcontents
\section{Conjugacy Classes in the $\NW_6$ Space-Time}
Classically symmetric D-branes are classified by the (twisted) conjugacy classes
of the 6 dimensional Nappi-Witten group.
\begin{equation}
  \label{eq:twistedconjugacy}
  C^r(g_0)=\{\omega(g)g_0g^{-1}|g\in {\cal N}\}
\end{equation}
In the case of normal conjugacy classes, $\omega$ is simply the identity.
However for the twisted conjugacy classes, it is the automorphism corresponding
to $(M,\varepsilon)$
\begin{eqnarray}
  \label{eq:automorphism}
  \omega(g)=g(Mu,\varepsilon\theta, \varepsilon t)=
  \begin{cases}
    g(Su, \theta, t) & \text{if $\varepsilon=1$}\\
    g(S\bar u, -\theta, -t) & \text{if $\varepsilon=-1$}
  \end{cases}
\end{eqnarray}
where $S\in\SU(2)$. We attempt to note the group $G^\omega$ for each twisted
conjugacy class, defined by
\begin{equation}
  \label{eq:Gomega}
  G^\omega:=\{g\in G|\omega(g)=g\}
\end{equation}
The metric and SUGRA fields in each case are given by
\begin{eqnarray}
  \label{eq:G}
  ds^2&=&|du'|^2-\frac i2 (\bar udu-ud\bar u)d\theta-2d\theta dt\\
  \label{eq:H}
  H&=&id\theta\wedge du\wedge d\bar u\\
  \label{eq:B}
  B&=&i\theta du\wedge du'
\end{eqnarray}

\subsection{Conjugacy Classes}
There are three untwisted conjugacy classes. In each case, due to the trivial
nature of $\omega=\1$, $G^\omega=G$.
\subsubsection{a)}
\begin{eqnarray}
  \label{eq:a}
  \theta_0&=&0\\\nonumber
  u'&=&e^{-i\theta}u_0\qquad\qquad\theta'=0\qquad t'=t_0-\im(\bar{u_0}e^{i\theta}u)\\\nonumber
  du'&=&-iu_0e^{-i\theta}d\theta\qquad d\theta'=0\qquad dt'=fd\theta+gdu\\\nonumber
  ds^2_a&=&u_0\bar{u_0}d\theta^2
\end{eqnarray}
which has a degenerate metric and cannot be straightforwardly interpreted as a
D-brane.

\subsubsection{b)}
\begin{eqnarray}
  \label{eq:b}
  \theta&=&\pi\\\nonumber
  u'&=&2u+e^{-i\theta}u_0\qquad\qquad\theta'=\pi\qquad t'=t_0\\\nonumber
  du'&=&2du-iu_0e^{-i\theta}d\theta\qquad d\theta'=0\qquad dt'=0\\\nonumber
  ds^2_b&=&|du'|^2\\\nonumber
  B_b&=&i\pi du'\wedge d\bar u'
\end{eqnarray}
Which may be interpreted as euclidean D3-branes.

\subsubsection{c)}
\begin{eqnarray}
  \label{eq:c}
  \theta&\neq&0,\pi\\\nonumber
  u'&=&u_0\qquad \theta'=\theta_0\qquad t'=t_0-\frac
  12\im(\bar{u_0}e^{i\theta}(1+e^{-i\theta_0}))\\\nonumber
  du'&=&0\qquad d\theta'=0\qquad dt'=fd\theta\\\nonumber
  ds^2_c&=&|du'|^2\\\nonumber
  B_c&=&i\theta_0 du'\wedge d\bar u'
\end{eqnarray}
Which may be interpreted as euclidean D3-branes. (However they do not seem to
make any sense in terms of a coordinate change as both $u'$ and $\theta'$ are
constant).

\subsection{Twisted Conjugacy Classes}
\subsubsection{d)}
\begin{eqnarray}
  \label{eq:d}
  \varepsilon=1\qquad u_0=0\qquad u=0\\\nonumber
  u'=0\qquad \theta'=\theta_0\qquad t'=t_0\\\nonumber
  du'=0\qquad d\theta'=0\qquad dt'=0
\end{eqnarray}
by the argument that the stabiliser is $g(0,\theta,t)$ it is claimed that the
sub-manifold is $g(u,0,0)$ which is a euclidean D3-brane.

\subsubsection{e)}
\begin{equation}
  \label{eq:e}
  \varepsilon=1\qquad u_0=0\qquad e^{i2\theta_0}\neq 1
\end{equation}
Euclidean D-String via more stabiliser arguments.

\subsubsection{f)}
\begin{eqnarray}
  \label{eq:f}
  \varepsilon&=&1\qquad u_0\neq 0\\\nonumber
  u'&=&\phi^{-1}(1-S)+(1+e^{-i\theta})u_0\qquad \theta=\theta_0\qquad t'=?
  \\\nonumber
  ds^2_f&=&|du'|^2\\\nonumber
  B_f&=&i\theta_0 du'\wedge d\bar u'
\end{eqnarray}
Stabiliser arguments again to say these are Euclidean D3-branes.

\subsubsection{g)}
\begin{eqnarray}
  \label{eq:g}
  \varepsilon&=&1\qquad S\neq\1\qquad e^{i\theta}=1\qquad u\in ker
  \phi\\\nonumber
  \bar u e^{i\theta_0}u_0&=&ue^{-i\theta_0}\bar{u_0}\\\nonumber
  \im[u_0^1+u_0^2]&=&0\\\nonumber
  \im[u^1+u^2]&=&0 \qquad or\qquad \re[u^1+u^2]=0
  \\\nonumber
  ds^2_g&=&|du'|^2
\end{eqnarray}
not forgetting the restriction on $u$.
\begin{eqnarray*}
  \label{eq:g:u}
  u'&=&Su+e^{-i\theta}u_0-e^{-i\theta_0}\\
  &=&e^{-i\theta}u_0\\
  du'&=&=-ie^{-i\theta}u_0d\theta
\end{eqnarray*}
which might imply the metric is degenerate (it seems i can either leave it as it
is which is a D3-brane $|du'|^2$, or convert to $\theta$ coordinates, which
implies a degenerate $fd\theta^2$ metric.)

\subsubsection{h)}
5D, degenerate metric.

\subsubsection{i)}
\begin{eqnarray}
  \label{eq:i}
  \varepsilon&=&1\qquad \text{$\phi$ not maximal}\\\nonumber
  du'&=&fd\theta
\end{eqnarray}
again, like g) with regards degeneracy.

\subsubsection{$G^\omega$ so far}
For all the previous twisted conjugacy classes, the group $G^\omega$ is given by
the group for which
\begin{equation}
  \label{eq:Gomega:1}
  g(Su,\theta,t)=g(u,\theta,t)
\end{equation}
since $S$ is not in general $=\1$, this means
\begin{equation}
  \label{eq:Gomega:2}
  G^\omega=\{(0,\theta,t)|t,\theta\in\real\}
\end{equation}

\subsubsection{j)}
$\varepsilon=-1$, space filling D5-branes

\subsubsection{k)}
$\varepsilon=-1$, lorentzian D3-branes

\subsubsection{$G^\omega$ for j) and k)}
For j) and k), the automorphism is different. In order to find $G^\omega$ we
require the group where
\begin{equation}
  \label{eq:Gomega:3}
  g(S\bar u,-\theta,-t)=g(u,\theta,t)
\end{equation}
which is empty unless $S=\1$, in which case
\begin{equation}
  \label{eq:Gomega:4}
  G^\omega=\{(u,0,0)|u\in\real^2\}
\end{equation}
or $S=-\1$, in which case
\begin{equation}
  \label{eq:Gomega:5}
  G^\omega=\{(u,0,0)|u\in i\real^2\}
\end{equation}

\begin{thebibliography}{99}
\bibitem{Figueroa-OFarrill:1999ie}
  J.~M.~Figueroa-O'Farrill and S.~Stanciu,
  ``More D-branes in the Nappi-Witten background,''
  JHEP {\bf 0001} (2000) 024
  [arXiv:hep-th/9909164].
  %%CITATION = HEP-TH 9909164;%%

\bibitem{Nappi:1993ie}
  C.~R.~Nappi and E.~Witten,
  ``A WZW model based on a nonsemisimple group,''
  Phys.\ Rev.\ Lett.\  {\bf 71} (1993) 3751
  [arXiv:hep-th/9310112].
  %%CITATION = HEP-TH 9310112;%%

\end{thebibliography}
\end{document}

% This is for Emacs
%%% Local Variables:
%%% TeX-command-default: "LaTeX"
%%% TeX-command-Show: "View"
%%% End:
