\section{The Twisted Heisenberg Algebra}
\label{TTHA}
In this section we will recall the definition of the six-dimensional
gravitational wave $\NW_6$ of Cahen-Wallach type and describe the manner in
which its geometry will be quantised in the subsequent sections.

\subsection{Definitions}
\label{Defs}
The spacetime $\NW_6$ is defined as the group manifold of the universal central
extension of the subgroup $\mathcal{S}:={\rm SO}(2)\ltimes\real^4$ of the
four-dimensional euclidean group ${\rm ISO}(4)={\rm SO}(4)\ltimes\real^4$. The
corresponding simply connected group $\mathcal N$ is homeomorphic to
six-dimensional Minkowski space $\eucl^{1,5}$. Its non-semisimple Lie algebra
$\mathfrak n$ is generated by elements $\J$, $\T$ and $\P^i_\pm$, $i=1,2$
obeying the non-vanishing commutation relations
\begin{eqnarray}
  \label{NW4algdef}
  \left[\P^i_+ , \P^j_-\right]&=&2 i \delta^{ij} \T   \nn\\
  \left[\J , \P^i_\pm\right]&=&\pm i \P^i_\pm
\end{eqnarray}
This is just the five-dimensional Heisenberg algebra extended by an outer
automorphism which rotates the noncommuting coordinates. The twisted Heisenberg
algebra may be regarded as defining the harmonic oscillator algebra of a
particle moving in two dimensions, with the additional generator $\J$ playing
the role of the number operator (or equivalently the oscillator hamiltonian). It
is this twisting that will lead to a noncommutative geometry that deviates from
the usual Moyal noncommutativity generated by the Heisenberg algebra. On the
other hand, $\mathfrak{n}$ is a solvable algebra whose properties are very
tractable. The subgroup $\mathcal{N}_0$ generated by $\P^1_\pm$, $\J$, $\T$ is
called the Nappi-Witten group and its four-dimensional group manifold is the
Nappi-Witten spacetime $\NW_4$.

The most general invariant, non-degenerate symmetric bilinear form
$\langle-,-\rangle:\mathfrak{n}\times\mathfrak{n}\to\real$
is defined by the non-vanishing values
\begin{eqnarray}
  \label{NW4innerprod}
  \left\langle\P^i_+ , \P^j_-\right\rangle&=&2 \delta^{ij}   \nn\\
  \left\langle\J , \T\right\rangle&=&1   \nn\\
  \left\langle\J , \J\right\rangle&=&b  
\end{eqnarray}
The arbitrary parameter $b\in\real$ can be set to zero by a Lie algebra
automorphism of $\mathfrak{n}$. This inner product has Minkowski signature, so
that the group manifold of $\mathcal N$ possesses a homogeneous, bi-invariant
Lorentzian metric defined by the pairing of the Cartan-Maurer left-invariant
$\mathfrak n$-valued one-forms $g^{-1} \dd g$ for $g\in\mathcal
N$,
\begin{equation}
  \label{NW4CM}
  \dd s^2=\left\langle g^{-1} \dd g , g^{-1} \dd g\right\rangle
\end{equation}
A generic group element $g\in\mathcal N$ may be parametrised as
\begin{equation}
  \label{NW4coords}
  g(u,v,\ma,\overline{\ma} )=\e^{a_i \P^i_++\overline{a}_i
    \P^i_-} \e^{\theta u \J} \e^{\theta^{-1} v \T}
\end{equation}
with $u,v,\theta\in\real$ and $\ma=(a_1,a_2)\in\complex^2$. In these global
coordinates, the metric \eqref{NW4CM} reads
\begin{equation}
  \label{NW4metricNW}
  \dd s^2=2 \dd u \dd v+|\dd\ma|^2+2 i \theta \left(\ma\cdot
    \dd\overline{\ma}-\overline{\ma}\cdot\dd\ma\right) \dd u  
\end{equation}
The metric \eqref{NW4metricNW} assumes the standard form of the plane wave
metric of a conformally flat, indecomposable Cahen-Wallach Lorentzian symmetric
spacetime $\CW_6$ in six dimensions upon introduction of Brinkman coordinates
$(x^+,x^-,\mz)$ defined by rotating the transverse space at a Larmor frequency
as $u=x^+$, $v=x^-$ and $\ma=\e^{ i \theta x^+/2} \mz$. In these coordinates the
metric assumes the stationary form
\begin{equation}
  \label{NW4metricBrink}
  \dd s^2=2 \dd x^+ \dd x^-+|\dd\mz|^2-\frac14  \theta^2 |\mz|^2 
  \left(\dd x^+\right)^2  
\end{equation}
revealing the pp-wave nature of the geometry. Note that on the null planes of
constant $u=x^+$, the geometry becomes that of flat four-dimensional euclidean
space $\eucl^4$. This is the geometry appropriate to the Heisenberg subgroup of
$\mathcal{N}$, and is what is expected in the Moyal limit when the effects of
the extra generator $\J$ are turned off.

The spacetime $\NW_6$ is further supported by a Neveu-Schwarz two-form field $B$
of constant field strength
\begin{eqnarray}
  \label{NS2formBrink}
  H&=&-\frac13  \bigl\langle g^{-1} \dd g , \dd
  \left(g^{-1} \dd g\right)\bigl\rangle
  = 2 i \theta \dd x^+\wedge\dd\mz^\top\wedge\dd\overline{\mz}
  = \dd B   \nn\\
  B&=&-\frac12  \bigl\langle g^{-1} \dd g , 
  \frac{\id+{\rm Ad}_g}{\id-{\rm Ad}_g} g^{-1} \dd g\bigl\rangle = 
  2 i \theta x^+ \dd\mz^\top\wedge\dd\overline{\mz}  
\end{eqnarray}
defined to be non-vanishing only on vectors tangent to the conjugacy class
containing $g\in\mathcal{N}$. It is the presence of this $B$-field that induces
time-dependent noncommutativity of the string background in the presence of
D-branes. Because its flux is constant, the noncommutative dynamics on this
space can still be formulated exactly, just like on other symmetric (fuzzy)
curved noncommutative spaces.

\subsection{Foliation by Fuzzy Spheres\label{D3Fuzzy}}

We will now clarify the role of the broken ${\rm U}(2)={\rm
  SU}(2)\times{\rm U}(1)$ symmetry on the noncommutative D3-branes. The ${\rm
U}(1)$ factor is generated by the vector
field $J+\overline{J}$ producing simultaneous rotations in the two
transverse planes, while the ${\rm SU}(2)$ generators can be
represented on each irreducible module $V^{p^+,p^-}$ over
$U(\mathfrak{n}_6)$ as
\begin{equation}
\hat I_a:=\mbox{$\frac1{(2\mu p^+)^2}$} 
\mcDp\left(\bigl( \Pu^-\bigr){}^\top 
\sigma_a^\top \Pu^+\right)
\label{SU2gensdef}\end{equation}
where $\sigma_a$, $a=1,2,3$ are the standard Pauli spin matrices. The
operators \eqref{SU2gensdef}) generate, up to a rescaling, the ${\rm
  su}(2)$ Lie algebra
\begin{equation}
\bigl[\hat I_a , \hat I_b\bigr]=
\mbox{$\frac i{2\mu p^+}$} \epsilon_{ab}^{   c} \hat I_c \ ,
\label{su2liealg}\end{equation}
and under their adjoint action the transverse space operators
transform in the fundamental and anti-fundamental representations of
${\rm SU}(2)$ as
\begin{eqnarray}
\left[\hat I_a , \mcDp\bigl( \Pu^+\bigr)
\right]&=&\mbox{$\frac1{2\mu p^+}$} 
\sigma_a \mcDp\bigl( \Pu^+\bigr) \ , \nn\\
\left[\hat I_a , \mcDp\bigl( \Pu^-\bigr)
\right]&=&-\mbox{$\frac1{2\mu p^+}$} 
\sigma_a^\top \mcDp\bigl( \Pu^-\bigr) \ .
\label{Pfundreps}\end{eqnarray}
The quadratic Casimir operator $\hat R^2:=\sum_a\hat I_a^2\in U({\rm
  su}(2))\subset{\rm End}(V^{p^+,p^-})$ may be written using
\eqref{C6irrep}) as
\begin{equation}
\hat R=\mbox{$\frac1{2\mu p^+}$} \Bigl( i\mcDp\bigl(\J \bigr)
+\mu^{-1} p^- \id\Bigr) \ .
\label{su2quadCas}\end{equation}

This ``hidden'' algebraic structure has the following geometric
interpretation. Corresponding to these operators, the functions on the
conjugacy classes $\mathcal{C}_{x_0^+,\chi_0}$ are given by
\begin{equation}
x_a:=\Delta_*^{-1}\bigl(\hat I_a\bigr)\left(\mz,
\overline{\mz} \right)=-\overline{\mz}{}^{ \top}
 \sigma_a^\top \mz \ ,    r^2:=
\Delta_*^{-1}\bigl(\hat R\bigr)\left(\mz,
\overline{\mz} \right)=|\mz|^2 \ .
\label{su2classfns}\end{equation}
The D3-brane worldvolume $\eucl^4$ admits a foliation into three-spheres
$\S^3$ defined by $|\mz|=r={\rm constant}$. The orbits are all
parallel and lie on two-spheres $\S^2\cong\complex{\rm
  P}^1\subset\eucl^4$ of radii $( \sum_ax_a^2)^{1/2}=r^2$. The corresponding
maps
$\S^3\to\S^2$ defined by \eqref{su2classfns}) are Hopf fibrations over
the coadjoint orbits ${\rm SU}(2) / {\rm U}(1)$ of the ${\rm SU}(2)$
symmetry group.

The ${\rm SU}(2)$ symmetry of the Nappi-Witten plane wave $\NW_6$
persists as well on the quantized conjugacy classes, giving a
noncommutative version of the Hopf fibration via the Jordan-Schwinger
map \cite{G-BLMV1,HLS-J1}. This is the residual symmetry ${\rm
  USp}(2)\times{\rm USp}(2) / {\rm U}(1)$ of the reduction of
$\eucl_\theta^4$. The corresponding
two-spheres $\S_{p^+,j}^2\subset\eucl_\theta^4$ are fuzzy spheres and
can be constructed for each $j\in\frac12 \nat_0$ by defining a
$(2j+1)$-dimensional module $W^{p^+}_j\subset V^{p^+,p^-}$ over
$U({\rm su}(2))$ as the linear span of the Schwinger basis vectors
\begin{equation}
\bigl|j,\ell \bigr\rangle\!\!\!
\bigm\rangle_{p^+}:=\bigl|j+\ell,j-\ell;p^+,p^-\bigr\rangle
\label{fuzzybasisdef}\end{equation}
for $\ell\in\{-j,-j+1,\dots,j-1,j\}$. The actions of the ${\rm su}(2)$
generators are given by
\begin{equation}
\hat I_\pm\bigl|j,\ell \bigr\rangle\!\!\!
\bigm\rangle_{p^+}=-\mbox{$\frac{j\pm\ell}
{\mu p^+}$} \bigl|j,\ell\mp1 \bigr\rangle\!\!\!\bigm\rangle_{p^+} \ ,   
\hat I_3\bigl|j,\ell \bigr\rangle\!\!\!\bigm\rangle_{p^+}
=-\mbox{$\frac\ell{\mu p^+}$} 
\bigl|j,\ell \bigr\rangle\!\!\!\bigm\rangle_{p^+}
\label{su2irrepaction}\end{equation}
where $\hat I_\pm:=\hat I_1\pm i\hat I_2$, while the operator
$\mcDp(\J )$ and hence the Casimir \eqref{su2quadCas}) are scalar
operators on $W_j^{p^+}$ with
\begin{equation}
\hat R\bigm|_{W_j^{p^+}}=R \id:=\mbox{$\frac j{\mu p^+}$} \id \ .
\label{quadCasscalar}\end{equation}
This module is the irreducible spin-$j$ representation of ${\rm
  SU}(2)$. The algebra of functions on the fuzzy sphere is given by
\begin{equation}
\mathcal{A}\bigl(\S^2_{p^+,j}\bigr)={\rm End}\bigl(W^{p^+}_j\bigr)
\label{fuzzyfns}\end{equation}
and it is isomorphic to the finite-dimensional algebra of
$(2j+1)\times(2j+1)$ matrices. It admits a natural action of the group
${\rm SU}(2)$ by conjugation with group elements evaluated in the
$(2j+1)$-dimensional representation of ${\rm SU}(2)$. Under this
action, the ${\rm SU}(2)$-module structure is determined by its
decomposition into irreducible representations as
\begin{equation}
\mathcal{A}\bigl(\S^2_{p^+,j}\bigr)=\bigoplus_{k=0}^{2j}W_k^{p^+} \ .
\label{fuzzyfnsmodule}\end{equation}

Since the basis \eqref{Vhighestex}) can be presented in terms of ${\rm
  SU}(2)$ module vectors as
$|n,m;p^+,p^-\rangle=|\frac{n+m}2,\frac{n-m}2 \rangle\!\rangle_{p^+}$, the
noncommutative worldvolume algebra may be expressed in terms of a
foliation by fuzzy spheres as
\begin{equation}
\mathcal{A}\bigl(\mathcal{C}_{x_0^+,\chi_0}\bigr)=
\bigoplus_{j\in\frac12 \nat_0}\Biggl(\mathcal{A}\bigl(\S^2_{p^+,j}\bigr)
{} \oplus \bigoplus_{\stackrel{\scriptstyle j'\in\frac12 \nat_0}
{\scriptstyle j'\neq j}}W_j^{p^+}\otimes\bigl(W_{j'}^{p^+}\bigr)^*
\Biggr) \ .
\label{D3algS2fol}\end{equation}
The first summand in \eqref{D3algS2fol}) represents the foliation of
noncommutative $\S^3\cong\real^3\cup\infty$, defined as the representation of
the universal enveloping algebra $U({\rm su}(2))$ in ${\rm End}(V^{p^+,p^-})$,
by fuzzy spheres of increasing quantized radii \eqref{quadCasscalar}). With
$\mbf x^\top=(x_a)\in\real^3$, the three-dimensional noncommutative space
$\mathcal{A}(\real_\theta^3):=
\bigoplus_{j\in\frac12 \nat_0}\mathcal{A}(\S^2_{p^+,j})$ may be
viewed as a deformation of the algebra ${\rm C}^\infty(\real^3)$ by
using \eqref{su2classfns}) to reduce the Voros product
\eqref{D3starfinal}) on $\eucl_\theta^4$ to the
star-product \cite{HLS-J1}
\begin{equation}
(f*_{\real^3}g)(\mbf x)=f(\mbf x) \exp\Bigl[\mbox{$\frac1{4\mu p^+}$} 
\mbox{$\frac{\overleftarrow{\partial}}{\partial x_a}$} \left(
\delta_{ab} r^4+ i\epsilon_{ab}^{   c} x_c\right) 
\mbox{$\frac{\overrightarrow{\partial}}{\partial x_b}$} \Bigr] 
g(\mbf x) \ .
\label{starprodR3}\end{equation}
The reduction to the fuzzy sphere \eqref{fuzzyfns}) is achieved explicitly by
introducing the orthogonal projection
\begin{equation}
\hat P_j=\mbox{$\frac1{(2j+1) \left(2\mu p^+\right)^{2j}}$} 
\sum_{\ell=-j}^j\mbox{$\frac1{(j+\ell)! (j-\ell)!}$} 
\bigl|j,\ell \bigr\rangle\!\!\!
\bigm\rangle_{p^+} {}_{p^+}\bigl\langle\!
\bigm\langle j,\ell \bigr|
\label{projspinj}\end{equation}
onto the spin-$j$ module $W_{p^+}^j$, so that
\begin{equation}
\mathcal{A}\bigl(\S^2_{p^+,j}\bigr)=\hat P_j\mathcal{A}\bigl
(\real_\theta^3\bigr)=\mathcal{A}\bigl
(\real_\theta^3\bigr)\hat P_j=\hat P_j\mathcal{A}\bigl(
\mathcal{C}_{x_0^+,\chi_0}\bigr)\hat P_j \ .
\label{fuzzyalgproj}\end{equation}
This projection onto $\S^2_{p^+,j}$ may be translated into the star-product
formalism on ordinary functions \cite{HLS-J1}, with the coordinate
generators of the deformation of ${\rm C}^\infty(\S^2)$ given by
$x_a*_{\real^3}P_j$. The function $P_j$ corresponding to the projection
operator
\eqref{projspinj}) is radially symmetric on $\eucl^4$ and may
be straightforwardly computed using \eqref{DeltainvD3}) to get
\begin{equation}
P_j\bigl(|\mz|\bigr):=\Delta_*^{-1}\bigl(\hat P_j\bigr)
=\mbox{$\frac{\left(2\mu p^+\right)^{2j}}{(2j+1)!}$} 
|\mz|^{4j}  e^{-2\mu p^+ |\mz|^2} \ .
\label{projspinjfn}\end{equation}
A similar foliation will be exploited in Section \ref{TDNC} to construct the
full noncommutative geometry of Nappi-Witten spacetime.

The second summand in \eqref{D3algS2fol}) is the foliation of $\eucl_\theta^4$
by noncommutative three-spheres of increasing radius, and the direct sum
decomposition \eqref{D3algS2fol}) reflects the ${\rm SU}(2)$ symmetry inherent
on the noncommutative D3-branes. It can be compared with the module
decomposition \eqref{D3module}) which suggests that the euclidean D3-branes in
$\NW_6$ are composed of elementary constituents with the
$\mathcal{N}_6$ quantum numbers of null branes. In the present case,
the algebra isomorphism \eqref{D3algS2fol}) suggests that the
elementary branes originate as objects expanded into fuzzy spheres in
$\eucl_\theta^4$. It is interesting at this stage to compare these
configurations with those which arise in the RR-supported pp-wave obtained as
the Penrose-G\"uven limit of $\AdS_5\times\S^5$ \cite{BFHP1}. Because
of the RR fluxes, giant gravitons, i.e. massless particles in
anti-de Sitter space which polarize into massive
branes, correspond to D-branes. In the strongly-coupled Type IIA plane wave,
degenerate vacua corresponding to fundamental strings blown up into fuzzy
spheres provide an exact microscopic description of giant gravitons in
$\AdS_7\times\S^4$ corresponding to M2-branes polarized along $\S^4$. These
latter objects become the fuzzy sphere configurations of the BMN matrix model
in the Penrose-G\"uven limit \cite{BMN1}. In the Type IIB case, the
gravitational waves expand into spherical D3-branes wrapping a fuzzy
three-sphere determined by the noncommutative Hopf fibration described
above \cite{JLR-G1,S-J1}. In the present case the
situation is a bit different. The intrinsic ${\rm SU}(2)$ symmetry of the NS
background implies that long strings, of light-cone momenta $p^+>1$, can move
freely in the two transverse planes to the Nappi-Witten pp-wave and correspond
to spectral-flowed null brane states \cite{BAKZ1}. The long strings in
$\NW_6$ thus correspond instead to fundamental string states. It would
be interesting to understand their origin more precisely through the
Penrose-G\"uven limit of the corresponding giant graviton
configurations in $\AdS_3\times\S^3$ which wrap fuzzy
cylinders \cite{JLR-G2}. A general definition of fuzzy spheres in
diverse dimensions is given in \cite{S-J1} and applied to the
construction of fuzzy three-spheres and four-spheres in \cite{S-JT1}
by embedding them into $\complex^4_\theta$, generalizing the
construction outlined above.

\subsection{Flat Space Limits\label{FlatLim}}

The spectral flow of long string states also implies that, unlike the
RR-supported pp-waves, the strong NS-field limit $\mu\to\infty$ gives a flat
space theory~\cite{DAK1}, just like the semiclassical limit $\mu\to0$
does. This flat space limit is described in terms of the fuzzy sphere
foliations above, with radii $R=\frac j{\mu\,p^+}$, as follows. If we
send the spin $j\to\infty$, along with $\mu\to\infty$ at fixed
light-cone momentum $p^+$, then the operators $\hat I_a$ in
(\ref{su2liealg}) commute and generate the commutative algebra ${\rm
  C}^\infty(\S^2)$ of functions on an ordinary sphere $\S^2$. This is
the intuitive expectation, as the noncommutativity parameter of
$\eucl_\theta^4$ vanishes in the limit $\mu\to\infty$. However,
although a flat space theory is attained, noncommutative dynamics
still persist signalling the remnant of the ${\rm SU}(2)$ symmetry of
the NS background, because the $j\to\infty$ limit of (\ref{su2liealg})
describes a noncommutative plane $\eucl_\vartheta^2$ under
stereographic projection~\cite{APS1,CMS1}. This limit is completely
analogous to the scaling of the volume and flux of $\AdS_2\times\S^2$ to
infinity that obtains the noncommutative space $\eucl_\theta^4$ in the
Penrose-G\"uven limit of (\ref{E4commdiag}).

The analog of stereographic projection for fuzzy coordinates is
defined by the operators~\cite{APS1,CMS1}
\beq
\hat Z:=\hat I_-\,\left(\id-\mbox{$\frac{\mu\,p^+}j$}\,\hat I_3
\right)^{-1} \ , ~~ \hat Z^\dag:=\left(\id-\mbox{$\frac{\mu\,p^+}j$}\,
\hat I_3\right)^{-1}\,\hat I_+
\label{fuzzystereo}\eeq
which for large $j$ have the commutator
\beq
\left[\hat Z\,,\,\hat Z^\dag\,\right]=\mbox{$\frac j{\left(\mu\,p^+
\right)^2}$}\,\left(\id-\mbox{$\frac{\mu\,p^+}j$}\,\hat I_3
\right)^{-2}+O\left(\mbox{$\frac1j$}\right) \ .
\label{Zcommj}\eeq
In the limit $j\to\infty$, the whole $\hat Z$-plane is covered by
projecting all operators onto the subspace of the eigenspace of the
hermitian operator $(\frac j{\mu\,p^+})^2\,(\id+\frac{\mu\,p^+}j\,\hat
I_3)$ corresponding to its eigenvalues which lie in the range
$[0,\frac{j^{3/2}}{(2\mu\,p^+)^2}]$. Thus for $\mu,j\to\infty$ with
$j/(\mu\,p^+)^2$ fixed, the commutation relation (\ref{Zcommj})
becomes
\beq
\left[\hat Z\,,\,\hat Z^\dag\,\right]=\vartheta:=
\mbox{$\frac j{\left(2\mu\,p^+\right)^2}$} \ .
\label{Zcommtheta}\eeq
Therefore, even though the foliation of $\eucl^4\subset\NW_6$ becomes
that by ordinary commutative spheres $\S^2$ as $\mu,j\to\infty$, the
quantization of the D3-brane worldvolumes persists in the
neighbourhoods of points on the two-spheres which can be
stereographically projected onto the noncommutative planes
$\eucl_\vartheta^2$.

\section{Noncommutative Geometry of $\mbf{\NW_6}$\label{TDNC}}

In this final section we will construct a noncommutative deformation
of the six-dimensional Nappi-Witten spacetime $\NW_6$. We will do so
by using the fact that the conjugacy classes foliate the
$\mathcal{N}_6$ group with respect to the standard Kirillov-Kostant
symplectic structure. Generally, from \eqref{twistconjhom}) it follows
that in a neighbourhood of $g\in\mathcal{G}$ one has
$\mathcal{G}=\mathcal{C}_g^\omega\times\mathcal{Z}_g^\omega$ and thus
the group manifold is foliated by hyperplanes corresponding to twisted
conjugacy classes. This construction is completely analogous to the
foliation by fuzzy spheres that we described in Section \ref{D3Fuzzy}, which
exhibits a foliation of the ${\rm SU}(2)$ group by its conjugacy
classes. In this way we will exhibit $\NW_6$ as a foliation by the
noncommutative D3-branes constructed in Section \ref{CGED3B}. We will
also compare this noncommutative geometry with that which arises in
the open string decoupling limit of Nappi-Witten spacetime, showing
that our foliation realizes an explicit quantization of the Lie
algebra $\mathfrak{n}_6$. Let us remark that in principle a
noncommutative deformation of $\NW_6$ can be induced via group
contraction of ${\rm SU}(1,1)\times{\rm SU}(2)$ (or of
$\real\times{\rm SU}(2)$ for $\NW_4$). However, the corresponding
star-products on $\real^{1,2}\times\real^3$ \cite{G-BLMV1,HLS-J1,HNT1}
are not well-defined under the contraction, which does not extend to
the corresponding universal enveloping algebras.

\subsection{Foliation by Noncommutative D3-Branes\label{D3Fol}}

Our starting point is the Peter-Weyl theorem which gives the
linear decomposition of the algebra of functions on the group $\mathcal{N}_6$
into its irreducible representations. With respect to the regular
action of $\mathcal{N}_6\times\overline{\mathcal{N}}_6$ given by left
and right multiplication of the group $\mathcal{N}_6$ itself, one has
\begin{equation}
{\rm C}^\infty\left(\mathcal{N}_6\right)= {\int\limits_0^\infty
\!\!\!\!\!\!\!\!\!\!\!\!  \bigodot\!\!\!\!\!\!\!\!\!\!\mbf-\!\!
\mbf-}  d q^+  {\int\limits_{-\infty}^\infty
\!\!\!\!\!\!\!\!\!\!\!\! \bigodot\!\!\!\!\!\!\!\!\!\!\mbf-\!\!
\mbf-}  d q^- \left[\bigl(V^{q^+,q^-}\otimes\widetilde{V}
{}^{q^+,q^-}\bigr)\oplus\bigl(\widetilde{V}
{}^{q^+,q^-}\otimes V^{q^+,q^-}\bigr)\right]
\label{PeterWeylthm}\end{equation}
where the $q^+=0$ contributions contain an implicit integration over all null
brane representations described in Section \ref{Null}. The explicit
decomposition is obtained by expanding any function in the complete
system of eigenfunctions of the scalar laplacian
$\Box_6=2 \partial_+ \partial_-+\frac{\mu^2}4 |\mz|^2 
\partial_-^2+|\md|^2$ in the plane wave background. The spectrum of
$\Box_6$ is organized into representations of
$\mathcal{N}_6\times\overline{\mathcal{N}}_6$ for $p^+\neq0$ as
\begin{equation}
\Box_6\Phi_{\mell,\mm}^{p^+,p^-}\left(x^+,x^-,\mz,\overline{\mz} \right)=
E_{\mell,\mm}^{p^+,p^-} \Phi_{\mell,\mm}^{p^+,p^-}\left(x^+,x^-,\mz,
\overline{\mz} \right)
\label{Box6eigeneq}\end{equation}
with $\mell^\top=(\ell_1,\ell_2)\in\nat_0^2$, $\mm^\top=(m_1,m_2)\in\zed^2$ and
$\mz^\top=(z_1,z_2)\in\complex^2$. As follows from the analysis of
Section \ref{Dynamics}, the eigenfunctions are given by
\begin{equation}
\Phi_{\mell,\mm}^{p^+,p^-}\left(x^+,x^-,\mz,\overline{\mz} \right)=
 e^{ i p^+x^-+ i
  p^-x^+} \varphi^{p^+}_{\mell,\mm}\left(\mz,\overline{\mz} \right)
\label{Box6eigenfns}\end{equation}
where $\varphi^{p^+}_{\mell,\mm}(\mz,\overline{\mz} )$ are the Landau
wavefunctions for a particle moving in four dimensions, with equal
magnetic fields $\omega=\frac12 \mu |p^+|$ through each transverse
plane, given by
\begin{eqnarray}
\varphi^{p^+}_{\mell,\mm}\left(\mz,\overline{\mz} \right)&=&
\mbox{$\frac{\mu \left|p^+\right|}{2\pi}$} 
\prod_{a=1,2}\sqrt{\mbox{$\frac{\ell_a!}{(\ell_a+|m_a|)!}$}} 
 e^{ i m_a {\rm arg} z_a}  e^{-\frac{\mu |p^+|}4 |z_a|^2}\nonumber\\&&
\times \left(\mbox{$\frac{\mu \left|p^+\right|}2$} |z_a|^2\right)^{|m_a|/2} 
L_{\ell_a}^{|m_a|}\left(\mbox{$\frac{\mu \left|p^+\right|}2$} |z_a|^2\right)
\label{Landauwavefns}\end{eqnarray}
with $L_\ell^m(x)$, $\ell,m\in\nat_0$ the associated Laguerre
polynomials.

The corresponding eigenvalues are given in terms of the energies of
Landau levels by
\begin{equation}
E_{\mell,\mm}^{p^+,p^-}=2p^+p^--\mu \left|p^+\right| 
\sum_{a=1,2}\bigl(2\ell_a+|m_a|+1\bigr) \ .
\label{Landaulevels}\end{equation}
They are matched with the quadratic Casimir eigenvalues
\eqref{C6irrep}) of the representation $V^{q^+,q^-}$ by relating the
light-cone momenta as
\begin{equation}
q^+=\left|p^+\right| \ ,    q^-=p^--\mu \sum_{a=1,2}\bigl(2\ell_a+
|m_a|\bigr) \ ,
\label{momid}\end{equation}
while the quantum numbers of a basis state
$|n,m;q^+,q^-\rangle\otimes|\widetilde{n},\widetilde{m} ;q^+,q^-\rangle$
of \eqref{PeterWeylthm}) are related through
\begin{equation}
\mell=\begin{pmatrix}\min\left(n,\widetilde{n} \right)\\
\min\left(m,\widetilde{m} \right)\end{pmatrix} \ ,   
\mm=\begin{pmatrix}n-\widetilde{n} \\m-\widetilde{m} 
\end{pmatrix} \ .
\label{quantnumid}\end{equation}
For $p^+=0$, the null brane representations of
Section \ref{Null} correspond to the zero modes of the laplacian
$\Box_6$ and are given as a product of Bessel functions yielding the
decomposition of a plane wave whose radial momentum in the two
transverse planes is $\alpha^2$ and $\beta^2$. They will play no
explicit role in the analysis of this section. Any function $f\in{\rm
  C}^\infty(\mathcal{N}_6)$ can be thereby expanded as
\begin{equation}
f\left(x^+,x^-,\mz,\overline{\mz} \right)=\int\limits_{-\infty}^\infty
\frac{ d p^+}{2\pi} \int\limits_{-\infty}^\infty
\frac{ d p^-}{2\pi}  e^{ i p^+x^-+ i
  p^-x^+} \sum_{\mell\in\nat_0^2} \sum_{\mm\in\zed^2}f_{\mell,\mm}^{p^+,p^-} 
\varphi^{p^+}_{\mell,\mm}\left(\mz,\overline{\mz} \right)
\label{fNW6exp}\end{equation}
with the appropriate integration over the $p^+=0$ zero modes again
implicitly understood throughout.

To quantize the algebra of functions on the $\mathcal{N}_6$ group
manifold, we use the Peter-Weyl decomposition of the group algebra
$\complex(\mathcal{N}_6)$ into matrix elements of irreducible
representations regarded as functions on $\mathcal{N}_6$. Thus we use
the canonical isomorphism \eqref{quantalgD3duals}) to
interpret \eqref{PeterWeylthm}) as an algebra decomposition into
the noncommutative D3-brane worldvolumes \eqref{quantalgD3}) given by
\begin{equation}
\mathcal{A}\left(\mathcal{N}_6\right)= {\int\limits_0^\infty
\!\!\!\!\!\!\!\!\!\!\!\!  \bigodot\!\!\!\!\!\!\!\!\!\!\mbf-\!\!
\mbf-}  d q^+  {\int\limits_{-\infty}^\infty
\!\!\!\!\!\!\!\!\!\!\!\! \bigodot\!\!\!\!\!\!\!\!\!\!\mbf-\!\!
\mbf-}  d q^- \left[{\rm End}\bigl(V^{q^+,q^-}\bigr)\oplus
{\rm End}\bigl(\widetilde{V}^{q^+,q^-}\bigr)\right] \ ,
\label{NW6quantalgdef}\end{equation}
so that any element $\hat f\in\mathcal{A}(\mathcal{N}_6)$ admits an
expansion
\begin{equation}
\hat f=\int\limits_{-\infty}^\infty
\frac{ d q^+}{2\pi} \int\limits_{-\infty}^\infty\frac{ d q^-}
{2\pi} \sum_{n,\widetilde{n},m,\widetilde{m}\in\nat_0}
f_{n,m;\widetilde{n},\widetilde{m}}^{q^+,q^-} 
\bigl|n,m;q^+,q^-\bigr\rangle\bigl\langle
\widetilde{n},\widetilde{m} ;q^+,q^-\bigr|
\label{hatfNW6exp}\end{equation}
with $|n,m;q^+,q^-\rangle$ the Fock space basis of
$\widetilde{V}^{|q^+|,q^-}$ when $q^+<0$ and
$f_{n,m;\widetilde{n},\widetilde{m}}^{-q^+,q^-}=
\overline{f_{n,m;\widetilde{n},\widetilde{m}}^{q^+,q^-}}$.
We can view the algebra
\eqref{NW6quantalgdef}) as a deformation of the algebra of functions
${\rm C}^\infty(\mathcal{N}_6)$ by using the fact that, with the
identifications \eqref{momid}) and \eqref{quantnumid}),
the images of the Landau wavefunctions \eqref{Landauwavefns}) under the
quantization map of Section \ref{CGED3B} coincide with rank $1$
operators on the Fock module \eqref{Vhighestex}) as \cite{LSZ1}
\begin{equation}
\Delta_\star\bigl(\varphi^{p^+}_{\mell,\mm}\bigr)=
\frac1{16\pi} \sqrt{\mbox{$\frac{\left(2\mu \left|p^+\right|
\right)^{2-n-\widetilde{n}-m-\widetilde{m}}}{n! \widetilde{n}! m! 
\widetilde{m}!}$}} \bigl|n,m;q^+,q^-\bigr\rangle\bigl\langle
\widetilde{n},\widetilde{m} ;q^+,q^-\bigr| \ ,
\label{LandauWigner}\end{equation}
where the linear isomorphism $\Delta_\star:{\rm
  C}^\infty(\mathcal{C}_{x_0^+,\chi_0})\to{\rm End}(V^{q^+,q^-})$
is defined via symmetric operator ordering. The implicit dependence on
the light-cone momentum $p^-$ occurs in \eqref{LandauWigner}) through
the level sets of the class function exhibiting the conjugacy classes,
as explained in \eqref{semiclasspos}), or equivalently through
\eqref{momid}). We extend this map to the foliation of $\NW_6$ by the
conjugacy classes of the $\mathcal{N}_6$ group via the identification
\begin{equation}
f_{n,m;\widetilde{n},\widetilde{m}}^{q^+,q^-}=
\frac1{16\pi} \sqrt{\mbox{$\frac{\left(2\mu \left|p^+\right|
\right)^{2-n-\widetilde{n}-m-\widetilde{m}}}{n! \widetilde{n}! m! 
\widetilde{m}!}$}} f_{\mell,\mm}^{p^+,p^-} \ ,
\label{fextid}\end{equation}
giving an isomorphism of underlying vector spaces
$\Delta_\star:{\rm
  C}^\infty(\mathcal{N}_6)\to\mathcal{A}(\mathcal{N}_6)$ which maps the
two expansions \eqref{fNW6exp}) and \eqref{hatfNW6exp}) into one another
as
\begin{equation}
\hat f = \Delta_\star(f) \ .
\label{DeltastarNW6}\end{equation}

At the classical level, the pull-back of a function \eqref{fNW6exp}) to
a conjugacy class is obtained by restricting its light-cone
coordinates as in \eqref{E4branesloc}). In the quantum geometry, it is
achieved via an orthogonal projection which maps operators $\hat
f\in\mathcal{A}(\mathcal{N}_6)$ onto functions $\hat f_{q^+,q^-}\in{\rm
  End}(V^{q^+,q^-})$ on the quantized conjugacy classes as
\begin{equation}
\hat f_{q^+,q^-}=\hat f \hat P_{q^+,q^-}=\hat P_{q^+,q^-} \hat f \ ,
\label{orthoprojfns}\end{equation}
where the hermitian projector
\begin{equation}
\hat P_{q^+,q^-}=\sum_{n,m\in\nat_0}\mbox{$\frac1{\left(2\mu \left|q^+
\right|\right)^{n+m} n! m!}$} \bigl|n,m;q^+,q^-\bigr\rangle
\bigl\langle n,m;q^+,q^-\bigr|
\label{projqdef}\end{equation}
is the identity operator on the Fock module \eqref{Vhighestex}). It
obeys
\begin{equation}
\hat P_{q^+,q^-} \hat P_{s^+,s^-}=\delta\left(q^+-s^+\right) 
\delta\left(q^--s^-\right) \hat P_{q^+,q^-}\ ,   
\int\limits_{-\infty}^\infty\frac{ d q^+}{2\pi} 
\int\limits_{-\infty}^\infty\frac{ d q^-}{2\pi} \hat P_{q^+,q^-}
=\id
\label{projqprops}\end{equation}
and has (uncountably) infinite rank. From \eqref{Landauwavefns}),
\eqref{momid}) and \eqref{quantnumid}) it follows that the function
corresponding to \eqref{projqdef}) is radially symmetric in each
transverse plane and can be expressed in terms of Laguerre polynomials
as
\begin{eqnarray}
P_{q^+,q^-}\left(x^+,x^-,|z_1|,|z_2|\right)&:=&\Delta_\star^{-1}\bigl(
\hat P_{q^+,q^-}\bigr)\left(x^+,x^-,\mz,\overline{\mz} \right)
\nonumber\\ &=&4  e^{- i q^+x^-- i q^-x^+}  e^{-\frac{\mu |q^+|}4 
|\mz|^2} \sum_{n=0}^\infty L_n\left(\mbox{$\frac{\mu \left|q^+\right|}2$}
|z_1|^2\right)\nonumber\\ &&\times 
\sum_{m=0}^\infty L_m\left(\mbox{$\frac{\mu \left|q^+\right|}2$}
|z_2|^2\right) \ .
\label{projqLaguerre}\end{eqnarray}
This construction is completely analogous to that sketched in
Section \ref{D3Fuzzy}. In particular, the star-projectors
\eqref{projspinjfn}) are special instances of \eqref{projqLaguerre}).

\subsection{Star-Product\label{NW6Star}}

Using the construction of the previous subsection, we can define an
associative star-product of fields on $\NW_6$ in the usual way by $f\star
g:=\Delta_\star^{-1}(\hat f \hat g )$. Written in terms of the
expansion \eqref{fNW6exp}), one has
\begin{eqnarray}
(f\star g)\left(x^+,x^-,\mz,\overline{\mz} \right)&=&
\int\limits_{\real^4} d\varrho\left(p^+,p^-,r^+,r^-\right) 
 e^{ i(p^++r^+)x^-+ i(p^-+r^-)x^+}\nonumber\\ &&\times 
\sum_{\mell,\mell'\in\nat_0^2} \sum_{\mm,\mm'\in\zed^2}
f_{\mell,\mm}^{p^+,p^-} 
g_{\mell',\mm'}^{r^+,r^-} \bigl(\varphi_{\mell,\mm}^{p^+}
\star\varphi_{\mell',\mm'}^{r^+}\bigr)\left(\mz,\overline{\mz} 
\right) \ .
\label{starNW6def}\end{eqnarray}
{}From \eqref{highestortho}), \eqref{momid}), \eqref{quantnumid}) and
\eqref{LandauWigner}) it follows that the Landau wavefunctions obey the
star-product projector relation \cite{LSZ1}
\begin{equation}
\varphi_{\mell,\mm}^{p^+}\star\varphi_{\mell',\mm'}^{r^+}=
\mbox{$\frac{\mu \left|p^+\right|}{8\pi}$} \delta_{\widetilde{n},n'} 
\delta_{\widetilde{m},m'} \delta\left(p^+-r^+
\right) \delta\left(p^--r^--\mu \tilde m_1'-\mu \tilde m_2'
\right) \varphi^{p^+}_{\tilde\mell,\tilde\mm'}
\label{Landaustarproj}\end{equation}
with
\begin{equation}
\tilde\mell=\begin{pmatrix}\min\left(n,\widetilde{n}^{ \prime} \right)\\
\min\left(m,\widetilde{m}^{ \prime} \right)\end{pmatrix} \ ,   
\tilde\mm'=\begin{pmatrix}n-\widetilde{n}^{ \prime} 
\\m-\widetilde{m}^{ \prime} \end{pmatrix} \ .
\label{quantnumtilde}\end{equation}
{}From \eqref{Landaustarproj}) it follows that the expansion coefficients
of the star-product \eqref{starNW6def}) vanish at
$p^+=0$. Thus the zero mode wavefunctions do not contribute to the
star-product, as expected from the commutativity of the null brane
worldvolume geometries. In particular, if either $f$ or $g$ is
independent of the light-cone position $x^-$, then $f\star
g=0$. Furthermore, from \eqref{Box6eigeneq}) it
follows that noncommutative scalar field theory on $\NW_6$ with this
star-product is of a similar type as the flat space noncommutative field
theories studied in \cite{LSZ1} which can be reformulated as exactly
solvable matrix models.

We can express this star-product in a form which does not relate to
the particular basis used to expand functions on $\NW_6$. For this, we
write the star-product of Landau wavefunctions in \eqref{starNW6def})
in terms of the Moyal bi-differential operator in \eqref{MoyalD3def})
to get
\begin{eqnarray}
(f\star g)\left(x^+,x^-,\mz,\overline{\mz} \right)&=&
\int\limits_{\real^4} d\varrho\left(p^+,p^-,r^+,r^-\right) 
 e^{ i(p^++r^+)x^-+ i(p^-+r^-)x^+}
\nonumber\\ &&\times \delta\left(p^+-r^+\right) \delta\bigl(p^--r^-+\mu 
\mbox{$\sum\limits_{a=1,2}$}(2\ell_a'+|m_a'|-2\ell_a-|m_a|)\bigr)
\nonumber\\ &&\times \sum_{\mell,\mell'\in\nat_0^2} 
\sum_{\mm,\mm'\in\zed^2}f_{\mell,\mm}^{p^+,p^-} 
g_{\mell',\mm'}^{r^+,r^-} \varphi_{\mell,\mm}^{p^+}
\left(\mz,\overline{\mz} \right)\nonumber\\ &&
\times \exp\left[\mbox{$\frac1{4\mu 
\left|p^+\right|}$} \Bigl(\overleftarrow{\md}{}^{ \top} 
\overrightarrow{\overline{\md}}
-\overleftarrow{\overline{\md}}{}^{ \top} 
\overrightarrow{\md} \Bigr)\right] 
\varphi_{\mell',\mm'}^{r^+}\left(\mz,\overline{\mz} \right) \ .
\label{starNW6bidiff}\end{eqnarray}
We have exploited the occurence of the Dirac delta-functions in
\eqref{Landaustarproj}) to write the star-product \eqref{MoyalD3def})
between representations of different light-cone momenta. For us to be
able to do so, it is important to use the Poisson bi-vector
\eqref{NCparD3}) associated with the field theory limit of large
$B$-field, not the string theoretic one of Section \ref{D3OSD}, as
\eqref{Landaustarproj}) holds {\it only} when $\theta$ is the inverse
of the magnetic field appearing in the Landau wavefunctions
\eqref{Landauwavefns}). The full stringy deformation of $\NW_6$
is much more complicated and it would not produce the nice explicit
formulas for the star-product that we derive here. It is also
important that the Moyal product is used, and not the Voros
product which leads to the same Hochschild cohomology of the
noncommutative algebra of functions.

Resolving the delta-functions in \eqref{starNW6bidiff}) and replacing
the light-cone momenta in the bi-differential operator by
bi-derivatives in the light-cone position $x^-$, we find
\begin{eqnarray}
(f\star g)\left(x^+,x^-,\mz,\overline{\mz} \right)&=&
\int\limits_{\real^4} d\varrho\left(p^+,p^-,r^+,r^-\right) 
\int\limits_{\real^2}
 d\varrho\left(\lambda^+,\lambda^-\right)  e^{ i\lambda^-(p^+-r^+)+
 i\lambda^+(p^--r^-)}\nonumber\\ &&\times 
\sum_{\mell,\mell'\in\nat_0^2} \sum_{\mm,\mm'\in\zed^2}
f_{\mell,\mm}^{p^+,p^-} g_{\mell',\mm'}^{r^+,r^-} 
 e^{ i\mu \lambda^+ \sum\limits_{a=1,2}
(2\ell_a'+|m_a'|-2\ell_a-|m_a|)}\nonumber\\ &&\times 
 e^{ i p^+x^-+ i p^-x^+} \varphi_{\mell,\mm}^{p^+}
\left(\mz,\overline{\mz} \right)\nonumber\\ &&\times 
\exp\left[\mbox{$\frac i{2\mu}$} 
\mbox{$\frac{\overleftarrow{\md}{}^{ \top} 
\overrightarrow{\overline{\md}}
-\overleftarrow{\overline{\md}}{}^{ \top} 
\overrightarrow{\md}}{|\overleftarrow{\partial_-}|+
|\overrightarrow{\partial_-}|}$}\right]  e^{ i r^+x^-+ i r^-x^+} 
\varphi_{\mell',\mm'}^{r^+}\left(\mz,\overline{\mz} \right) \ .
\label{starNW6resolve}\end{eqnarray}
For the exponentials of the Landau level quantum numbers in
\eqref{starNW6resolve}), we use the eigenvalue problem
\eqref{Box6eigeneq},\ref{Landaulevels}) to replace
$\mu |p^+| \sum_{a=1,2}(2\ell_a+|m_a|)$ with the differential
operator $\Box_6-2 \partial_+ \partial_-$ when acting on the
eigenfunctions \eqref{Box6eigenfns}). The
light-cone momentum integrals can then be performed explicitly to
recover the original functions, leading to the desired basis
independent form of the star-product in terms of an
integro-bidifferential operator as
\begin{eqnarray}
&&(f\star g)\left(x^+,x^-,\mz,\overline{\mz} \right) = 
\int\limits_{\real^2} d\varrho\left(\lambda^+,\lambda^-\right) 
f\left(x^++\lambda^+,x^-+\lambda^-,\mz,\overline{\mz} \right)
\nonumber\\ &&               \times \exp\left[\lambda^+ 
\mbox{$\frac{\frac{\mu^2}4 |\mz|^2 
|\overleftarrow{\partial_-}|^{2}+|\overleftarrow{\md}|^2}
{|\overleftarrow{\partial_-}|}$}\right]
\exp\left[\mbox{$\frac i{2\mu}$} 
\mbox{$\frac{\overleftarrow{\md}{}^{ \top} 
\overrightarrow{\overline{\md}}
-\overleftarrow{\overline{\md}}{}^{ \top} 
\overrightarrow{\md}}{|\overleftarrow{\partial_-}|+
|\overrightarrow{\partial_-}|}$}\right] 
\exp\left[-\lambda^+ \mbox{$\frac{\frac{\mu^2}4 |\mz|^2 
|\overrightarrow{\partial_-}|^{2}+|\overrightarrow{\md}|^2}
{|\overrightarrow{\partial_-}|}$}\right]
\nonumber\\ &&               \times 
g\left(x^+-\lambda^+,x^--\lambda^-,\mz,\overline{\mz} \right) \ .
\label{starNW6final}\end{eqnarray}
Note the particular ordering of bi-differential operators in this
expression. To recover the usual Moyal star-product on the quantized
coadjoint orbits, one uses the projector functions
\eqref{projqLaguerre}) to map functions $f\in{\rm C}^\infty(\NW_6)$ to
functions $f_{q^+,q^-}$ on conjugacy classes as
\begin{equation}
f_{q^+,q^-}=f\star P_{q^+,q^-}=P_{q^+,q^-}\star f \ .
\label{fconjclassprojq}\end{equation}
With this definition one has $(f\star g)_{q^+,q^-}=f_{q^+,q^-}\star
g_{q^+,q^-}$. The quantum space $\mathcal{A}(\mathcal{N}_6)$ does {\it
  not} reduce to the classical Nappi-Witten spacetime in the
commutative limit, as \eqref{starNW6final}) is not a deformation
quantization of the pointwise product of functions in ${\rm
  C}^\infty(\NW_6)$ (an analogous statement applies to the foliation of
$\real_\theta^3$ by fuzzy spheres in
Section \ref{D3Fuzzy} \cite{HLS-J1}).

For generic functions on $\NW_6$ the star-product
\eqref{starNW6final}) will be divergent, and so it is only well-defined
on a (relatively small) subalgebra of ${\rm
  C}^\infty(\NW_6)$. Formally, we may remove this divergence by
multiplying the right-hand side of \eqref{starNW6def}) by the rank
$\Tr \hat P_{q^+,q^-}$ of the projector \eqref{projqdef}). Using
zeta-function regularization on the sums over Landau levels, this rank
is given in terms of the volume of the light-cone momentum space as
$({\rm vol} \real^2)^{-1}$, which will cancel against the divergences
coming from \eqref{starNW6final}). With this regularization understood,
a simple set of non-vanishing star-products is given as
\begin{equation}
\left(z_a x^-\right)\star\left( \overline{z}_b x^-\right)=
-\left( \overline{z}_a x^-\right)\star\left(z_b x^-\right)=
\mbox{$\frac{ i\mu}2$} \delta_{ab}
\label{simplestarprods}\end{equation}
for $a,b=1,2$. By changing to the new transverse space coordinates
$\mw:=x^- \mz$, we see that one of the new features of the
noncommutative geometry of the pp-wave, compared to the flat space
case, is the dependence of the noncommutativity on the light-cone
position $x^-$.

\section{open string}
\nts{possibly cut from here to the end of this section}
We can gain more insight into this description by appealing to the
exact boundary conformal field theory description of the conjugacy
classes \cite{DK2,Hikida1}, which can be done on the limiting planes
$w=0$ whereby the D3-branes restrict to the symmetric euclidean
D-strings of the four-dimensional Nappi-Witten spacetime
$\NW_4$ \cite{FS1}. With this restriction understood everywhere, the
zero-mode Hilbert space for open strings which start and end on a
D1-brane labelled by light-cone momenta $p^\pm$ coincides as a vector
space with the worldvolume algebra
\eqref{quantalgD3duals},\ref{D3module}) (at $\beta=0$). To each open
string boundary condition corresponding to an irreducible module
$V_{\alpha,0}^{0,0}$ in \eqref{D3module}), there is associated a
collection of boundary vertex operators $\Psi_\alpha^{p^+,p^-}(x)$,
$x\in\real$ of conformal dimension $h_\alpha=\alpha^2$ (the quadratic
Casimir eigenvalue) in one-to-one correspondence with the vectors of
$V_{\alpha,0}^{0,0}$. The operator product expansion of two such ${\rm
  U}(1)$ boundary primary fields reads \cite{DK2}
\begin{eqnarray}
\Psi_{\alpha_1}^{p^+,p^-}(x_1) \Psi_{\alpha_2}^{p^+,p^-}(x_2)&=&
\int\limits_0^\infty d\alpha \alpha (x_1-x_2)^{h_\alpha-
h_{\alpha_1}-h_{\alpha_2}} \delta\left(\alpha^2-
\alpha_1^2+\alpha_2^2+2\alpha_1 \alpha_2 \cos\phi\right)
\nn\\ &&\times  {\rm F}^{p^+,p^-;\alpha}_{\alpha_1,\alpha_2} 
\Psi_{\alpha}^{p^+,p^-}(x_2)+\dots
\label{OPEbdryfields}\end{eqnarray}
for $x_1<x_2$ and up to contributions involving descendant fields. The
structure constants ${\rm F}^{p^+,p^-;\alpha}_{\alpha_1,\alpha_2}$
coincide with the (quantum) Racah coefficients of $\mathcal{N}_4$ and
are given explicitly by
\begin{equation}
{\rm F}^{p^+,p^-;\alpha}_{\alpha_1,\alpha_2}=\frac{ e^{\frac{ i\pi}2 
\alpha_1 \alpha_2 \sin\phi \cot\frac{\mu p^+}2}}{
\pi \alpha_1 \alpha_2 \sin\phi}
\label{Racahquantum}\end{equation}
with
$\alpha^2:=\alpha_1^2+\alpha_2^2-2\alpha_1 \alpha_2 \cos\phi$
expressing momentum conservation in the two-dimensional transverse
plane to the $\NW_4$ wave.

An element $\psi_{\mbf\alpha}^{p^+,p^-}$, $\mbf\alpha\in
T\mathcal{C}_{x_0^+,\chi_0}$ of the vector space
$V_{\alpha,0}^{0,0}$ is an eigenfunction of the momentum operators
\eqref{6CWKilling}) pulled back to the brane worldvolume. It can be
represented as a plane wave whose radial momentum in the transverse
plane is $\alpha=|\mbf\alpha|$, and the association
$\Psi_{\alpha}^{p^+,p^-}:={\rm V}[\psi_{\mbf\alpha}^{p^+,p^-}]$ yields
the standard, flat space open string tachyon vertex
operators \cite{Schom1}. The operator product expansion
\eqref{OPEbdryfields}) written in the form
\begin{equation}
{\rm V}\bigl[\psi_{\mbf\alpha_1}^{p^+,p^-}\bigr](x_1) {\rm V}
\bigl[\psi_{\mbf\alpha_2}^{p^+,p^-}\bigr](x_2):={\rm V}
\bigl[\psi_{\mbf\alpha_1}^{p^+,p^-}*\psi_{\mbf\alpha_2}^{p^+,p^-}
\bigr](x_2)+\dots
\label{OPEstarform}\end{equation}
can then be used to define a star-product in the zero-slope limit by
\begin{equation}
  \label{psistardef}
  \psi_{\mbf\alpha_1}^{p^+,p^-}*\psi_{\mbf\alpha_2}^{p^+,p^-}=
  e^{\frac{ i\pi}2 \mbf\alpha_1\wedge\mbf\alpha_2 \cot\frac{\mu p^+}2}
  {} \psi_{\mbf\alpha_1+\mbf\alpha_2}^{p^+,p^-} \ \def\R{{\sf R}}
  {}.
\end{equation}
The star-product of any two functions on the classical worldvolume algebra
is then given by expressing them, according to \eqref{D3module}, as
expansions $f=\int_0^\infty d\alpha \alpha \tilde
f_\alpha \psi_{\mbf\alpha}^{p^+,p^-}$. According to the standard
formulas for D-branes in flat space with a magnetic field on their
worldvolume \cite{SW1}, the conformal weight power in
\eqref{OPEbdryfields} should be identified as
$\alpha_1 \alpha_2 \cos\phi=\mbf\alpha_1^\top \mbf\alpha_2:=
G_{\rm o}^{-1}(\mbf k_1,\mbf k_2)$. In the present case the open
string metric \eqref{SWopenmetgen} is given by $G_{\rm
  o}=\csc^2\frac{\mu x_0^+}2 | d\mz|^2$, yielding
$\mbf\alpha_i=\sin\frac{\mu x_0^+}2 \mbf k_i$ for
$i=1,2$. Identifying now the phase factor in \eqref{psistardef} as
$ e^{\frac{ i\pi}2 \mbf\alpha_1\wedge\mbf\alpha_2 
\cot\frac{\mu p^+}2}:=
 e^{-\frac{ i\pi}2 \Theta(\mbf k_1,\mbf k_2)}$ leads
immediately to the anticipated result \eqref{ThetaD3} \cite{DK2}.


%%% Local Variables: 
%%% mode: latex
%%% TeX-master: "main.tex"
%%% End: 
