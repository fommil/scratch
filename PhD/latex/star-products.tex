\documentclass[11pt, a4paper]{article}
\usepackage[notref,notcite]{showkeys}
\usepackage{amsmath,amsfonts,amssymb,bbold,color}
\usepackage[latin1]{inputenc}
%\setlength{\parindent}{0pt}
\setlength\arraycolsep{2pt}
\renewcommand{\thefootnote}{\fnsymbol{footnote}}

%% Local defines
\def\P{{\sf P}} \def\K{{\sf K}} \def\J{{\sf J}} \def\T{{\sf FIXME}}
\def\d{\partial}
\DeclareMathOperator{\AdS}{AdS}
\DeclareMathOperator{\Sphere}{S}
\let\S\Sphere
\DeclareMathOperator{\NW}{NW}
\DeclareMathOperator{\CW}{CW}
\DeclareMathOperator{\weyl}{{\cal W}}
\DeclareMathOperator{\real}{{\mathbb R}}
\newcommand{\comp}{\boxplus}
\newcommand{\compa}{\mkern+4mu\square\mkern-17.4mu{\mbox{
      \protect\raisebox{0.2ex}{$\ast$}}\mkern+5mu}}
\newcommand{\compb}{\mkern+4mu\square\mkern-17.4mu{\mbox{
      \protect\raisebox{0.2ex}{$\bullet$}}\mkern+5mu}}
\newcommand{\compc}{\mkern+4mu\square\mkern-17.4mu{\mbox{
      \protect\raisebox{0.2ex}{$\star$}}\mkern+5mu}}
\newcommand{\cb}[2]{\left[#1,#2\right]}
\newcommand{\scb}[2]{\left[#1,#2\right]_\star}
\newcommand{\NO}{\mbox{$\substack{\circ\\\circ}$}}
\newcommand{\NOa}{\mbox{$\substack{\ast\\\ast}$}}
\newcommand{\NOb}{\mbox{$\substack{\bullet\\\bullet}$}}
\newcommand{\nts}[1]{\textcolor{red}{TODO: #1}}
\newcommand{\1}{\mathbb{1}}
\newcommand{\grad}{\nabla}
\newcommand{\opd}{\hat \d}
\newcommand{\xh}{\hat x}
\newcommand{\shouldid}{\stackrel{!}{=}}

\begin{document}
\tableofcontents
%\clearpage

\section{$\star$-Products for the Nappi-Witten Algebra}
\subsection{The $\star$-Product}
\label{star}
The Weyl operator ${\hat\Omega}[f]$ can be thought of as a one-to-one map from
local coordinates $x^i$ (an algebra of fields on $\real^D$) to Hermitian
operators $\hat{x}^i$ (on an appropriate Hilbert space) exhibiting the algebra
\begin{equation}
  \label{eq:algebra}
  \cb{\hat{x}^i}{\hat{x}^j}=i\theta^{ij}
\end{equation}
where $\theta^{ij}$ is not necessarily constant. The inverse map
${\hat\Omega}^{-1}[f]$ pulls operators back to local coordinates; the functions
obtained in this way are called Wigner distribution functions and their
$\star$-commutator brackets, defined as
\begin{equation}
  \label{eq:poisson}
  \scb{x^i}{x^j}=x^i\star x^j-x^j\star x^i
\end{equation}
replicate the operator algebra
\begin{equation}
  \label{eq:falgebra}
  \scb{x^i}{x^j}=i\theta^{ij}
\end{equation}
where we understand that the $\theta$ in \eqref{eq:falgebra} may depend on
Wigner distribution functions associated to the operators on the Hilbert space.
If the fields live in an appropriate Schwartz space of functions, any function
may be described by its Fourier transform
\begin{equation}
  \label{eq:fourier}
  \widetilde{f}(k)=\int^\infty_{-\infty}e^{-ik_ix^i}f(x)d^Dx
\end{equation}
allowing us to define the symbol for $f(x)$
\begin{equation}
  \label{eq:weylsymb}
  {\hat\Omega}[f]=\frac{1}{(2\pi)^D}\int^\infty_{-\infty}
  \widetilde{f}(k)\NO e^{ik_i\hat{x}^i}\NO d^Dk
\end{equation}
We are free to choose a convenient ordering, denoted by $\NO~\NO$. When the
ordering is symmetric, ${\hat\Omega}[f]$ is often called the Weyl symbol. The
products of ${\hat\Omega}[f]$ symbols may be computed allowing us to define the
$\star$-product
\begin{eqnarray}
  \label{eq:weylproduct}
  {\hat\Omega}[f]{\hat\Omega}[g]&=&{\hat\Omega}[f\star g]\\
  \label{eq:starproduct}
  f\star g &=& \iint
  \frac{\widetilde{f}(k)\widetilde{g}(k')}{(2\pi)^{2D}}{\hat\Omega}^{-1}[\NO
  e^{ik_i\hat{x}^i}e^{ik'_i\hat{x}^i}\NO]d^Dkd^Dk'
\end{eqnarray}
The product of exponentials is calculated by using the
Baker-Campbell\--Hausdorff (BCH) formula (which we will explain in the
proceeding section). The $\star$-product is a deformation of point-wise
multiplication of functions on $\real^D$ to an associative but non-commutative
algebra. For $\theta=0$, the $\star$-product reduces to the ordinary product of
functions and if $\theta$ is constant we may employ simplifications in the BCH
formula such that we get the Groenewold-Moyal $\star$-product
\begin{eqnarray}
  \label{eq:star:constanttheta}
  f(x)\star g(x)&=&f(x)\exp\left(\frac{i}{2}\overleftarrow\d_i
    \theta^{ij}\overrightarrow\d_j\right)g(x)\\ \nonumber
  &=& f(x)g(x)+
  \sum_{n=1}^\infty
  \frac{i^n}{2^nn!}\theta^{i_1j_1}\ldots\theta^{i_nj_n}
  \d_{i_1}\ldots\d_{i_n}f(x)\d_{j_1}\ldots\d_{j_n}g(x)
\end{eqnarray}
For non-constant $\theta$ the expansion is generally more complicated. However,
an easy example is for an algebra
\begin{equation}
  \label{eq:example:nonconstant}
  \cb{\hat{x}_i}{\hat{x}_j}=f^k_{ij}\hat{x}_k
\end{equation}
where we might wish to find the $\star$-products between the local coordinates
$x_i$. Since we know
\begin{equation}
  \label{eq:example:integral}
  \int x e^{ik x}dk=-i\frac{\d}{\d k}\delta(k)
\end{equation}
we can insert this into \eqref{eq:starproduct} obtaining
\begin{equation}
  \label{eq:example:star}
  x_i\star x_j = -\frac{\d^2}
  {\d k_i\d k'_j}e^{F_nx^n}\
  \Bigr|_{\boldsymbol{k}=\boldsymbol{k}'=0}
\end{equation}
where $F_n=F_n(\boldsymbol{k},\boldsymbol{k}')$ are functions arising from the
BCH formula.

More generally, there is a standard $\star$-product for Lie algebras
\cite{Kathotia:1998}
\begin{equation}
  \label{eq:star:lie}
  f\star g(z)=f(z)\exp\left(ix^i\left[F_i-k_i-k_i'\right]\right)g(z)
\end{equation}
which can easily be written in position space by replacing
\begin{eqnarray}
  \label{eq:star:position}
  k_\nu\rightarrow -i\overleftarrow\d_\nu \\ \nonumber
  k_\nu'\rightarrow-i\overrightarrow\d_\nu
\end{eqnarray}
There is no general closed form for $F_n(\boldsymbol{k},\boldsymbol{k}')$ (as we
will see in the next section). It is often convenient to scale all of the
generators $x^i\rightarrow \frac 1\alpha x^i$ resulting in some non-trivial
change in the $F_n$ but allowing a power expansion of \eqref{eq:star:lie} in
$\alpha$. As $\alpha\rightarrow 0$ the normal commutative product is recovered.

\subsection{Baker-Campbell-Hausdorff Formula}
The BCH formula is used to calculate the product of exponentials when the
exponents do not commute
\begin{equation}
  \label{eq:BCH:define}
  e^Ae^B=e^{\mathrm{C}(A:B)}
\end{equation}
where $\mathrm{C}(A:B)$ is often an infinite sum, calculated through the
recursion formula
\begin{eqnarray}
  \label{eq:BCH}
  (n+1)\mathrm{C}_{n+1}(X:Y)&&=\frac 12 \cb{X-Y}{\mathrm{C}_n(X:Y)}
  \\ \nonumber + \sum_{p \ge 1, 2p \le n} K_{2p}\!\!\!\!\!
  \sum_{\substack{k_1,\ldots,k_{2p}> 0 \\ k_1+\ldots+k_{2p}=n }}&&
  \!\!\!\!\!\!\!\!\!\!\cb{\mathrm{C}_{k_1}(X:Y)}
  {\left[\ldots,\cb{\mathrm{C}_{k_{2p}}(X:Y)}{X+Y}\ldots\right]}
\end{eqnarray}
where ($n \ge 1 ; X,Y\in \mathfrak g$) and $\mathrm{C}_1(X:Y)=X+Y$. The $K$s
are defined by
\begin{equation}
  \label{eq:BCH:K}
  \frac{z}{1-e^{-z}}-\frac z2-1=\sum_{p=1}K_{2p}z^{2p}
\end{equation}
which are closely related to the Bernoulli numbers ($B_n$), defined by
\begin{equation}
  \label{eq:BCH:bernoulli}
  \sum_{n=0}\frac{B_nx^n}{n!}=\frac{x}{e^x-1}
\end{equation}
The first few terms of the BCH formula may be written concisely as
\begin{eqnarray}
  \label{eq:BCH:1}
  \mathrm{C}_1(X:Y)&=& X+Y\\
  \label{eq:BCH:2}
  \mathrm{C}_2(X:Y)&=& \frac 12 \cb XY \\
  \label{eq:BCH:3}
  \mathrm{C}_3(X:Y)&=& \frac 1{12} \cb X {\cb XY}
  -\frac 1{12} \cb Y {\cb XY}\\
  \label{eq:BCH:4}
  \mathrm{C}_4(X:Y)&=& -\frac 1{24} \cb Y{ \cb X {\cb XY}}
\end{eqnarray}
but beyond about 4th order, the terms get lengthy to write explicitly due to
the partition sum in \eqref{eq:BCH}.

\nts{need to define $F_i$ properly.}

\subsection{The Nappi-Witten Algebra}
We wish to find the $\star$-product of the $\NW_4$ algebra, a centrally extended
Poincar� algebra. \cite{Nappi:1993ie} define the algebra of anti-Hermitian operators
as
\begin{eqnarray}
  \label{eq:NW:origalg}
  \cb{\hat\J}{\hat\P_i}&=&\epsilon_{ij}\hat\P_j \\ \nonumber
  \cb{\hat\P_i}{\hat\P_j}&=&\epsilon_{ij}\hat\K
\end{eqnarray}
which may be written as an algebra of Hermitian operators by replacing each
operator with its complex equivalent: ${\sf X}_j\rightarrow i{\sf X}_j$
\begin{eqnarray}
  \label{eq:NW:hermalg}
  \cb{\hat\J}{\hat\P_i}&=&-i\epsilon_{ij}\hat\P_j \\ \nonumber
  \cb{\hat\P_i}{\hat\P_j}&=&-i\epsilon_{ij}\hat\K
\end{eqnarray}
By defining the conjugate operators
\begin{eqnarray}
  \label{eq:NW:defineP+-}
  \hat\P^+&=&\hat\P_1+i\hat\P_2 \\ \nonumber
  \hat\P^-&=&\hat\P_1-i\hat\P_2
\end{eqnarray}
we have an algebra which looks much like a centrally extended $\kappa$-Minkowski
algebra\footnote{In fact, the algebra is so closely related that this approach
  may be used as an alternative derivation of the Weyl ordered $\kappa$-Minkowski
  $\star$-product, see Appendix \ref{app:kappaminsk}.}
\begin{eqnarray}
  \label{eq:NW:algebra}
  \cb{\hat{\P}^+}{\hat{\P}^-} &=& -2 \hat{\K} \\ \nonumber
  \cb{\hat{\J}}{\hat{\P}^\pm} &=& \mp\hat{\P}^\pm
  % \\ \cb{\hat{\K}}{\Diamond} &=& 0
\end{eqnarray}
Using the parameterisation (where $u$ and $v$ are real numbers, $w$ is a complex
with conjugate $\bar w$)
\begin{eqnarray}
  \label{eq:NW:param:R}
  R(u) &=& e^{i u\hat{\J}} \\
  \label{eq:NW:param:T}
  T(w) &=& e^{i\bar w\hat\P^++iw\hat\P^-} \\
  \label{eq:NW:param:Z}
  Z(v) &=& e^{iv\hat\K}
\end{eqnarray}
we know that $R(u)$ is the rotation by an angle $u$ in the complex plane:
$R(u)\cdot z=e^{iu}z$ and $T(w)$ is the translation by $w$: $T(w)\cdot z = z+w$.
$Z(v)$ is a one-parameter subgroup acting as $Z(v)\cdot z = e^{iv}z$. The group
multiplication law \cite{Figueroa-OFarrill:1999ie} tells us
\begin{eqnarray}
  \label{eq:NW:mult:RR}
  R(u_1)R(u_2)&=&R(u_1+u_2) \\
  \label{eq:NW:mult:RT}  
  R(u)T(w)&=&T(we^{iu})R(u) \\
  \label{eq:NW:mult:TT}
  T(w_1)T(w_2)&=&T(w_1+w_2)Z(iw\bar w'-i\bar w w')
\end{eqnarray}
which is invaluable when attempting to calculate exponential products such as
\begin{eqnarray}
  \label{eq:NW:mult:JP}
  e^{iu\hat\J}e^{i\bar w\hat\P^++iw\hat\P^-}&=&e^{i\bar w e^{-i\beta
      u}\hat\P^+ +iwe^{i\beta u}\hat\P^-}e^{iu\hat\J} \\
  \label{eq:NW:mult:PJ}
  e^{i\bar w\hat\P^++iw\hat\P^-}e^{iu\hat\J}&=&e^{iu\hat\J}e^{i\bar w e^{i\beta
      u}\hat\P^+ +iwe^{-i\beta u}\hat\P^-} \\
  \label{eq:NW:mult:PP}
  e^{i\bar w\hat\P^++iw\P^-}e^{i\bar w'\hat\P^++iw'\hat\P^-}&=&
  e^{i(\bar w+\bar w')\hat\P^++i(w+w')\hat\P^-}e^{\left(\bar ww'-w\bar
      w'\right)\hat\K}
\end{eqnarray}
where we mean to expand the exponentials around $\beta=1$.

\nts{add a note about the $\Im$ notation}

The more general $\alpha$-scaled generators of \eqref{eq:star:lie} are given by
\begin{eqnarray*}
  \cb{\hat{\P}^+}{\hat{\P}^-} &=& -2 \alpha\hat{\K} \\ \nonumber
  \cb{\hat{\J}}{\hat{\P}^\pm} &=& \mp\alpha\hat{\P}^\pm
\end{eqnarray*}
with the corresponding exponential products
\begin{eqnarray*}
  e^{iu\hat\J}e^{i\bar w\hat\P^++iw\hat\P^-}&=&e^{i\bar w e^{-i\alpha\beta
      u}\hat\P^+ +iwe^{i\alpha\beta u}\hat\P^-}e^{iu\hat\J} \\
  e^{i\bar w\hat\P^++iw\hat\P^-}e^{iu\hat\J}&=&e^{iu\hat\J}e^{i\bar w
    e^{i\alpha\beta u}\hat\P^+ +iwe^{-i\alpha\beta u}\hat\P^-} \\
  e^{i\bar w\hat\P^++iw\P^-}e^{i\bar w'\hat\P^++iw'\hat\P^-}&=&
  e^{i(\bar w+\bar w')\hat\P^++i(w+w')\hat\P^-}e^{\alpha\left(\bar ww'-w\bar
      w'\right)\hat\K}
\end{eqnarray*}

\subsubsection{Time Ordering}
As mentioned, we are free to choose a convenient exponential ordering for
\eqref{eq:starproduct}. Here we will use an ordering such that
\begin{equation}
  \label{eq:time:defn}
  \NOa e^{ik_i\hat x^i}\NOa=e^{i\bar w\hat\P^++iw\hat\P^-}e^{iu\hat\J}e^{iv\hat\K}
\end{equation}
We call this ``Time-Ordering'' as the time coordinate ($u$) is always kept
separated and to the right of the $\P$s. The group multiplication law is well
studied and it is easy to calculate the exponential product explicitly
\begin{eqnarray}
  \label{eq:time:mult}
  &&\NOa e^{i\bar w \hat\P^++iw \hat\P^-}e^{iu \hat\J}e^{iv \hat\K}
  e^{i\bar w'\hat\P^++iw'\hat\P^-}e^{iu'\hat\J}e^{iv'\hat\K}\NOa
  =\\ \nonumber
  &&e^{i(\bar w+\bar w'e^{-i u})\hat\P^+
    +i(w + w'e^{i u})\hat\P^-}
  e^{i(u+u')\hat\J}
  e^{\left\{i(v+v')+\bar ww'e^{i u}-w\bar w'e^{-i
        u}\right\}\hat\K}
\end{eqnarray}
so that the $\ast$-product between any two functions is
\begin{eqnarray}
  \label{eq:time:star}
  f\ast g(z)&=&f(z)e^{ix^i(F^\ast_i-k_i-k_i')}g(z)\\ \nonumber
  F^\ast_1&=& \bar w+\bar w'e^{-i\alpha u}\\ \nonumber
  F^\ast_2&=& w + w'e^{i\alpha u}\\ \nonumber
  F^\ast_3&=& u+u'\\ \nonumber
  F^\ast_4&=& v+v'-i\alpha \bar ww'e^{i\alpha u}+i\alpha w\bar
  w'e^{-i\alpha u}
\end{eqnarray}
where $k$ runs over $\bar w,w,u,v$ and $x$ over $\P^+,\P^-,\J,\K$. We see from
\eqref{eq:example:star} and \eqref{eq:star:lie} that the operator algebra is
recovered in the functional space with the $\ast$-product between the generators
being
\begin{eqnarray}
  \label{eq:time:star:generators}
  \P^+\ast\P^-&=&\P^+\P^--\K\\ \nonumber
  \P^-\ast\P^+&=&\P^+\P^-+\K\\ \nonumber
  \J\ast\P^+&=&\J\P^+-\P^+\\ \nonumber
  \P^+\ast\J&=&\J\P^+\\ \nonumber
  \J\ast\P^-&=&\J\P^-+\P^-\\ \nonumber
  \P^-\ast\J&=&\J\P^-
\end{eqnarray}
To second order in $\alpha$ and written in position space, the $\ast$-product
between any two functions is
\begin{eqnarray}
  \label{eq:time:positionspace}
  \nonumber
  f\ast g(x)=fg
  &&+\alpha
  \left(
    \K\d_-f\d_+g
    -\K\d_+f\d_-g
    +\P^-\d_\J f\d_-g
    -\P^+\d_\J f\d_+g
  \right)\\\nonumber
  &&+\alpha^2
  \biggl(
  \frac 12\K^2\d_-^2f\d_+^2g
  -\K^2\d_+\d_-f\d_+\d_-g
  +\frac 12\K^2\d_+^2f\d_-^2g\\ \nonumber &&\qquad\quad\!\!
  +\frac 12{\P^+}^2d_\J^2f\d_+^2g
  +\frac 12\P^+\d_\J^2f\d_+g\\ \nonumber &&\qquad\quad\!\!
  +\frac 12\P^-\d_\J^2f\d_-g
  +\frac 12{\P^-}^2\d_\J^2f\d_-^2g\\ \nonumber &&\qquad\quad\!\!
  -\P^+\P^-\d_\J^2f\d_+\d_-g
  -\K\d_-\d_\J f\d_+g
  -\K\d_+\d_\J f\d_-g\\ \nonumber &&\qquad\quad\!\!
  +\K\P^+\d_+\d_\J f\d_+\d_-g
  -\K\P^+\d_-\d_\J f\d_+^2g
  \\ \nonumber &&\qquad\quad\!\!
  +\K\P^-\d_-\d_\J f\d_+\d_-g
  -\K\P^-\d_+\d_\J f\d_-^2g
  \biggr)\\ &&
  +\mathcal{O}(\alpha^3)
\end{eqnarray}

\subsubsection{Symmetric Time Ordering}
The ordering used in \cite{DAppollonio:2003dr} is of interest as it allows us
to recover the plane wave coordinates. Called ``Symmetric Time Ordering'' due to
the time coordinate on either side of the translation
\begin{equation}
  \label{eq:symtime:defn}
  \NOb e^{ik_i\hat x^i}\NOb=e^{i\frac u2\hat\J}e^{i\bar w\hat\P^++iw\hat\P^-}
  e^{i\frac u2\hat\J}e^{iv\hat\K}
\end{equation}
From \eqref{eq:NW:mult:JP},\eqref{eq:NW:mult:PJ} and \eqref{eq:NW:mult:PP} we
know how to calculate all the exponential products required to find
\begin{eqnarray}
  \label{eq:symtime:mult}
  \NOb e^{\frac i2u\hat\J}&&e^{i\bar w\hat\P^++iw\hat\P^-}
  e^{\frac i2u\hat\J}e^{iv\hat\K}
  e^{\frac i2u'\hat\J}e^{i\bar w'\hat\P^++iw'\hat\P^-}
  e^{\frac i2u'\hat\J}e^{iv'\hat\K}\NOb= \\ \nonumber
  &&e^{\frac i2(u+u')\hat\J}
  e^{i\left(\bar we^{\frac i2 u'}+\bar w'e^{-\frac i2 u}\right)\hat\P^+
    +i\left(we^{-\frac i2 u'}+w'e^{\frac i2 u}\right)\hat\P^-}\\ \nonumber
  &&e^{\frac i2(u+u')\hat\J}e^{\left[i(v+v')+\bar ww'e^{\frac
        i2(u+u')}-w\bar w'e^{-\frac i2
        (u+u')}\right]\hat\K}
\end{eqnarray}
so that the $\bullet$-product between any two functions is

\begin{eqnarray}
  \label{eq:symtime:star}
  f\bullet g(z)&=&f(z)e^{ix^i(F^\bullet_i-k_i-k_i')}g(z)\\ \nonumber
  F^\bullet_1&=&\bar we^{\frac i2\alpha u'}+\bar w'e^{-\frac i2\alpha u}\\ \nonumber
  F^\bullet_2&=&we^{-\frac i2\alpha u'}+w'e^{\frac i2\alpha u}\\ \nonumber
  F^\bullet_3&=&u+u'\\ \nonumber
  F^\bullet_4&=&v+v'-i\alpha\bar ww'e^{\frac i2\alpha(u+u')}
  +i\alpha w\bar w'e^{-\frac i2\alpha(u+u')}
\end{eqnarray}
where $k$ runs over $\bar w,w,u,v$ and $x$ over $\P^+,\P^-,\J,\K$. We see from
\eqref{eq:example:star} and \eqref{eq:star:lie} that the operator algebra is
recovered in the functional space with the $\bullet$-product between the
generators being
\begin{eqnarray}
  \label{eq:symtime:generators}
  \P^+\bullet\P^-&=&\P^+\P^--\K\\ \nonumber
  \P^-\bullet\P^+&=&\P^+\P^-+\K\\ \nonumber
  \P^+\bullet\J&=&\J\P^++\frac 12\P^+\\ \nonumber
  \J\bullet\P^+&=&\J\P^+-\frac 12\P^+\\ \nonumber
  \P^-\bullet\J&=&\J\P^--\frac 12\P^-\\ \nonumber
  \J\bullet\P^-&=&\J\P^-+\frac 12\P^-
\end{eqnarray}
To second order in $\alpha$ and written in position space, the $\bullet$-product
between any two functions is
\begin{eqnarray}
  \label{eq:symtime:positionspace}
  &&f\bullet g(x)=fg\\ \nonumber &+&\frac\alpha2
  \Biggl(
  - 2\K\d_+f\d_-g
  + 2\K\d_-f\d_+g
  - \P^-\d_-f\d_\J g\\ \nonumber &&\qquad\quad\!\!
  + \P^-\d_\J f\d_-g
  + \P^+\d_+f\d_\J g
  - \P^+\d_\J f\d_+g
  \Biggr)\\ \nonumber
  &+&\frac{\alpha^2}{2}
  \Biggl(
  -\K\d_-f\d_+\d_\J g
  -\K\d_+f\d_-\d_\J g
  -\K\d_+\d_\J f\d_-g
  -\K\d_-\d_\J f\d_+g\\ \nonumber &&
  +\K\P^+\d_+\d_-f\d_+\d_\J g
  -\K\P^+\d_-\d_\J f\d_+^2g
  +\K\P^-\d_-\d_\J f\d_+\d_-g\\ \nonumber &&
  +\K\P^+\d_+\d_\J f\d_+\d_-g
  -\K\P^-\d_-^2f\d_+\d_\J g
  -\K\P^+\d_+^2f\d_-\d_\J g\\ \nonumber &&
  +\K\P^-\d_+\d_-f\d_-\d_\J g
  -\K\P^-\d_+\d_\J f\d_-^2g
  -\frac 12\P^+\P^-\d_\J ^2f\d_+\d_-g\\ \nonumber &&
  -\frac 12\P^+\P^-\d_+\d_-f\d_\J^2g
  +\frac 12\P^+\P^-\d_+\d_\J f\d_-\d_\J g
  +\frac 12\P^+\P^-\d_-\d_\J f\d_+\d_\J g\\ \nonumber &&
  -\frac 12{\P^+}^2\d_+\d_\J f\d_+\d_\J g
  -\frac 12{\P^-}^2\d_-\d_\J f\d_-\d_\J g
  +\frac 14{\P^-}^2\d_\J^2f\d_-^2g\\ \nonumber &&
  +\frac 14{\P^+}^2\d_+^2f\d_\J^2g
  +\frac 14{\P^+}^2\d_\J^2f\d_+^2g
  +\frac 14{\P^-}^2\d_-^2f\d_\J^2g\\ \nonumber &&
  +\K^2\d_+^2f\d_-^2g
  +\K^2\d_-^2f\d_+^2g
  -2\K^2\d_+\d_-f\d_+\d_-g\\ \nonumber &&
  +\frac 14\P^+\d_\J^2f\d_+g
  +\frac 14\P^+\d_+f\d_\J^2g
  +\frac 14\P^-\d_-f\d_\J^2g
  +\frac 14\P^-\d_\J^2f\d_-g
  \Biggr)\\ \nonumber
  &+&\mathcal{O}(\alpha^3)
\end{eqnarray}

\subsubsection{Weyl Ordering}
The most obvious, but least physically enlightening parameterisation is given by
the Weyl ordering
\begin{equation}
  \label{eq:weyl:defn}
  \NO e^{ik_i\hat x^i}\NO=e^{i\bar{w}\hat \P^++iw\hat \P^-+iu\hat \J+iv\hat\K}
\end{equation}
It is a difficult task to calculate the product directly using the BCH formula
\eqref{eq:BCH} as it is a complicated, unrecognisable, infinite sum. We can
however investigate the BCH formula for
\begin{equation}
  \label{eq:weyl:start}
  e^{ip_1\hat\P^++ip_2\hat\P^-}e^{ip_3\hat\J}=e^{G_1(\boldsymbol{p})\hat\P^+
    +G_2(\boldsymbol{p})\hat\P^-+G_3(\boldsymbol{p})\hat\J+G_4(\boldsymbol{p})\hat\K}
\end{equation}
where $G_i(\boldsymbol{p})$ may be calculated. This lets us write a time ordered
product using the Weyl ordering coefficients. Using \eqref{eq:time:mult} the
full exponential product can be calculated, and we may then use
\eqref{eq:weyl:start} again to get back to the Weyl ordering.

We notice that the only surviving $\hat \P^+$ terms are given by the commutators
\begin{equation*}
  \cb{\hat \J}{\left[\ldots,\cb{\hat \J}{\hat \P^+}\right]\ldots} 
\end{equation*}
Gathering all the terms, we find
\begin{eqnarray}
  \nonumber
  G_1&=&\sum_{n=0}\frac{B_n}{n!}\cb{ip_3\hat\J_n}{\left[\ldots,\cb{ip_3\hat
        \J_1}{ip_1\hat\P^+}\right]\ldots} \\ \nonumber
  &=& p_1\sum_{n=0}\frac{B_n}{n!}(-ip_3)^n \\
  \label{eq:weyl:G1}
  &=&\frac{ip_1}{\Phi(-ip_3)}
\end{eqnarray}
where we have introduced the generating function
\begin{equation}
  \label{eq:weyl:Phi}
  \Phi(a)=\frac{e^{\beta a}-1}a
\end{equation}
Similarly for $G_2$
\begin{eqnarray}
  \nonumber
  G_2&=&\sum_{n=0}\frac{B_n}{n!}\cb{ip_3\hat\J_n}{\left[\ldots,\cb{ip_3\hat
        \J_1}{ip_2\hat \P^-}\right]\ldots} \\ \nonumber
  &=& p_2\sum_{n=0}\frac{B_n}{n!}(ip_3)^n \\
  \label{eq:weyl:G2}
  &=&\frac{ip_2}{\Phi(ip_3)}
\end{eqnarray}
The $G_3$ term is trivial
\begin{equation}
  \label{eq:weyl:G3}
  G_3=ip_3
\end{equation}
For $G_4$ there are several non-vanishing commutators
\begin{eqnarray}
  \nonumber
  G_4&=&\sum_{n=1}\frac{B_{n+1}}{n!}
  \cb{ip_1\hat \P^+}{\left[ip_3\hat\J_n,\ldots\cb{ip_3\hat\J_1}{ip_2\hat
        \P^-}\ldots\right]}\\ \nonumber
  &+&\sum_{n=1}\frac{B_{n+1}}{n!}\cb{ip_2\hat\P^-}{\left[ip_3\hat
      \J_n,\ldots\cb{ip_3\hat\J_1}{ip_1 \hat\P^+}\ldots\right]}\\ \nonumber
  &=&-2p_1p_2\sum_{n=1}\frac{B_{n+1}}{n!}(ip_3)^n\\
  \label{eq:weyl:G4}
  &=&-2p_1p_2\Upsilon(ip_3)
\end{eqnarray}
where we have introduced the generating function (see Appendix
\ref{app:generating} for a derivation)\footnote{Note that the $\alpha$-scaling
  also requires, in addition to the exponential product changes, that
  $\Phi(\lambda)\rightarrow\Phi(\alpha\lambda)$ and
  $\Upsilon(\lambda)\rightarrow\alpha\Upsilon(\alpha\lambda)$}
\begin{equation}
  \label{eq:weyl:Upsilon}
  \Upsilon(a)=\frac 12 +\frac{e^{\beta a}-1-\beta a
    e^{\beta a}}{\left(e^{\beta a}-1\right)^2}
\end{equation}
Now that we know the $G_i$ in \eqref{eq:weyl:start}, we can rewrite the Weyl
ordering as a product of exponentials. In preparation, we define $v=\widetilde
v+q_4$ and write the Weyl ordering as
\begin{equation}
  \label{eq:weyl:rewritten:1}
  e^{i\bar{w}\hat\P^++iw\hat \P^-+iu\hat\J+i\widetilde v\hat\K}e^{iq_4\hat\K}
\end{equation}
We proceed to define $q$s by
\begin{eqnarray}
  \label{eq:weyl:qs}
  q_1&=&\bar{w}\Phi(-iu)\\ \nonumber
  q_2&=&w\Phi(iu)\\ \nonumber
  q_3&=&u\\ \nonumber
  q_4&=&v-2i\bar{w}w\Upsilon(iu)\Phi(iu)\Phi(-iu)
\end{eqnarray}
allowing us to rewrite the Weyl ordering as
\begin{equation}
  \label{eq:weyl:time}
  e^{iq_1\hat\P^++iq_2\hat\P^-}e^{iq_3\hat\J}e^{iq_4\hat\K}
\end{equation}
It should be clear that $\bar q_1=q_2$ and $u$ is a real angle. Using
\eqref{eq:time:mult} to calculate the exponential product of the time ordering
form and then \eqref{eq:weyl:start} to get back into the Weyl ordering form, we
find the $\star$-product to be
\begin{eqnarray}
  \label{eq:weyl:star}
  f\star g(z)&=&f(z)e^{ix^i(F^\star_i-k_i-k_i')}g(z)\\ \nonumber
  F^\star_1&=&\frac{\bar{w}\Phi(-i\alpha u)+\bar{w}'\Phi(-i\alpha
    u')e^{-i\alpha  u}}{\Phi(-i\alpha u-i\alpha u')}\\ \nonumber
  F^\star_2&=&\frac{w\Phi(i\alpha u)+w'\Phi(i\alpha u')e^{i\alpha
      u}}{\Phi(i\alpha u-i\alpha u')}\\ \nonumber
  F^\star_3&=&u+u'\\ \nonumber
  F^\star_4&=&-2i\alpha\bar{w}w\Phi(i\alpha u)\Phi(-i\alpha u)\Upsilon(i\alpha
  u)+i\alpha w\bar{w}'\Phi(-i\alpha u)\Phi(-i\alpha u')\\ \nonumber
  &&-2i\alpha\bar{w}'w'\Phi(i\alpha u')\Phi(-i\alpha u')\Upsilon(i\alpha u')
  -i\alpha\bar{w}w'\Phi(i\alpha u)\Phi(i\alpha u')\\ \nonumber
  &&+2i\alpha\left(\bar{w}\Phi(i\alpha u)+\bar{w}'\Phi(-i\alpha u')\right)
  \left(w\Phi(-i\alpha u)+w'\Phi(i\alpha u')\right)
  \\ \nonumber && \quad\!\! \times \Upsilon(i\alpha u+i\alpha u')
  +v+v'
\end{eqnarray}
where $x$ runs over $\P^+,\P^-,\J,\K$. We see from \eqref{eq:example:star} and
\eqref{eq:star:lie} that the operator algebra is recovered in the functional
space with the $\star$-product between the generators being
\begin{eqnarray}
  \label{eq:weyl:generators}
  \P^+\star\P^-&=&\P^+\P^--\K\\ \nonumber
  \P^-\star\P^+&=&\P^+\P^-+\K\\ \nonumber
  \P^+\star\J&=&\J\P^++\frac 12\P^+\\ \nonumber
  \J\star\P^+&=&\J\P^+-\frac 12\P^+\\ \nonumber
  \P^-\star\J&=&\J\P^--\frac 12\P^-\\ \nonumber
  \J\star\P^-&=&\J\P^-+\frac 12\P^-
\end{eqnarray}
To second order in $\alpha$ and written in position space, the $\star$-product
between any two functions is
\begin{eqnarray}
  \label{eq:weyl:positionspace}
  f\star g(x)&=&fg+\frac{\alpha}{2}\Biggl(
  2\K\d_-f\d_+g
  +\P^-\d_\J f\d_-g
  -\P^-\d_-f\d_\J g\\ \nonumber &&\qquad\quad\!\!
  -2\K\d_+f\d_-g
  -\P^+\d_\J f\d_+g
  +\P^+\d_+f\d_\J g
  \Biggr) \\ \nonumber
  &+&\frac{\alpha^2}{2}\Biggl(
  \K^2\d_+^2f\d_-^2g
  +\K\P^+\d_+\d_-f\d_+\d_\J g
  +\K^2\d_-^2f\d_+^2g\\ \nonumber &&
  +\frac 16\P^+\d_\J^2f\d_+g
  +\frac 16\P^+\d_+f\d_\J^2g
  +\frac 16\P^-\d_-f\d_\J^2g\\ \nonumber &&
  -\K\P^+\d_+^2f\d_-\d_\J g
  +\K\P^-\d_-\d_\J f\d_+\d_-g
  -\K\P^-\d_-^2f\d_+\d_\J g\\ \nonumber &&
  -\K\P^+\d_-\d_\J f\d_+^2g
  -2\K^2\d_+\d_-f\d_+\d_-g
  +\K\P^-\d_+\d_-f\d_-\d_\J g\\ \nonumber &&
  +\K\P^+\d_+\d_\J f\d_+\d_-g
  -\K\P^-\d_+\d_\J f\d_-^2g
  +\frac 16\P^-\d_\J^2f\d_-g\\ \nonumber &&
  -\frac 12{\P^+}^2\d_+\d_\J f\d_+\d_\J g
  -\frac 12{\P^-}^2\d_-\d_\J f\d_-\d_\J g\\ \nonumber &&
  -\frac 16\P^-\d_\J f\d_-\d_\J g
  -\frac 16\P^+\d_+\d_\J f\d_\J g
  -\frac 16\P^+\d_\J f\d_+\d_\J g\\ \nonumber &&
  -\frac 13\K\d_+f\d_-\d_\J g
  +\frac 23\K\d_\J f\d_+\d_-g
  +\frac 23\K\d_+\d_-f\d_\J g\\ \nonumber &&
  +\frac 14{\P^+}^2\d_\J^2f\d_+^2g
  +\frac 12\P^+\P^-\d_-\d_\J f\d_+\d_\J g
  -\frac 12\P^+\P^-\d_+\d_-f\d_\J^2g\\ \nonumber &&
  +\frac 14{\P^-}^2\d_\J ^2f\d_-^2g
  +\frac 14{\P^-}^2\d_-^2f\d_\J ^2g
  -\frac 16\P^-\d_-\d_\J f\d_\J g\\ \nonumber &&
  -\frac 26\K\d_-\d_\J f\d_+g
  -\frac 26\K\d_-f\d_+\d_\J g
  -\frac 13\K\d_+\d_\J f\d_-g\\ \nonumber &&
  +\frac 12\P^+\P^-\d_+\d_\J f\d_-\d_\J g
  +\frac 14{\P^+}^2\d_+^2f\d_\J^2g
  -\frac 12\P^+\P^-\d_\J^2f\d_+\d_-g
  \Biggr)\\ \nonumber
  &+&\mathcal{O}(\alpha^3)
\end{eqnarray}

\clearpage
\section{Weyl Systems}
\subsection{Standard Weyl Systems}
A Weyl system is a map $\weyl$ from a real, finite dimensional, symplectic
vector space $S$ to the set of unitary operators on a suitable Hilbert space,
with the property
\begin{equation}
  \label{eq:nw:weyl:st:prop}
  \weyl(k)\weyl(k')=e^{i\omega(k,k')}\weyl(k')\weyl(k)
\end{equation}
where $\omega$ is the symplectic, translationally invariant form on $S$.
According to Stone's theorem, $\weyl$ is the exponential of a Hermitian operator
on $\cal H$
\begin{equation}
  \label{eq:nw:weyl:st:exp}
  \weyl(ak)=e^{iaX(k)}
\end{equation}
The original motivation was to avoid the presence of unbounded operators in the
Q.M. formalism and is satisfied by property \eqref{eq:nw:weyl:st:prop} as it can
be considered the exponentiated version of the commutation relations. The usual
form of the commutators can be recovered with a series expansion
\begin{equation}
  \label{eq:nw:weyl:st:series}
  \cb{X(k)}{X(k')}=-i\omega(k,k')
\end{equation}
The usual identification has $S=\real^{2n}$ and $\omega=dq^i\wedge dp^j$ where
$(q^i,p_j)$ are canonical coordinates, giving the explicit realisation
\begin{equation}
  \label{eq:nw:weyl:st:realise}
  \weyl(q,p)=e^{i\left(q^iQ_j+p_iP^j\right)}
\end{equation}
where $Q$ and $P$ are the usual operators that represent position and momentum
observables.

It is then possible to formulate the Groenewold-Moyal $\star$-product given in
section \ref{star}, which is the case when $\omega(k,k')$ and $\Theta$ are
constant.

\subsection{Generalised Weyl Systems}
The previous construction of a $\star$-product for a Weyl system is not valid
for the case when $\Theta$ is non-constant, simply because property
\eqref{eq:nw:weyl:st:prop} does not hold.

It is possible to define a class of deformed products in a set of functions on
$\real^n$ without using an explicit realisation of $\weyl$, thus generalising
the concept. A \textit{generalised Weyl system} \cite{Agostini:2002} is a map in
the set of operators $\weyl$ with composition rule
\begin{equation}
  \label{eq:weyl:gen:prop}
  \weyl(k)\weyl(k')=e^{\frac i2\omega(k,k')}\weyl(k\comp k')
\end{equation}
$\comp$ forms a group on $\real^n$ satisfying
\begin{eqnarray}
  \label{eq:weyl:comp:group}
  (k\comp k')\comp k'' &=& k\comp(k'\comp k'') \\
  k\comp \underline k &=& 0
\end{eqnarray}
where $\underline k$ is the inverse of $k$ and $0$ is the neutral element. When
$\comp$ is the normal addition on $\real^n$ this is a standard Weyl system.

In analogy to the standard Weyl system, we define the symbol
\begin{equation}
  \label{eq:weyl:gen:symb}
  {\hat\Omega}[f]\equiv F=\frac{1}{(2\pi)^n}\int_{-\infty}^\infty \widetilde
  f(k)\weyl(k)dk
\end{equation}
(from here on we use capital Roman letters to denote elements of the deformed
algebra). The product is
\begin{equation}
  \label{eq:weyl:gen:prod}
  FG=\frac{1}{(2\pi)^{2n}}\int_{-\infty}^\infty \widetilde f(k)\widetilde g(k')
  e^{\frac i2 \omega(k,k')}\weyl(k\comp k')dkdk'
\end{equation}
and it can be shown that
\begin{equation}
  \label{eq:weyl:gen:star}
  f\star g(x)=\frac{1}{(2\pi)^{2n}}\int_{-\infty}^\infty \widetilde f(k)\widetilde g(k')
  e^{\frac i2 \omega(k,k')}e^{i\left(k\comp k'\right)x}dkdk'
\end{equation}
where
\begin{equation}
  \label{eq:weyl:gen:exp}
  e^{ikx}={\hat\Omega}^{-1}[\weyl(k)]
\end{equation}
Associativity of this product is a consequence of the associativity of $\comp$.
The Hermitian conjugate of $F$ is defined as
\begin{equation}
  \label{eq:weyl:gen:herm}
  F^\dagger = \frac{1}{(2\pi)^{2n}}\int_{-\infty}^\infty\widetilde f^*(\underline
  k)\weyl(k)dk
\end{equation}
and
\begin{equation}
  \label{eq:weyl:gen:herm:2}
  (FG)^\dagger=G^\dagger F^\dagger
\end{equation}
an extra condition
\begin{equation}
  \label{eq:weyl:gen:lizzicond}
  \underline{(k\comp k')}=(\underline{k'}\comp\underline{k})
\end{equation}
is introduced in \cite{Agostini:2002} which is automatically satisfied by the
associativity of $\comp$ since
\begin{eqnarray}
  \label{eq:weyl:gen:assoc}
  (k\comp k')\comp\underline{(k\comp k')}&=&0 \\ \nonumber
  (k\comp k')\comp(\underline{k'}\comp\underline{k})&=&
  k\comp(k'\comp\underline{k'})\comp\underline k=0
\end{eqnarray}

\subsection{Nappi-Witten Generalised Weyl System}
As the Nappi-Witten algebra cannot satisfy property \eqref{eq:nw:weyl:st:prop},
it does not meet the conditions required to be a standard Weyl system. It can
however be presented as a generalised Weyl system. We find the compositions
$\comp$ for each ordering by inspection of the previously derived
$\star$-products \eqref{eq:time:star}, \eqref{eq:symtime:star} and
\eqref{eq:weyl:star}. In each case $\omega(k,k')=0$ and the composition can be
shown to be associative.

\subsubsection{Time Ordering}
By inspection of \eqref{eq:time:star}, we find the composition $\compa$ to be
\begin{eqnarray}
  \label{eq:weyl:nw:time}
  (k \compa k')^1&=& \bar w+\bar w'e^{-i\alpha u}\\ \nonumber
  (k \compa k')^2&=& w + w'e^{i\alpha u}\\ \nonumber
  (k \compa k')^3&=& u+u'\\ \nonumber
  (k \compa k')^4&=& v+v'-i\alpha\bar ww'e^{i\alpha u}
  +i\alpha w\bar w'e^{-i\alpha u}
\end{eqnarray}
where we define $k$ and it's inverse $\underline k$, component-wise as
\begin{eqnarray}
  \label{eq:weyl:gen:time}
  k&=&(\bar w,w, u, v) \\ \nonumber
  \underline k&=&(-e^{iu}\bar w,-e^{-iu}w,-u,-v)
\end{eqnarray}

\subsubsection{Symmetric Time Ordering}
By inspection of \eqref{eq:symtime:star}, we find the composition $\compb$ to be
\begin{eqnarray}
  \label{eq:weyl:nw:symtime}
  (k \compb k')^1&=&\bar we^{\frac i2\alpha u'}+\bar w'e^{-\frac i2\alpha u}\\
  \nonumber
  (k \compb k')^2&=&we^{-\frac i2\alpha u'}+w'e^{\frac i2\alpha u}\\ \nonumber
  (k \compb k')^3&=&u+u'\\ \nonumber
  (k \compb k')^4&=&v+v'-i\alpha\bar ww'e^{\frac i2\alpha(u+u')}
  +i\alpha w\bar w'e^{-\frac i2\alpha(u+u')}
\end{eqnarray}
where we define $k$ and it's inverse $\underline k$, component-wise as
\begin{eqnarray}
  \label{eq:weyl:gen:symtime}
  k&=&(\bar w,w, u, v) \\ \nonumber
  \underline k&=&(-\bar w,-w,-u,-v)
\end{eqnarray}

\subsubsection{Weyl Ordering}
By inspection of \eqref{eq:weyl:star}, we find the composition $\compc$ to be
\begin{eqnarray}
  \label{eq:weyl:nw:weyl}
  (k \compc k')^1&=&\frac{\bar{w}\Phi(-i\alpha u)+\bar{w}'\Phi(-i\alpha
    u')e^{-i\alpha  u}}{\Phi(-i\alpha u-i\alpha u')}\\ \nonumber
  (k \compc k')^2&=&\frac{w\Phi(i\alpha u)+w'\Phi(i\alpha u')e^{i\alpha
      u}}{\Phi(i\alpha u-i\alpha u')}\\ \nonumber
  (k \compc k')^3&=&u+u'\\ \nonumber
  (k \compc k')^4&=&-2i\alpha\bar{w}w\Phi(i\alpha u)\Phi(-i\alpha u)
  \Upsilon(i\alpha u)+i\alpha w\bar{w}'\Phi(-i\alpha u)\Phi(-i\alpha u')
  \\ \nonumber
  &&-2i\alpha\bar{w}'w'\Phi(i\alpha u')\Phi(-i\alpha u')\Upsilon(i\alpha u')
  -i\alpha\bar{w}w'\Phi(i\alpha u)\Phi(i\alpha u')\\ \nonumber
  &&+2i\alpha\left(\bar{w}\Phi(i\alpha u)+\bar{w}'\Phi(-i\alpha u')\right)
  \left(w\Phi(-i\alpha u)+w'\Phi(i\alpha u')\right)
  \\ \nonumber && \quad\!\! \times \Upsilon(i\alpha u+i\alpha u')
  +v+v'
\end{eqnarray}
where we define $k$ and it's inverse $\underline k$, component-wise as
\begin{eqnarray}
  \label{eq:weyl:gen:weyl}
  k&=&(k_1,k_2,k_3,k_4) \\ \nonumber
  \underline k&=&(-k_1,-k_2,-k_3,-k_4)
\end{eqnarray}

\clearpage
\section{Derivatives}
\subsection{Derivative Commutators}
For any Lie Algebra
\begin{equation}
  \label{eq:lie}
  \cb{\hat x^\mu}{\hat x^\nu}=iC_\lambda^{\mu\nu}\hat x^\lambda
\end{equation}
we may demand linear derivatives \cite{Dimitrijevic:2004vv} which look like
\begin{equation}
  \label{eq:deriv}
  \cb{\opd_\mu}{\hat x^\nu}=\delta_\mu^\nu
  +i\rho_\mu^{\nu\lambda}\opd_\lambda
\end{equation}
The Jacobi identity between $\opd_\mu$, $\hat x^\nu$, $\hat x^\lambda$
supplies the sufficient conditions
\begin{eqnarray}
  \label{eq:deriv:rho}
  \rho_\lambda^{\mu\nu}-\rho_\lambda^{\nu\mu}&=&C_\lambda^{\mu\nu}\\ \nonumber
  \rho_\lambda^{\mu\nu}\rho_\nu^{\kappa\sigma}
  -\rho_\lambda^{\kappa\nu}\rho_\nu^{\mu\sigma}
  &=&C_\nu^{\mu\kappa}\rho_\lambda^{\nu\sigma}
\end{eqnarray}

\subsection{$\star$-Derivatives}
This allows us to define a $\star$-derivative by elevating the $\opd$ to
$\d^\star$ composed of normal derivatives and satisfying the following
conditions
\begin{eqnarray}
  \nonumber
  \scb{x^\mu}{x^\nu}&=&iC_\lambda^{\mu\nu}x^\lambda\\
  \label{eq:lie:starcomm}
  \scb{\d_\mu^\star}{x^\nu}f(x)&=&\d_\mu^\star(x^\nu\star
  f(x)) -x^\nu\star(\d_\mu^\star f(x))\\ \nonumber
  &=&(\delta_\mu^\nu+i\rho_\mu^{\nu\lambda}\d_\lambda^\star)f(x)
\end{eqnarray}

\nts{I see why the awful notation by the Germans exists now... using this
  cleaner notation means that strictly speaking we mean $\d_\mu^\star
  \star (\ldots)$ instead of $\d_\mu^\star(\ldots)$ in
  \eqref{eq:lie:starcomm}}

\subsection{Leibniz Rule}
These $\opd$, $\d^\star$ do not necessarily follow the normal
Leibniz rule. However we may obtain a Leibniz rule from the co-product, given by
the Generalised Weyl System

\nts{I am still unclear about the details of \emph{why} this is true. speak to
  Berndt maybe?}

For example, \eqref{eq:weyl:nw:time} implies the co-product
\begin{equation}
  \label{eq:coproduct}
    \Delta(\opd_1)=\opd_1\otimes \1
    +e^{\alpha\opd_3}\otimes\opd_1
\end{equation}
giving the Leibniz rule
\begin{eqnarray}
  \label{eq:leibniz}
  \hat\grad_1(\hat f\hat g)&=&i\mu\Delta(\opd_1)(\hat f\otimes\hat g)
  \\ \nonumber
  &=&(\opd_1\hat f)\hat g +(e^{\alpha\opd_3}\hat
  f)(\opd_1\hat g)
\end{eqnarray}
with the rule for the $\star$-derivatives being inherited
\begin{equation}
  \label{eq:leibtiz:star}
  \grad^\star_1(f\star g)=(\d_1^\star f)\star g
  +(e^{\alpha\d_3^\star}f)\star(\d_1^\star g)
\end{equation}
It is important to note that normal Leibniz does not necessarily hold even when
$fg$ is normal multiplication
\begin{equation}
  \label{eq:leibniz:normal}
  \grad^\star(fg)\neq f\grad^\star g+g\grad^\star f
\end{equation}

\subsection{Nappi-Witten $\star$-Derivatives}
There are an infinite number of solutions to $\rho_\lambda^{\mu\nu}$
\eqref{eq:deriv:rho}, but we choose an intuitive one giving
\begin{align}
  \nonumber
  \cb{\opd_1}{\xh^1}&=1 & \cb{\opd_2}{\xh^1}&=0
  &\quad\cb{\opd_3}{\xh^1}=0 &\quad \cb{\opd_4}{\xh^1}=-\alpha\opd_2
  \\ \nonumber
  \cb{\opd_1}{\xh^2}&=0 & \cb{\opd_2}{\xh^2}&=1
  &\quad\cb{\opd_3}{\xh^2}=0 &\quad \cb{\opd_4}{\xh^2}=\alpha\opd_1
  \\ \nonumber
  \cb{\opd_1}{\xh^3}&=-\alpha\opd_1 & \cb{\opd_2}{\xh^3}&=\alpha\opd_2
  &\quad\cb{\opd_3}{\xh^3}=1 &\quad \cb{\opd_4}{\xh^3}=0
  \\ \nonumber
  \cb{\opd_1}{\xh^4}&=0 & \cb{\opd_2}{\xh^4}&=0
  &\quad\cb{\opd_3}{\xh^4}=0 &\quad \cb{\opd_4}{\xh^4}=1
  \\ \label{eq:rho:nw4}
\end{align}

\subsubsection{Time Ordering}
From \eqref{eq:weyl:nw:time} we deduce the co-products
\begin{eqnarray}
  \label{eq:deriv:leibniz:time}
  \Delta^\ast(\opd_1)&=&\opd_1\otimes\1+e^{\alpha\opd_3}\otimes\opd_1\\ \nonumber
  \Delta^\ast(\opd_2)&=&\opd_2\otimes\1+e^{-\alpha\opd_3}\otimes\opd_2\\ \nonumber
  \Delta^\ast(\opd_3)&=&\opd_3\otimes\1 + \1\otimes\opd_3\\ \nonumber
  \Delta^\ast(\opd_4)&=&\opd_4\otimes\1 + \1\otimes\opd_4
        -i\alpha\opd_1e^{-\alpha\opd_3}\otimes\opd_2
        +i\alpha\opd_2e^{\alpha\opd_3}\otimes\opd_1
\end{eqnarray}
Using \eqref{eq:time:star} we find the $x\ast f(x)$ products
\begin{eqnarray}
  \label{eq:time:xstar}
  x^1\ast f(x)&=&(x^1-\alpha x^4\d_2)f(x)\\ \nonumber
  x^2\ast f(x)&=&(x^2+\alpha x^4\d_1)f(x)\\ \nonumber
  x^3\ast f(x)&=&(x^3+\alpha x^2\d_2
        -\alpha x^1\d_1)f(x)\\ \nonumber
  x^4\ast f(x)&=&x^4f(x)
\end{eqnarray}
which we may insert into \eqref{eq:lie:starcomm} to show that the
$\ast$-derivatives are simply the normal derivatives on functions
\begin{equation}
  \label{eq:time:derivs}
  \d^\ast_i=\d_i
\end{equation}

\subsubsection{Symmetric Time Ordering}
From \eqref{eq:weyl:nw:symtime} we deduce the co-products
\begin{eqnarray}
  \label{eq:deriv:leibniz:symtime}
  \Delta^\bullet(\opd_1)&=&\opd_1\otimes e^{-\frac \alpha 2\opd_3}
        +e^{\frac \alpha 2\opd_3}\otimes\opd_1\\ \nonumber
  \Delta^\bullet(\opd_2)&=&\opd_2\otimes e^{\frac \alpha 2\opd_3}
        +e^{-\frac \alpha 2\opd_3}\otimes\opd_2\\ \nonumber
  \Delta^\bullet(\opd_3)&=&\opd_3\otimes\1 + \1\otimes\opd_3\\ \nonumber
  \Delta^\bullet(\opd_4)&=&\opd_4\otimes\1 + \1\otimes\opd_4
        -i\alpha\opd_1e^{-\frac \alpha 2\opd_3}\otimes e^{-\frac \alpha 2\opd_3}\opd_2
        +i\alpha\opd_2e^{\frac \alpha 2\opd_3}\otimes e^{\frac \alpha 2\opd_3}\opd_1
\end{eqnarray}
Using \eqref{eq:symtime:star} we find the $x\bullet f(x)$ products
\begin{eqnarray}
  \label{eq:symtime:xstar}
  x^1\bullet f(x)&=&\left((x^1-\alpha x^4\d_2)
    e^{\frac \alpha 2\d_3}\right)f(x)\\ \nonumber
  x^2\bullet f(x)&=&\left((x^2+\alpha x^4\d_1)
    e^{-\frac \alpha 2\d_3}\right)f(x)\\ \nonumber
  x^3\bullet f(x)&=&(x^3+\frac \alpha 2 x^2\d_2
        -\frac \alpha 2 x^1\d_1)f(x)\\ \nonumber
  x^4\bullet f(x)&=&x^4f(x)
\end{eqnarray}
which we may insert into \eqref{eq:lie:starcomm} to show that the
$\bullet$-derivatives are non-trivial
\begin{eqnarray}
  \label{eq:symtime:derivs}
  \d^\bullet_1&=&\d_1e^{-\frac \alpha 2\d_3}\\ \nonumber
  \d^\bullet_2&=&\d_2e^{\frac \alpha 2\d_3}\\ \nonumber
  \d^\bullet_3&=&\d_3\\ \nonumber
  \d^\bullet_4&=&\d_4
\end{eqnarray}

\subsubsection{Weyl Ordering}
From \eqref{eq:weyl:nw:weyl} we deduce the co-products
\begin{eqnarray}
  \label{eq:deriv:leibniz:weyl}
  \Delta^\star(\opd_1)&=&\frac{\opd_1\Phi(\alpha \opd_3)\otimes\1
  +e^{\alpha\opd_3}\otimes\opd_2\Phi(\alpha \opd_3)}
{\Phi(\alpha \opd_3\otimes\1+\1\otimes\alpha\opd_3)}\\ \nonumber
  \Delta^\star(\opd_2)&=&\frac{\opd_2\Phi(-\alpha\opd_3)\otimes\1
  +e^{-\alpha\opd_3}\otimes\opd_1\Phi(-\alpha \opd_3)}
{\Phi(-\alpha \opd_3\otimes\1+\1\otimes\alpha\opd_3)}\\ \nonumber
  \Delta^\star(\opd_3)&=&\opd_3\otimes\1+\1\otimes\opd_3
\\ \nonumber
  \Delta^\star(\opd_4)&=&-2\alpha\opd_1\opd_2\Phi(-\alpha\opd_3)\Phi(\alpha\opd_3)
  \Upsilon(-\alpha\opd_3)\otimes\1 \\ \nonumber
  &&-\1\otimes2\alpha\opd_2\opd_1\Phi(-\alpha\opd_3)\Phi(\alpha\opd_3)
  \Upsilon(-\alpha\opd_3)\\ \nonumber
  &&+\alpha\opd_2\Phi(\alpha\opd_3)\otimes\opd_1\Phi(\alpha\opd_3)
  -\alpha\opd_1\Phi(-\alpha\opd_3)\otimes\opd_2\Phi(-\alpha\opd_3)\\ \nonumber
  &&+2\alpha\left(\opd_1\Phi(-\alpha\opd_3)\otimes\1
    +\1\otimes\opd_1\Phi(\alpha\opd_3)\right)
  \\ \nonumber && \qquad
  \times \left(\opd_2\Phi(\alpha\opd_3)\otimes\1
    +\1\otimes\opd_2\Phi(-\alpha\opd_3)\right)
  \\ \nonumber && \qquad
  \times \Upsilon(-\alpha\opd_3\otimes\1-\1\otimes\alpha\opd_3)
  +\opd_4\otimes\1+\1\otimes\opd_4
\end{eqnarray}

Using \eqref{eq:weyl:star} we find the $x\star f(x)$ products
\begin{eqnarray}
  \label{eq:weyl:xstar}
  x^1\star f(x)&=&\left\{\frac{x^1}{\Phi(-\alpha\d_3)}
    +2x^4\left(1-\frac 1{\Phi(-\alpha\d_3)}\right)
      \frac{\d_2}{\d_3}\right\}f(x)\\ \nonumber
  x^2\star f(x)&=&\left\{\frac{x^2}{\Phi(\alpha\d_3)}
    +2x^4\left(1-\frac 1{\Phi(\alpha\d_3)}\right)
      \frac{\d_1}{\d_3}\right\}f(x)\\ \nonumber
  x^3\star f(x)&=&\left\{
    x^1\left(1-\frac{1}{\Phi(-\alpha\d_3)}\right)
        \frac{\d_1}{\d_3}
    +x^2\left(1-\frac{1}{\Phi(\alpha\d_3)}\right)
        \frac{\d_2}{\d_3}\right.\\ \nonumber
    &&+x^3-2\alpha x^4\left(\frac{2}{\alpha\d_3}
        \left.-\frac{e^{\alpha\d_3}+1}{e^{\alpha\d_3}-1}\right)
        \frac{\d_1\d_2}{\d_3}\right\}f(x)\\ \nonumber
  x^4\star f(x)&=&x^4f(x)
\end{eqnarray}
which we may insert into \eqref{eq:lie:starcomm}, showing that the
following $\star$-derivatives are non-trivial
\begin{eqnarray}
  \label{eq:weyl:derivs}
  \d^\star_1&=&\Phi(-\alpha\d_3)\d_1\\ \nonumber
  \d^\star_2&=&\Phi(\alpha\d_3)\d_2\\ \nonumber
  \d^\star_3&=&\d_3
\end{eqnarray}
however the ansatz
\begin{equation}
  \label{eq:weyl:derivs:4}
  \d^\star_4=\d_4+\frac{\d_1\d_2}{\d_3}
  \left(2-\Phi(-\alpha\d_3)-\Phi(\alpha\d_3)\right)
\end{equation}
is only confirmed for three of the $\d^\star_4$ relationships given in
\eqref{eq:lie:starcomm}. In order for \eqref{eq:weyl:derivs:4} to be true we
require
\begin{eqnarray}
  \nonumber
  \scb{\d^\star_4}{x^3}f(x)&=& \left\{2\left(1-\Phi(-\alpha\d_3)
      -\Phi(\alpha\d_3)\right)+e^{\alpha\d_3}+e^{-\alpha\d_3}
      \right\}\frac{\d_1\d_2}{\d_3^2}f(x)\\ \nonumber
  &&+\frac 1{\d_3}\left(2-\Phi(-\alpha\d_3)
      -\Phi(\alpha\d_3)\right)x^3\d_1\d_2f(x)\\ \nonumber
  &&-x^3\left(2-\Phi(-\alpha\d_3)-\Phi(\alpha\d_3)\right)
    \frac{\d_1\d_2}{\d_3}f(x) \\
  \label{eq:weyl:derivs:4:cond}
  &\shouldid&0
\end{eqnarray}
which is very difficult to calculate due to the presence of an ill-defined
$\frac 1{\d_3}$ term acting on $x^3\d_if(x)$.

\section{Integrals}
\subsection{$\star$-Integral Measure}
In order to derive field equations by means of a variational principle, we
require an integral. Algebraically an integral is a linear map of the algebra
into complex numbers
\begin{eqnarray}
  \label{eq:nw:int:alg}
  \int:{\cal \hat A}(\hat x)\rightarrow \mathbb C&&\\
  \int (c_1\hat f+c_2\hat g)=c_1\int\hat f+c_2\int\hat g\qquad&&
  \forall \hat f,\hat g\in {\cal \hat A}(\hat x), c_i\in\mathbb C
\end{eqnarray}
We also demand the cyclic property \cite{Dimitrijevic:2003wv,Agostini:2004cu}
\begin{equation}
  \label{eq:nw:int:cyc}
  \int\hat f\hat g=\int\hat g\hat f
\end{equation}
We define the integral in the $\star$-product formalism using the usual
definitions of an integral of commuting functions. It has been shown in
\cite{Dietz:2001} that a measure can be introduced to achieve property
\eqref{eq:nw:int:cyc}
\begin{equation}
  \label{eq:nw:int:cyc:measure}
  \int d^nx\mu(x) f(x)\star g(x)=\int d^nx\mu(x) g(x)\star f(x)
\end{equation}
It is important to note that a $\mu$ which satisfies
\eqref{eq:nw:int:cyc:measure} gives the integral the additional
property \nts{show why this is true in more detail}
\begin{equation}
  \label{eq:nw:int:cyc:nostar}
  \int d^nx\mu(x) f(x)\star g(x) =\int d^nx\mu(x) f(x)g(x)
\end{equation}

Since the functions $x^i$ form a basis, it is necessary for a cyclic action
functional to satisfy the following condition
\begin{equation}
  \label{eq:nw:int:condition}
  \int d^4x \mu_\star(x)\cb{(x^i)^n}{f(x)}_\star=0
\end{equation}
for all the natural numbers $n$ and $i=1\ldots4$.

We may rewrite any commutator of the form $\cb{\hat x^n}{\hat y}$ using the
following identity 
\begin{equation}
  \label{eq:nw:int:samstheory}
  \cb{\hat x^{n+1}}{\hat y} = \sum_{m=0}^n\binom nm
  \hat x^{n-m}\cb{\hat x}{\hat y}\hat x^m
\end{equation}
allowing us to simplify \eqref{eq:nw:int:condition} to
\begin{equation}
  \label{eq:nw:int:condition2}
  \int d^4x \mu_\star(x)\sum_{m=0}^n\binom nm
  (x^i)^{n-m}\cb{x^i}{f(x)}_\star(x^i)^m=0
\end{equation}
It is then a matter of inserting the explicit form of $\cb{x^i}{f(x)}_\star$ to
find restraints upon $\mu(x)$ for general $f(x)$, making use of integration by
parts in the case where $\int d(fg)=0$\footnote{As we are in a Schwartz space,
  all functions go to zero at infinity.}.
\begin{equation}
  \label{eq:nw:int:parts}
  \int \left[fg\d^nh\right] dx=
  (-1)^n\int\left[f(\d^ng)h+g(\d^nf)h\right]dx
\end{equation}

\subsection{Anti-Hermitian Derivatives}
\nts{this should not be its own section, encorporate this text into the previous
  section}

The variational principle requires that the derivatives be anti-hermitian by
the following definiton
\begin{equation}
  \label{eq:nw:int:ah}
  \int d^nx\mu(x) \overline f\star\d^\star g =
  -\int d^nx\mu(x) \overline{\d^\star f}\star g
\end{equation}
Which allows for a generalised integration by parts.

\subsection{Nappi-Witten $\star$-Integrals}
\subsubsection{Time Ordering}
In addition to \eqref{eq:time:xstar}, the $f(x)\ast x^i$ products are
\begin{eqnarray}
  \label{eq:time:starx}
  f(x)\ast x^1&=&(x^1+\alpha x^4\d_2)e^{-\alpha\d_3}f(x)\\ \nonumber
  f(x)\ast x^2&=&(x^2-\alpha x^4\d_1)e^{\alpha\d_3}f(x)\\ \nonumber
  f(x)\ast x^3&=&x^3f(x)\\ \nonumber
  f(x)\ast x^4&=&x^4f(x)
\end{eqnarray}
giving the following $\ast$-commutators
\begin{eqnarray}
  \label{eq:time:comm}
  \cb{x^1}{f(x)}_\ast&=&-\alpha x^4\left(1
    +e^{-\alpha\d_3}\right)\d_2f(x)\\ \nonumber
  \cb{x^2}{f(x)}_\ast&=&\alpha x^4\left(1
    +e^{\alpha\d_3}\right)\d_1f(x)\\ \nonumber
  \cb{x^3}{f(x)}_\ast&=&\alpha\left(x^2\d_2
    -x^1\d_1\right)f(x)\\ \nonumber
  \cb{x^4}{f(x)}_\ast&=&0
\end{eqnarray}
When inserted into \eqref{eq:nw:int:condition2}, they give the restraints on
$\mu_\ast(x)$
\begin{eqnarray}
  \label{eq:time:mu:all}
  \left(1+e^{\alpha\d_3}\right)\d_2\mu_\ast(x)&=&0\\ \nonumber
  \left(1-e^{-\alpha\d_3}\right)\d_1\mu_\ast(x)&=&0\\ \nonumber
  x^1\d_1\mu_\ast(x)&=&x^2\d_2\mu_\ast(x)
\end{eqnarray}
satisfied by
\begin{equation}
  \label{eq:time:mu}
  \d_{1,2}\mu_\ast(x)=0
\end{equation}

However, the $\d_4^\ast$ derivative does not satisfy \eqref{eq:nw:int:ah},
but it may be remedied by introducing the translation
$\d_4\rightarrow\d_4+\frac{\d_4\mu_\ast(x)}{2\mu_\ast(x)}$ so that the
$\ast$-derivatives \eqref{eq:time:derivs} are now
\begin{eqnarray}
  \label{eq:time:d}
  \widetilde\d^\ast_1&=&\d_1 \\ \nonumber
  \widetilde\d^\ast_2&=&\d_2 \\ \nonumber
  \widetilde\d^\ast_3&=&\d_3 \\ \nonumber
  \widetilde\d^\ast_4&=&\d_4 + \frac{\d_4\mu_\ast(x)}{2\mu_\ast(x)}
\end{eqnarray}
with no adverse effects on \eqref{eq:rho:nw4} \nts{but affects linearity. go
  into more detail}.

\subsubsection{Symmetric Time Ordering}
In addition to \eqref{eq:symtime:xstar}, the $f(x)\bullet x^i$ products are
\begin{eqnarray}
  \label{eq:symtime:starx}
  f(x)\bullet x^1&=&\left(x^1+\alpha x^4\d_2\right)
        e^{-\frac \alpha 2 \d_3}f(x)\\ \nonumber
  f(x)\bullet x^2&=&\left(x^2-\alpha x^4\d_1\right)
        e^{\frac \alpha 2 \d_3}f(x)\\ \nonumber
  f(x)\bullet x^3&=&-\alpha\left(x^3+\frac \alpha 2x^1\d_1
    -\frac \alpha 2x^2\d_2\right)f(x)\\ \nonumber
  f(x)\bullet x^4&=&x^4f(x)
\end{eqnarray}
giving the following $\bullet$-commutators
\begin{eqnarray}
  \label{eq:symtime:comm}
  \cb{x^1}{f(x)}_\bullet&=&\left(x^1-\alpha x^4\d_2\right)
  \left(e^{\frac\alpha 2\d_3}-e^{-\frac\alpha 2\d_3}\right)f(x)\\ \nonumber
  \cb{x^2}{f(x)}_\bullet&=&\left(x^2+\alpha x^4\d_1\right)
  \left(e^{-\frac\alpha 2\d_3}-e^{\frac\alpha 2\d_3}\right)f(x)\\ \nonumber
  \cb{x^3}{f(x)}_\bullet&=&\alpha\left(x^2\d_2
    -x^1\d_1\right)f(x)\\ \nonumber
  \cb{x^4}{f(x)}_\bullet&=&0
\end{eqnarray}
When inserted into \eqref{eq:nw:int:condition2}, they give the restraints on
$\mu_\bullet(x)$
\begin{eqnarray}
  \label{eq:symtime:mu:all}
  (1-\d_2)\left(e^{-\frac\alpha 2\d_3}
    -e^{\frac\alpha 2\d_3}\right)\mu_\bullet(x)&=&0\\ \nonumber
  (1+\d_1)\left(e^{\frac\alpha 2\d_3}
    -e^{-\frac\alpha 2\d_3}\right)\mu_\bullet(x)&=&0\\ \nonumber
  x^1\d_1\mu_\bullet(x)&=&x^2\d_2\mu_\bullet(x)
\end{eqnarray}
satisfied by
\begin{equation}
  \label{eq:symtime:mu:rest}
  x^1\d_1\mu_\bullet(x)=x^2\d_2\mu_\bullet(x)
  \qquad\qquad \d_3\mu_\bullet(x)=0
\end{equation}

However, the $\d_{1,2,4}^\bullet$ derivatives do not satisfy
\eqref{eq:nw:int:ah}. The introduction of translations to $\d_{1,2}$ breaks the
derivative $\bullet$-commutator relationships\footnote{Article
  \cite{Dimitrijevic:2003wv} suggests that such a change does not effect the
  canonical commutators, but this is not true for the function commutators.
  Consistency between operator and function space commutators would only be
  possible by demanding that multiplication from the left follow a Leibniz like
  rule for the translation part.} (\ref{eq:lie:starcomm}, \ref{eq:rho:nw4}) and
in order to satisfy both sets of restraints, we are forced to further require
$\mu_\bullet(x)$ to be independent of $x^{1,2}$
\begin{equation}
  \label{eq:symtime:mu}
  \d_{1,2,3}\mu_\bullet(x)=0
\end{equation}

The translation $\d_4\rightarrow\d_4+\frac{\d_4\mu_\bullet(x)}{2\mu_\bullet(x)}$
must still be applied in order to ensure the derivatives are anti-Hermitian
under integration. The $\bullet$-derivatives \eqref{eq:symtime:derivs} are now
\begin{eqnarray}
  \label{eq:symtime:d}
  \widetilde\d^\bullet_1&=&\d_1e^{-\frac \alpha 2\d_3}\\ \nonumber
  \widetilde\d^\bullet_2&=&\d_2e^{\frac \alpha 2\d_3}\\ \nonumber
  \widetilde\d^\bullet_3&=&\d_3\\ \nonumber
  \widetilde\d^\bullet_4&=&\d_4+\frac{\d_4\mu_\bullet(x)}{2\mu_\bullet(x)}
\end{eqnarray}
with no adverse effects on \eqref{eq:rho:nw4}.

\subsubsection{Weyl Ordering}
In addition to \eqref{eq:weyl:xstar}, the $f(x)\star x^i$ products are
\begin{eqnarray}
  \label{eq:weyl:starx}
  f(x)\star x^1&=&\left\{\frac{x^1}{\Phi(\alpha\d_3)}
    +2x^4\left(1-\frac 1{\Phi(\alpha\d_3)}\right)
      \frac{\d_2}{\d_3}\right\}f(x)\\ \nonumber
  f(x)\star x^2&=&\left\{\frac{x^2}{\Phi(-\alpha\d_3)}
    +2x^4\left(1-\frac 1{\Phi(-\alpha\d_3)}\right)
      \frac{\d_1}{\d_3}\right\}f(x)\\ \nonumber
  f(x)\star x^3&=&\left\{
    x^1\left(1-\frac{1}{\Phi(\alpha\d_3)}\right)
        \frac{\d_1}{\d_3}
    +x^2\left(1-\frac{1}{\Phi(-\alpha\d_3)}\right)
        \frac{\d_2}{\d_3}\right.\\ \nonumber
    &&+x^3-2\alpha x^4\left(\frac{2}{\alpha\d_3}
        \left.-\frac{e^{\alpha\d_3}+1}{e^{\alpha\d_3}-1}\right)
        \frac{\d_1\d_2}{\d_3}\right\}f(x)\\ \nonumber
  f(x)\star x^4&=&x^4f(x)
\end{eqnarray}
giving the following $\star$-commutators
\begin{eqnarray}
  \label{eq:weyl:comm}
  \cb{x^1}{f(x)}_\star&=&\alpha\left(x^1\d_3
    -2x^4\d_2\right)f(x)\\ \nonumber
  \cb{x^2}{f(x)}_\star&=&\alpha\left(x^2\d_3
    +2x^4\d_1\right)f(x)\\ \nonumber
  \cb{x^3}{f(x)}_\star&=&\alpha\left(x^2\d_2
    -x^1\d_1\right)f(x)\\ \nonumber
  \cb{x^4}{f(x)}_\star&=&0
\end{eqnarray}
When inserted into \eqref{eq:nw:int:condition2}, they give the restraints
\begin{eqnarray}
  \label{eq:weyl:mu:all}
  x^1\d_3\mu_\star(x)&=&2x^4\d_2\mu_\star(x)\\ \nonumber
  x^2\d_3\mu_\star(x)&=&-2x^4\d_1\mu_\star(x)\\ \nonumber
  x^1\d_1\mu_\star(x)&=&x^2\d_2\mu_\star(x)
\end{eqnarray}
satisfied by
\begin{equation}
  \label{eq:weyl:mu}
  \d_{1,2,3}\mu_\star(x)=0
\end{equation}

However, the $\d_4^\star$ derivative does not satisfy
\eqref{eq:nw:int:ah}, but it may be remedied by introducing the translation
$\d_4\rightarrow\d_4+\frac{\d_4\mu_\star(x)}{2\mu_\star(x)}$ so that the
$\star$-derivatives \eqref{eq:weyl:derivs} are now
\begin{eqnarray}
  \label{eq:weyl:d}
  \widetilde\d^\star_1&=&\Phi(-\alpha\d_3)\d_1\\ \nonumber
  \widetilde\d^\star_2&=&\Phi(\alpha\d_3)\d_2\\ \nonumber
  \widetilde\d^\star_3&=&\d_3\\ \nonumber
  \widetilde\d^\star_4&=&\d_4+\frac{\d_4\mu_\star(x)}{2\mu_\star(x)}
  +\left(2-\Phi(-\alpha\d_3)-\Phi(\alpha\d_3)\right)
  \frac{\d_1\d_2}{\d_3}
\end{eqnarray}
with no adverse effects on \eqref{eq:rho:nw4}.

\section{Free Scalar Field Theory}
We now have all the necessary tools in order to construct a free scalar field
theory using the action principle for the non-commuting Nappi-Witten space-time.
For our Lagrangian density, we use the familiar
\begin{equation}
  \label{eq:lagdens}
  {\cal L} = \frac 12 \d^\star_\mu\phi\star\d^\star_\mu\phi
  + \frac 12 m^2\phi\star\phi
\end{equation}
where we have replaced multiplication by $\star$-multiplication, and derivatives
by $\star$-derivatives. Using \eqref{eq:nw:int:cyc:nostar}, the Lagrangian under
integration ${\cal L}_\star$ simplifies to
\begin{equation}
  \label{eq:lagdens:simp}
  {\cal L}_\star = \frac 12 (\d^\star_\mu\phi)^2 + \frac 12 m^2\phi^2
\end{equation}
The action is defined with an appropriate $\mu_\star(x)$
\begin{eqnarray}
  \label{eq:action}
  S[\phi] &=& \int d^4x\mu_\star(x){\cal L}_\star \\ \nonumber
  &=& \frac 12 \int d^4x\mu_\star(x)\left\{(\d^\star_\mu\phi)^2 +
    m^2\phi^2\right\}
\end{eqnarray}
We should be able to recover a familiar $\star$-Klein-Gordon equation by finding
the extremum of $S$ by demanding the functional derivative be zero
\begin{equation}
  \label{eq:action:zero}
  \frac{\delta S[\phi]}{\delta\phi}=0
\end{equation}
We define the functional derivative by
\begin{equation}
  \label{eq:action:funderiv}
  S[\phi,\delta\phi]-S[\phi]\equiv \frac{\delta S[\phi]}{\delta\phi}\delta\phi
\end{equation}
With this definition, we get to first order in variation
\begin{eqnarray}
  \nonumber
  S[\phi,\delta\phi]-S[\phi]
  &=&\frac 12 \int d^4x\mu_\star(x)\left\{
    \left(\d^\star_\mu(\phi+\delta\phi)\right)^2 + m^2(\phi+\delta\phi)^2
    \right.\\ \nonumber
    &&\qquad\qquad\qquad\quad -(\d^\star_\mu\phi)^2 - m^2\phi^2
  \Big\} \\ \nonumber
  &=&\int d^4x\mu_\star(x)\left\{
    \d^\star_\mu\phi\d^\star_\mu\delta\phi+m^2\phi\delta\phi
  \right\} \\
  \label{eq:action:min}
  &=&-\int d^4x\mu_\star(x)\left\{
    \overline{\d^\star_\mu}\d^\star_\mu\phi - m^2\phi
  \right\}\delta\phi
\end{eqnarray}
In the final line we make use of the anti-hermitian property
\eqref{eq:nw:int:ah} which is integration by parts when the boundary terms are
zero.

This allows us to identify
\begin{eqnarray}
  \label{eq:nw:starminaction}
  \frac{\delta S[\phi]}{\delta\phi}&=&0 \\ \nonumber
  &=& \overline{\d^\star_\mu}\d^\star_\mu\phi - m^2\phi
\end{eqnarray}
and therefore define\footnote{Note that this differs from the definition of
  $\Box^\star$ in the $\kappa$-Minkowski literature \cite{Dimitrijevic:2003wv}.
  Using this definition we obtain
  \begin{equation*}
    \Box^\star_\kappa\phi = \frac{2}{\lambda^2\d_n^2}
    \left(1-\cos(\lambda\d_n)\right)\d_i^2\phi + \d_n^2\phi
  \end{equation*}
  whereas the literature obtains the similar
  \begin{equation*}
    \Box^\star_\kappa\phi = \frac{2}{\lambda^2\d_n^2}
    \left(1-\cos(\lambda\d_n)\right)\left(\d_i^2+\d_n^2\right)\phi
  \end{equation*}
  as a result of defining the box operator in terms of the invariant ``Dirac''
  derivative. The standard box operator is recovered for both definitions in the
  commutative geometry $\lambda\rightarrow 0$ limit.}
\begin{equation}
  \label{eq:nw:kg}
  \Box^\star\phi\equiv\overline{\d^\star_\mu}\d^\star_\mu\phi
\end{equation}

We may now explicitly calculate the box operator for all three of our orderings,
where, for brevity, we use $\d_\star$ to mean $\widetilde{\d_\star}$. \nts{this
  could be fixed by using a better notation earlier on.}

\subsection{Time Ordering}
Using \eqref{eq:time:d}, and making the simplificion
\begin{equation}
  \label{eq:time:eta}
  \eta_\ast(x) = \frac{\d_4\mu_\ast(x)}{2\mu_\ast(x)}
\end{equation}
we can write out the box operator
\begin{equation}
  \label{eq:time:box}
  \Box^\ast\phi = \d_\mu^2\phi +(\overline{\eta_\ast(x)} + \eta_\ast(x))\d_4\phi
  + \phi\d_4\eta_\ast(x) + |\eta_\ast(x)|^2\phi
\end{equation}

\subsubsection*{Example: $\mu_\ast(x)=1$}
\begin{equation}
  \label{eq:time:box:unitmeasure}
  \Box^\ast\phi = \d_\mu^2\phi
\end{equation}

%\subsubsection{Example: $\mu_\ast(x)=\sqrt{\det(g)}$}
%\nts{}

\subsection{Symmetric Time Ordering}
Using \eqref{eq:symtime:d}, and making the simplificion
\begin{equation}
  \label{eq:symtime:eta}
  \eta_\bullet(x) = \frac{\d_4\mu_\bullet(x)}{2\mu_\bullet(x)}
\end{equation}
we can write out the box operator for $\alpha\in \mathbb R$
\begin{eqnarray}
  \label{eq:symtime:box:1}
  \Box^\bullet_{\alpha}\phi &=& e^{-\alpha\d_3}\d_1^2\phi
  + e^{\alpha\d_3}\d_2^2\phi + \d_3^2\phi + \d_4^2\phi \\ \nonumber
  &&+(\overline{\eta_\bullet(x)} + \eta_\bullet(x))\d_4\phi
  + \phi\d_4\eta_\bullet(x) + |\eta_\bullet(x)|^2\phi
\end{eqnarray}
and for the Wick rotated $\alpha \rightarrow i\alpha$, $\alpha\in \mathbb R$
\begin{equation}
  \label{eq:symtime:box:2}
  \Box^\bullet_{i\alpha}\phi = \d_\mu^2\phi
  +(\overline{\eta_\bullet(x)} + \eta_\bullet(x))\d_4\phi
  + \phi\d_4\eta_\bullet(x) + |\eta_\bullet(x)|^2\phi
\end{equation}

\subsubsection*{Example: $\mu_\bullet(x)=1$}
For $\alpha\in \mathbb R$
\begin{equation}
  \label{eq:symtime:box:1:unitmeasure}
  \Box^\bullet_{\alpha}\phi = e^{-\alpha\d_3}\d_1^2\phi
  + e^{\alpha\d_3}\d_2^2\phi + \d_3^2\phi + \d_4^2\phi
\end{equation}
and for the Wick rotated $\alpha \rightarrow i\alpha$, $\alpha\in \mathbb R$
\begin{equation}
  \label{eq:symtime:box:2:unitmeasure}
  \Box^\bullet_{i\alpha}\phi = \d_\mu^2\phi
\end{equation}
The standard box operator is recovered in the commutative geometry
$\alpha\rightarrow 0$ limit.

%\subsubsection{Example: $\mu_\bullet(x)=\sqrt{\det(g)}$}
%\nts{}

\subsection{Weyl Ordering}
Using \eqref{eq:weyl:d}, and making the simplificions
\begin{eqnarray}
  \label{eq:weyl:etasigma}
  \eta_\star(x) &=& \frac{\d_4\mu_\star(x)}{2\mu_\star(x)}\\ \nonumber
  \d_\triangle &=& \left(2-\Phi(-\alpha\d_3)-\Phi(\alpha\d_3)\right)
  \frac{\d_1\d_2}{\d_3}
\end{eqnarray}
we can write out the box operator for $\alpha\in \mathbb R$
\begin{eqnarray}
  \label{eq:weyl:box:1}
  \Box^\star_{\alpha}\phi &=&
    \frac{2}{\alpha^2\d_3^2}\left(1-\cos(\alpha\d_3)\right)
    \left(e^{-\alpha\d_3}\d_1^2\phi + e^{\alpha\d_3}\d_2^2\phi\right)\\ \nonumber
  &&  + \d_3^2\phi + \d_4^2\phi \\ \nonumber
  && + \left(2\d_\triangle + \eta_\star(x) +
    \overline{\eta_\star(x)}\right)\d_4\phi
  + |\eta_\star(x)|^2\phi + \phi\d_4\eta_\star(x)\\ \nonumber
  && + (\eta_\star(x) + \overline{\eta_\star(x)})\d_\triangle\phi +
  (\d_\triangle)^2\phi
\end{eqnarray}
and for the Wick rotated $\alpha \rightarrow i\alpha$, $\alpha\in \mathbb R$
\begin{eqnarray}
  \label{eq:weyl:box:2}
  \Box^\star_{i\alpha}\phi &=&
  \frac{2}{\alpha^2\d_3^2}\left(1-\cos(\alpha\d_3)\right)
  \left(\d_1^2\phi + \d_2^2\phi\right) + \d_3^2\phi + \d_4^2\phi \\ \nonumber
  && + \left(\d_\triangle + \overline{\d_\triangle} + \eta_\star(x)
    + \overline{\eta_\star(x)}\right)\d_4\phi
  + |\eta_\star(x)|^2\phi + \phi\d_4\eta_\star(x)\\ \nonumber
  && + (\eta_\star(x)\overline{\d_\triangle} +
  \overline{\eta_\star(x)}\d_\triangle)\phi
  + \d_\triangle\overline{\d_\triangle}\phi
\end{eqnarray}
which is hard to solve, again due to $\frac{1}{\d_3}$ terms.

\subsubsection*{Example: $\mu_\star(x)=1$}
For $\alpha\in \mathbb R$
\begin{eqnarray}
  \label{eq:weyl:box:1:unitmeasure}
  \Box^\star_{\alpha}\phi &=& \Phi(-\alpha\d_3)^2\d_1^2\phi
  + \Phi(\alpha\d_3)^2\d_2^2\phi + \d_3^2\phi + \d_4^2\phi \\ \nonumber
  &&+ 2\d_\triangle\d_4\phi
  + (\d_\triangle)^2\phi
\end{eqnarray}
and for the Wick rotated $\alpha \rightarrow i\alpha$, $\alpha\in \mathbb R$
\begin{eqnarray}
  \label{eq:weyl:box:2:unitmeasure}
  \Box^\star_{i\alpha}\phi &=&
  \frac{2}{\alpha^2\d_3^2}\left(1-\cos(\alpha\d_3)\right)
  \left(\d_1^2\phi + \d_2^2\phi\right) + \d_3^2\phi + \d_4^2\phi \\ \nonumber
  && + \left(\d_\triangle + \overline{\d_\triangle}\right)\d_4\phi
  + \d_\triangle\overline{\d_\triangle}\phi
\end{eqnarray}
The standard box operator is recovered in the commutative geometry
$\alpha\rightarrow 0$ limit.

%\subsubsection{Example: $\mu_\star(x)=\sqrt{\det(g)}$}
%\nts{}

% \section{Free Spinor Field Theory}
% We need the Dirac derivative to do this... need to look at the
% NW-deformed Lorentz group to define it. i.e. another paper of work

% \section{Interacting $\phi^4$ Theory}
% We need to know how to calculate $e^{iq_ix^i}\star e^{iq_i'x^i}$ to be able to
% do this

\clearpage
\appendix
\section{Generating Function of Equation \eqref{eq:weyl:G4}}
\label{app:generating}
\begin{eqnarray}
  \label{eq:app:1}
  \sum_{n=1}\frac{B_{n+1}
    y^n}{n!}&=&\sum_{n=1}\frac{(n+1)B_{n+1}}{(n+1)!}\frac{y^{n+1}}{y}
  \\ \nonumber
  &=&\frac 1y\sum_{m=2}\frac{mB_m}{m!}y^m\\ \nonumber
  &=&\frac{d}{dy}\sum_{m=2}\frac{B_m}{m!}y^m
\end{eqnarray}
where $m=n+1$. Since
\begin{eqnarray}
  \label{eq:app:2}
  \sum_{m=2}\frac{B_m}{m!}y^m&=&-B_0-B_1y+\sum_{m=0}\frac{B_m}{m!}y^m \\ \nonumber
  &=&-B_0-B_1y+\frac{y}{e^{xy}-1}\\ \nonumber
  &=&\frac{y}{e^{xy}-1}+\frac y2 -1
\end{eqnarray}
around $x=1$ and $B_0=1$, $B_1=-\frac 12$, we can calculate the generating
function of the original sum
\begin{eqnarray}
  \label{eq:app:3}
  \sum_{n=1}\frac{B_{n+1} y^n}{n!}&=&\frac 12 +
  \frac{d}{dy}\left(\frac{y}{e^{xy}-1}\right) \\ \nonumber
  &=&\frac 12+\frac{e^{xy}-1-ye^{xy}}{\left(e^{xy}-1\right)^2}  
\end{eqnarray}

\section{$\kappa$-Minkowski Weyl Ordered $\star$-Product}
\label{app:kappaminsk}
A $\kappa$-Minkowski algebra is one that satisfies
\begin{eqnarray}
  \label{eq:app:kmink}
  \cb{\hat x_0}{\hat x_i}&=&\lambda\hat x_i\\ \nonumber
  \cb{\hat x_i}{\hat x_j}&=&0
\end{eqnarray}
Such an algebra can be thought of as a simplified Nappi-Witten algebra
\eqref{eq:NW:algebra} with $\hat\K=0$, $\hat x_0$ taking the role of $\hat\J$
and the $\hat x_i$ acting as independent $\hat\P^-$s. The only non-trivial
exponential products for an $n$-dimensional algebra with a representation
similar to the Nappi-Witten scenario in the main text are
\begin{eqnarray}
  \label{eq:app:kmink:mult}
  e^{ik_l\hat x^l}e^{ik_0\hat x^0}&=&e^{ik_0\hat x^0}e^{ik_le^{-i\beta\lambda
      k_0}\hat x^l} \\ \nonumber
  e^{ik_0\hat x^0}e^{ik_l\hat x^l}&=&e^{ik_le^{i\beta\lambda k_0}\hat
    x^n}e^{ik_0\hat x^0}
\end{eqnarray}
In analogy to \eqref{eq:weyl:start} we find the BCH formula for
\begin{eqnarray}
  \label{eq:app:kmink:start}
  e^{ik_l\hat x^l}e^{ik_0\hat x^0}&=&e^{G_0\hat x^0+G_l\hat x^l} \\ \nonumber
  G_0&=&ik_0 \\ \nonumber
  G_l&=&\frac{ik_l}{\Phi(i\lambda k_0)}
\end{eqnarray}
where we use the same $\Phi$ as in \eqref{eq:weyl:Phi}
\begin{equation*}
  \Phi(a)=\frac{e^{\beta a}-1}a
\end{equation*}
and define
\begin{eqnarray}
  \label{eq:app:kmink:qs}
  q_0&=&k_0 \\ \nonumber
  q_l&=&k_l\Phi(i\lambda k_0)
\end{eqnarray}
allowing us to rewrite the Weyl ordering in terms of the $q$s
\begin{equation}
  \label{eq:app:kmink:ordering}
  \NO e^{ik_l\hat x^i}\NO=e^{ik_0\hat x^0+ik_l\hat x^l}=e^{iq_l\hat x^l}e^{iq_0\hat x^0}
\end{equation}
Using \eqref{eq:app:kmink:mult},\eqref{eq:app:kmink:start}, we calculate the Weyl
ordered $\star$-product for $\kappa$-Minkowski as
\begin{eqnarray}
  \label{eq:app:kmink:star}
  f\star g(z)&=&f(z)e^{ix^i(F_i-k_i-k_i')}g(z)\\ \nonumber
  F_0&=&k_0+k_0'\\ \nonumber
  F_l&=&\frac{k_l\Phi(i\lambda k_0)+k_l'\Phi(i\lambda k_0')
    e^{i\lambda k_0}}{\Phi(i\lambda(k_0+k_0'))}
\end{eqnarray}
where $l$ runs over $1,\ldots,n-1$. This result may be compared with
\cite{Agostini:2002}. We see from \eqref{eq:example:star} and \eqref{eq:star:lie}
that the operator algebra is recovered in the functional space with deformed
$\star$-product between the generators being
\begin{eqnarray}
  \label{eq:app:kmink:generators}
  x_l\star x_0&=&x_lx_0-\frac\lambda2x_l\\ \nonumber
  x_0\star x_l&=&x_0x_l+\frac \lambda2x_l
\end{eqnarray}
Using \eqref{eq:star:lie} to second order in
$\lambda$\footnote{$\kappa$-Minkowski is by definition a scaled algebra, so
  there is no need to introduce a parameter such as $\alpha$} and written in
position space, the $\star$-product between any two functions for $n=3$ is
\begin{eqnarray}
  \label{eq:app:kmink:position}
  \nonumber
  f\star g(x)=fg&+&\frac{\lambda}{2}\Biggl(
  x_1\d_0f\d_1g
  -x_1\d_1f\d_0g
  +x_2\d_0f\d_2g
  -x_2\d_2f\d_0g
  \Biggr) \\ \nonumber
  &+&\frac{\lambda^2}{4}\Biggl(
  \frac 13x_1\d_1f\d_0^2g
  +\frac 13x_2\d_0^2f\d_2g
  +\frac 13x_2\d_2f\d_0^2g\\ \nonumber &&\qquad
  -x_1^2\d_0\d_1f\d_0\d_1g
  -\frac 13x_1\d_0f\d_0\d_1g
  -\frac 13x_2\d_0\d_2f\d_0g\\ \nonumber &&\qquad
  +\frac 12x_2^2\d_0^2f\d_2^2g
  +x_1x_2\d_0^2f\d_1\d_2g
  -x_1x_2\d_0\d_1f\d_0\d_2g\\ \nonumber &&\qquad
  +\frac 12x_2^2\d_2^2f\d_0^2g
  +\frac 12x_1^2\d_0^2f\d_1^2g
  -\frac 13x_1\d_0\d_1f\d_0g\\ \nonumber &&\qquad
  +\frac 12x_1^2\d_1^2f\d_0^2g
  -x_2^2\d_0\d_2f\d_0\d_2g
  +x_1x_2\d_1\d_2f\d_0^2g\\ \nonumber &&\qquad
  +\frac 13x_1\d_0^2f\d_1g
  -\frac 13x_2\d_0f\d_0\d_2g
  -x_1x_2\d_0\d_2f\d_0\d_1g
  \Biggr) \\
  &+&\mathcal{O}(\lambda^3)
\end{eqnarray}
In this proof, we have used a real Lie algebra \eqref{eq:app:kmink}, as opposed
to the anti-Hermitian definition often used. It should be noted that in order to
directly compare this $\star$-product with those given in
\cite{Dimitrijevic:2003wv},\cite{Dimitrijevic:2004vv} then the definition of the
algebra should be changed (using $\lambda\rightarrow i\Lambda$) to
\begin{eqnarray}
  \label{eq:app:kmink:alt}
  \cb{\hat x_0}{\hat x_i}&=&i\Lambda\hat x_i\\ \nonumber
  \cb{\hat x_i}{\hat x_j}&=&0
\end{eqnarray}
giving
\begin{eqnarray}
  \label{eq:app:kmink:star:alt}
  f\star g(z)&=&f(z)e^{ix^i(F_i-k_i-k_i')}g(z)\\ \nonumber
  F_0&=&k_0+k_0'\\ \nonumber
  F_l&=&\frac{k_l\Phi(-\Lambda k_0)+k_l'\Phi(-\Lambda k_0')
    e^{-\Lambda  k_0}}{\Phi(-\Lambda(k_0+k_0'))}
\end{eqnarray}

\section{A ``Not so Obvious'' Mistake}
A mistake which is not so obvious to spot is in assuming
\begin{equation*}
  e^{i\hat A}e^{i\hat B}=e^{iC(\hat A:\hat B)}
\end{equation*}
To show why this is not the case, a simple counterexample is used. For an
algebra $\cb{A}{B}=c$, we would have
\begin{eqnarray*}
  e^{i\hat A}e^{i\hat B}&=&e^{i\hat A+i\hat B+\frac 12\cb{i\hat A}{i\hat B}} \\
  &=&e^{i\hat A+i\hat B-\frac c2}
\end{eqnarray*}
whereas
\begin{equation*}
  e^{iC(\hat A:\hat B)}=e^{i(\hat A+\hat B+\frac c2)}
\end{equation*}
which is not equivalent.

\section{Some Non-Trivial Derivative Rules}
we note that
\begin{eqnarray*}
  \d_0^nx^0\d_1f &=& \d_0^{n-1}(\d_1f+x^0\d_0\d_1f) \\
  &=& \d_0^{n-1}\d_1f+\d_0^{n-2}(\d_0\d_1f+x^0\d_0^2\d_1f)\\
  && \vdots \\
  &=& n\d_0^{n-1}\d_1f+x^0\d_0^n\d_1f
\end{eqnarray*}
allowing us to construct the following rules
\begin{eqnarray*}
  e^{\alpha\d_0}x^0\d_1f &=& \left(\sum_{n=0}^\infty\frac{(\alpha\d_0)^n}{n!}
  \right)x^0\d_1f \\
  &=&\left(1+\sum_{n=1}^\infty\frac{(\alpha\d_0)^n}{n!}\right)x^0\d_1f \\
  &=&x^0\d_1f+\sum_{n=1}^\infty\frac 1{n!}\left(
    n\alpha^n\d_0^{n-1}\d_1f + x^0(\alpha\d_0)^n\d_1f\right)\\
  &=&\left(\alpha+x^0\right)e^{\alpha\d_0}\d_1f
\end{eqnarray*}
\begin{eqnarray*}
  \Phi(\alpha\d_0)x^0\d_1f &=&
  \left(\frac{e^{\alpha\d_0}-1}{\alpha\d_0}\right)x^0\d_1f \\
  &=& \left(\sum_{n=1}^\infty\frac{(\alpha\d_0)^{n-1}}{n!}\right)x^0\d_1f \\
  &=& \sum_{n=1}^\infty\frac 1{n!}\left(
    (n-1)\alpha^{n-1}\d_0^{n-2}+x^0\alpha^{n-1}\d_0^{n-1}\right)\d_1f \\
  &=& \left\{\frac{e^{\alpha\d_0}}{\d_0}-
    \frac{\Phi(\alpha\d_0)}{\d_0}+x^0\Phi(\alpha\d_0)\right\}\d_1f
\end{eqnarray*}

\section{Some Helpful Identities}
\begin{eqnarray*}
  \Phi(a)e^{-a}&=&\Phi(-a) \\
  \Phi(ia)\Phi(-ia)&=&\frac{2}{a^2}\left(1-\cos(a)\right)
\end{eqnarray*}

\begin{thebibliography}{99}
\bibitem{Nappi:1993ie}
  C.~R.~Nappi and E.~Witten,
  ``A WZW model based on a nonsemisimple group,''
  Phys.\ Rev.\ Lett.\  {\bf 71} (1993) 3751
  [arXiv:hep-th/9310112].
  %%CITATION = HEP-TH 9310112;%%

\bibitem{Kathotia:1998}
  V.~Kathotia,
  ``Kontsevichs Universal Formula for Deformation Quantization and the
  Campbell-Baker-Hausdorff Formula, I''
  UC Davis Math 1998-16
  [arXiv:math.QA/9811174].
  %%CITATION = MATH-QA 9811174;%%

\bibitem{Figueroa-OFarrill:1999ie}
  J.~M.~Figueroa-O'Farrill and S.~Stanciu,
  ``More D-branes in the Nappi-Witten background,''
  JHEP {\bf 0001} (2000) 024
  [arXiv:hep-th/9909164].
  %%CITATION = HEP-TH 9909164;%%

\bibitem{DAppollonio:2003dr}
  G.~D'Appollonio and E.~Kiritsis,
  ``String interactions in gravitational wave backgrounds,''
  Nucl.\ Phys.\ B {\bf 674} (2003) 80
  [arXiv:hep-th/0305081].
  %%CITATION = HEP-TH 0305081;%%

\bibitem{Agostini:2002}
  A.~Agostini, F.~Lizzi, A.~Zampini,
  ``Generalized Weyl Systems and Kappa Minkowski Space''
  Mod.\ Phys.\ Lett.\ {\bf A17} (2002) 2105-2126
  [arXiv:hep-th/0209174].
  %%CITATION = HEP-TH 0209174;%%

\bibitem{Dimitrijevic:2003wv}
  M.~Dimitrijevic, L.~Jonke, L.~Moller, E.~Tsouchnika, J.~Wess and
  M.~Wohlgenannt,
  ``Deformed field theory on kappa-spacetime,''
  Eur.\ Phys.\ J.\ C {\bf 31} (2003) 129
  [arXiv:hep-th/0307149].
  %%CITATION = HEP-TH 0307149;%%

\bibitem{Dimitrijevic:2004vv}
  M.~Dimitrijevic, L.~Moller and E.~Tsouchnika,
  ``Derivatives, forms and vector fields on the kappa-deformed Euclidean space,''
  [arXiv:hep-th/0404224].
  %%CITATION = HEP-TH 0404224;%%

\bibitem{Agostini:2004cu}
  A.~Agostini, G.~Amelino-Camelia, M.~Arzano and F.~D'Andrea,
  ``Action functional for kappa-Minkowski noncommutative spacetime,''
  [arXiv:hep-th/0407227].
  %%CITATION = HEP-TH 0407227;%%

\bibitem{Dietz:2001}
  M.~Dietz,
  ``Symmetrische Formen auf Quantenalgebren'',
  Diploma thesis at the University of Hamburg (2001).

\end{thebibliography}
\end{document}
      
% This is for Emacs
%%% Local Variables:
%%% TeX-command-default: "LaTeX"
%%% TeX-command-Show: "View"
%%% End:
