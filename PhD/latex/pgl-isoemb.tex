\documentclass[11pt, a4paper, titlepage]{article}
\usepackage{amsmath,amsfonts,amssymb,amsthm,amstext,amscd,array,bbold}
%\usepackage[latin1]{inputenc}
\DeclareMathOperator{\AdS}{AdS}
\DeclareMathOperator{\Sphere}{S}
\DeclareMathOperator{\NW}{NW}
\DeclareMathOperator{\CW}{CW}
\let\S\Sphere

\begin{document}
\section{Penrose-G\"{u}ven Limits and Isometric Embeddings}
\subsection{The Penrose-G\"{u}ven Plane Wave Limit}
In \cite{pl}, Penrose showed that any Lorentzian spacetime has a limiting
spacetime which is a plane wave. This limit can be thought of as a ``first order
approximation'' along a null geodesic. The limiting spacetime depends on the
choice of null geodesic, and hence any spacetime can have more than one Penrose
limit. More recently, G\"{u}ven extended Penrose's argument to show that any
solution of a supergravity theory has plane wave limits which are also
solutions \cite{guven}.

Let $(M,G)$ be a $D$-dimensional Lorentzian spacetime. By \cite{penrose} we can
always introduce local coordinates $(U,V,Y^i)$ in the neighbourhood of a portion
of a null geodesic $\gamma$ which contains no conjugate points. Such a metric
takes the form
\begin{equation}
  \label{eq:bpg:initial:G}
  G = \left(dU + \alpha dV + \beta_i dY^i \right)dV + C_{ij} dY^i dY^j
\end{equation}
where $\alpha$, $\beta_i$ and $C_{ij}$ are functions of the coordinates and
$C_{ij}$ is a symmetric positive-definite matrix, $i,j=1,2,\ldots,D-2$. The
coordinates break down when $\det{C}=0$, signalling the existence of a conjugate
point. This coordinate system has the advantage that a null geodesic congruence
is singled out by $V,Y^i\rightarrow$ constant, with $U$ being an affine
parameter along these geodesics. The geodesic $\gamma$ is the one for which
$V=Y^i=0$.

In supergravity theories there are also general $p$-form potentials $A$ with
$(p+1)$-form field strengths $F=dA$. The potentials are defined up to gauge
transformations $A\mapsto A+d\Lambda$ in such a way that $F$ is gauge invariant.
G\"{u}ven extended the Penrose limit to supergravity theories by noting that in
order to ensure well defined potentials in the target space, a gauge must be
chosen for the potentials such that
\begin{equation}
  \label{eq:bpg:guven}
  A_{U(i_1\ldots i_{p-1})}=0
\end{equation}
By imposing this condition on $A$, we can now write general potential and
associated field strengths on $M$
\begin{eqnarray}
  \label{eq:bpg:initial:A}
  A &=& a_{i_1 \ldots i_p} dV\wedge dY^{i_1}\wedge\ldots\wedge
  dY^{i_{p-1}} \\ \nonumber
  &&+b_{i_1\ldots i_p} dY^{i_1}\wedge\ldots\wedge dY^{i_p}\\ \nonumber
  &&+c_{i_1\ldots i_p} dU\wedge dV\wedge dY^{i_1}\wedge\ldots\wedge
  dY^{i_{p-2}} \\
  \label{eq:bpg:initial:F}
  F &=& \left( \frac{\partial b_{i_1 \ldots i_{p+1}}}{\partial U}
  \right) dU\wedge dY^{i_1}\wedge\ldots\wedge dY^{i_{p}} \\ \nonumber
  &&+ d_{i_1 \ldots i_{p+1}} dY^{i_1}\wedge\ldots\wedge dY^{i_{p+1}} \\ \nonumber
  &&+ e_{i_1 \ldots i_{p+1}} dU\wedge dV\wedge dY^{i_1}\wedge\ldots\wedge
  dY^{i_{p-1}} \\ \nonumber
  &&+ f_{i_1 \ldots i_{p+1}} dV\wedge dY^{i_1}\wedge\ldots\wedge dY^{i_{p}}  
\end{eqnarray}
Where $a$, $b$, $c$, $d$, $e$ and $f$ are functions of the coordinates. The
Penrose limit is in practise the rescaling of the coordinates
\begin{equation}
  \label{eq:bpg:rescale}
  U\rightarrow u\qquad V\rightarrow\Omega^2v\qquad Y^i\rightarrow\Omega y^i
\end{equation}
where $\Omega\rightarrow 0$. Let us define the new fields on the target space as
\begin{eqnarray}
  \label{eq:bpg:tildefields}
  \widetilde{G}=\lim_{\Omega\rightarrow 0} \Omega^{-2}\psi^*G \\ \nonumber
  \widetilde{A}=\lim_{\Omega\rightarrow 0} \Omega^{-p}\psi^*A \\ \nonumber
  \widetilde{F}=\lim_{\Omega\rightarrow 0} \Omega^{-p}\psi^*F
\end{eqnarray}
where $\psi^*$ denotes the local diffeomorphism defined by
\eqref{eq:bpg:rescale}. These new fields are related to the original ones by a
diffeomorphism, a rescaling and (in the case of the potentials) possibly a gauge
transformation. We obtain the metric and fields in Rosen \cite{rosen}
coordinates $(u,v,y^i)$
\begin{eqnarray}
  \label{eq:bpg:target:G}
  \widetilde{G}&=& dudv + C(u)_{ij}dy^i dy^j \\
  \label{eq:bpg:target:A}
  \widetilde{A}&=& b(u)_{i_1\ldots i_p}dy^{i_1}
  \wedge\ldots\wedge dy^{i_p} \\ \nonumber
  &&+c(u)_{i_1\ldots i_p} du\wedge
  dv\wedge dy^{i_1}\wedge\ldots\wedge dy^{i_{p-2}}\\
  \label{eq:bpg:target:F}
  \widetilde{F}&=&\frac{\partial b(u)_{i_1\ldots i_{p+1}}}{\partial u}
  du\wedge dy^{i_1}\wedge\ldots\wedge dy^{i_p}
\end{eqnarray}
Due to \eqref{eq:bpg:rescale}, we can see that $C(u)_{ij}=C(U,0,0)_{ij}$ which
is just the value of $C$ along the geodesic $\gamma$. This is also true of
$b(u)$ and $c(u)$.

\subsection{Isometric Embedding Diagrams}
\label{bpg:ie}
If we wish to generate a commutative isometric embedding diagram, whereby the
embeddings of $N$'s into the $M$'s are denoted by vertical arrows and
Penrose-G\"{u}ven limits (PGL) by horizontal arrows, we would have the following
picture:
\begin{equation}
  \label{eq:bpg:ie:generic}
  \begin{CD}
    @.\\
    M @>\text{PGL}>>                      \widetilde{M}\\
    \text{$\imath$}@AAA @AAA\text{$\widetilde{\imath}$}\\
    N @>\text{PGL}>>                      \widetilde{N}\\
    @.
  \end{CD}
\end{equation}
In order to ensure that the metric and fields commute in such a diagram (i.e.
that $M\rightarrow \widetilde{M} \rightarrow \widetilde{N}$ yields the same
result as $M\rightarrow N\rightarrow\widetilde{N}$), we must place some
restrictions on the kind of projections $\imath^*$, $\widetilde{\imath}^*$ we
can use. If we choose to set $m$ coordinates to zero\footnote{We require the
  coordinates to go to zero in order that the PGL is along the same null
  geodesic in each case, but requiring the coordinates to go constant rather
  than zero is often sufficient.} in each projection, then we can define
\begin{eqnarray}
  \label{eq:bpg:ie:i1}
  \imath^*&:& Y^{q_1},\ldots,Y^{q_m} \rightarrow 0\\
  \label{eq:bpg:ie:i2}
  \widetilde{\imath}^*&:& y^{r_1},\ldots,y^{r_m} \rightarrow 0
\end{eqnarray}
If $I=\{1,2,\ldots,D-2\}$ is the set of all indexes and the sets of indexes of
the coordinates set to zero are $Q=\{q_1,\ldots,q_m\}$ and
$R=\{r_1,\ldots,r_m\}$, then $Q,R\subset I$. We can now define two new sets of
indexes, which are the indexes of the coordinates which are not set to zero;
$S=I\backslash Q=\{s_1,\ldots,s_{D-2-m}\}$ and $T=I\backslash
R=\{t_1,\ldots,t_{D-2-m}\}$.

The following condition is required for the metric to be commutative in such a
diagram
\begin{equation}
  \label{eq:bpg:ie:metriccond}
  C(U,0,0)_{is} = C(U,0,0)_{i\pi(t)}
\end{equation}
where $\pi(t)$ is any permutation of the indexes $t$. A similar condition exists
for field strengths
\begin{equation}
  \label{eq:bpg:ie:fieldcond}
  b(U,0,0)_{(i_1\ldots i_{p+1})s} = b(U,0,0)_{(i_1\ldots i_{p+1})\pi(t)}
\end{equation}
For potentials we require the condition \eqref{eq:bpg:ie:fieldcond} (with
indexes up to $i_p$ rather than $i_{p+1}$) and also a second condition
\begin{equation}
  \label{eq:bpg:ie:potcond}
  c(U,0,0)_{(i_1\ldots i_{p})s} = c(U,0,0)_{(i_1\ldots i_{p})\pi(t)}
\end{equation}
It is the particular pairing of $S$ and $T$ which we must choose carefully in
order to satisfy \eqref{eq:bpg:ie:metriccond}, \eqref{eq:bpg:ie:fieldcond} and
\eqref{eq:bpg:ie:potcond}. It may even be possible that no such choice exists,
in which case such a commutative diagram cannot be constructed. If we wish the
metric and fields to both commute simultaneously, we require the pairing of $S$
and $T$ to be the same in all the conditions.

These conditions are essentially just a trivial statement that $\imath^*$ and
$\widetilde{\imath}^*$ must be equivalent.

\subsection{An Example}
\subsubsection{Broken Commutativity}
We will now show an example of a diagram such as \eqref{eq:bpg:ie:generic} which
does not commute
\begin{equation}
  \label{eq:bpg:ex:jose}
  \begin{CD}
    @.\\
    \AdS_3 \times \S^3    @>\text{PGL}>>          \NW_6\\
    \text{$\imath$}@AAA @AAA\text{$\widetilde{\imath}$}\\
    \AdS_2 \times \S^2    @>\text{PGL}>>          \NW_4\\
    @.
  \end{CD}
\end{equation}
This diagram was originally studied in \cite{plsbd, pllbnwb} and was conjectured
to be commutative. As we will now show, the metric is commutative but the fields
are not.

The restrictions of $\AdS_3$ and $\S^3$ in $\mathbb{R}^4$ are separately given by
\begin{eqnarray}
  \label{eq:bpg:ex:jose:ads3}
  \left(X^0\right)^2+\left(X^1\right)^2-\left(X^2\right)^2-\left(X^3\right)^2&=&R\\
  \label{eq:bpg:ex:jose:s3}
  \left(X^4\right)^2+\left(X^5\right)^2+\left(X^6\right)^2+\left(X^7\right)^2&=&R
\end{eqnarray}
We choose to keep the radii $R$ the same for each space with the explicit
parameterisation
\begin{eqnarray}
  \label{eq:bpg:ex:jose:param}
  X^0&=\sqrt{1+r^2}\sin{\tau}    \qquad &X^1=\cos{\tau}\cosh{\beta} \\ \nonumber
  X^2&=\cos{\tau}\sinh{\beta}    \qquad &X^3=r\sin{\tau} \\ \nonumber
  X^4&=\cos{\phi}\sin{\theta}    \qquad &X^5=\chi\sin{\phi} \\ \nonumber
  X^6&=\sqrt{1-\chi^2}\sin{\phi} \qquad &X^7=\cos{\phi}\cos{\theta}
\end{eqnarray}
Using these coordinates, we may write the metric, NS-NS $B$ field and strength $H$
on $\AdS_3\times\S^3$ as
\begin{eqnarray}
  \label{eq:bpg:ex:jose:G}
  G &=& - d\tau^2 + \sin^2{\tau}\left( \frac{dr^2}{1+r^2} +
    \left(1+r^2\right)d\beta^2\right) \\ \nonumber
  &&+d\phi^2+\sin^2{\phi}\left(\frac{d\chi^2}{1-\chi^2}+
    \left(1-\chi^2\right)d\theta^2\right)\\
  \label{eq:bpg:ex:jose:H}
  -H/2 &=& \cos^2{\tau} d\tau\wedge dr\wedge d\beta + \sin^2{\phi}d\phi\wedge
  d\chi\wedge d\theta\\
  \label{eq:bpg:ex:jose:B}
  -2B &=& \left(\sin{2\tau}+2\tau\right)dr\wedge d\beta
  +\left(\sin{2\phi}-2\phi\right)d\chi\wedge d\theta
\end{eqnarray}
where $\AdS_3$ has coordinates $(\tau,r,\beta)$ and $\S^3$ has
$(\phi,\chi,\theta)$ so that $D=6$. Making a change of coordinates so that we
can write everything in the generalised form of \eqref{eq:bpg:initial:G},
\eqref{eq:bpg:initial:A} and \eqref{eq:bpg:initial:F}
\begin{eqnarray}
  \label{eq:bpg:ex:jose:changecoords}
  2\tau=U-V &\qquad& 2\phi=U+V \\ \nonumber
  r=Y^1\quad\beta=Y^2 &\qquad& \chi=Y^3\quad\theta=Y^4
\end{eqnarray}
we can now express the metric and fields along the geodesic $\gamma$ as
\begin{eqnarray}
  \label{eq:bpg:ex:jose:parts}
  C(u)_{ii} &=& \sin^2{\frac{u}{2}} \\ \nonumber
  b(u)_{12}&=&-\frac{\sin{u}+u}{2}\\ \nonumber
  b(u)_{34}&=&\frac{u-\sin{u}}{2}
\end{eqnarray}
with everything else zero. The projection $\imath^*$ consists of setting
$Y^q\rightarrow 0$ where $q=1,3$ and $\widetilde{\imath}$ as $y^r\rightarrow 0$
where $r=1,2$. We can now observe the sets $S=\{2,4\}$, $T=\{3,4\}$ and we see
clearly that there is no such pairing between $S$ and $T$ such that
\eqref{eq:bpg:ie:fieldcond} is satisfied, as it will result in requiring
$b(u)_{12}=b(u)_{1j}$ where $j$ is either $3$ or $4$. We can however see that
\eqref{eq:bpg:ie:metriccond} will always be satisfied since all the components
of $C(u)_{ii}$ are identical. So we expect the metric but not the fields to
commute in such a diagram.

We can confirm these predictions by following each path explicitly\footnote{It
  is easier to work in Brinkman \cite{brinkman} coordinates $(x^-,x^+,z^i)$ at
  this stage, we have made the transformation from Rosen coordinates by
  \begin{equation}
    \label{eq:ex:jose:brinkman}
    u=2x^- \qquad v=x^+ + \frac{\cot{x^-}}{2}\vec{z}^2 \qquad y^i=\frac{z^i}{\sin{x^-}}
  \end{equation}}. Firstly, the path
$\AdS_3\times\S^3\rightarrow\CW_6\rightarrow\CW_4$ yields
\begin{eqnarray}
  \label{eq:bpg:ex:jose:path1:G}
  \widetilde{G}&=&2dx^-dx^+ +\frac{\vec{z}^2}{4}\left(d
    x^-\right)^2 + \left(d\vec{z}\right)^2 \\
  \label{eq:bpg:ex:jose:path1:H}
  \widetilde{H}&=&-dx^-\wedge dz^1\wedge dz^2 \\
  \label{eq:bpg:ex:jose:path1:B}
  \widetilde{B}&=&-x^- dz^1\wedge dz^2
\end{eqnarray}
which is in fact, the Nappi-Witten WZW model \cite{nw}. The alternative path
$\AdS_3\times\S^3\rightarrow\AdS_2\times\S^2\rightarrow\CW_4$ yields
\begin{eqnarray}
  \label{eq:bpg:ex:jose:path2:G}
  \widetilde{G}&=&2dx^-dx^+ +\frac{\vec{z}^2}{4}\left(d
    x^-\right)^2 + \left(d\vec{z}\right)^2 \\
  \label{eq:bpg:ex:jose:path2:fields}
  \widetilde{H}&=&d\widetilde{B}=0
\end{eqnarray}
confirming the prediction that the metric commutes, but the fields do not.

\subsubsection{Working Commutativity}
We can now see why \eqref{eq:bpg:ex:jose} breaks down and we may attempt to
remedy it. The essential problem is in choosing a suitable $\imath^*$ projection
where we may pair up the elements of $S$ and $T$. The easiest and most obvious
way to fix this is to make the projection $\widetilde{\imath}^*$ equivalent to
$\imath^*$, i.e. so that we choose $\imath^*: Y^l\rightarrow 0$ where $l=1,2$.
Since the sets $S=\{3,4\}$ and $T=\{3,4\}$ are now identical, the commutative
conditions \eqref{eq:bpg:ie:metriccond}, \eqref{eq:bpg:ie:fieldcond},
\eqref{eq:bpg:ie:potcond} are trivially satisfied. This has however changed the
diagram somewhat, from \eqref{eq:bpg:ex:jose} into the following
\begin{equation}
  \label{eq:bpg:ex:fixed1}
  \begin{CD}
    @.\\
    \AdS_3 \times \S^3             @>\text{PGL}>> \NW_6\\
    \text{$\imath$}@AAA @AAA\text{$\widetilde{\imath}$}\\
    \mathbb{R}^1\times\S^3         @>\text{PGL}>> \NW_4\\
    @.
  \end{CD}
\end{equation}
The final space of each path is the $\NW_4$ model and both metric and fields
commute in the diagram. It is interesting to note that the PGL on the bottom of
the diagram has also been studied in \cite{kiritsis}, albeit with slight
technical differences from the PGL used here.

We could also have fixed \eqref{eq:bpg:ex:jose} by choosing the projection
$\widetilde{\imath}^*: y^l\rightarrow 0$ where $l=1,3$ so that it is equivalent
to $\imath$ and that the diagram will again trivially commute. Again this
involves a change in the diagram
\begin{equation}
  \label{eq:bpg:ex:fixed2}
  \begin{CD}
    @.\\
    \AdS_3 \times \S^3             @>\text{PGL}>> \NW_6\\
    \text{$\imath$}@AAA @AAA\text{$\widetilde{\imath}$}\\
    \AdS_2 \times \S^2             @>\text{PGL}>> \CW_4\\
    @.
  \end{CD}
\end{equation}
where $\CW_N$ is used to denote a Cahen-Wallach space \cite{cw} in $N$
dimensions. This diagram more closely resembles \eqref{eq:bpg:ex:jose}, since
$\NW_4$ is just $\CW_4$ at the level of the geometry, neglecting the
supergravity fields.

\begin{thebibliography}{100}
\bibitem{pl} R. Penrose, ``Any Space-Time has a Plane Wave as a Limit'',
  \textit{Differential Geometry and Relativity}, Reidel, Dordrecht (1976)
  271-275
\bibitem{guven} R. G\"{u}ven, ``Plane Wave Limits and T-Duality'', \textit{Phys.
    Lett.} \textbf{B482} (2000) 255-263, \textit{hep-th/0005061}
\bibitem{penrose} D. Lerner, R. Penrose, ``Technique of Differential Topology in
  Relativity'', \textit{SIAM}, Philadelphia (1971)
\bibitem{rosen} N. Rosen, \textit{Phys. Z. Sowjetunion} \textbf{12} (1937) 366
\bibitem{plsbd} M. Blau, J. Figueroa-O'Farrill, G. Papadopoulos, \textit{Class.
    Quant. Grav} \textbf{19} (2002) 4753, \textit{hep-th/0202111}
\bibitem{pllbnwb} S. Stanciu, J. Figueroa-O'Farrill, ``Penrose limits of Lie
  Branes and a Nappi-Witten braneworld'', \textit{JHEP} \textbf{0306} (2003)
  025, \textit{hep-th/0303212}
\bibitem{brinkman} H. W. Brinkman, \textit{Proc. Natl. Acad. Sci. (U.S.)}
  \textbf{9} (1923) 1
\bibitem{nw} C. R. Nappi, E. Witten, ``A WZW model based on a non-semi-simple
  group'', \textit{Phys. Rev.} \textbf{L71} (1993) 3751, \textit{hep-th/9310112}
\bibitem{kiritsis} G. D'Appollonio, E. Kiritsis, ``String interactions in
  gravitational wave backgrounds'', \textit{hep-th/0305081}
\bibitem{cw} M. Cahen, N. Wallach, ``Lorentzian Symmetric Spaces'',
  \textit{Bull. Am. Math. Soc.} \textbf{76} (1970) 585-591
\end{thebibliography}
\end{document}
