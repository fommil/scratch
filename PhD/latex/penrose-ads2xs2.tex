\documentclass[11pt, a4paper, titlepage]{article}
\usepackage{amsmath,amsfonts,amssymb,amsthm,amstext,amscd,array,bbold}
\usepackage[latin1]{inputenc}
\DeclareMathOperator{\AdS}{AdS}
\DeclareMathOperator{\Sphere}{S}
\DeclareMathOperator{\NW}{NW}
\let\S\Sphere

\begin{document}
\section{The Noncommutativity Parameter of $\NW_4$ as a Penrose-G�ven Limit
of $\AdS_3 \times \S^3$... NOT!}
\label{sec:pg}
Calculating the non-commutativity parameter, $\Theta$, for a general space-time
has been documented \cite{stncg,ho} for the case when the background NS-NS field
$H$ is zero (this is true of all flat metrics). For space-times with non-zero
$H$, the calculation is not so straightforward; attempts to generalise the
technique to curved backgrounds have been summarised in \cite{lbcb}, but fail
for non-compact manifolds. These complications manifest themselves when
calculating $\Theta$ for the open string version of the $\NW_4$ WZW model
developed in \cite{nw}. A calculation has been attempted in \cite{dn}, but is
however approximate.

As the $\NW_4$ space-time can be viewed as a Cahan-Wallach space \cite{cw}, it
will therefore arise as a Penrose limit \cite{pl} of some other metrics. The
most notable example is that of $\AdS_2 \times \S^2$ studied in \cite{plms,
  plsbd}. A six-dimensional analogy of the $\NW_4$ space-time also arises as a
Penrose limit of $\AdS_3 \times \S^3$ \cite{pllbnwb}. The Penrose limit was
extended by G�ven \cite{pgltd} in order to apply to supergravity fields (we
will refer to this limit as the PGL for brevity). It is hoped that the limit
also extends to $\Theta$, as we observe the remarkable fact that $H$ is zero on
$\AdS_2 \times \S^2$.

\subsection{$\AdS_3 \times \S^3$}
\label{sec:pg:ads3xs3}
Combining the studies of $\AdS_3$ and $\S^3$ in \cite{ads3,s3} respectively, we
may write the metric and supergravity fields of $\AdS_3 \times \S^3$ as a direct
sum:
\begin{eqnarray}
  \label{eq:pg:ads3xs3:G}
  G &=& \alpha' k \left(
    d\phi^2 - d\tau^2 + \sin^2{\tau}\left[ \frac{dr^2}{1+r^2} + (1+r^2)d\beta^2
    \right] \right.\\ \nonumber
  &&\left. + \sin^2{\phi} \left[ \frac{d\chi^2}{1-\chi^2} +
    (1-\chi^2)d\theta^2\right]\right)\\
  \label{eq:pg:ads3xs3:H}
  H &=& -2\alpha' k \left( \cos^2{\tau}d\tau\wedge dr\wedge d\beta +
  \sin^2{\phi}d\phi\wedge d\chi\wedge d\theta\right)\\
  \label{eq:pg:ads3xs3:B}
  B &=& -\frac{\alpha' k}{2}\left( \left(\sin{2\tau}+2 \tau\right)dr\wedge d\beta
  + \left( \sin{2\phi}- 2\phi\right)d\chi\wedge d\theta\right)\\
  \label{eq:pg:ads3xs3:F}
  F &=& \alpha' n\left( dr\wedge d\beta + d\chi\wedge d\theta\right)
\end{eqnarray}
where $k$ is an integer related to the radius of the space by $R^2=k\alpha'$.
$n$ is the `magnetic monopole' number ($0<n<k$). It is the flux $F$ which is
quantised, not the invariant quantity ${\cal F}=B+F$.

From $\AdS_3 \times \S^3$, we may arrive at $\NW_4$ from two paths:
\begin{equation}
  \label{eq:pg:ads3xs3:diagram}
  \begin{CD}
    @.\\
    \AdS_3 \times \S^3 @>\text{PGL}>> \NW_6\\
    @AAA          @AAA\\
    \AdS_2 \times \S^2 @>\text{PGL}>> \NW_4\\
    @.
  \end{CD}
\end{equation}
the first being via $\NW_6$, and the other being through $\AdS_2\times
\S^2$. Each path requires a Penrose limit and finding an embedding.

\subsubsection{$\AdS_3 \times \S^3 \rightarrow \NW_6 \rightarrow \NW_4$}
Performing the Penrose limit from $\AdS_3\times \S^3$ to $\NW_6$ is achieved by
redefining our coordinates
\begin{eqnarray}
  \label{eq:pg:ads3xs3:pl}
  \tau = \frac{U-V}{2} &\qquad& \phi=\frac{U+V}{2} \\ \nonumber
  r=Y^1 \quad \beta=Y^2 &\quad& \chi=Y^3 \quad \theta=Y^4
\end{eqnarray}
and scaling $U=u$, $V=\Omega^2v$, $Y^i=\Omega y^i$ sending $\Omega\rightarrow 0$
and $k\rightarrow \infty$ such that $\Omega^2 k = 1$.
In practise the PGL is achieved by finding a suitable gauge such that
$B_{ui}=0$, i.e. that there are no ``time'' components in the supergravity potentials.
This condition is already satisfied as there are no $d\tau$ or $d\phi$
components in $B$. We arrive at
\begin{eqnarray}
  \label{eq:pg:nw6:G}
  \widetilde{G} &=& \alpha' \left( dudv + \sin^2{\left(\frac{u}{2}\right)}{d\vec{y}}^2\right)\\
  \label{eq:pg:nw6:H}
  \widetilde{H} &=& - \alpha' \left(\sin^2{\left(\frac{u}{2}\right)} du\wedge dy^3\wedge dy^4
    + \cos^2{\left(\frac{u}{2}\right)} du\wedge dy^1\wedge dy^2\right)\\
  \label{eq:pg:nw6:B}
  \widetilde{B} &=& - \frac{\alpha'}{2}\left(\left(u-\sin{u}\right)dy^1\wedge dy^2 -
  \left(u-\sin{u}\right) dy^3\wedge dy^4\right)\\
  \label{eq:pg:nw6:F}
  \widetilde{F} &=& \alpha' n \left( dy^1\wedge dy^2 + dy^3\wedge dy^4\right)
\end{eqnarray}
in Rosen coordinates. By changing to Brinkman coordinates given by
\begin{equation}
  \label{eq:pg:nw6:brinkman}
  u=2x^- \qquad v=x^+ + \frac{\cot{x^-}}{2}\vec{z}^2 \qquad
  y^i=\frac{z^i}{\sin{x^-}}
\end{equation}
and dropping the dependence on the $z^{1,2}$ coordinates, we recover $\NW_4$.
\begin{eqnarray}
  \label{eq:pg:nw6:nw4:G}
  \widetilde{G} &=& \alpha' \left(2dx^-dx^+ + \frac{\vec{z}^2}{4}{dx^-}^2 + {d\vec{z}}^2\right)\\
  \label{eq:pg:nw6:nw4:H}
  \widetilde{H} &=& -2\alpha' dx^-\wedge dz^1\wedge dz^2\\
  \label{eq:pg:nw6:nw4:B}
  \widetilde{B} &=& -2\alpha' x^-dz^1\wedge dz^2\\
  \label{eq:pg:nw6:nw4:F}
  \widetilde{F} &=& \frac{\alpha'  n}{\sin^3{x^-}}\left(z^2\cos{x^-} dx^-\wedge dz^1
    \right.\\ \nonumber
  &&\left. + z^1\cos{x^-} dx^-\wedge dz^2 + \sin{x^-}dz^1\wedge dz^2\right)
\end{eqnarray}

\subsubsection{$\AdS_3 \times \S^3 \rightarrow \AdS_2 \times \S^2 \rightarrow
  \NW_4$}
We create the $\AdS_2 \times \S^2$ sub-manifold by fixing the length of
the $\tau$ and $\phi$ coordinates. We then perform the Penrose limit given by
the redefinition
\begin{eqnarray}
  \label{eq:pg:ads2xs2:pl}
  r = \frac{U-V}{2} &\qquad& \chi=\frac{U+V}{2} \\ \nonumber
  \beta=Y^1 &\qquad& \theta=Y^2
\end{eqnarray}
and using the same scaling as in the $\AdS_3 \times \S^3$ case. Due to the fixing
of $\tau$ and $\phi$, the $H$ field trivially goes to zero. Which we may use to
our advantage in calculating Poisson brackets between the coordinates, using the
technique outlined in \cite{ho}.
\begin{equation}
  \label{eq:pg:ads2xs2:poisson}
  check these
\end{equation}
After another suitable change to Brinkman coordinates, similar to
(\ref{eq:pg:nw6:brinkman}), we arrive at the $\NW_4$ metric, but unfortunately
neither $H$ nor $B$
\begin{eqnarray}
  \label{eq:pg:ads2xs2:nw4:G}
  \widetilde{G} &=& \alpha'' k \left(2dx^-dx^+ + \frac{\vec{z}^2}{4}{dx^-}^2 + {d\vec{z}}^2\right)\\
  \label{eq:pg:ads2xs2:nw4:H}
  \widetilde{H} &=& d\widetilde{B} = 0\\
  \label{eq:pg:ads2xs2:nw4:F}
  \widetilde{F} &=& \frac{2\alpha' n}{\sin{x^-}}\left( dx^-\wedge dz^1 + dx^-\wedge
  dz^2\right)
\end{eqnarray}
It should also be noted that we obtain a different flux, $\widetilde{F}$, in
(\ref{eq:pg:ads2xs2:nw4:F}) than we do in (\ref{eq:pg:nw6:nw4:F}).

We may be able to understand why the forms (\ref{eq:pg:ads2xs2:nw4:H},
\ref{eq:pg:nw6:nw4:H}) do not agree, by following the coordinates from
$\AdS_3\times\S^3$ through to $\NW_4$. For the path $\AdS_3\times\S^3
\rightarrow \NW_6 \rightarrow \NW_4$, we may summarise the coordinate change
which induces the PGL by
\begin{eqnarray}
  \label{eq:pg:broken:nw6}
  \tau = \frac{u-\Omega^2v}{2} &\qquad& \phi=\frac{u+\Omega^2 v}{2}\\ \nonumber
  r=\Omega y^1 \quad \beta=\Omega y^2 &\quad& \chi=\Omega y^3 \quad
  \theta=\Omega y^4
\end{eqnarray}
setting $y^1$, $y^2$ constant. Similarly for $\AdS_3\times\S^3\rightarrow
\AdS_2\times\S^2\rightarrow\NW_4$ we may use the same coordinate change, but
instead setting $r$, $\chi$ ($y^1$, $y^3$) constant. It is obvious that
these two paths to $\NW_4$ are not equivalent at this level of the coordinates,
possibly having an adverse effect on the forms which is to be investigated further.

\subsection{$\mathbb{R}^1\times\S^3\rightarrow\NW_4$}
\label{eq:pg:rxs3}
A similar path to $\NW_4$ via a Penrose limit has been studied in
\cite{kiritsis}, starting from an initial space-time of $\mathbb{R}^1 \times
\S^3$ ($\mathbb{R}^1$ generating the time coordinate). We may view this
space as a subspace of $\AdS_3\times\S^3$ and construct a new diagram, which we
suspect to be commutative with respect to the forms.
\begin{equation}
  \label{eq:pg:rxs3:diagram}
  \begin{CD}
    @.\\
    \AdS_3 \times \S^3 @>\text{PGL}>> \NW_6\\
    @AAA          @AAA\\
    \mathbb{R}^1\times\S^3 @>\text{PGL}>> \NW_4\\
    @.
  \end{CD}
\end{equation}
We choose a different parametrisation than in the previous section
(\ref{eq:pg:ads3xs3:G},\ref{eq:pg:ads3xs3:H},\ref{eq:pg:ads3xs3:B}) and write
the metric and associated forms as
\begin{eqnarray}
  \label{eq:pg:rxs3:ads3xs3:G}
  G &=& \frac{\alpha' k}{4}\left(
    {d\phi_1}^2 + {d\phi_2}^2 - {d\phi_3}^2 + {d\phi_4}^2 +
    {d\phi_5}^2 + {d\phi_6}^2 \right.\\ \nonumber
    &&\left.+ 2\sinh{\phi_1}d\phi_2 d\phi_3
    +  2\cos{\phi_4}d\phi_5 d\phi_6 \right)\\
  \label{eq:pg:rxs3:ads3xs3:H}
  H &=& -\frac{\alpha' k}{4}\left(
    \cosh{\phi_1}d\phi_1 \wedge d\phi_2 \wedge d\phi_3
    + \sin{\phi_4}d\phi_4 \wedge d\phi_5 \wedge d\phi_6
    \right)\\
  \label{eq:pg:rxs3:ads3xs3:B}
  B &=& -\frac{\alpha' k}{4}\left(
    \sinh{\phi_1}d\phi_2 \wedge d\phi_3
    -\cos{\phi_4} d\phi_5 \wedge d\phi_6
  \right)
\end{eqnarray}
the $\phi_{1,2,3}$ coming from $\AdS_3$ and the $\phi_{4,5,6}$ coming from
$\S^3$. We obtain $\mathbb{R}^1\times\S^3$ simply by setting $\phi_{1,2}$
constant.

This time, we may summarise both Penrose limits in the same way. We apply a
change of coordinates
\begin{eqnarray}
  \label{eq:pg:rxs3:coord}
  \phi_3 = 2t            &\qquad& \phi_4 = 2r \\ \nonumber
  \phi_5 = \alpha + \varphi &\qquad& \phi_6 = \alpha - \varphi
\end{eqnarray}
and then scale
\begin{eqnarray}
  \label{eq:pg:rxs3:pl}
  \alpha = \Omega^2 v - \frac{u}{2} &\qquad& t = \Omega^2 v + \frac{u}{2} \\ \nonumber
  r = \Omega \rho &\qquad& \phi_i = \Omega \theta_i
\end{eqnarray}
where $i=1,2$. We send $\Omega\rightarrow 0$ and $k\rightarrow \infty$ such that
$\Omega^2 k = 1$. From here on, we set $\alpha'=1$. The two PGLs of
(\ref{eq:pg:rxs3:diagram}) are equivalent at the level of the coordinates as we
initially set $\phi_{1,2}$ constant for the path through
$\mathbb{R}^1\times\S^3$ whereas we set $\theta_{1,2}$ constant for the path
through $\NW_6$, which is essentially the same restriction. We obtain the metric
and associated forms on $NW_4$ as
\begin{eqnarray}
  \label{eq:pg:rxs3:nw4:G}
  G &=& -dudv - \frac{\rho^2}{4}du^2 +d\rho^2 + \rho^2 d\varphi^2 \\
  \label{eq:pg:rxs3:nw4:H}
  H &=& -\rho d\rho \wedge du \wedge d\varphi \\
  \label{eq:pg:rxs3:nw4:B}
  B &=& - \frac{\rho^2}{2}du \wedge d\varphi
\end{eqnarray}

\begin{thebibliography}{100}
\bibitem{stncg} N. Seiberg, E. Witten, ``String Theory and Noncommutative
  Geometry'', \textit{JHEP} \textbf{9909} (1999) 032, \textit{hep-th/9908142}
\bibitem{ho} Pei-Ming Ho, Yu-Ting Yeh, ``Noncommutative D-Brane in
  Non-Constant NS-NS B Field Background'', \textit{Phys. Rev.}
  \textbf{L85} (2000) 5523, \textit{hep-th/0005159}
\bibitem{lbcb} V. Schomerus, ``Lectures on Branes in Curved Backgrounds'',
  \textit{Class. Quant. Grav.}, \textbf{19} (2002) 5781, \textit{hep-th/0209241}
\bibitem{nw} C. R. Nappi, E. Witten, ``A WZW model based on a non-semi-simple
  group'', \textit{Phys. Rev.} \textbf{L71} (1993) 3751, \textit{hep-th/9310112}
\bibitem{dn} L. Dolan, C. R. Nappi, ``Noncommutativity in a Time-Dependent
  Background'', \textit{Phys. Lett.} \textbf{B551} (2003) 369-377,
  \textit{hep-th/0210030}
\bibitem{cw} M. Cahen, N. Wallach, ``Lorentzian Symmetric Spaces'',
  \textit{Bull. Am. Math. Soc.} \textbf{76} (1970) 585-591
\bibitem{pl} R. Penrose, ``Any Space-Time has a Plane Wave as a Limit'',
  \textit{Differential Geometry and Relativity}, Reidel, Dordrecht (1976)
  271-275
\bibitem{plms} M. Blau, J. Figueroa-O'Farrill, C. Hull, G.  Papadopoulos,
  ``Penrose limits and maximal supersymmetry'', \textit{Class. Quant. Grav.}
  \textbf{19} (2002) 19, \textit{hep-th/0201081}
\bibitem{plsbd} M. Blau, J. Figueroa-O'Farrill, G. Papadopoulos, \textit{Class.
    Quant. Grav} \textbf{19} (2002) 4753, \textit{hep-th/0202111}
\bibitem{pllbnwb} S. Stanciu, J. Figueroa-O'Farrill, ``Penrose limits of Lie
  Branes and a Nappi-Witten braneworld'', \textit{JHEP} \textbf{0306} (2003)
  025, \textit{hep-th/0303212}
\bibitem{pgltd} R. G�ven, ``Plane Wave Limits and T-Duality'', \textit{Phys.
    Lett.} \textbf{B482} (2000) 255-263, \textit{hep-th/0005061}
\bibitem{ads3} C. Bachas, M. Petropoulos, ``Anti-de-Sitter D-branes'',
  \textit{JHEP} \textbf{0102} (2001) 025, \textit{hep-th/0012234}
\bibitem{s3} C. Bachas, M. Douglas, C. Schweigert, ``Flux
  Stabilization of D-branes'', \textit{JHEP} \textbf{0005} (2000) 048,
  \textit{hep-th/0003037}
\bibitem{kiritsis} G. D'Appollonio, E. Kiritsis, ``String interactions in
  gravitational wave backgrounds'', \textit{hep-th/0305081}
\end{thebibliography}
\end{document}
