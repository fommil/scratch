\documentclass[11pt,a4paper]{article}
\usepackage{amsmath,amsfonts,amssymb,amsthm,amstext,amscd,array,bbold}
%\usepackage[latin1]{inputenc}
\DeclareMathOperator{\AdS}{AdS}
\DeclareMathOperator{\Sphere}{S}
\DeclareMathOperator{\NW}{NW}
\DeclareMathOperator{\CW}{CW}
\let\S\Sphere
\DeclareMathOperator{\weyl}{{\sf W}}                   %% \weyl =
%% Weyl operator

\newcommand{\comp}{\boxplus}                            %% \comp =
%% composition
\newcommand{\compa}{\mkern+4mu\square\mkern-17.4mu{\mbox{
    \protect\raisebox{0.2ex}{$\ast$}}\mkern+5mu}}       %% \compa =
%% composition style a 
\newcommand{\compb}{\mkern+4mu\square\mkern-17.4mu{\mbox{
    \protect\raisebox{0.2ex}{$\bullet$}}\mkern+5mu}}    %% \compb =
%% composition style b 
\newcommand{\compc}{\mkern+4mu\square\mkern-17.4mu{\mbox{
    \protect\raisebox{0.2ex}{$\star$}}\mkern+5mu}}      %% \compc =
%% composition style c 
\newcommand{\cb}[2]{\left[#1,#2\right]}                 %% \cb{a}{b} =
%% [ a, b ] 
\newcommand{\scb}[2]{\left[#1,#2\right]_\star}          %% \scb{a}{b}
%% = [ a, b ]_star 
\newcommand{\NO}{\mbox{$\substack{\circ\\\circ}$}}      %% Normal
%% ordering 
\newcommand{\NOa}{\mbox{$\substack{\ast\\\ast}$}}       %% Normal
%% ordering alt A 
\newcommand{\NOb}{\mbox{$\substack{\bullet\\\bullet}$}} %% Normal
%% ordering alt B 

\def\baselinestretch{1.1}
\textheight 24cm
\textwidth 16cm
\parskip 1ex

\oddsidemargin 0pt
\evensidemargin 0pt
\topmargin -60pt
\jot = .5ex
\newcommand{\appx}{\approx}
\newcommand{\cD}{{\mathcal D}}

\newcommand{\ee}[1]{{\rm e}^{#1}}
\newcommand{\ii}{{\rm i}}
\newcommand{\f}{\frac}

\newcommand{\sgn}{{\rm sgn}}
\newcommand{\eps}{\varepsilon}
\newcommand{\half}{\mbox{$\frac{1}{2}$}}
\newcommand{\behalf}{\frac{\beta}{2}}
\def\Dirac{{D\!\!\!\!/\,}} % Dirac operator

\newcommand{\C}{\complex}
\newcommand{\Z}{\zed}
\newcommand{\N}{\nat}

\newcommand{\QED}{\hfill QED}
\newcommand{\mbf}[1]{{\boldsymbol {#1} }}
\def\ii{{\,{\rm i}\,}}
\def\dd{{\rm d}}
\def\DD{{\rm D}}
\def\CC{{\rm C}}
\def\P{{\sf P}}
\def\B{{\sf B}}
\def\a{{\sf a}}
\def\b{{\sf b}}
\def\A{{\sf A}}
\def\V{{\sf V}}
\def\K{{\sf K}}
\def\T{{\sf T}}
\def\X{{\sf X}}
\def\Y{{\sf Y}}
\def\Q{{\sf Q}}
\def\C{{\sf C}}
\def\J{{\sf J}}

\def\mz{{\mbf z}}
\def\mw{{\mbf w}}
\def\mx{{\mbf x}}
\def\my{{\mbf y}}
\def\mk{{\mbf k}}
\def\mq{{\mbf q}}
\def\mD{{\mbf D}}
\def\mG{{\mbf G}}
\def\mdell{{\mbf\partial}}

\def\mfn{{\mathfrak n}}
\def\mfg{{\mathfrak g}}
\def\mfs{{\mathfrak s}}
\def\mcN{{\mathcal N}}
\def\mcS{{\mathcal S}}

\newcommand{\eq}{\begin{equation}}
\newcommand{\eqend}{\end{equation}}
\newcommand{\eqa}{\begin{eqnarray}}
\newcommand{\nonueqa}{\begin{eqnarray*}}
\newcommand{\eqaend}{\end{eqnarray}}
\newcommand{\nonueqaend}{\end{eqnarray*}}
\newcommand{\nonu}{\nonumber \\ \nopagebreak}
\newcommand{\bma}[1]{\begin{array}{#1}}
\newcommand{\ema}{\end{array}}
\newcommand{\bc}{\begin{center}}
\newcommand{\ec}{\end{center}}

\newcommand{\U}{{\rm U}}
\newcommand{\vect}{{\rm Vect}}
\newcommand{\Ref}[1]{(\ref{#1})}
\newcommand{\YM}{{\mathcal Y\!M}}
\newcommand{\tet}{\theta}
\newcommand{\cL}{{\mathcal L}}
\newcommand{\sine}{{\sf K}}

\newcommand{\R}{\real}

\renewcommand{\theequation}{\thesection.\arabic{equation}}
\renewcommand{\thefootnote}{\fnsymbol{footnote}}
\newcommand{\newsection}{\setcounter{equation}{0}\section}

\def\appendix#1{\addtocounter{section}{1}\setcounter{equation}{0}
\renewcommand{\thesection}{\Alph{section}}
\section*{Appendix \thesection\protect\indent \parbox[t]{11.715cm} {#1}}
\addcontentsline{toc}{section}{Appendix \thesection\ \ \ #1} }

\newcommand{\complex}{{\mathbb C}} %% complex numbers
\newcommand{\zed}{{\mathbb Z}} %% integers
\newcommand{\nat}{{\mathbb N}} %% naturals
\newcommand{\real}{{\mathbb R}} %% real numbers
\newcommand{\eucl}{{\mathbb E}}
\newcommand{\rat}{{\mathbb Q}} %% rational numbers
\newcommand{\mat}{{\mathbb M}} %% matrix algebra
%\newcommand{\NO}{\,\mbox{$\circ\atop\circ$}\,} % Normal ordering
\newcommand{\id}{{1\!\!1}} %% identity matrix
\def\Dirac{{D\!\!\!\!/\,}} % Dirac operator
\def\semiprod{{\,\rhd\!\!\!<\,}} % crossed product
\def\alg{{\mathcal A}}
\def\hil{{\mathcal H}}
\def\ota{\otimes_{ A}}
\def\otc{\otimes_{\complexs}}
\def\otr{\otimes_{reals}}
\newif\ifold             \oldtrue            \def\new{\oldfalse}

\font\mathsm=cmmi9

\def\nn{\nonumber}
\newcommand{\tr}[1]{\:{\rm tr}\,#1}
\newcommand{\Tr}[1]{\:{\rm Tr}\,#1}
\newcommand{\sdet}[1]{\:{\rm SDet}\,#1}
\def\e{{\,\rm e}\,}
\newcommand{\rf}[1]{(\ref{#1})}
\newcommand{\non}{\nonumber \\*}
\newcommand{\vgr}{{\vec \nabla}}
\newcommand{\vx}{{\vec x}}
\hyphenation{pre-print}
\hyphenation{pre-prints}
\hyphenation{di-men-sion-al}
\hyphenation{di-men-sion-al-ly}
\def\be{\begin{equation}}
\def\ee{\end{equation}}
\def\bea{\begin{eqnarray}}
\def\eea{\end{eqnarray}}
\def\bd{\begin{displaymath}}
\def\ed{\end{displaymath}}
\def\const{{\rm const}}
\def\s{\sigma}
\def\vcl{\varphi_{\rm cl}}
\def\la{\left\langle}
\def\ra{\right\rancle}
\def\d{\partial}
\def\se{S_{\rm eff}}
\def\E{F}

\newcommand{\fr}[2]{{\textstyle {#1 \over #2}}}

\newcommand{\fintf}[1]{\int \frac{d^{2d} #1}{(2\pi)^{2d}}}
\newcommand{\fintt}[1]{\int \frac{d^2 #1}{(2\pi)^2}}

\def\sepand{\rule{14cm}{0pt}\and}
\newcommand{\beq}{\begin{eqnarray}}
\newcommand{\eeq}{\end{eqnarray}}
\newcommand{\MM}{\bf \mathcal M}
\newcommand{\g}{\gamma}
\newcommand{\G}{\Gamma}
\newcommand{\rh}{\rho}
\newcommand{\sg}{\sigma}
\newcommand{\Sg}{\Sigma}
\newcommand{\pr}{\partial}
\newcommand{\tSg}{\tilde{\Sigma}}
\newcommand{\hr}{\hat{r}}
\newcommand{\hq}{\hat{q}}
\newcommand{\hk}{\hat{K}}
\newcommand{\ho}{\hat{\omega}}
\newcommand{\del}{\partial}
\newcommand{\ka}{ e}
\newcommand{\co}{{\mathcal O}}
\newcommand{\z}{\zeta}
\newcommand{\upa}{\uparrow}
\newcommand{\doa}{\downarrow}
\newcommand{\ZZ}{{\mathcal Z}}

\makeatletter
\newdimen\normalarrayskip              % skip between lines
\newdimen\minarrayskip                 % minimal skip between lines
\normalarrayskip\baselineskip
\minarrayskip\jot
\newif\ifold             \oldtrue            \def\new{\oldfalse}
%
\def\arraymode{\ifold\relax\else\displaystyle\fi} % mode of array entries
\def\eqnumphantom{\phantom{(\theequation)}}     % right phantom in eqnarray
\def\@arrayskip{\ifold\baselineskip\z@\lineskip\z@
     \else
     \baselineskip\minarrayskip\lineskip2\minarrayskip\fi}
%
\def\@arrayclassz{\ifcase \@lastchclass \@acolampacol \or
\@ampacol \or \or \or \@addamp \or
   \@acolampacol \or \@firstampfalse \@acol \fi
\edef\@preamble{\@preamble
  \ifcase \@chnum
     \hfil$\relax\arraymode\@sharp$\hfil
     \or $\relax\arraymode\@sharp$\hfil
     \or \hfil$\relax\arraymode\@sharp$\fi}}
%
\def\@array[#1]#2{\setbox\@arstrutbox=\hbox{\vrule
     height\arraystretch \ht\strutbox
     depth\arraystretch \dp\strutbox
     width\z@}\@mkpream{#2}\edef\@preamble{\halign \noexpand\@halignto
\bgroup \tabskip\z@ \@arstrut \@preamble \tabskip\z@ \cr}%
\let\@startpbox\@@startpbox \let\@endpbox\@@endpbox
  \if #1t\vtop \else \if#1b\vbox \else \vcenter \fi\fi
  \bgroup \let\par\relax
  \let\@sharp##\let\protect\relax
  \@arrayskip\@preamble}
\makeatother

\allowdisplaybreaks

\begin{document}
\begin{titlepage}
\begin{flushright}

\baselineskip=12pt

HWM--04--xx\\
EMPG--04--xx\\
hep--th/yymmnnn\\
\hfill{ }\\
\today
\end{flushright}

\begin{center}

\vspace{2cm}

\baselineskip=24pt

{\Large\bf Isometric Embeddings and Noncommutative Branes \\
in Homogeneous Gravitational Waves}

\baselineskip=14pt

\vspace{1cm}

{\bf Sam Halliday} and {\bf Richard J. Szabo}
\\[4mm]
{\it Department of Mathematics\\ School of Mathematical and Computer
  Sciences\\ Heriot-Watt University\\ Scott Russell Building,
  Riccarton, Edinburgh EH14 4AS, U.K.}
\\{\tt samuel@ma.hw.ac.uk} , {\tt R.J.Szabo@ma.hw.ac.uk}
\\[40mm]

\end{center}

\begin{abstract}

\baselineskip=12pt

We explicitly construct the time-dependent noncommutative geometry of
a class of pp-wave string backgrounds supported by a constant Neveu-Schwarz
flux, and develop the necessary tools to analyse quantum field
theories defined thereon.

\end{abstract}

\end{titlepage}
\setcounter{page}{2}

\newpage

\renewcommand{\thefootnote}{\arabic{footnote}} \setcounter{footnote}{0}

\newsection{Introduction \label{Intro}}

\newsection{The Nappi-Witten Plane Wave \label{NWPW}}

In this section we will briefly describe some of the general results
concerning the definition and geometry of the Nappi-Witten spacetime
NW$_4$. Its noteworthy properties are that it is a four-dimensional
homogeneous spacetime of Minkowski signature, which may be regarded as
a monochromatic plane wave. It is further equipped with a supergravity
NS $B$-field of constant flux, which in the presence of D-branes will
eventually be responsible for the spacetime noncommutativity of the
pp-wave. We will emphasize the simple, time-independent harmonic
oscillator form of the dynamics in this background, as it will play a
crucial role in subsequent sections.

\subsection{Definitions \label{Defs}}

The spacetime NW$_4$ is defined as the group manifold of
the Nappi-Witten group, the universal central extension of the
two-dimensional euclidean group ${\rm ISO}(2)={\rm
  SO}(2)\ltimes\real^2$. The corresponding simply connected group
$\mathcal N_4$ is homeomorphic to four-dimensional Minkowski space
$\eucl^{1,3}$. Its non-semisimple Lie algebra $\mathfrak n_4$ is
generated by elements $\P^\pm$, $\J$, $\T$ obeying the commutation
relations
\bea
\left[\P^+\,,\,\P^-\right]&=&2\,\T \ , \nn\\
\left[\J\,,\,\P^\pm\right]&=&\pm\,\P^\pm \ , \nn\\
\left[\T\,,\,\J\right]&=&\left[\T\,,\,\P^\pm\right]~=~0 \ .
\label{NW4algdef}\eea
This is just the three-dimensional Heisenberg algebra extended by an
outer automorphism which rotates the noncommuting coordinates. The
twisted Heisenberg algebra
may be regarded as defining the harmonic oscillator algebra of a
particle moving in one-dimension, with the additional generator $\J$
playing the role of the number operator (or equivalently the
oscillator hamiltonian). It is this twisting that will
lead to a noncommutative geometry that deviates from the usual Moyal
noncommutativity generated by the Heisenberg algebra. On the other
hand, $\mathfrak{n}_4$ is a solvable algebra whose properties will be
much more tractable than, for instance, those of the
${\rm su}(2)$ or ${\rm sl}(2,\real)$ Lie algebras which are
at the opposite extreme.

The center of the universal enveloping algebra $U(\mathfrak n_4)$
contains the central element $\T$ of the Lie algebra $\mathfrak{n}_4$
and also the quadratic Casimir element
\beq
\C_4=\mbox{$\frac12$}\,\left(\P^+\,\P^-+\P^-\,\P^+\right)+2\,
\J\,\T \ .
\label{NW4Casimir}\eeq
The most general invariant, non-degenerate symmetric bilinear form
$\langle\,\cdot\,,\,\cdot\,\rangle:\mathfrak{n}_4\times\mathfrak{n}_4\to\real$
is defined by
\bea
\left\langle\P^+\,,\,\P^-\right\rangle&=&2 \ , \nn\\
\left\langle\J\,,\,\T\right\rangle&=&1 \ , \nn\\
\left\langle\J\,,\,\J\right\rangle&=&b \ , \nn\\
\left\langle\P^\pm\,,\,\P^\pm\right\rangle&=&
\left\langle\T\,,\,\T\right\rangle~=~0 \ , \nn\\
\left\langle\J\,,\,\P^\pm\right\rangle&=&\left\langle\T\,,\,
\P^\pm\right\rangle~=~0
\label{NW4innerprod}\eea
for any $b\in\real$. This inner product has Minkowski signature, so that the
group manifold of $\mathcal N_4$ possesses a
homogeneous, bi-invariant lorentzian metric defined by the pairing of
the Cartan-Maurer left-invariant $\mathfrak n_4$-valued one-forms
$g^{-1}~\dd g$ for $g\in\mathcal N_4$,
\beq
\dd s_4^2=\left\langle g^{-1}~\dd g\,,\,g^{-1}~\dd g\right\rangle \ .
\label{NW4CM}\eeq
A generic group element $g\in\mathcal N_4$ may be parametrized as
\beq
g(u,v,a)=\e^{a\,\P^++\overline{a}\,\P^-}~
\e^{u\,\J+v\,\T}
\label{NW4coords}\eeq
with $u,v\in\real$ and $a\in\complex$. In these
global coordinates, the metric (\ref{NW4CM}) reads
\beq
\dd s_4^2=2~\dd u~\dd v+|\dd a|^2+2\ii\left(a~
\dd\overline{a}-\overline{a}~\dd a\right)~\dd u+b~\dd u^2 \ .
\label{NW4metricNW}\eeq

The metric (\ref{NW4metricNW}) assumes the standard form of the plane
wave metric of a conformally flat, indecomposable Cahen-Wallach
lorentzian symmetric spacetime CW$_4$ in four dimensions upon
introduction of Brinkmann coordinates $(x^+,x^-,z)$ defined by
rotating the transverse plane at a Larmor frequency as $u=x^+$,
$v=x^-$ and $a=\e^{\ii x^+/2}\,z$. In these coordinates the metric
assumes the stationary form
\beq
\dd s_4^2=2~\dd x^+~\dd x^-+|\dd z|^2+\left(b-\mbox{$\frac14$}\,|z|^2\right)~
\left(\dd x^+\right)^2 \ ,
\label{NW4metricBrink}\eeq
revealing the pp-wave nature of the geometry for $b=0$. The physical
meaning of the arbitrary parameter $b$ will be elucidated below. It
may be set to zero by exploiting the translational symmetry of the
geometry in $x^-$ to shift $x^-\to x^--\frac b2\,x^+$, which
corresponds to a Lie algebra automorphism of $\mathfrak n_4$. Note
that on the null planes of constant $u=x^+$, the geometry becomes that of
flat two-dimensional euclidean space $\eucl^2$. This is the geometry
appropriate to the Heisenberg subgroup of $\mathcal{N}_4$, and is what
is expected in the Moyal limit when the effects of the extra generator
$\J$ are turned off.

Thus far, the Nappi-Witten spacetime has been described geometrically
as a four-dimensional Cahen-Wallach space CW$_4$. The spacetime NW$_4$
is further supported by a Neveu-Schwarz two-form field $B_4$ of constant
field strength
\bea
H_4&=&-\mbox{$\frac13$}\,\bigl\langle g^{-1}~\dd g\,,\,\dd
\left(g^{-1}~\dd g\right)\bigl\rangle
~=~2\ii~\dd x^+\wedge\dd z\wedge\dd\overline{z}~=~\dd B_4 \ , \nn\\
B_4&=&-\mbox{$\frac12$}\,\bigl\langle g^{-1}~\dd g\,,\,
\frac{\id+{\rm Ad}_g}{\id-{\rm Ad}_g}\,g^{-1}~\dd g\bigl\rangle~=~
2\ii x^+~\dd z\wedge\dd\overline{z} \ .
\label{NS2formBrink}\eea
The corresponding contracted two-form $H_4^2$ compensates exactly the
constant Riemann curvature of the metric (\ref{NW4metricBrink}), so
that NW$_4$ provides a viable supergravity background. In fact, in
this case the cancellation is exact at the level of the full string
equations of motion, so that the plane wave is an exact background of
string theory. It is the presence of
this $B$-field that will induce time-dependent noncommutativity of the
string background in the presence of D-branes. Because its flux is
constant, the noncommutative dynamics on this space can still be
formulated exactly, just like on other symmetric (fuzzy) curved
noncommutative spaces.

\subsection{Isometries \label{Isoms}}

The realization of the geometry of NW$_4$ as a standard plane wave of
Cahen-Wallach type enables us to study its isometry group using the
standard classification. Writing $\partial_\pm:=\partial/\partial
x^\pm$, the metric (\ref{NW4metricBrink}) has the obvious null Killing
vector
\beq
T=\partial_-
\label{ZKilling}\eeq
generating translations in $x^-$ and characterizing a pp-wave, and
also the null Killing vector
\beq
J=\partial_+
\label{HKilling}\eeq
generating translations in $x^+$. An
analysis of the Killing equations shows that there are also four extra
Killing vectors $P^{(k)}$, $P^{\prime\,(k)}$, $k=1,2$ which generate
twisted translations in the transverse plane $z\in\complex$ to
the motion of the plane wave. Denoting $\partial:=\partial/\partial
z$, they are given in the form
\bea
P^{(k)}&=&c^{(k)}(x^+)\,\partial+\overline{c}^{\,(k)}(x^+)\,
\overline{\partial}-\left(\dot c^{(k)}(x^+)\,\overline{z}+
\dot{\overline{c}}^{\,(k)}(x^+)\,z\right)\,\partial_- \ , \nn\\
P^{\prime\,(k)}&=&c^{\prime\,(k)}(x^+)\,\partial+\overline{c}^{\,
\prime\,(k)}(x^+)\,
\overline{\partial}-\left(\dot c^{\prime\,(k)}(x^+)\,\overline{z}+
\dot{\overline{c}}^{\,\prime\,(k)}(x^+)\,z\right)\,\partial_- \ ,
\label{XkXprimegen}\eea
where the dots denote differentiation with respect to the plane
wave time coordinate $u=x^+$, and the complex-valued coefficient functions in
(\ref{XkXprimegen}) solve the harmonic oscillator differential
equation
\beq
\ddot c(x^+)=-\mbox{$\frac14$}\,c(x^+) \ .
\label{HODE}\eeq
The four linearly independent solutions of (\ref{HODE}) are
characterized by their initial conditions on the null surface $x^+=0$,
\bea
c^{(k)}(0)~=~\delta_{k1}+\ii\delta_{k2} ~~~~ &,& ~~~~ \dot c^{(k)}(0)~=~0
\ , \nn\\c^{\prime\,(k)}(0)~=~0 ~~~~ &,& ~~~~ \dot
c^{\prime\,(k)}(0)~=~\delta_{k1}+\ii\delta_{k2} \ .
\label{cinitialconds}\eea

The solutions of (\ref{HODE}) and (\ref{cinitialconds}) are given by
\bea
c^{(1)}(x^+)~=~\cos\mbox{$\frac{x^+}2$} ~~~~ &,& ~~~~
c^{(2)}(x^+)~=~\ii\cos\mbox{$\frac{x^+}2$} \ , \nn\\
c^{\prime\,(1)}(x^+)~=~2\sin\mbox{$\frac{x^+}2$} ~~~~ &,& ~~~~
c^{\prime\,(2)}(x^-)~=~2\ii\sin\mbox{$\frac{x^+}2$} \ .
\label{cexplsoln}\eea
An interesting feature of these functions is that they generate the
Rosen form of the plane wave metric (\ref{NW4metricBrink}). It is
defined by the transformation to local coordinates $(u,v,y^1,y^2)$
given by
\bea
u&=&x^+ \ , \nn\\v&=&x^--\mbox{$\frac14$}\,(z+\overline{z}\,)^2\,
\tan\mbox{$\frac{x^+}2$}-\mbox{$\frac14$}\,(z-\overline{z}\,)^2\,
\cot\mbox{$\frac{x^+}2$} \ , \nn\\y^1&=&\mbox{$\frac12$}\,
\left((z+\overline{z}\,)\sec\mbox{$\frac{x^+}2$}+\ii(z-\overline{z}\,)
\,\csc\mbox{$\frac{x^+}2$}\right) \ , \nn\\y^2&=&\mbox{$\frac12$}\,
\left((z+\overline{z}\,)\sec\mbox{$\frac{x^+}2$}-\ii(z-\overline{z}\,)
\,\csc\mbox{$\frac{x^+}2$}\right) \ ,
\label{Rosen}\eeq
under which the metric becomes
\beq
\dd s_4^2=2~\dd u~\dd v+C_{ij}(u)~\dd y^i~\dd y^j+b~\dd u^2
\label{NW4metricRosen}\eeq
where
\beq
C(u)=\bigl(C_{ij}(u)\bigr)=\begin{pmatrix}1&\cos u\\\cos
  u&1\end{pmatrix} \ .
\label{Cumatrix}\eeq
This form of the metric is degenerate at the singular points $\cos u=\pm\,1$.
The harmonic oscillator solutions (\ref{cexplsoln}) then generate an
orthonormal frame for the transverse metric~(\ref{Cumatrix}),
\beq
C(u)=E(u)\,E^\top(u) \ ,
\label{CQrel}\eeq
with
\beq
E=\begin{pmatrix}c^{(1)}+\mbox{$\frac12$}\,c^{\prime\,(2)}+
\overline{c}^{\,(1)}+\mbox{$\frac12$}\,\overline{c}^{\,\prime
\,(2)}& &c^{(2)}-\mbox{$\frac12$}\,c^{\prime\,(1)}+
\overline{c}^{\,(2)}-\mbox{$\frac12$}\,\overline{c}^{\,\prime
\,(1)}\\\ii\left(c^{(1)}+\mbox{$\frac12$}\,c^{\prime\,(2)}+
\overline{c}^{\,(1)}+\mbox{$\frac12$}\,\overline{c}^{\,\prime
\,(2)}\right)& &\ii\left(c^{(2)}-\mbox{$\frac12$}\,c^{\prime\,(1)}+
\overline{c}^{\,(2)}-\mbox{$\frac12$}\,\overline{c}^{\,\prime
\,(1)}\right)\end{pmatrix}
\label{Qvielbein}\eeq
satisfying the symmetry condition
\beq
\dot E(u)\,E^\top(u)=E(u)\,\dot E^\top(u) \ .
\label{Esymcond}\eeq
Note that in contrast to the Brinkmann coordinate system, in the Rosen
form (\ref{NW4metricRosen}) two extra commuting translational
symmetries in the transverse plane $(y^1,y^2)$ are manifest, while
time translation symmetry is lost.

By defining $P^\pm:=P^{\prime\,(1)}\pm\ii P^{(1)}$ and
$\overline{P}^{\,\pm}:=P^{\prime\,(2)}\pm\ii P^{(2)}$, the six
Killing vectors generated by the basic Cahen-Wallach structure of the
plane wave may be summarized as
\bea
T&=&\partial_- \ , \nn\\J&=&\partial_+ \ , \nn\\
P^\pm&=&\left(\sin\mbox{$\frac{x^+}2$}\pm\ii\e^{\mp\ii x^+/2}\right)
\left(\partial+\overline{\partial}\,\right)-\e^{\mp\ii x^+/2}\,\left(
z+\overline{z}\,\right)~\partial_- \ , \nn
\\\overline{P}^{\,\pm}&=&\left(\ii\sin\mbox{$\frac{x^+}2$}\mp
\e^{\mp\ii x^+/2}\right)
\left(\partial-\overline{\partial}\,\right)+\ii\e^{\mp\ii x^+/2}\,\left(
z-\overline{z}\,\right)~\partial_- \ .
\label{6CWKilling}\eea
Together, they generate the harmonic oscillator algebra of a particle
moving in {\it two} dimensions,
\bea
\left[P^\alpha\,,\,\overline{P}^{\,\beta}\right]&=&0 \ , ~~
\alpha,\beta=\pm \ , \nn\\\left[T\,,P^\pm\right]&=&\left[T
\,,\,\overline{P}^{\,\pm}\right]~=~\left[T\,,\,J\right]~=~0 \ ,
\nn\\\left[P^+\,,\,P^-\right]&=&\left[\overline{P}^{\,+}\,,\,
\overline{P}^{\,-}\right]~=~2\ii T \ , \nn\\\left[J\,,\,
P^\pm\right]&=&\pm\ii P^\pm \ , \nn\\\left[J\,,\,\overline{P}^{\,
\pm}\right]&=&\pm\ii\overline{P}^{\,\pm} \ .
\label{NW4isomalg}\eea
This isometry algebra acts transitively on the null planes of constant
  $x^+$, and it generates a double extension ${\mathcal
  N}_6$ of the euclidean group ${\rm ISO}(2)$, defined by extending the
  commutation relations (\ref{NW4algdef}) by generators
  $\overline{\P}^{\,\pm}$ obeying relations as in (\ref{NW4isomalg}). In
  this context, we shall refer to $\mathcal N_6$ as the generic pp-wave
  isometry group.

Following the analysis of the previous section, one can show that the
group manifold of ${\mathcal N}_6$ is a six-dimensional Cahen-Wallach
space CW$_6$, with Brinkmann metric
\beq
\dd s_6^2=2~\dd x^+~\dd x^-+|\dd\mz|^2+\left(b-\mbox{$\frac14$}\,|\mz|^2\right)~
\left(\dd x^+\right)^2
\label{NW6metricBrink}\eeq
where $\mz^\top=(z,w)\in\complex^2$, which carries a constant
Neveu-Schwarz three-form flux
\beq
H_6=-2\ii~\dd x^+\wedge\dd\overline{\mz}^{\,\top}
\wedge\dd\mz=\dd B_6 \ , ~~
B_6=-2\ii x^+~\dd\overline{\mz}^{\,\top}\wedge\dd\mz \ .
\label{NW6Bfield}\eeq
It thereby defines a six-dimensional version NW$_6$ of the
Nappi-Witten pp-wave. This observation will be exploited in the
ensuing sections to view the Nappi-Witten wave as an isometrically
embedded D-submanifold ${\rm NW}_4\hookrightarrow{\rm NW}_6$. From an
algebraic point of view, this will be tantamount to regarding the
Nappi-Witten group $\mathcal N_4$ as a twisted conjugacy class inside
its basic pp-wave isometry group $\mathcal N_6$. In this
setting, it corresponds to a symmetric D3-brane in a non-zero $H$-flux
whose worldvolume will contain the noncommutative geometry that we are
looking for.

However, for the Nappi-Witten wave this is not the end of the
story. Because of the bi-invariance of the metric (\ref{NW4CM}), the
actual isometry group is the direct product $\mathcal
N_4\times\overline{\mathcal N_4}$ acting from the left and right on
the group $\mathcal N_4$ itself. Because the left and right actions of
the central generator $\T$ coincide, the isometry group is
seven-dimensional. The missing generator from the list
(\ref{6CWKilling}) is the left-moving copy $\overline{J}$ of the
oscillator Hamiltonian, and it is straightforward to compute that it
is given by
\beq
\overline{J}=-\partial_+-\ii\left(z\,\partial-\overline{z}\,
\overline{\partial}\,\right) \ .
\label{extraHKilling}\eeq
The vector field $J+\overline{J}$ generates rotations in the
transverse plane.

\subsection{Commutative Dynamics \label{Dynamics}}

Standard covariant quantization of a massless relativistic particle in
NW$_4$ leads to the Klein-Gordon equation in the curved background,
\beq
\Box_4\phi~=~0 \ ,
\label{KGeqn}\eeq
where
\beq
\Box_4=2\,\partial_+\,\partial_--\left(b-\mbox{$\frac14$}\,|z|^2\right)\,
\partial_-^2+|\partial|^2
\label{BoxNW4def}\eeq
is the Laplacian corresponding to the Brinkmann metric
(\ref{NW4metricBrink}). It coincides with the Casimir
(\ref{NW4Casimir}) expressed in terms of left or right isometry
generators,
\beq
\Box_4=\mbox{$\frac12$}\,\left(P^+\,P^-+P^-\,P^+\right)+2\,J\,T=
\mbox{$\frac12$}\,\left(
\overline{P}^{\,+}\,\overline{P}^{\,-}+\overline{P}^{\,-}\,\overline{P}^{\,+}
\right)+2\,\overline{J}\,\overline{T} \ .
\label{BoxCasimirNW4}\eeq
The dependence on the light-cone coordinates $x^\pm$ drops out of
the Klein-Gordon equation because of the isometries generated by the
Killing vectors (\ref{ZKilling}) and (\ref{HKilling}).

By using a Fourier transformation of the covariant Klein-Gordon field
$\phi$ along the $x^-$ direction,
\beq
\phi\left(x^+,x^-,z\right)=\int\limits_{-\infty}^\infty
\dd p^+~\psi\left(x^+,z;p^+\right)~\e^{\ii p^+x^-} \ ,
\label{KGphiFT}\eeq
we may write (\ref{KGeqn}) equivalently as
\beq
\left[|\partial|^2+2\ii p^+\,\partial_++\left(b-
\mbox{$\frac14$}\,|z|^2\right)\,\left(p^+\right)^2\right]\psi\left(x^+,z
;p^+\right)=0 \ .
\label{KGmomsp}\eeq
Introducing the time parameter $\tau$ through
\beq
u=x^+=p^+\,\tau \ ,
\label{utaudef}\eeq
the differential equation (\ref{KGmomsp}) becomes the standard
Schr\"odinger wave equation
\beq
\ii\,\frac{\partial\psi\left(\tau,z;p^+\right)}{\partial\tau}=
\left[-\mbox{$\frac12$}\,|\partial|^2+\mbox{$\frac12$}\,\left(p^+/2\right)^2\,
|z|^2-\mbox{$\frac b2$}\,\left(p^+\right)^2\right]\psi\left(\tau,z;p^+\right)
\label{SchHO}\eeq
for the non-relativistic two-dimensional harmonic oscillator with a
time independent frequency given by the light-cone momentum as
$\omega=|p^+|/2$. The only role of the arbitrary parameter $b$ is to
shift the zero-point energy of the harmonic oscillator, and it thereby
carries no physical significance.

Let us remark that the same Hamiltonian that appears in the
Schr\"odinger equation (\ref{SchHO}) could also have been derived in
light-cone gauge in the plane wave metric (\ref{NW4metricBrink})
starting from the massless relativistic particle Lagrangian
\beq
L=\dot x^+\,\dot x^-+\mbox{$\frac12$}\,\left(b-\mbox{$\frac14$}\,
|z|^2\right)\,\left(\dot x^+\right)^2+\mbox{$\frac12$}\,\left|\dot z
\right|^2
\label{masslessLag}\eeq
describing free geodesic motion in the Nappi-Witten spacetime. In the
light-cone gauge, the light-cone momentum is $p^+=p_-=\partial
L/\partial\dot x^-=\dot x^+=1$, while the Hamiltonian is
$J=p_+=\partial L/\partial\dot x^+$. Imposing the mass-shell
constraint $L=0$ at $\dot x^+=1$ gives the equation of motion for
$x^-$, which when substituted into $J$ yields exactly the Hamiltonian
appearing on the right-hand side of (\ref{SchHO}) with $p=\dot
z=-\ii\partial$. These calculations give the light-cone quantization
of a particle in NW$_4$ only in the commutative geometry limit,
i.e. in the spacetime CW$_4$,
because they do not incorporate the supergravity $B$-field supported
by the Nappi-Witten spacetime. In the following we will describe how to incorporate
the deformation of CW$_4$ caused by the non-trivial NS-sector and how to properly
describe dynamics in the Nappi-Witten spacetime. Henceforth we will
drop the zero-point energy and set $b=0$.

\newsection{Isometric Embeddings of Branes\label{IsomEmb}}

A remarkable feature of the Nappi-Witten spacetime is the extent to
which it shares common features of many of the more ``standard''
curved spaces. It is formally similar to the spacetimes built on the
${\rm SL}(2,\real)$ and ${\rm SU}(2)$ group manifolds, but in many ways is much
simpler. As a twisted Heisenberg group, it lies somewhere in between
these curved spaces and the flat space based on the usual Heisenberg
algebra. We will encounter this aspect profoundly through the
noncommutative geometry of this space. One way to see this feature at
a quantitative level is by examining Penrose-G\"uven limits involving the
${\rm SL}(2,\real)$ and ${\rm SU}(2)$ group manifolds which produce the
spacetime NW$_4$. While this feature will not be particularly helpful
for the explicit construction of the noncommutative geometry, it will
provide an aid in understanding various structures which arise and
will point out which D-branes to look for as noncommutative
submanifolds of the Nappi-Witten spacetime.

In looking for such D-submanifolds, we are interested in those
D-embeddings which are NS-supported and thereby carry a noncommutative
geometry. As we will discuss, this involves certain
important subtleties that must be carefully taken into account. As the
Nappi-Witten spacetime can be viewed as a Cahen-Wallach space, i.e. as
a plane wave, its geometry will arise as Penrose limits of other
metrics. This opens up the possibility of extracting features of
$\NW_4$ by mapping them directly from properties of simpler, better
studied noncommutative spaces. In
this section we will begin with a thorough general analysis of the interplay
between Penrose-G\"uven limits and isometric embeddings of lorentzian
manifolds, and derive simple criteria for the limit and embedding to
commute. Then we apply these results to derive the possible limits
that can be used to describe the NS-supported D-embeddings of NW$_4$.

\subsection{Penrose-G\"uven Limits and Isometric Embeddings: \\ General
  Considerations \label{PGLIE}}

Any lorentzian spacetime has a limiting spacetime which is a plane
wave, whereby the limit can be thought of as a ``first order
approximation'' along a null geodesic in that spacetime. The limiting
spacetime depends on the choice of null geodesic, and hence any
spacetime can have more than one Penrose limit. Let $(M,G)$
be a $d$-dimensional lorentzian spacetime. We can always introduce
local Penrose coordinates $(U,V,Y^i)$, $i=1,\dots,d-2$ in the
neighbourhood of a segment of a null geodesic $\gamma\subset M$ which
contains no conjugate points, whereby the metric assumes the form
\bea
G&=&2~\dd U~\dd V+\alpha\left(U,V,Y^k\right)~\dd V^2+2\beta_i
\left(U,V,Y^k\right)~\dd V~\dd Y^i+C_{ij}\left(U,V,
Y^k\right)~\dd Y^i~\dd Y^j \nn\\&&
\label{GPenrose}\eeq
with $C=(C_{ij})$ a positive definite symmetric matrix. This
coordinate system breaks down at the conjugate points where $C$
becomes degenerate, $\det C=0$. It singles out a twist-free null
geodesic congruence given by constant $V$ and $Y^i$, with $U$ being
the affine parameter along these geodesics. The geodesic $\gamma(U)$
is the one at $V=Y^i=0$.

In string backgrounds one also has to consider generic supergravity $p$-form
gauge potentials $A$ with $(p+1)$-form field strengths $F=\dd A$ in
order to compensate non-trivial spacetime curvature effects. The
G\"uven extension of the Penrose limit to general supergravity fields
shows that any supergravity background has plane wave limits which are
also supergravity backgrounds. It requires the local temporal gauge
choice
\beq
A_{Ui_1\cdots i_{p-1}}=0
\label{PGgaugechoice}\eeq
in order to ensure well-defined potentials in the limit. With this
gauge choice, which can always be achieved via a gauge transformation
$A\mapsto A+\dd\Lambda$ leaving the flux $F$ invariant, we can write
general potentials and field strengths in the neighbourhood of a null
geodesic $\gamma$ on $M$ as
\bea
A&=&a_{i_1\cdots i_{p-1}}\left(U,V,Y^k\right)~\dd V\wedge\dd
Y^{i_1}\wedge\cdots\wedge\dd Y^{i_{p-1}}+b_{i_1\cdots i_{p}}
\left(U,V,Y^k\right)~\dd Y^{i_1}\wedge\cdots\wedge\dd Y^{i_p}
\nn\\&&+\,c_{i_1\cdots i_{p-2}}\left(U,V,Y^k\right)
~\dd U\wedge\dd V\wedge\dd Y^{i_1}\wedge
\cdots\wedge\dd Y^{i_{p-2}} \ , \label{Anull}\\&&{~~~~}_{~~}^{~~}\nn\\
F&=&\frac{\partial b_{i_1\cdots i_p}\left(U,V,Y^k\right)}
{\partial U}~\dd U\wedge\dd Y^{i_1}\wedge\cdots\wedge\dd Y^{i_p}\nn\\&&+\,
\frac{\partial b_{i_2\cdots i_{p+1}}\left(U,V,Y^k\right)}
{\partial Y^{i_1}}~\dd Y^{i_1}\wedge\cdots\wedge\dd Y^{i_{p+1}}
\nn\\&&+\,\left(\frac{\partial a_{i_1\cdots i_{p-1}}
\left(U,V,Y^k\right)}{\partial U}+\frac{\partial c_{i_2\cdots
  i_{p-1}}\left(U,V,Y^k\right)}{\partial Y^{i_1}}\right)~
\dd U\wedge\dd V\wedge\dd Y^{i_1}\wedge\cdots\wedge\dd Y^{i_{p-1}}
\nn\\&&+\,\left(\frac{\partial b_{i_1\cdots i_p}
\left(U,V,Y^k\right)}{\partial V}-\frac{\partial a_{i_2\cdots
  i_p}\left(U,V,Y^k\right)}{\partial Y^{i_1}}\right)~
\dd V\wedge\dd Y^{i_1}\wedge\cdots\wedge\dd Y^{i_p} \ .
\label{Fnull}\eea

The Penrose-G\"uven limit starts with the one-parameter family of
local diffeomorphisms $\psi_\lambda:M\to M$, $\lambda\in\real$ defined
by a rescaling of the Penrose coordinates as
\beq
\psi_\lambda\left(U,V,Y^k\right)=\left(u,\lambda^2\,v,
\lambda\,y^k\right) \ .
\label{psiOmegadef}\eeq
One then defines new fields which are related to the original ones by
a diffeomorphism, a rescaling, and (in the case of potentials) possibly
a gauge transformation, by the well-defined limits
\bea
\widetilde{G}&=&\lim_{\lambda\to0}\,\lambda^{-2}\,\psi^*_\lambda G \ ,
\nn\\\widetilde{A}&=&\lim_{\lambda\to0}\,\lambda^{-p}\,\psi^*_\lambda A
\ , \nn\\\widetilde{F}&=&\lim_{\lambda\to0}\,\lambda^{-p}\,\psi^*_\lambda
F \ .
\label{PGlimitsdef}\eea
Due to (\ref{psiOmegadef}), the only functions in (\ref{GPenrose}),
(\ref{Anull}) and (\ref{Fnull}) which survive this limit are
$C_{ij}(u)=C_{ij}(U,0,0)$, $b_{i_1\cdots i_p}(u)=b_{i_1\cdots
  i_p}(U,0,0)$ and $c_{i_1\cdots i_{p-2}}(u)=c_{i_1\cdots
  i_{p-2}}(U,0,0)$, which are just the restrictions of the tensor fields $C$,
$b$ and $c$ to the null geodesic $\gamma$. Explicitly, we
obtain a pp-wave metric and supergravity fields in Rosen
coordinates $(u,v,y^i)$ as
\bea
\widetilde{G}&=&2~\dd u~\dd v+C_{ij}(u)~\dd y^i~\dd y^j \ ,
\label{GPGlim}\\&&{~~~~}_{~~}^{~~}\nn\\
\widetilde{A}&=&b_{i_1\cdots i_p}(u)~\dd y^{i_1}
\wedge\cdots\wedge\dd y^{i_p}+c_{i_1\cdots i_{p-2}}(u)~\dd u
\wedge\dd v\wedge\dd y^{i_1}\wedge\cdots\wedge\dd y^{i_{p-2}} \ ,
\label{APGlim}\\&&{~~~~}_{~~}^{~~}\nn\\
\widetilde{F}&=&\frac{\partial b_{i_1\cdots i_p}(u)}
{\partial u}~\dd u\wedge\dd y^{i_1}\wedge\cdots\wedge\dd y^{i_p} \ .
\label{FPGlim}\eea
The physical effect of this limit is to blow up a neighbourhood of the
null geodesic $\gamma$, and it can be thought of as an infinite volume
limit. We may set $c_{i_1\cdots i_{p-2}}(u)=0$ in (\ref{APGlim}) via
the local gauge transformation
\beq
\widetilde{A}~\longmapsto~\widetilde{A}+\dd\widetilde{\Lambda} \ , ~~
\widetilde{\Lambda}=-\left(\,\mbox{$\int^u$}\,\dd u'~c_{i_1\cdots
    i_{p-2}}(u'\,)\right)~\dd v\wedge\dd y^{i_1}\wedge\cdots\wedge
\dd y^{i_{p-2}} \ .
\label{APGcset0}\eeq

We are now interested in a smooth, local isometric embedding
$\imath:W\subset N\hookrightarrow M$ of a Lorentzian manifold $(N,g)$, also
possibly supported by non-trivial $p$-form fields, from an open subset
$W$ of $N$ onto a submanifold of $M$ in codimension $m$. Thus we require that
$\imath:W\to\imath(W)$ be a diffeomorphism in the induced ${\rm C}^\infty$
structure, and that the derivative map $\dd\imath_x:T_xN\to
T_{\imath(x)}M$ be injective for all $x\in W$. The Lorentzian metric
$g$ of $N$ is related to that of $M$ through the pull-back
\beq
g_x\bigl(\,\cdot\,,\,\cdot\,\bigr)=G_{\imath(x)}\bigl(\dd\imath(\,
\cdot\,)\,,\,\dd\imath(\,\cdot\,)\bigr)
\label{gNpullback}\eeq
on $T_xW\otimes T_xW\to\real$, and $p$-form fields $a$ on $N$ are similarly
related to those on $M$ by
\beq
a_x\bigl(\,\cdot\,,\ldots,\,\cdot\,\bigr)=A_{\imath(x)}
\bigl(\dd\imath(\,\cdot\,)\,,\,\ldots,\,\dd\imath(\,\cdot\,)\bigr)
\label{aNpullback}\eeq
on $\otimes^p\,T_xW\to\real$. In some cases of interest $N$ will correspond to
an embedded D-submanifold of the spacetime $M$, but this need not be
the case. In what follows we will usually write $\imath(N)$ for the
injection, with the implicit understanding that it need only be
defined with respect to a local coordinatization of the manifolds.

We shall examine situations in which the Penrose-G\"uven limit will
simultaneously induce the Penrose-G\"uven limits of the ambient spacetime and
of the embedded submanifold. This automatically restricts the types of
embeddings $\imath$ possible. We need to ensure the existence of a
null geodesic on $M$ which starts from a point in $\imath(N)$, whose
initial velocity is tangent to $\imath(N)$, and
which stays on $\imath(N)$ for all time. In certain instances we would
also have to impose further constraints on the geodesic restriction
depending on which target spacetime is desired from the Penrose
limit. A natural restriction is to assume that, in the Penrose
coordinates of (\ref{GPenrose}) adapted to a fixed null geodesic
$\gamma\subset M$ with $\imath(W)$ contained in the set of
conjugate-free points, $\imath$~is the inclusion map defined by fixing
$m$ of the transverse coordinates $Y^{j_1},\dots,Y^{j_m}$ so that
$\imath(N)$ is the intersection of $M$ with the
hypersurface $Y^{j_1},\dots,Y^{j_m}={\rm constant}$. This defines a
null geodesic on $\imath(N)$ which is embedded into the congruence
of null geodesics defined by (\ref{GPenrose}), and the family of such submanifolds
foliates the spacetime $M$ in its local adapted coordinate system.

In what follows we will simplify matters somewhat by taking the Penrose-G\"uven
limit of $(N,g)$ along the same null geodesic $\gamma$ as that used on
$(M,G)$. The embedded submanifold $\imath(N)$ is then the intersection of
$M$ with the hypersurface $Y^i=0~~\forall i\in{\mathcal
  I}:=\{j_1,\dots,j_m\}$. With the additional requirement that the metric $G$
restricts non-degenerately on $\imath(N)$, there is an orthogonal
tangent space decomposition
\beq
T_xM=T_xN\oplus T_xN^\perp \ .
\label{taudecomp}\eeq
With $\partial_i$ a local basis of tangent vectors dual to the one-forms
$\dd Y^i$ in an open neighbourhood of $x\in M$, the fibers of the
normal bundle $TN^\perp\to\imath(N)$ over the embedding are given by
$T_xN^\perp={\rm span}_{{\rm
    C}^\infty(W)}^{~}\{\partial_i\}_{i\notin{\mathcal I}}$. While
these requirements are not the most general ones that one can
envisage, they will suffice for the examples that we consider in this
paper.

Suppose now that we are given another local isometric embedding
$\widetilde{\imath}:\widetilde{W}\subset\widetilde{N}\hookrightarrow\widetilde{M}$ of
lorentzian manifolds $(\,\widetilde{N},\widetilde{g}\,)$ and
$(\,\widetilde{M},\widetilde{G}\,)$, again possibly in the presence of
other supergravity fields. We are interested in the conditions under
which the isometric embedding diagram
\begin{equation}
  \begin{CD}
    @.\\
    M @>\text{PGL}>>                      \widetilde{M}\\
    \text{$\imath$}@AAA @AAA\text{$\widetilde{\imath}$}\\
    N @>\text{PGL}>>                      \widetilde{N}\\
    @.
  \end{CD}
\label{isomembdiag}\end{equation}
commutes. The horizontal arrows denote Penrose-G\"uven limits (PGL) of
the respective lorentzian spacetimes, defined in the manner explained
above. In fact such a commutative diagram can only be written down
under very stringent symmetry constraints on the geodesic restrictions
of the transverse plane metric $C_{ij}$ of (\ref{GPenrose}) and
supergravity tensor field $b_{i_1\cdots i_p}$ of
(\ref{Anull}) in the directions normal to the embeddings
$\imath(N)\subset M$ and
$\widetilde{\imath}\,(\,\widetilde{N}\,)\subset\widetilde{M}$.

To formulate these symmetry requirements, let
$\widetilde{\imath}\,(\,\widetilde{N}\,)$ be realized as the
intersection of $\widetilde{M}$ with the hypersurface $y^i=0~~\forall
i\in\widetilde{\mathcal I}:=\{\,\tilde j_1,\dots,\tilde j_m\}$. This
realization of the isometric embedding on the right-hand side of
(\ref{isomembdiag}) is dictated by that on the left-hand side and the
Penrose limit. Consider the submanifold, also denoted
$\widetilde{N}$, defined by the intersection of $M$ with the
hypersurface $Y^i=0~~\forall i\in\widetilde{\mathcal I}$. Denoting the
normal bundle fibers as $T_{\tilde x}\widetilde{N}^{\,\perp}:={\rm
  span}^{~}_{{\rm C}^\infty(W)}\{\partial_i\}_{i\notin\widetilde{\mathcal I}}$ for
$\tilde x\in\widetilde{W}$, there is an orthogonal tangent space
decomposition
\beq
T_{\tilde x}M=T_{\tilde x}\widetilde{N}\oplus T_{\tilde x}
\widetilde{N}^{\,\perp}
\label{tautildedecomp}\eeq
analogous to (\ref{taudecomp}). Along the light-like null geodesic
$\gamma$, where $x=\tilde x=(U,0,0)$, we fix $p$ tangent vectors
$X,X_1,\dots,X_{p-1}\in T_{(U,0,0)}M$, and use the lorentzian metric
and $p$-form gauge potentials to define the linear transformations
\bea
G_{(U,0,0)}(X,\,\cdot\,)\,:\,T_{(U,0,0)}M&\longrightarrow&
T_{(U,0,0)}M \ , \label{Gelinmap}\\&&{~~~~}^{~~}_{~~}\nn\\
A_{(U,0,0)}(X_1,\dots,X_{p-1},\,\cdot\,)\,:\,T_{(U,0,0)}M&
\longrightarrow&T_{(U,0,0)}M \ .
\label{Aelinmap}\eea

The isometric embedding diagram (\ref{isomembdiag}) then commutes if,
for every collection of tangent vectors $X,X_1,\dots,X_{p-1}\in
T_{(U,0,0)}M$, the restrictions of the linear maps (\ref{Gelinmap})
and (\ref{Aelinmap}) to the corresponding orthogonal projections in
(\ref{taudecomp}) and (\ref{tautildedecomp}) agree,
\bea
\bigl.C_{(U,0,0)}(X,\,\cdot\,)\bigr|_{T_{(U,0,0)}N^\perp}&=&
\bigl.C_{(U,0,0)}(X,\,\cdot\,)\bigr|_{T_{(U,0,0)}
\widetilde{N}^{\,\perp}} \ , \label{Ccommcond}\\&&{~~~~}^{~~}_{~~}
\nn\\\bigl.b_{(U,0,0)}(X_1,\ldots,X_{p-1},\,\cdot\,)
\bigr|_{T_{(U,0,0)}N^\perp}&=&
\bigl.b_{(U,0,0)}(X_1,\ldots,X_{p-1},\,\cdot\,)\bigr|_{T_{(U,0,0)}
\widetilde{N}^{\,\perp}} \ ,
\label{bcommcond}\eea
in the following sense. The normal subspaces $T_{(U,0,0)}N^\perp$ and
$T_{(U,0,0)}\widetilde{N}^{\,\perp}$ of $T_{(U,0,0)}M$ are non-canonically
isomorphic as vector spaces. Fixing one such isomorphism, there is
then a one-to-one correspondence between normal vectors $\xi^\perp\in
T_{(U,0,0)}N^\perp$ and $\tilde\xi^\perp\in
T_{(U,0,0)}\widetilde{N}^{\,\perp}$, under which we require the
transverse plane metric and $p$-form fields to coincide,
$C_{(U,0,0)}(X,\xi^\perp)=C_{(U,0,0)}(X,\tilde\xi^\perp)$ and
$b_{(U,0,0)}(X_1,\ldots,X_{p-1},\xi^\perp)=b_{(U,0,0)}(X_1,\ldots,X_{p-1},\tilde
\xi^\perp)$. These symmetry conditions together ensure that the same
supergravity fields are induced on
$\widetilde{\imath}\,(\,\widetilde{N}\,)$ along the two different
paths of the diagram (\ref{isomembdiag}), i.e. that the
Penrose-G\"uven limit of $M$, along the null geodesic described above,
induces simultaneously the Penrose-G\"uven limit of $N$. We stress that
(\ref{Ccommcond}) and (\ref{bcommcond}) are required to simultaneously
hold under only a single such isomorphism of $m$-dimensional vector
spaces, and in all there are thus $\frac12\,m\,(m+1)$ such commuting
isometric embedding diagrams that can potentially be constructed for
appropriate plane wave profiles.

These symmetry conditions are essentially just the simple statement
that the restrictions of the embeddings $\imath$ and $\widetilde{\imath}$ to
the null geodesic $\gamma(U)$ are equivalent. Nevertheless, there are
many examples whereby the Penrose limit of the metric carries through
in (\ref{isomembdiag}), but not the G\"uven extension to generic
$p$-form supergravity fields, i.e. (\ref{Ccommcond}) can hold with
(\ref{bcommcond}) being violated. Conversely, there can be exotic
isometric embeddings whereby the transverse metric violates the
requirement (\ref{Ccommcond}), leading to target spacetimes with
distinct pp-wave profiles induced by the Penrose limit of essentially
the same lorentzian structure. An interesting example of this would
be a situation wherein the metric is not preserved, but the other
supergravity $p$-form fields are.

\subsection{Hpp-Wave Limits\label{PGLNW}}

Let us now specialize the analysis of the previous subsection to a broad
class of examples that are important to the analysis of this
paper. Consider a connected Lie group $\mathcal{G}$ possessing a bi-invariant
metric. On the Lie algebra $\mathfrak g$ of $\mathcal{G}$, this induces an
invariant, non-degenerate inner product
$\langle\,\cdot\,,\,\cdot\,\rangle:\mathfrak{g}\times\mathfrak{g}\to\real$.
We will also write $\mathcal{G}$ for the group manifold.

Symmetric D-branes wrapping submanifolds $D\subset \mathcal{G}$ are
described by twisted conjugacy classes of the group. Let
$\Omega:\mathfrak{g}\to\mathfrak{g}$ be an outer automorphism of the
Lie algebra of $\mathcal{G}$ preserving its inner product
$\langle\,\cdot\,,\,\cdot\,\rangle$, and let $\omega:\mathcal{G}\to
\mathcal{G}$ be the corresponding Lie group automorphism. The map
$\omega$ is an isometry of the bi-invariant metric on $\mathcal G$, and so it
generates an orbit of any point $g\in \mathcal{G}$ under the
twisted adjoint action of the group, $g\mapsto{\rm
  Ad}_g^\omega(h)=h\,g\,\omega(h^{-1})$, $h\in \mathcal{G}$. We may
thereby identify $D$ with such an orbit as
\beq
D={\mathcal C}_g^\omega=\bigl\{h\,g\,\omega\left(h^{-1}\right)\,
\bigm|\,h\in \mathcal{G}\bigr\} \ .
\label{Dtwistconj}\eeq
This is essentially the generic situation, since additional inner
automorphisms would give rise to twisted conjugacy classes which are
simply translates of one another. In this way, to each such $\omega$
we can associate an equivalence class of D-branes foliating $\mathcal{G}$.

Since the metric on $\mathcal{G}$ is bi-invariant, the twisted conjugacy class
(\ref{Dtwistconj}) may be exhibited as a homogeneous space
\beq
{\mathcal C}_g^\omega=\mathcal{G}\,/\,{\mathcal Z}_g^\omega
\label{twistconjhom}\eeq
where ${\mathcal Z}_g^\omega\subset \mathcal{G}$ is the stabilizer subgroup of the point
$g$ under the twisted adjoint action of $\mathcal{G}$,
\beq
{\mathcal Z}_g^\omega=\bigl\{h\in \mathcal{G}\,\bigm|\,h\,g\,\omega\left(h^{-1}
\right)=g\bigr\} \ .
\label{stabdef}\eeq
By this homogeneity, it will always suffice to determine the geometry
(and any $\mathcal{G}$-invariant fluxes) at a single point, as all other points
are related by the twisted adjoint action of the group, which is an
isometry. A natural physical assumption is that the bi-invariant
metric of $\mathcal{G}$ restricts non-degenerately to the twisted conjugacy
classes (\ref{twistconjhom}). The normal bundle over the D-submanifold
then has fibers given by
\beq
\left(T_g\mathcal{C}_g^\omega\right)^\perp=T_g\mathcal{Z}_g^\omega
\label{twistconjnorm}\eeq
with $T_g\mathcal{C}_g^\omega\cap T_g\mathcal{Z}_g^\omega=\{0\}$, so
that there is an orthogonal direct sum decomposition of the tangent
bundle of $\mathcal{G}$,
\beq
T_g\mathcal{G}=T_g\mathcal{C}_g^\omega\oplus T_g\mathcal{Z}_g^\omega \ .
\label{TgGdecomp}\eeq
These D-submanifolds are also NS-supported with $B$-field
\beq
B_{g_0}=-\bigl\langle\dd h\,h^{-1}\,,\,{\rm Ad}_g\,\Omega\left(
\dd h\,h^{-1}\right)\bigr\rangle
\label{Bfieldg0}\eeq
at $g_0=h\,g\,\omega(h^{-1})\in{\mathcal C}_g^\omega$.

There are two ways in which the Penrose limit may be achieved within
this setting. In both instances it can be understood as an
In\"on\"u-Wigner group contraction of $\mathcal{G}$ whose limit
$\widetilde{\mathcal{G}}$ is a non-compact, non-semisimple Lie group
admitting a bi-invariant metric. In the first case we assume $\mathcal
G$ is simple and consider a one-parameter subgroup
$\mathcal{H}\subset\mathcal{G}$, which is necessarily geodesic
relative to the bi-invariant metric. If it is also null, then it gives
rise to a null geodesic and hence the Penrose limit can be taken as
prescribed in the previous subsection. In the second case we consider
a product group $\mathcal{G}=\mathcal{G}'\times\mathcal{H}$, where
$\mathcal{G}'$ is a simple Lie group with a bi-invariant metric and
$\mathcal{H}\subset \mathcal{G}'$ is a compact subgroup, with Lie subalgebra
$\mathfrak{h}\subset\mathfrak{g}'$, which inherits a bi-invariant
metric from $\mathcal{G}$ by restriction. The product group
$\mathcal{G}'\times \mathcal{H}$ then carries the bi-invariant product
metric corresponding to the bilinear form
$\langle\,\cdot\,,\,\cdot\,\rangle\oplus
(-\langle\,\cdot\,,\,\cdot\,\rangle|_{\mathfrak{h}})$ on
$\mathfrak{g}'\oplus\mathfrak{h}$. The submanifold $\mathcal{H}\subset
\mathcal{H}\times\mathcal{H}\subset\mathcal{G}'\times \mathcal{H}$
given by the diagonal embedding is a Lie subgroup, and hence it is
totally geodesic and maximally isotropic. The (generalized) Penrose
limit of $\mathcal{G}'\times\mathcal{H}$ along $\mathcal{H}$ thereby
yields a non-semisimple Lie group with a bi-invariant metric. The
related group contraction can also be understood as an infinite volume limit
along the compact directions of $\mathcal{G}'\times \mathcal{H}$,
giving a semi-classical picture of the string dynamics in this
background.

We may now attempt to specialize the isometric embedding diagram
(\ref{isomembdiag}) to the diagram
\begin{equation}
  \begin{CD}
    @.\\
    \mathcal{G} @>\text{PGL}>>                      \widetilde{\mathcal{G}}\\
    \text{$\imath$}@AAA @AAA\text{$\widetilde{\imath}$}\\
    \mathcal{C}_g^\omega @>\text{PGL}>> \mathcal{C}_{\tilde g}^{\tilde\omega}\\
    @.
  \end{CD}
\label{twistconjembdiag}\end{equation}
specifying the Penrose-G\"uven limit between twisted D-branes. To
formulate the symmetry conditions (\ref{Ccommcond}) and
(\ref{bcommcond}) within this algebraic setting, we identify the
space of tangent vectors (\ref{twistconjnorm}) with the Lie algebra of
the stabilizer subgroup (\ref{stabdef}) given by
\beq
\mathfrak{z}_g^\omega=\bigl\{X\in\mathfrak{g}\,\bigm|\,g^{-1}\,X\,g=
\Omega(X)\bigr\} \ .
\label{stabLiealg}\eeq
For any $h\in\mathcal{H}$ and any Lie algebra element
$X\in\mathfrak{g}$, the isometric embedding diagram
(\ref{twistconjembdiag}) then commutes if and only if
\bea
\bigl.\langle X,\,\cdot\,\rangle\bigr|_{\mathfrak{z}_h^\omega}&=&
\bigl.\langle X,\,\cdot\,\rangle\bigr|_{\mathfrak{z}_h^{\tilde\omega}}
\ , \label{innprodtwistcomm}\\{~~~~}_{~~}^{~~}\nn\\\bigl.B_{h_0}(X,\,\cdot\,)\bigr|_{
\mathfrak{z}_h^\omega}&=&\bigl.B_{h_0}(X,\,\cdot\,)\bigr|_{
\mathfrak{z}_h^{\tilde\omega}} \ ,
\label{Bfieldtwistcomm}\eea
with the analogous restrictions on higher-degree $p$-form fields.

\subsection{NW Limits\label{ApplNW}}

We will now apply these considerations to examine the possible
D-embeddings of Nappi-Witten spacetimes. We start with the
six-dimensional lorentzian manifold $M=\AdS_3\times\Sphere^3$
describing the near horizon geometry of the self-dual string in $d=6$
$(2,0)$ supergravity. We assume that both factors share a common radius
of curvature $R$. We can embed $\AdS_3\times\Sphere^3$ in the pseudo-euclidean
space $\eucl^{2,6}$ as the intersection of the two quadrics
\bea
\left(x^0\right)^2+\left(x^1\right)^2-\left(x^2\right)^2-\left(x^3\right)^2
&=&R^2\ , \label{AdS3quadric}\\&&{~~~~}^{~~}_{~~}\nn\\
\left(x^4\right)^2+\left(x^5\right)^2+\left(x^6\right)^2+\left(x^7\right)^2
&=&R^2
\label{S3quadric}\eea
with the induced metric. An explicit parametrization is given by
\beq
\begin{matrix}
x^0&=&R\,\sqrt{1+r^2}\,\sin\tau ~~~~ &,& ~~~~
x^1&=&R\,\cos\tau\,\cosh\beta \ , \\x^2&=&R\,\cos\tau\,\sinh\beta
~~~~ &,& ~~~~ x^3&=&R\,r\,\sin\tau \ , \\x^4&=&R\,\cos\phi\,\sin\psi
~~~~ &,& ~~~~ x^5&=&R\,\chi\,\sin\phi \ , \\x^6&=&R\,\sqrt{1-\chi^2}\,
\sin\phi ~~~~ &,& ~~~~ x^7&=&R\,\cos\phi\,\cos\psi \ .
\end{matrix}
\label{AdS3S3param}\eeq
In these coordinates the metric, NS--NS $B$-field and three-form flux
on $\AdS_3\times\Sphere^3$ are given by
\bea
\mbox{$\frac1{R^2}$}\,
G&=&-\dd\tau^2+\sin^2\tau\,\left(\frac{\dd r^2}{1+r^2}+\left(1-r^2
\right)~\dd\beta^2\right)\nn\\&&+\,\dd\phi^2+\sin^2\phi\,\left(
\frac{\dd\chi^2}{1-\chi^2}+\left(1-\chi^2\right)~\dd\psi^2\right) \ ,
\label{AdS3S3metric}\\&&{~~~~}^{~~}_{~~}\nn\\
-\mbox{$\frac1{2R^2}$}\,H&=&\cos^2\tau~\dd\tau\wedge
\dd r\wedge\dd\beta+\sin^2\phi~\dd\phi\wedge\dd\chi\wedge\dd\psi \ ,
\label{AdS3S3flux}\\&&{~~~~}_{~~}^{~~}\nn\\-\mbox{$\frac2{R^2}$}\,
B&=&(\sin2\tau+2\tau)~\dd r\wedge\dd\beta+
(\sin2\phi-2\phi)~\dd\chi\wedge\dd\psi \ .
\label{AdS3S3Bfield}\eea

Viewing $M$ as the group manifold of the Lie group ${\rm
  SU}(1,1)\times{\rm SU}(2)$ with its usual bi-invariant metric, its
  embedded submanifolds wrapped by maximally symmetric D-branes are given
  by twisted conjugacy classes of the group. Here we will focus on the
  family of D3-branes which are isometric to $N=\AdS_2\times\Sphere^2$
  and are given by the intersections of the hyperboloids
  (\ref{AdS3quadric},\ref{S3quadric}) with the affine hyperplanes
  $x^3,x^5={\rm constant}$. In this way we may
  exhibit a foliation of $\AdS_3\times\S^3$ consisting of twisted
  D-branes, each of which is isometric to
  $\AdS_2\times\S^2$. Within this family, the intersection of
  $\AdS_3\times\S^3$ with the hyperplane defined by $x^3=x^5=0$ is
  special for a variety of reasons. It
  corresponds to the fixed point set of the reflection isometry of
  $\eucl^{2,6}$ defined by $x^3\mapsto-x^3$, $x^5\mapsto-x^5$ while
  leaving fixed all other coordinates. This isometry preserves the
  embedding (\ref{AdS3quadric},\ref{S3quadric}) and hence induces an
  isometry of $\AdS_3\times\S^3$. The metric (\ref{AdS3S3metric})
  restricts nondegenerately to the corresponding $\AdS_2\times\S^2$
  submanifold, which is thereby totally geodesic and has equal radii of
  curvature $R$.

When we consider such embedded D-submanifolds, we should also add to
the list of supergravity fields
(\ref{AdS3S3metric})--(\ref{AdS3S3Bfield}) the
constant ${\rm U}(1)$ gauge field flux
\beq
F=\alpha'\,(q~\dd r\wedge\dd\beta+p~\dd\chi\wedge\dd\psi) \ ,
\label{AdS3S3U1flux}\eeq
where $p,q\in\zed$ are magnetic and electric ``monopole'' numbers, with
$0<p,q<R^2/\alpha'$, and $\alpha'$ is the string slope. This two-form
has quantized periods around $\AdS_2\times\Sphere^2$ submanifolds of
$M$ which prevent the wrapped D3-branes from collapsing. The quantity
${\mathcal F}=B+F$ is invariant under two-form gauge transformations
of the $B$-field, but it is only the flux (\ref{AdS3S3U1flux}) which
is quantized. The addition of such worldvolume electric fields
proportional to the volume form of $\AdS_2$ along with worldvolume
magnetic fields proportional to the volume form of $\S^2$ still
preserves supersymmetry. The sources of such fluxes are $(p,q)$
strings connecting the D3-branes to an NS5/F1 black string background
in the near horizon region.

Let us now consider the Penrose-G\"uven limit of $M$ which produces the
six-dimensional Nappi-Witten spacetime $\widetilde{M}=\NW_6$. As Lie
groups, the Penrose limit can be interpreted as an In\"on\"u-Wigner
group contraction of ${\rm SU}(1,1)\times{\rm SU}(2)$ onto
$\mathcal{N}_6$. Geometrically, it can
be achieved along any null geodesic which has a
non-vanishing velocity component tangent to the sphere $\S^3$. For
this, we change coordinates in the $(\tau,\phi)$ plane to
\beq
U=\phi+\tau \ , ~~ V=\mbox{$\frac12$}\,(\phi-\tau) \ ,
\label{UVAdS3S3def}\eeq
and relabel the remaining coordinates as
\beq
Y^1=r \ , ~~ Y^2=\beta \ , ~~ Y^3=\chi \ , ~~ Y^4=\psi \ .
\label{Yrelabel}\eeq
This enables us to represent the fields in
(\ref{AdS3S3metric})--(\ref{AdS3S3Bfield}) in the adapted coordinate
forms (\ref{GPenrose}), (\ref{Anull}) and (\ref{Fnull}),  which
thereby exhibits $\frac\partial{\partial
  U}=\frac12\,(\frac\partial{\partial\phi}+\frac\partial{\partial\tau})$
as the null geodesic vector field with $G(\frac\partial{\partial
  U},\frac\partial{\partial U})=0$. After the Penrose-G\"uven limit,
the metric and Neveu-Schwarz fields along the geodesic $\gamma(U)$ are
given by
\bea
\mbox{$\frac1{R^2}$}\,\widetilde{G}&=&2~\dd u~\dd v+\sin^2\mbox{$
\frac u2$}~\dd{\mbf y}^2 \ , \label{NW6tildemetric}\\&&{~~~~}^{~~}_{~~}\nn\\
\mbox{$\frac1{R^2}$}\,\widetilde{H}&=&\cos^2\mbox{$\frac u2$}~\dd u
\wedge\dd y^1\wedge\dd y^2-\sin^2\mbox{$\frac u2$}~\dd u\wedge\dd y^3
\wedge\dd y^4 \ , \label{NW6tildeflux}\\&&{~~~~}^{~~}_{~~}\nn\\
\mbox{$\frac4{R^2}$}\,
\widetilde{B}&=&-(u+\sin u)~\dd y^1\wedge\dd y^2+(u-\sin u)~
\dd y^3\wedge\dd y^4 \ ,
\label{NW6tildeBfield}\eea
with ${\mbf y}^\top=(y^i)\in\real^4$. At this stage it is convenient to
transform from Rosen coordinates to Brinkmann coordinates as
\bea
u&=&x^+ \ , \nn\\v&=&x^-+\mbox{$\frac12$}\,\bigl(|z|^2+|w|^2\bigr)\,
\cot\mbox{$\frac{x^+}2$} \ , \nn\\y^1+\ii y^2&=&w\,\csc\mbox{$\frac{x^+}2$} \ ,
\nn\\y^3+\ii y^4&=&z\,\csc\mbox{$\frac{x^+}2$} \ .
\label{BrinkfromAdS3S3}\eea
It is then straightforward to compute that one recovers the standard
NS-supported geometry of $\NW_6$, with supergravity fields
$\frac1{R^2}\,\widetilde{G}=\dd s_6^2$,
$\frac1{2R^2}\,\widetilde{H}=H_6$ and $\frac2{R^2}\,\widetilde{B}=B_6$
given by (\ref{NW6metricBrink},\ref{NW6Bfield}).

The foliating hyperplanes $w=w_0\in\complex$ isometrically embed
$\NW_4$ in $\NW_6$ with its standard geometry
(\ref{NW4metricBrink},\ref{NS2formBrink}) and zero-point energy
$b=-\frac14\,|w_0|^2$. With this, one can also
straightforwardly compute the Penrose-G\"uven limit of the worldvolume flux
(\ref{AdS3S3U1flux}) induced on $\NW_6$, and one finds
\beq
F_6~:=~\widetilde{F}&=&2\ii\alpha'\,
\csc^2x^+\,\left[\cot x^+~\dd x^+\wedge\bigl(\,p\,(z~\dd z-\overline{z}~
\dd\overline{z})+q\,(w~\dd w-\overline{w}~\dd\overline{w}\,)\bigr)
\right.\nn\\&&\bigl.+\,p~
\dd z\wedge\dd\overline{z}+q~\dd w\wedge\dd\overline{w}\,\bigr] \ .
\label{NW6flux}\eea
This induces a worldvolume flux $F_4:=F_6|_{w=w_0}$
on $\NW_4$. The submanifold $\NW_4\subset\NW_6$ is in fact a maximally
symmetric lorentzian D3-brane, while $\NW_6$ itself is wrapped by
symmetric spacetime filling D5-branes. We will describe these and
other symmetric D-embeddings in more detail later on.

Let us now examine the isometric embedding of the totally geodesic
$\AdS_2\times\S^2$ D-brane in $\AdS_3\times\S^3$, which may be
defined as the hyperplane $Y^1=Y^3=0$. The geodetic property
ensures that the Penrose limit of $\AdS_3\times\S^3$ induces that of
$\AdS_2\times\S^2$, which yields the Cahen-Wallach symmetric space
$\CW_4$. However, using the general analysis of the
previous section, one can see immediately that the G\"uven extension
breaks down. In this case the normal bundle fiber
$T_{(U,0,0)}N^\perp$ is locally spanned by the vector fields
$\partial_2,\partial_4$, while $T_{(U,0,0)}\widetilde{N}^{\,\perp}$ is
spanned by $\partial_3,\partial_4$. From the local form of the
$B$-field (\ref{NW6tildeBfield}), we see for instance that
$b_{12}(u)=-\frac12\,(u+\sin u)$ and $b_{13}(u)=b_{14}(u)=0$ are
clearly distinct. There is thus no gauge transformation such that
(\ref{bcommcond}) is satisfied. On the other hand, since the
transverse plane metric $C_{ij}(u)=\sin^2\frac u2~\delta_{ij}$
in (\ref{NW6tildemetric}) is proportional to the identity,
(\ref{Ccommcond}) is trivially satisfied and so the Penrose limit of
the $\AdS_2\times\S^2$ metric coincides with that of $\CW_4$. This can
also be checked by explicit calculation. Indeed, the restrictions of
the Neveu-Schwarz fields (\ref{AdS3S3flux},\ref{AdS3S3Bfield}) and the
${\rm U}(1)$ flux (\ref{AdS3S3U1flux}) to constant $Y^1$ and $Y^3$
both vanish, and hence so do their Penrose-G\"uven limits.

One may think that this problem could be rectified by choosing
an alternative embedding of $\AdS_2\times\S^2$ in which the $B$-field
is non-vanishing, such as that with $\tau,\phi={\rm
  constant}$. However, the induced NS-flux (\ref{AdS3S3flux}) still
vanishes, and indeed this leads to the standard fuzzy geometry of
$\AdS_2\times\S^2$. The Penrose limit of $\AdS_2\times\S^2$ could be
taken now in the adapted coordinates $U=\chi+r$, $V=\frac12\,(\chi-r)$,
$Y^1=\beta$, $Y^2=\psi$, and after another suitable change to
Brinkmann coordinates it leads to the anticipated $\CW_4$
geometry. However, now the only non-vanishing components of the
$B$-field are $B_{Ui}$ and $B_{Vi}$, which both necessarily vanish
after the Penrose-G\"uven limit is taken. Similarly, one easily finds
an induced worldvolume flux which differs from $F_4$ obtained via
pull-back of (\ref{NW6flux}). We are forced to conclude that there is
no way to obtain the fully NS-supported geometry of $\NW_4$ as a plane
wave limit from that of $\AdS_2\times\S^2$.

\subsection{Embedding Diagrams for NW Spacetimes\label{DiagNW}}

Let us now describe two simple and obvious remedies to the problem
which we have raised in the previous subsection. The first one modifies the
embedding $\widetilde{\imath}$ on the right-hand side of the diagram
(\ref{isomembdiag}) to be the intersection of $\widetilde{M}=\NW_6$
with the hyperplane $y^1=y^3=0$, so that
 $T_{(U,0,0)}N^\perp=T_{(U,0,0)}\widetilde{N}^{\,\perp}$ and the
conditions (\ref{Ccommcond}) and (\ref{bcommcond}) are always
trivially satisfied. Now the pull-backs of the Neveu-Schwarz fields
(\ref{NW6tildeflux},\ref{NW6tildeBfield}) vanish, as does the
pull-back of the ${\rm U}(1)$ flux (\ref{NW6flux}), while the pull-back
of the metric (\ref{NW6tildemetric}) is still the standard metric on
$\CW_4$. This embedding thereby preserves the basic Cahen-Wallach
structure $\CW_4\subset\NW_6$, and the vanishing of the other
supergravity form fields on $\CW_4$ is indeed induced now by the
Penrose-G\"uven limit from $\AdS_2\times\S^2$. We may thereby write
the commuting embedding diagram
\begin{equation}
  \label{CW4commdiag}
  \begin{CD}
    @.\\
    \AdS_3 \times \S^3             @>\text{PGL}>> \NW_6\\
    \text{$\imath$}@AAA @AAA\text{$\widetilde{\imath}$}\\
    \AdS_2 \times \S^2             @>\text{PGL}>> \CW_4\\
    @.
  \end{CD}
\end{equation}
which describes the Penrose-G\"uven limit between {\it commutative},
maximally symmetric lorentzian D3-branes in $\AdS_3\times\S^3$ and the
six-dimensional Nappi-Witten spacetime $\NW_6$. Due to the vanishing
of the worldvolume flux and the fact that $\pi_2(\S^3)=0$, these
D-branes can be unstable. They may either completely shrink to zero
size corresponding to point-like D-instantons, or to D-strings induced
at a point in $\S^3$ with worldvolume geometries
$\AdS_2\subset\AdS_3$ on the left-hand side of the diagram
(\ref{CW4commdiag}).

Alternatively, we may choose to modify the embedding $\imath$ on the
left-hand side of the diagram (\ref{isomembdiag}) to be the
intersection of $M=\AdS_3\times\S^3$ with the hyperplane
$Y^1=Y^2=0$. Again
$T_{(U,0,0)}N^\perp=T_{(U,0,0)}\widetilde{N}^{\,\perp}$ and so the
conditions (\ref{Ccommcond}) and (\ref{bcommcond}) are trivially
satisfied. This hyperplane corresponds to the intersection of the
hyperboloid (\ref{AdS3quadric}) with $x^2=x^3=0$, while
(\ref{S3quadric}) is left unchanged. It thereby defines a totally
geodesic embedding of $\S^{1,0}\times\S^3$ in
$\AdS_3\times\S^3$. This does not define a twisted conjugacy class of
the Lie group ${\rm SU}(1,1)\times{\rm SU}(2)$ and so does not
correspond to a symmetric D-brane. Instead, it arises in the near
horizon geometry of a stack of NS5-branes. The pull-backs of the NS--NS fields
(\ref{AdS3S3flux},\ref{AdS3S3Bfield}) are non-vanishing, and the null
geodesic defined by (\ref{UVAdS3S3def}) spins along an equator of the
sphere~$\S^3$. The Penrose-G\"uven limit thus induces the complete
NS-supported geometry of $\NW_4$, and it can be thought of as a group
contraction of ${\rm U}(1)\times{\rm SU}(2)$ along ${\rm U}(1)$ onto
$\mathcal{N}_4$. We may thereby write the commuting embedding diagram
\begin{equation}
  \label{NW4commdiag}
  \begin{CD}
    @.\\
    \AdS_3 \times \S^3             @>\text{PGL}>> \NW_6\\
    \text{$\imath$}@AAA @AAA\text{$\widetilde{\imath}$}\\
    \S^{1,0}\times\S^3         @>\text{PGL}>> \NW_4\\
    @.
  \end{CD}
\end{equation}
describing {\it noncommutative} branes. Although
$\S^{1,0}\times\S^3\subset\AdS_3\times\S^3$ is not a maximally
symmetric D-embedding, $\NW_4\subset\NW_6$ is a symmetric lorentzian
D3-brane supported by non-trivial worldvolume fields. It is stabilized
against decay by its quantized worldvolume flux $F_4$. Similarly,
while $\AdS_3\times\S^3$ is not a symmetric D-brane, $\NW_6$ is the
worldvolume of a spacetime-filling twisted D5-brane.

\newsection{Quantization of the Twisted Heisenberg Algebra
  \label{QNWA}}

In this section we will begin working our way towards describing how
the worldvolumes of D-branes in the Nappi-Witten spacetime $\NW_4$ are
deformed by the non-trivial $B$-field background. Our starting point
is at the algebraic level. We will consider the deformation
quantization of the dual $\mathfrak{n}_6^*$ to the Lie
algebra $\mathfrak{n}^{~}_6$ of Killing vectors of $\NW_4$. By setting
$\overline{\P}^{\,\pm}=0$, this will also automatically give the
quantization of the four-dimensional Nappi-Witten dual $\mfn_4^*$. Naively,
one may think that the easiest way to carry this out is to exploit
the isometric embedding diagram (\ref{NW4commdiag}) and compute star
products on the NW spacetimes by taking the Penrose limits of the
standard ones on $\S^3$ and $\AdS_3$ (or equivalently by contracting
the standard quantizations of the Lie algebras ${\rm su}(2)$ and ${\rm
  sl}(2,\real)$). However, some quick calculations show that the
induced star-products obtained in this way are divergent in the
infinite volume limit $\lambda\to0$, and the reason why is
simple. While the standard In\"on\"u-Wigner contractions hold at the
level of the Lie algebras, they need not necessarily map the
corresponding universal enveloping algebras, on
which the quantizations are performed. We must therefore resort to a
more direct approach to quantizing the Nappi-Witten spacetime.

For notational ease, we will write the algebra $\mathfrak{n}_6$ in the
generic form
\beq
[\X_a,\X_b]=\theta\,C_{ab}^{~~c}\,\X_c \ ,
\label{n6genform}\eeq
where $\X^\top=(\X_a):=(\J,\T,\P^\pm,\overline{\P}^{\,\pm}\,)$ are the
generators of $\mathfrak{n}_6$ and the structure constants
$C_{ab}^{~~c}$ can be gleamed off from (\ref{NW4isomalg}). We have
inserted a constant $\theta\in\real$ which will play the role of a
deformation parameter. The algebra (\ref{n6genform}) can be regarded
as a formal deformation quantization of the Kirillov-Kostant Poisson bracket on
$\mathfrak{n}_6^*$ in the standard coadjoint orbit method. Let us
identify $\mathfrak{n}_6^*$ as the vector space $\real^6$ with basis
$\X_a^*:=\langle X_a,\,\cdot\,\rangle:\mfn_6\to\real$ dual to the
$\X_a^{~}$, where $\langle\,\cdot\,,\,\cdot\,\rangle$ is the obvious
extension of the bilinear form (\ref{NW4innerprod}) to $\mathcal{N}_6$
with $b=0$. In the algebra of polynomial functions
$\complex(\mathfrak{n}_6^*)=\complex(\real^6)$, we may then identify
the generators $\X_a$ themselves with the coordinate functions
\beq
\X_\J(\mx)=x_\T \ , ~~ \X_\T(\mx)=x_\J \ , ~~ \X_{\Q^\pm}(\mx)=
2x_{\Q^\mp}
\label{Xacoordfns}\eeq
for any $\mx\in\mathfrak{n}_6^*$ with component $x_a$ in the $\X_a^*$
direction, where we have introduced the short-hand notation
$(\Q^\pm)^\top:=(\P^\pm,\overline{\P}^{\,\pm}\,)$. These functions
generate the whole coordinate algebra and their Poisson bracket $\Pi$
is defined by
\beq
\Pi(\X_a,\X_b)(\mx)=\mx\bigl([\X_a,\X_b]\bigr) ~~~~ \forall\mx\in
\mathfrak{n}_6^* \ .
\label{KKXadef}\eeq
Therefore, when viewed as functions on $\real^6$ the Lie algebra
generators have a Poisson bracket given by the Lie bracket, and their
quantization is provided by (\ref{n6genform}) with deformation
parameter~$\theta$. Heuristically, this can be regarded as a
quantization of the ``momentum space'' of the Nappi-Witten spacetime,
and one of the goals of the subsequent sections will be to translate this
into a noncommutative deformation of $\NW_4$ itself. In this section
we will explore various aspects of this quantization and derive
several (equivalent) star products on $\mfn_6^*$.

\subsection{Gutt Products\label{StarProds}}

The completion of the space of polynomials $\complex(\mfn_6^*)$ is the
algebra ${\rm C}^\infty(\mathfrak{n}_6^*)$ of smooth functions on
$\mathfrak{n}_6^*$. There is a natural way to construct a star-product
  on the cotangent bundle $T^*\mathcal{N}_6$, which naturally induces
  an associative product on ${\rm C}^\infty(\mfn_6^*)$. This induced
  product is called the Gutt product. The Poisson bracket defined by
  (\ref{KKXadef}) naturally extends to a Poisson structure
  $\Pi:\CC^\infty(\mfn_6^*)\times\CC^\infty(\mfn_6^*)\to\CC^\infty(\mfn_6^*)$
  defined by the Kirillov-Kostant bi-vector
\beq
\Pi=\mbox{$\frac12$}\,C_{ab}^{~~c}\,x_c\,\partial^a\wedge\partial^b \
,
\label{KKbivector}\eeq
where $\partial^a:=\partial/\partial x_a$. The Gutt product constructs a
  quantization of this Poisson structure. It is equivalent to the
  Kontsevich star-product in this case, and by construction it keeps
  that part of the Kontsevich formula which is associative. In
  general, the Gutt and Kontsevich deformation quantizations are only
  identical for nilpotent Lie algebras.

The algebra $\complex(\mfn_6^*)$ of polynomial functions on the dual
to the Lie algebra is naturally isomorphic to the symmetric tensor algebra
$S(\mfn_6)$ of $\mfn_6$. By the Poincar\'e-Birkhoff-Witt theorem,
there is a natural isomorphism $\Delta:S(\mfn_6)\to
U(\mfn_6)$. Using the above identifications, this extends to a
canonical isomorphism
\beq
\Delta\,:\,\CC^\infty\left(\real^6\right)~\longrightarrow~U(\mfn_6)^c
\label{Sigmaiso}\eeq
defined by specifying an ordering for the elements of the
basis of monomials for $S(\mfn_6)$, where $U(\mfn_6)^c$ denotes a
formal completion of the complexified universal enveloping algebra
$U(n_6)\otimes\complex$. Denoting this ordering by
$\NO\,\cdot\,\NO$, we may write this isomorphism symbolically as
\beq
\Delta(x_{a_1}\cdots x_{a_n})=\NO\,\X_{a_1}\cdots\X_{a_n}\,\NO \ .
\label{Sigmasymbol}\eeq
The original Gutt construction takes the isomorphism
$\Delta$ on $S(\mfn_6)$ to be symmetrization of monomials. In this
case $\Delta(f)$ is usually called the Weyl symbol of
$f\in\CC^\infty(\real^6)$ and the symmetric ordering $\NO\,\cdot\,\NO$
of symbols $\Delta(f)$ is called Weyl ordering. In this subsection we
shall work with three natural orderings appropriate to the
Nappi-Witten algebra.

The isomorphism (\ref{Sigmaiso}) can be used to transport the
algebraic structure on the universal enveloping algebra $U(\mfn_6)$ of
$\mfn_6$ to the algebra of smooth functions on $\mfn_6^*\cong\real^6$
to give the star-product defined by
\beq
f\star g:=\Delta^{-1}\bigl(\,\NO\,\Delta(f)\cdot\Delta(g)\,\NO
\,\bigr) \ , ~~ f,g\in\CC^\infty\left(\real^6\right) \ .
\label{fstargSigma}\eeq
The product on the right-hand side of the formula (\ref{fstargSigma})
is taken in $U(\mfn_6)$, and it follows that $\star$ defines an
associative, noncommutative product. Moreover, it represents a
deformation quantization of the Kirillov-Kostant Poisson structure on
$\mfn_6^*$, in the sense that
\beq
[x,y]_\star:=x\star y-y\star x=\theta\,\Pi(x,y) \ , ~~ x,y\in\complex_{(1)}
\left(\mfn_6^*\right) \ ,
\label{xyPoisson}\eeq
where $\complex_{(1)}(\mfn_6^*)$ is the subspace of homogeneous
polynomials of degree~$1$ on $\mfn_6^*$. In particular, the Lie
algebra relations (\ref{n6genform}) are reproduced by star-commutators
of the coordinate functions as
\beq
[x_a,x_b]_\star=\theta\,C_{ab}^{~~c}\,x_c \ ,
\label{xaxbstarcomm}\eeq
in accordance with the definition (\ref{KKXadef}).

Let us now describe how to write the star-product (\ref{fstargSigma})
explicitly in terms of a bi-differential operator
$\hat{\mathcal{D}}:\CC^\infty(\mfn_6^*)\times\CC^\infty(\mfn_6^*)\to
\CC^\infty(\mfn_6^*)$. Using the Kirillov-Kostant Poisson structure as
before, we identify the generators of $\mfn_6$ as coordinates on
$\mfn_6^*$. This establishes, for small $s\in\real$, a one-to-one
correspondence between group elements $\e^{s\,\X}$, $\X\in\mfn_6$ and
functions $\e^{s\,x}$ on $\mfn_6^*$. Pulling back the group
multiplication of elements $\e^{s\,\X}\in\mathcal{N}_6$ via this
correspondence induces a bi-differential operator $\hat{\mathcal{D}}$
acting on the functions $\e^{s\,x}$. Since these functions separate
the points on $\mfn_6^*$, this extends to a star-product on the whole
of $\CC^\infty(\mfn_6^*)$.

To apply this construction explicitly, we will use the following trick
which will be useful for later considerations. By restricting to an
appropriate Schwartz subspace of functions $f\in\CC^\infty(\real^6)$,
we may use a Fourier representation
\beq
f(\mx)=\int\limits_{\real^6}\frac{\dd\mk}{(2\pi)^6}~\tilde f(\mk)~
\e^{\ii\mk^\top\mx} \ .
\label{Fouriertransfdef}\eeq
This establishes a correspondence between (Schwartz) functions on
$\mfn_6^*$ and elements of the complexified Nappi-Witten group
$\mathcal{N}\otimes\complex$. The products of symbols $\Delta(f)$
may be computed using (\ref{Sigmasymbol}), and the star-product
(\ref{fstargSigma}) can be represented in terms of a product of group
elements in $\mathcal{N}_6\otimes\complex$ as
\beq
f\star g=\int\limits_{\real^6}\frac{\dd\mk}{(2\pi)^6}~
\int\limits_{\real^6}\frac{\dd\mq}{(2\pi)^6}~\tilde f(\mk)\,
\tilde g(\mq)~\Delta^{-1}\left(\,\NO~~\NO\,\e^{\ii\mk^\top\X}\,\NO\cdot
\NO\,\e^{\ii\mq^\top\X}\,\NO~~\NO\,\right) \ .
\label{fstargFourier}\eeq
Using the Baker-Campbell-Hausdorff formula, to be discussed below, we
may write
\beq
\NO~~\NO\,\e^{\ii\mk^\top\X}\,\NO\cdot\NO\,\e^{\ii\mq^\top\X}\,\NO~~\NO=
\NO\,\e^{\ii\mD(\mk,\mq)^\top\X}\,\NO
\label{NOproductsBCH}\eeq
for some function $\mD^\top=(D^a):\real^6\times\real^6\to\real^6$. This
enables us to rewrite the star-product (\ref{fstargFourier}) in terms
of a bi-differential operator $f\star g:=\hat{\mathcal{D}}(f,g)$ given
explicitly by
\beq
f\star
g=f~\e^{\ii\mx^\top[\mD(\,-\ii\overleftarrow{\mdell}
\,,\,-\ii\overrightarrow{\mdell}\,)+\ii\overleftarrow{\mdell}+\ii
\overrightarrow{\mdell}\,]}~g
\label{fstargbidiff}\eeq
with $\mdell^\top:=(\partial^a)$. In particular, the star-products of
the coordinate functions themselves may be computed from the formula
\beq
x_a\star x_b=\left.-\frac{\partial}{\partial k^a}\frac\partial
{\partial q^b}\e^{\ii\mD(\mk,\mq)^\top\mx}\right|_{\mk=\mq=\mbf0} \ .
\label{xastarxb}\eeq

Finally, let us describe how to explicitly compute the functions
$D^a(\mk,\mq)$ in (\ref{NOproductsBCH}). For this, we consider the
Dynkin form of the Baker-Campbell-Hausdorff formula which is given for
$\X,\Y\in\mfn_6$ by
\begin{equation}
  \label{eq:BCH:define}
\e^\X~\e^\Y=\e^{\mathrm{H}(\X:\Y)} \ ,
\end{equation}
where $\mathrm{H}(\X:\Y)=\sum_{n\geq1}\mathrm{H}_n(\X:\Y)\in\mfn_6$ is
generically an infinite series whose terms may be calculated through the
recurrence relation
\begin{eqnarray}
  \label{eq:BCH}
&&(n+1)~\mathrm{H}_{n+1}(\X:\Y)~=~\mbox{$\frac 12$}\,\bigl[\X-\Y\,,\,
\mathrm{H}_n(\X:\Y)\bigr]
\nonumber  \\ &&~~~~~~~~~~~~~~~~~~~~
+\,\sum_{p=1}^{\lfloor n/2\rfloor}\frac{B_{2p}}{(2p)!}~
\sum_{\substack{k_1,\ldots,k_{2p}> 0 \\ k_1+\ldots+k_{2p}=n }}
\bigl[\mathrm{H}_{k_1}(\X:\Y)\,,\,\bigl[\,\ldots\,,\,\bigl[
\mathrm{H}_{k_{2p}}(\X:\Y)\,,\,\X+\Y\bigr]\ldots\bigr]\,\bigr]\nonumber\\
\end{eqnarray}
with $\mathrm{H}_1(\X:\Y):=\X+\Y$. The coefficients $B_{2p}$ are the
Bernoulli numbers which are defined by the generating function
\begin{equation}
  \label{eq:BCH:K}
  \frac{s}{1-\e^{-s}}-\frac s2-1=\sum_{p=1}^\infty\frac{B_{2p}}{(2p)!}
  ~s^{2p} \ .
\end{equation}
The first few terms of the formula (\ref{eq:BCH:define}) may be
written explicitly as
\begin{eqnarray}
  \label{eq:BCH:1}
  \mathrm{H}_1(\X:\Y)&=& \X+\Y \ , \nonumber\\
  \mathrm{H}_2(\X:\Y)&=&\mbox{$\frac 12$}\,\cb \X\Y \ , \nonumber \\
  \mathrm{H}_3(\X:\Y)&=&\mbox{$\frac 1{12}$}\,\bigl[\X\,,\,\cb \X\Y\,\bigr]
  -\mbox{$\frac 1{12}$}\,\bigl[\Y\,,\,\cb \X\Y\,\bigr] \ , \nonumber\\
  \mathrm{H}_4(\X:\Y)&=& -\mbox{$\frac 1{24}$}\,\bigl[\Y\,,\,\bigl[\X
  \,,\,\cb \X\Y\,\bigr]\,\bigr] \ .
\end{eqnarray}
Terms in the series grow increasingly complicated due to
the sum over partitions in \eqref{eq:BCH}, and in general there is no
closed symbolic form, as in the case of the Moyal product based on the
ordinary Heisenberg algebra, for the functions $D^a(\mk,\mq)$ appearing in
(\ref{NOproductsBCH}). However, at least for certain ordering
prescriptions, the solvability of the Lie algebra $\mfn_6$ enables one
to find explicit expressions for the star-product (\ref{fstargbidiff})
in this fashion. We will now proceed to construction three such
products.

\subsubsection{Time Ordering\label{TOP}}

The simplest Gutt product is obtained by choosing a ``time ordering''
prescription in (\ref{Sigmasymbol}) whereby all factors of the time
translation generator $\J$ occur to the far right in any monomial in
$U(\mfn_6)$. It coincides precisely with the global coordinatization
(\ref{NW4coords}) of the Cahen-Wallach spacetime, and written on
elements of the complexified Nappi-Witten group
$\mathcal{N}_6\otimes\complex$ it is defined by
\begin{equation}
  \label{eq:time:defn}
\Delta_*\left(\e^{\ii\mk^\top\mx}\right)=
\NOa\,\e^{\ii\mk^\top\X}\,\NOa:=\e^{\ii(\,\overline{\mz}^{\,\top}\Q^+
+\mz^\top\Q^-)}~\e^{\ii u\,\J}~\e^{\ii v\,\T} \ ,
\end{equation}
where we have denoted
$\mk^\top:=(u,v,\mz^\top,\overline{\mz}^{\,\top}\,)$ with
$u,v\in\real$ and $\mz\in\complex^2$. In the following we will also
adopt the explicit notation $\mx^\top:=(j,t,p_1^\pm,p_2^\pm)$, with
$j,t\in\real$ and $p_1^\pm,p_2^{\pm}\in\complex$, for the
canonical basis of the dual $\mfn_6^*$. To calculate the corresponding
star-product $*$, we have to compute the group products
\bea
\NOa~~\NOa\,\e^{\ii\mk^\top\X}\,\NOa\cdot\NOa\,\e^{\ii\mk^{\prime\,\top}\X}
\,\NOa~~\NOa&=&\NOa\,\e^{\ii(\,\overline{\mz}^{\,\top}\Q^+
+\mz^\top\Q^-)}~\e^{\ii u\,\J}~\e^{\ii v\,\T}\nonumber\\&&\times~
\e^{\ii(\,\overline{\mz}^{\,\prime\,\top}\Q^+
+\mz^{\prime\,\top}\Q^-)}~\e^{\ii u'\,\J}~\e^{\ii v'\,\T}\,\NOa \ .
\label{TOgpprods}\eea

The simplest way to compute these products is to realize the
six-dimensional Nappi-Witten Lie algebra $\mfn_6$ as a central
extension of a subalgebra $\mfs$ of the four-dimensional euclidean algebra
${\rm iso}(4)={\rm so}(4)\ltimes\real^4$. Regarding $\real^4$ as
$\complex^2$ (with respect to a chosen complex structure), for generic
$\theta\neq0$ the generators of $\mfn_6$ act on $\mw\in\complex^2$
according to the affine transformations $\e^{\ii
  u\,\J}\cdot\mw=\e^{\ii\theta\,u}\,\mw$ and
$\e^{\ii(\,\overline{\mz}^{\,\top}\Q^++\mz^\top\Q^-)}\cdot\mw=\mw+\theta\,\mz$,
corresponding to a combined rotation in the $(12)$, $(34)$ planes and
translations in $\real^4\cong\complex^2$. The central element
generates an abstract one-parameter subgroup acting as $\e^{\ii
  v\,\T}\cdot\mw=\e^{\ii\theta\,v}\,\mw$ in this representation. From
this action we can read off the group multiplication laws
\bea
\e^{\ii u\,\J}~\e^{\ii u'\,\J}&=&\e^{\ii(u+u'\,)\,\J} \ ,
\label{JJgpmultlaw}\\
{~~~~}^{~~}_{~~}\nn\\\e^{\ii u\,\J}~\e^{\ii(\,\overline{\mz}^{\,\top}\Q^+
+\mz^\top\Q^-)}&=&\e^{\ii(\,\e^{\ii\theta\,u}\,\overline{\mz}^{\,\top}\Q^+
+\e^{-\ii\theta\,u}\,\mz^\top\Q^-)}~\e^{\ii u\,\J} \ ,
\label{JQgpmultlaw}\\
{~~~~}^{~~}_{~~}\nn\\\e^{\ii(\,\overline{\mz}^{\,\top}\Q^+
+\mz^\top\Q^-)}~\e^{\ii(\,\overline{\mz}^{\,\prime\,\top}\Q^+
+\mz^{\prime\,\top}\Q^-)}&=&\e^{\ii[\,\overline{(\mz+\mz'\,)}^{\,\top}\Q^+
+(\mz+\mz'\,)^{\top}\Q^-]}~\e^{-2\ii\theta~{\rm Im}(\,
\overline{\mz}^{\,\top}\mz'\,)\,\T}
\label{QQgpmultlaw}\eea
where the formula (\ref{JQgpmultlaw}) displays the semi-direct product
nature of the euclidean group, while (\ref{QQgpmultlaw}) displays the
group cocycle of the projective representation of the subgroup of
${\rm ISO}(4)$, arising from the central extension, which is computed
from the Baker-Campbell-Hausdorff formula.

Using (\ref{JJgpmultlaw})--(\ref{QQgpmultlaw}) we may now compute the
products (\ref{TOgpprods}) and one finds
\bea
\NOa~~\NOa\,\e^{\ii\mk^\top\X}\,\NOa\cdot\NOa\,\e^{\ii\mk^{\prime\,\top}\X}
\,\NOa~~\NOa&=&\e^{\ii[\,\overline{(\mz+\e^{-\ii\theta\,u}\,\mz'\,)}^{\,\top}
\Q^++(\mz+\e^{-\ii\theta\,u}\,\mz'\,)^\top\Q^-]}~\e^{\ii(u+u'\,)\,\J}
\nonumber\\&&\times~
\e^{\ii[v+v'-2\theta~{\rm Im}(\e^{-\ii\theta\,u}\,
\overline{\mz}^{\,\top}\mz'\,)]\,\T} \ .
\label{TOgpprodexpl}\eea
From (\ref{xastarxb}) we may compute the star-products between the
coordinate functions on $\mfn_6^*$ and easily verify the commutation
relations of the Nappi-Witten algebra,
\bea
x_a*x_a&=&(x_a)^2 \ , \nonumber\\x_a*t&=&t*x_a~=~x_a\,t \ ,
\nonumber\\p_1^\alpha*p_2^\beta&=&p_2^\beta*p_1^\alpha
~=~p_1^\alpha\,p_2^\beta \ , \nonumber\\
j*p_\ell^\pm&=&j\,p_\ell^\pm\pm\theta\,p_\ell^\pm \ , \nonumber\\
p_\ell^\pm*j&=&j\,p_\ell^\pm \ , \nonumber\\
p_\ell^\pm*p_\ell^\mp&=&p_\ell^\pm\,p_\ell^\mp\pm\theta\,t \ ,
\label{TOcoordstarprods}\eea
with $a=1,\dots,6$, $\alpha,\beta=\pm$ and $\ell=1,2$. From
(\ref{NOproductsBCH},\ref{fstargbidiff}) we find the star-product $*$
of generic functions $f,g\in\CC^{\infty}(\mfn_6^*)$ given by
\bea
f*g&=&\mu\circ\exp\left[\theta\,t\,\left(\e^{-\theta\,\partial_j}\,
\partial_{p_\ell^-}\otimes\partial_{p_\ell^+}-\e^{\theta\,\partial_j}\,
\partial_{p_\ell^+}\otimes\partial_{p_\ell^-}\right)\right.\nonumber
\\&&\qquad\qquad+\left.p_\ell^+\,\left(\e^{\theta\,\partial_j}-1\right)\otimes
\partial_{p_\ell^-}+p_\ell^-\,\left(\e^{-\theta\,\partial_j}-1\right)
\otimes\partial_{p_\ell^+}\right]f\otimes g \ ,
\label{TOstargen}\eea
where $\mu(f\otimes g)=f\,g$ is the pointwise product and an implicit
sum over $\ell=1,2$ is understood. To second order in the deformation
parameter $\theta$ we obtain
\begin{eqnarray}
  \label{eq:time:positionspace}
  \nonumber
  f\ast g&=&f\,g
  -\theta\,\left[
    t\,\left(\d_{p_\ell^+}f\,\d_{p_\ell^-}g
    -\d_{p_\ell^-}f\,\d_{p_\ell^+}g\right)
    -p_\ell^+\,\d_j f\,\d_{p_\ell^-}g
    +p_\ell^-\,\d_j f\,\d_{p_\ell^+}g
  \right]\\\nonumber
  &&+\,\theta^2\,\left[
  \mbox{$\frac12$}\,t^2\,\left(\d_{p_\ell^-}^2f\,\d_{p_\ell^+}^2g
  -2\d_{p_\ell^+}\d_{p_\ell^-}f\,\d_{p_\ell^+}\d_{p_\ell^-}g
  +\d_{p_\ell^+}^2f\,\d_{p_\ell^-}^2g\right)\right.
  \\\nonumber&&\qquad\quad-\,t\,\left(\d_{p_\ell^-}\d_j f\,\d_{p_\ell^+}g
  -\d_{p_\ell^+}\d_j f\,\d_{p_\ell^-}g\right)
  -t\,p_\ell^+\,\left(\d_{p_\ell^+}\d_j f\,\d_{p_\ell^-}^2g
  -\d_{p_\ell^-}\d_j f\,\d_{p_\ell^+}\d_{p_\ell^-}g\right)\\
  \nonumber &&\qquad\quad
  +\,t\,p_\ell^-\,\left(\d_{p_\ell^+}\d_j f\,\d_{p_\ell^+}\d_{p_\ell^-}g
  -\d_{p_\ell^-}\d_j f\,\d_{p_\ell^+}^2g\right)
  -p_\ell^+\,p_\ell^-\,\d_j^2f\,\d_{p_\ell^+}\d_{p_\ell^-}g
  \\ \nonumber &&\qquad\quad
  +\left.\mbox{$\frac12$}\,\left((p_\ell^+)^2\,\d_j^2f\,\d_{p_\ell^-}^2g
  +p_\ell^+\,\d_j^2f\,\d_{p_\ell^-}g
  +p_\ell^-\,\d_j^2f\,\d_{p_\ell^+}g
  +(p_\ell^-)^2\,\d_j^2f\,\d_{p_\ell^+}^2g\right)
  \right]\\&&+\,O\left(\theta^3\right) \ . 
\end{eqnarray}

\subsubsection{Symmetric Time Ordering\label{TSOP}}

Our next Gutt product is obtained by taking a ``symmetric time
ordering'' whereby any monomial in $U(\mfn_6)$ is the symmetric sum
over the two time orderings obtained by placing $\J$ to the far right
and to the far left. This ordering is induced by the group contraction of
${\rm U}(1)\times{\rm SU}(2)$ onto $\mathcal{N}_4$, and it thereby
induces the coordinatization of $\NW_4$ that is obtained from the
Penrose-G\"uven limit of the spacetime $\S^{1,0}\times\S^3$ in
(\ref{NW4commdiag}). On elements of $\mathcal{N}_6\otimes\complex$ it
is defined by
\beq
\Delta_\bullet\left(\e^{\ii\mk^\top\mx}\right)=
\NOb\,\e^{\ii\mk^\top\X}\,\NOb:=\e^{\frac\ii2\,u\,\J}~
\e^{\ii(\,\overline{\mz}^{\,\top}\Q^+
+\mz^\top\Q^-)}~\e^{\frac\ii2\,u\,\J}~\e^{\ii v\,\T} \ .
\label{TOsymgpprods}\eeq
From (\ref{JJgpmultlaw})--(\ref{QQgpmultlaw}) we can again easily
compute the required group products to get
\bea
\NOb~~\NOb\,\e^{\ii\mk^\top\X}\,\NOb\cdot\NOb\,\e^{\ii\mk^{\prime\,\top}\X}
\,\NOb~~\NOb&=&\e^{\frac\ii2\,(u+u'\,)\,\J}\nonumber\\&&\times~
\e^{\ii[\,\overline{(\e^{\frac{\ii\theta}2\,u'}\,\mz+
\e^{-\frac{\ii\theta}2\,u}\,\mz'\,)}^{\,\top}
\Q^++(\e^{\frac{\ii\theta}2\,u'}\,\mz+
\e^{-\frac{\ii\theta}2\,u}\,\mz'\,)^\top\Q^-]}
\nonumber\\&&\times~\e^{\frac\ii2\,(u+u'\,)\,\J}~
\e^{\ii[v+v'-2\theta~{\rm Im}(\e^{-\frac{\ii\theta}2\,(u+u'\,)}\,
\overline{\mz}^{\,\top}\mz'\,)]\,\T} \ .
\label{TOsymgpprodexpl}\eea

With the same conventions as above, from (\ref{xastarxb}) we may now
compute the star-products $\bullet$ between the coordinate functions
on $\mfn_6^*$ and again verify the commutation relations of the
Nappi-Witten algebra,
\bea
x_a\bullet x_a&=&(x_a)^2 \ , \nonumber\\
x_a\bullet t&=&t\bullet x_a~=~x_a\,t \ ,
\nonumber\\p_1^\alpha\bullet p_2^\beta&=&p_2^\beta\bullet p_1^\alpha
~=~p_1^\alpha\,p_2^\beta \ , \nonumber\\
j\bullet p_\ell^\pm&=&j\,p_\ell^\pm\pm\mbox{$\frac12$}\,
\theta\,p_\ell^\pm \ , \nonumber\\
p_\ell^\pm\bullet j&=&j\,p_\ell^\pm\mp\mbox{$\frac12$}\,
\theta\,p_\ell^\pm \ , \nonumber\\
p_\ell^\pm\bullet p_\ell^\mp&=&p_\ell^\pm\,p_\ell^\mp\pm\theta\,t \ .
\label{TOsymcoordstarprods}\eea
From (\ref{NOproductsBCH},\ref{fstargbidiff}) we find for generic
functions the formula
\bea
f\bullet g&=&\mu\circ\exp\left\{\theta\,t\,\left(\e^{-\frac\theta2
\,\partial_j}\,
\partial_{p_\ell^-}\otimes\e^{-\frac\theta2\,\partial_j}\,
\partial_{p_\ell^+}-\e^{\frac\theta2\,\partial_j}\,
\partial_{p_\ell^+}\otimes\e^{\frac\theta2\,\partial_j}\,
\partial_{p_\ell^-}\right)\right.\nonumber
\\&&\qquad\qquad+\,p_\ell^+\,\left[\partial_{p_\ell^-}\otimes
\left(\e^{-\frac\theta2\,\partial_j}-1\right)
+\left(\e^{\frac\theta2\,\partial_j}-1\right)\otimes
\partial_{p_\ell^-}\right]\nonumber\\&&\qquad\qquad+\left.
p_\ell^-\,\left[\partial_{p_\ell^+}\otimes
\left(\e^{\frac\theta2\,\partial_j}-1\right)
+\left(\e^{-\frac\theta2\,\partial_j}-1\right)
\otimes\partial_{p_\ell^+}\right]\right\}f\otimes g \ .
\label{TOsymstargen}\eea
To second order in $\theta$ we obtain
\begin{eqnarray}
  \label{eq:symtime:positionspace}\nonumber
  f\bullet g&=&f\,g-\frac\theta2\,\left[
  2t\,\left(\d_{p_\ell^+}f\,\d_{p_\ell^-}g
  - \d_{p_\ell^-}f\,\d_{p_\ell^+}g\right)\right.\\\nonumber
  &&\qquad\qquad+\left.p_\ell^+\,\left(\d_{p_\ell^-}f\,\d_j g
  - \d_j f\,\d_{p_\ell^-}g\right)+ p_\ell^-\,\left(\d_j f\,\d_{p_\ell^+}g
   - \d_{p_\ell^+}f\,\d_j g\right)\right]\\ \nonumber
  &&+\,\frac{\theta^2}{2}\,
  \left[t^2\,\left(\d_{p_\ell^+}^2f\,\d_{p_\ell^-}^2g
  +\d_{p_\ell^-}^2f\,\d_{p_\ell^+}^2g
  -2\d_{p_\ell^+}\d_{p_\ell^-}f\,\d_{p_\ell^+}\d_{p_\ell^-}g
  \right)\right.\\ \nonumber &&\qquad\quad
  -\,t\,\left(\d_{p_\ell^-}f\,\d_{p_\ell^+}\d_j g
  +\d_{p_\ell^+}f\,\d_{p_\ell^-}\d_j g
  +\d_{p_\ell^+}\d_j f\,\d_{p_\ell^-}g
  +\d_{p_\ell^-}\d_j f\,\d_{p_\ell^+}g\right)\\ \nonumber
  &&\qquad\quad
  +\,t\,p_\ell^+\,\left(\d_{p_\ell^+}\d_{p_\ell^-}f\,\d_{p_\ell^-}\d_j g
  -\d_{p_\ell^+}\d_j f\,\d_{p_\ell^-}^2g
  +\d_{p_\ell^-}\d_j f\,\d_{p_\ell^+}\d_{p_\ell^-}g
  -\d_{p_\ell^-}^2f\,\d_{p_\ell^+}\d_j g\right)
  \\ \nonumber &&\qquad\quad
  +\,t\,p_\ell^-\,\left(\d_{p_\ell^+}\d_{p_\ell^-}f\,\d_{p_\ell^+}\d_j g
  -\d_{p_\ell^-}\d_j f\,\d_{p_\ell^+}^2g
  +\d_{p_\ell^+}\d_j f\,\d_{p_\ell^+}\d_{p_\ell^-}g
  -\d_{p_\ell^+}^2f\,\d_{p_\ell^-}\d_j g\right)\\ \nonumber
  &&\qquad\quad
  +\,\mbox{$\frac12$}\,p_\ell^+\,p_\ell^-\,\left(
  \d_{p_\ell^+}\d_j f\,\d_{p_\ell^-}\d_j g
  +\d_{p_\ell^-}\d_j f\,\d_{p_\ell^+}\d_j g
  -\d_j ^2f\,\d_{p_\ell^+}\d_{p_\ell^-}g
  -\d_{p_\ell^+}\d_{p_\ell^-}f\,\d_j^2g\right)
  \\ \nonumber &&\qquad\quad
  +\,\mbox{$\frac14$}\,\left(p_\ell^+\right)^2\,
  \left(\d_{p_\ell^-}^2f\,\d_j^2g
  -2\d_{p_\ell^-}\d_j f\,\d_{p_\ell^-}\d_j g
  +\d_j^2f\,\d_{p_\ell^-}^2g\right)\\ \nonumber &&\qquad\quad
  +\,\mbox{$\frac14$}\,\left(p_\ell^-\right)^2\,
  \left(\d_{p_\ell^+}^2f\,\d_j^2g
  -2\d_{p_\ell^+}\d_j f\,\d_{p_\ell^+}\d_j g
  +\d_j^2f\,\d_{p_\ell^+}^2g\right)
  \\ \nonumber &&\qquad\quad
  +\left.\mbox{$\frac14$}\,p_\ell^+\,\left(\d_{p_\ell^-}f\,\d_j^2g
  +\d_j^2f\,\d_{p_\ell^-}g\right)
  +\mbox{$\frac14$}\,p_\ell^-\,\left(\d_j^2f\,\d_{p_\ell^+}g
  +\d_{p_\ell^+}f\,\d_j^2g\right)
  \right]+O\left(\theta^3\right) \ . \\&&
\end{eqnarray}

\subsubsection{Weyl Ordering\label{WOP}}

The original Gutt product is based on the ``Weyl ordering''
prescription whereby all monomials  in $U(\mfn_6)$ are completely
symmetrized over all elements of $\mfn_6$. On
$\mathcal{N}_6\otimes\complex$ it is defined by
\beq
\Delta_\star\left(\e^{\ii\mk^\top\mx}\right)=
\NO\,\e^{\ii\mk^\top\X}\,\NO:=\e^{\ii\mk^\top\X} \ .
\label{Weylgpprods}\eeq
While this ordering is usually thought of as the ``canonical''
ordering for the construction of star-products, in our case it turns
out to be drastically more complicated than the other
orderings. Nevertheless, we shall present here its explicit
construction for the sake of completeness and for later comparisons.

It is an extremely arduous task to compute products of the group
elements (\ref{Weylgpprods}) directly from the
Baker-Campbell-Hausdorff formula (\ref{eq:BCH}). Instead, we shall
construct an isomorphism $\mathcal{G}:U(\mfn_6)^c\to
U(\mfn_6)^c$ which sends the time-ordered product defined by
(\ref{TOgpprods}) into the Weyl-ordered product defined by
(\ref{Weylgpprods}), i.e.
\beq
\mathcal{G}\circ\Delta_*=\Delta_\star \ .
\label{1ststudy}\eeq
Then by defining
$\mathcal{G}_\Delta:=\Delta_*^{-1}\circ\mathcal{G}\circ\Delta^{~}_\star$,
the star-product $\star$ associated with the Weyl ordering
prescription (\ref{Weylgpprods}) may be computed as
\beq
f\star g=\mathcal{G}^{~}_\Delta\bigl(\mathcal{G}_\Delta^{-1}(f)*
\mathcal{G}_\Delta^{-1}(g)\bigr) \ , ~~ f,g\in\CC^\infty(\mfn_6^*) \ .
\label{WeylTOrel}\eeq
Explicitly, if
\beq
\NOa\,\e^{\ii\mk^\top\X}\,\NOa=\e^{\ii\mG(\mk)^\top\X}
\label{1ststudyexpl}\eeq
for some function $\mG=(G^a):\real^6\to\real^6$, then the isomorphism
$\mathcal{G}_\Delta:\CC^\infty(\mfn_6^*)\to\CC^\infty(\mfn_6^*)$ may
be represented as the invertible differential operator
\beq
\mathcal{G}_\Delta=\e^{\ii\mx^\top[\mG(-\ii\mbf\partial)+\ii\mbf
\partial]} \ .
\label{Gdiffop}\eeq
This relation justs reflects the fact that the time-ordered and
Weyl-ordered star-products, although not identical, simply represent different
ordering prescriptions for the same algebra and are therefore {\it
  equivalent}. We will elucidate this property more thoroughly in
Section~\ref{WeylSystems}. Thus once the map (\ref{1ststudyexpl}) is
known, the Weyl ordered star-product $\star$ can be computed in terms
of the time-ordered star-product $*$ of Section~\ref{TOP}.

The functions $G^a(\mk)$ appearing in (\ref{1ststudyexpl}) are readily
calculable through the Baker-Campbell-Hausdorff formula. It is clear
from (\ref{TOgpprods}) that the coefficient of the time translation
generator $\J\in\mfn_6$ is simply
\beq
G^u(u,v,\mz,\overline{\mz}\,)=u \ .
\label{Gj}\eeq
From (\ref{eq:BCH}) it is also clear that the only terms proportional
to $\Q^+$ come from commutators of the form
$[\J,[\dots,[\J,\Q^+]\,]\dots]$, and gathering all terms we find
\bea
G^{\overline{\mz}}(u,v,\mz,\overline{\mz}\,)^\top~\Q^+&=&-\ii\sum_{n=0}^\infty
\frac{B_n}{n!}~\bigl[\,\underbrace{\ii u\,\J\,,\,\bigl[\dots\,,\,
\bigl[\ii u\J}_n\,,\,\ii\overline{\mz}^{\,\top}\Q^+\,\bigr]\,\bigr]
\dots\bigr]\nonumber\\&=&
\overline{\mz}^{\,\top}\,\sum_{n=0}^\infty\frac{B_n}{n!}\,
(\ii\theta\,u)^n~\Q^+ \ .
\label{GomzBn}\eea
Since $B_0=1$, $B_1=-\frac12$ and $B_{2k+1}=0~~\forall k\geq1$, from
(\ref{eq:BCH:K}) we thereby find
\beq
G^{\overline{\mz}}(u,v,\mz,\overline{\mz}\,)=\frac{\overline{\mz}}
{\phi_\theta(u)}
\label{Gomz}\eeq
where we have introduced the function
\beq
\phi_\theta(u)=\frac{\e^{\ii\theta\,u}-1}{\ii\theta\,u} \ .
\label{phithetadef}\eeq
In a completely analogous way one finds the coefficient of the $\Q^-$
term given by
\beq
G^{{\mz}}(u,v,\mz,\overline{\mz}\,)=\frac{{\mz}}
{\overline{\phi_\theta(u)}} \ .
\label{Gmz}\eeq
Finally, the non-vanishing contributions to the central
element $\T\in\mfn_6$ are given by
\bea
G^v(u,v,\mz,\overline{\mz}\,)~\T&=&
v~\T-\ii\sum_{n=1}^\infty\frac{B_{n+1}}{n!}\,
\left(\bigl[\ii\overline{\mz}^{\,\top}\Q^+\,,\,\bigl[\,
\underbrace{\ii u\,\J\,,\,\dots\bigl[\ii u\,\J}_n\,,\,\ii\mz^\top
\Q^-\,\bigr]\dots\bigr]\,\bigr]\right.\nonumber\\&&\qquad\qquad\qquad
+\left.\bigl[\ii\mz^\top\Q^-\,,\,\bigl[\,
\underbrace{\ii u\,\J\,,\,\dots\bigl[\ii u\,\J}_n\,,\,
\ii\overline{\mz}^{\,\top}\Q^+\,\bigr]\dots\bigr]\,\bigr]\right)
\nonumber\\&=&v~\T-4\ii\theta\,|\mz|^2\,\sum_{n=1}^\infty\frac{B_{n+1}}
{n!}\,(\ii\theta\,u)^n~\T \ .
\label{GtBn}\eea
By differentiating (\ref{GomzBn}) and (\ref{phithetadef}) we arrive
finally at
\beq
G^v(u,v,\mz,\overline{\mz}\,)=v-4\ii\theta\,|\mz|^2\,
\gamma_\theta(u)
\label{Gt}\eeq
where we have introduced the function
\beq
\gamma_\theta(u)=\frac12+\frac{(1-\ii\theta\,u)~\e^{\ii\theta\,u}-1}
{\left(\e^{\ii\theta\,u}-1\right)^2} \ .
\label{gammathetadef}\eeq

From (\ref{Gdiffop}) we may now write down the explicit form of the
differential operator implementing the equivalence between the
star-products $*$ and $\star$ as
\bea
\mathcal{G}_\Delta&=&\exp\left[-2\theta\,t\,\partial_{p_\ell^+}
\partial_{p_\ell^-}\left(1+\frac{2(1-\theta\,\partial_j)~
\e^{\theta\,\partial_j}-1}{\left(\e^{\theta\,\partial_j}-1
\right)^2}\right)\right.\nonumber\\&&\qquad\quad+\left.
p_\ell^+\,\partial_{p_\ell^-}\left(\frac{\theta\,\partial_j}
{\e^{\theta\,\partial_j}-1}-1\right)-p_\ell^-\,\partial_{p_\ell^+}
\left(\frac{\theta\,\partial_j}
{\e^{-\theta\,\partial_j}-1}+1\right)\right] \ .
\label{Gdiffopexpl}\eea
From (\ref{TOgpprodexpl}) and (\ref{1ststudyexpl}) we may readily
compute the products of Weyl symbols with the result
\bea
\NO~~\NO\,\e^{\ii\mk^\top\X}\,\NO\cdot\NO\,\e^{\ii\mk^{\prime\,\top}\X}
\,\NO~~\NO&=&\exp\ii\left\{\frac{\phi_\theta(u)\,\overline{\mz}^{\,\top}+
\e^{\ii\theta\,u}\,\phi_\theta(u'\,)\,\overline{\mz}^{\,\prime\,\top}}
{\phi_\theta(u+u'\,)}~\Q^+\right.\nonumber\\&&\qquad\quad
+\,\frac{\overline{\phi_\theta(u)}\,{\mz}^{\top}+
\e^{-\ii\theta\,u}\,\overline{\phi_\theta(u'\,)}\,\mz^{\prime\,\top}}
{\overline{\phi_\theta(u+u'\,)}}~\Q^-+(u+u'\,)~\J\nonumber\\&&
\qquad\quad+\,\left[v+v'+2\theta~{\rm Im}\bigl(\,\overline{\phi_\theta(u)\,
\phi_\theta(u'\,)}\,\overline{\mz}^{\,\top}\mz'\,\bigr)\right.
\nonumber\\&&\quad\qquad+\,4\ii\theta\,\left(\gamma_\theta(u+u'\,)
\,\bigl|\,\overline{\phi_\theta(u)}\,
\mz+\e^{-\ii\theta\,u}\,\overline{\phi_\theta(u'\,)}\,\mz'\,\bigr|^2
\right.\nonumber\\&&\qquad\quad\qquad-\biggl.\left.\left.
\gamma_\theta(u)\,\bigl|\phi_\theta(u)\bigr|^2\,|\mz|^2-
\gamma_\theta(u'\,)\,\bigl|\phi_\theta(u'\,)\bigr|^2\,|\mz'\,|^2
\right)\right]~\T\biggr\} \ . \nonumber\\&&
\label{Weylgpprodexpl}\eea
From (\ref{xastarxb}) we may now compute the star-products $\star$
between the coordinate functions on $\mfn_6^*$ to be
\bea
x_a\star x_a&=&(x_a)^2 \ , \nonumber\\
x_a\star t&=&t\star x_a~=~x_a\,t \ ,
\nonumber\\p_1^\alpha\star p_2^\beta&=&p_2^\beta\star p_1^\alpha
~=~p_1^\alpha\,p_2^\beta \ , \nonumber\\
j\star p_\ell^\pm&=&j\,p_\ell^\pm\pm\mbox{$\frac12$}\,
\theta\,p_\ell^\pm \ , \nonumber\\
p_\ell^\pm\star j&=&j\,p_\ell^\pm\mp\mbox{$\frac12$}\,
\theta\,p_\ell^\pm \ , \nonumber\\
p_\ell^\pm\star p_\ell^\mp&=&p_\ell^\pm\,p_\ell^\mp\pm\theta\,t \ .
\label{Weylsymcoordstarprods}\eea
These products are identical to those of the symmetric time ordering
prescription (\ref{TOsymcoordstarprods}). After some computation, from
(\ref{NOproductsBCH},\ref{fstargbidiff}) we find for generic functions
$f,g\in\CC^\infty(\mfn_6^*)$ the formula
\bea
f\star g&=&\mu\circ\exp\left\{\ii\theta\,t\,\left[~~
\frac{1\otimes1+\bigl(\theta\,(\partial_j\otimes1+1\otimes\partial_j)-1\otimes1
\bigr)~\e^{\theta\,\partial_j}\otimes\e^{\theta\,\partial_j}}
{\left(\e^{\theta\,\partial_j}\otimes\e^{\theta\,\partial_j}-1\otimes1
\right)^2}\right.\right.\nonumber\\&&\times\,
\left(\frac{4\partial_{p_\ell^-}\left(\e^{-\theta\,\partial_j}-1
\right)}{\theta\,\partial_j}\otimes
\frac{\partial_{p_\ell^+}\left(\e^{-\theta\,\partial_j}-1
\right)}{\theta\,\partial_j}-\frac{3\partial_{p_\ell^+}
\left(\e^{\theta\,\partial_j}-1
\right)}{\theta\,\partial_j}\otimes
\frac{\partial_{p_\ell^-}\left(\e^{\theta\,\partial_j}-1
\right)}{\theta\,\partial_j}\right.\nonumber\\&&-\left.
\frac{4\partial_{p_\ell^+}\partial_{p_\ell^-}\sinh^2\left(\frac\theta2
\,\partial_j\right)}{\theta^2\,\partial_j^2}\otimes1-1\otimes
\frac{4\partial_{p_\ell^+}\partial_{p_\ell^-}\sinh^2\left(\frac\theta2
\,\partial_j\right)}{\theta^2\,\partial_j^2}\right)
\nonumber\\&&+\,
\frac{4\partial_{p_\ell^+}\partial_{p_\ell^-}}{\theta\,\partial_j}
\left(\frac{\sinh^2\left(\frac\theta2
\,\partial_j\right)}{\theta\,\partial_j\left(
\e^{\theta\,\partial_j}-1\right)}-1\right)\otimes1+1\otimes
\frac{4\partial_{p_\ell^+}\partial_{p_\ell^-}}{\theta\,\partial_j}
\left(\frac{\sinh^2\left(\frac\theta2
\,\partial_j\right)}{\theta\,\partial_j\left(
\e^{\theta\,\partial_j}-1\right)}-1\right)\nonumber\\&&
+\left.\frac{3\partial_{p_\ell^+}\left(\e^{\theta\,\partial_j}-1
\right)}{\theta\,\partial_j}\otimes\frac{
\partial_{p_\ell^-}\left(\e^{\theta\,\partial_j}-1
\right)}{\theta\,\partial_j}+\frac{\partial_{p_\ell^-}
\left(\e^{-\theta\,\partial_j}-1
\right)}{\theta\,\partial_j}\otimes\frac{\partial_{p_\ell^+}
\left(\e^{-\theta\,\partial_j}-1\right)}{\theta\,\partial_j}~~\right]
\nonumber\\&&
+\,\frac{p_\ell^+}{1\otimes\e^{-\theta\,
\partial_j}-\e^{\theta\,\partial_j}\otimes1}\,\left[~~
\frac{\partial_{p_\ell^-}}{\partial_j}\,\left(1-\e^{\theta\,\partial_j}
\right)\otimes\partial_j-\partial_j\otimes\frac{\partial_{p_\ell^-}}
{\partial_j}\,\left(1-\e^{-\theta\,\partial_j}\right)\right.\nonumber\\&&
+\Biggl.\,1\otimes\partial_{p_\ell^-}~
\e^{-\theta\,\partial_j}-\partial_{p_\ell^-}~\e^{\theta\,\partial_j}
\otimes1-1\otimes2\partial_{p_\ell^-}~~\Biggr]\nonumber\\&&
+\,\frac{p_\ell^-}{1\otimes\e^{\theta\,
\partial_j}-\e^{-\theta\,\partial_j}\otimes1}\,\left[~~
\frac{\partial_{p_\ell^+}}{\partial_j}\,\left(1-\e^{-\theta\,\partial_j}
\right)\otimes\partial_j-\partial_j\otimes\frac{\partial_{p_\ell^+}}
{\partial_j}\,\left(1-\e^{\theta\,\partial_j}\right)\right.\nonumber\\&&
+\left.\Biggl.1\otimes\partial_{p_\ell^+}~
\e^{\theta\,\partial_j}-\partial_{p_\ell^+}~\e^{-\theta\,\partial_j}
\otimes1-1\otimes2\partial_{p_\ell^+}~~\Biggr]\right\}f\otimes g \ .
\label{Weylstargen}\eea
To second order in the deformation parameter $\theta$ we obtain
\begin{eqnarray}
  \label{eq:weyl:positionspace}
  f\bullet g&=&f\,g-\frac\theta2\,\left[
  2t\,\left(\d_{p_\ell^+}f\,\d_{p_\ell^-}g
  - \d_{p_\ell^-}f\,\d_{p_\ell^+}g\right)\right.\\\nonumber
  &&\qquad\qquad+\left.p_\ell^+\,\left(\d_{p_\ell^-}f\,\d_j g
  - \d_j f\,\d_{p_\ell^-}g\right)+ p_\ell^-\,\left(\d_j f\,\d_{p_\ell^+}g
   - \d_{p_\ell^+}f\,\d_j g\right)\right]\\ \nonumber
  &&+\,\frac{\theta^2}{2}\,\left[t^2\,
  \left(\d_{p_\ell^+}^2f\,\d_{p_\ell^-}^2g
  +\d_{p_\ell^-}^2f\,\d_{p_\ell^+}^2g
  -2\d_{p_\ell^+}\d_{p_\ell^-}f\,\d_{p_\ell^+}\d_{p_\ell^-}g
  \right)\right.\\ \nonumber &&\qquad\quad
  -\,\mbox{$\frac13$}\,t\,\left(\d_{p_\ell^-}f\,\d_{p_\ell^+}\d_j g
  +\d_{p_\ell^+}f\,\d_{p_\ell^-}\d_j g
  +\d_{p_\ell^+}\d_j f\,\d_{p_\ell^-}g
  +\d_{p_\ell^-}\d_j f\,\d_{p_\ell^+}g\right.\\ \nonumber
  &&\qquad\qquad\qquad\qquad\quad-\left.2\partial_jf\,\partial_{p_\ell^+}
  \partial_{p_\ell^-}g-2\partial_{p_\ell^+}\partial_{p_\ell^-}f\,
  \partial_jg\right)\\\nonumber&&\qquad\quad
  +\,t\,p_\ell^+\,\left(\d_{p_\ell^+}\d_{p_\ell^-}f\,\d_{p_\ell^-}\d_j g
  -\d_{p_\ell^+}\d_j f\,\d_{p_\ell^-}^2g
  +\d_{p_\ell^-}\d_j f\,\d_{p_\ell^+}\d_{p_\ell^-}g
  -\d_{p_\ell^-}^2f\,\d_{p_\ell^+}\d_j g\right)
  \\ \nonumber &&\qquad\quad
  +\,t\,p_\ell^-\,\left(\d_{p_\ell^+}\d_{p_\ell^-}f\,\d_{p_\ell^+}\d_j g
  -\d_{p_\ell^-}\d_j f\,\d_{p_\ell^+}^2g
  +\d_{p_\ell^+}\d_j f\,\d_{p_\ell^+}\d_{p_\ell^-}g
  -\d_{p_\ell^+}^2f\,\d_{p_\ell^-}\d_j g\right)\\ \nonumber
  &&\qquad\quad
  +\,\mbox{$\frac12$}\,p_\ell^+\,p_\ell^-\,\left(
  \d_{p_\ell^+}\d_j f\,\d_{p_\ell^-}\d_j g
  +\d_{p_\ell^-}\d_j f\,\d_{p_\ell^+}\d_j g
  -\d_j ^2f\,\d_{p_\ell^+}\d_{p_\ell^-}g
  -\d_{p_\ell^+}\d_{p_\ell^-}f\,\d_j^2g\right)
  \\ \nonumber &&\qquad\quad
  +\,\mbox{$\frac14$}\,\left(p_\ell^+\right)^2\,
  \left(\d_{p_\ell^-}^2f\,\d_j^2g
  -2\d_{p_\ell^-}\d_j f\,\d_{p_\ell^-}\d_j g
  +\d_j^2f\,\d_{p_\ell^-}^2g\right)\\ \nonumber &&\qquad\quad
  +\,\mbox{$\frac14$}\,\left(p_\ell^-\right)^2\,
  \left(\d_{p_\ell^+}^2f\,\d_j^2g
  -2\d_{p_\ell^+}\d_j f\,\d_{p_\ell^+}\d_j g
  +\d_j^2f\,\d_{p_\ell^+}^2g\right)
  \\ \nonumber &&\qquad\quad
  +\,\mbox{$\frac16$}\,p_\ell^+\,\left(\d_{p_\ell^-}f\,\d_j^2g
  +\d_j^2f\,\d_{p_\ell^-}g-\partial_jf\,\partial_{p_\ell^-}
  \partial_jg-\partial_{p_\ell^-}\partial_jf\,\partial_jg\right)
  \\&&\qquad\quad
  +\left.\mbox{$\frac16$}\,p_\ell^-\,\left(\d_j^2f\,\d_{p_\ell^+}g
  +\d_{p_\ell^+}f\,\d_j^2g-\partial_jf\,\partial_{p_\ell^+}\d_jg-
  \d_{p_\ell^+}\d_jf\,\d_jg\right)
  \right]+O\left(\theta^3\right) \ .
\end{eqnarray}

Although extremely cumbersome in form, the Weyl-ordered product has
several desirable features over the simpler time-ordered products. For
instance, it is hermitean owing to the property
\beq
\overline{f\star g}=\overline{g}\star\overline{f} \ .
\label{Weylstarherm}\eeq
Moreover, while the $\mfn_6$-covariance condition (\ref{xyPoisson})
holds for all of our star-products, the Weyl product is in fact
$\mfn_6$-invariant, because for any $x\in\complex_{(1)}(\mfn_6^*)$ one
has the stronger compatibility condition
\beq
[x,f]_\star=\theta\,\Pi(x,f) ~~~~ \forall f\in\CC^\infty(\mfn_6^*)
\label{strongcompcond}\eeq
with the action of the Lie algebra $\mfn_6$. In the next
subsection we shall see that the Weyl-ordered star-product is, in a
certain sense, the generator of all other star-products making it the
``universal'' product for the quantization of the Nappi-Witten
algebra.

\subsection{Weyl Systems\label{WeylSystems}}

In this subsection we will use the notion of a generalized Weyl system
introduced in~\cite{ALZ1} to describe some more formal aspects of the
star-products that we have constructed and to analyse the interplay
between them. This generalizes the standard Weyl systems which may be
used to provide a purely operator theoretic characterization of the
Moyal product, associated to the (untwisted) Heisenberg algebra. In
this case, it can be regarded as a projective representation of the
translation group in an even-dimensional real vector space. However,
for the twisted Heisenberg algebra such a representation is not
possible, since by definition the appropriate arena should be a
central extension of a non-abelian subgroup of the full euclidean
group ${\rm ISO}(4)$. This requires a generalization of the standard
notion which we will now describe and use it to obtain a very useful
characterization of the noncommutative geometry induced by the
Nappi-Witten algebra.

Let $\mathcal{S}$ be a five-dimensional real vector space. In a
suitable (canonical) basis, vectors
$\mk\in\mathcal{S}\cong\real\times\complex^2$ will be denoted (with
respect to a chosen complex structure) as
\beq
\mk=\begin{pmatrix}u\\\mz\\\,\overline{\mz}\,\end{pmatrix}
\label{Svectors}\eeq
with $u\in\real$ and $\mz\in\complex^2$. As the notation suggests, we
regard $\mathcal{S}$ as the ``momentum space'' of the dual
$\mfn_6^*$. Note that we do not explicitly incorporate the component
corresponding to the central element $\T$, as it will instead appear
through the appropriate projective representation that we construct,
similarly to the Moyal case. As an abelian group,
$\mathcal{S}\cong\real^5$ with the usual addition $+$ and identity
$\mbf0$. Corresponding to a deformation parameter
$\theta\in\real$, we deform this abelian Lie group structure to a
generically non-abelian one. The deformed composition law is denoted
$\comp$. It is associative and in general will depend on
$\theta$. The identity element with respect to $\comp$ is still
defined to be $\mbf0$, and the inverse of any element $\mk\in\mathcal{S}$
is denoted $\underline{\mk}$, so that
\beq
\mk\comp\underline{\mk}=\underline{\mk}\comp\mk=\mbf0 \ .
\label{compinverse}\eeq
Being a deformation of the underlying abelian group structure on
$\mathcal{S}$ means that the composition of any two vectors
$\mk,\mq\in\mathcal{S}$ has a formal small $\theta$ expansion of the
form
\beq
\mk\comp\mq=\mk+\mq+O(\theta) \ ,
\label{compsmalltheta}\eeq
from which it follows that
\beq
\underline{\mk}=-\mk+O(\theta) \ .
\label{compinvsmalltheta}\eeq
In other words, rather than introducing star-products that deform the
pointwise multiplication of functions on $\mfn_6^*$, we now deform the
``momentum space'' of $\mfn_6^*$ to a non-abelian Lie group.
We will see below that the two notions of quantization are in fact the
same.

Given such a group, we now define a (generalized) Weyl system for the
Nappi-Witten algebra as a quadruple
$(\mathcal{S},\comp,\weyl,\omega)$, where the map
\beq
\weyl\,:\,\mathcal{S}~\longrightarrow~U(\mfn_6)^c
\label{genweylmapdef}\eeq
is a projective representation of the group
$(\mathcal{S},\comp)$ with projective phase
$\omega:\mathcal{S}\times\mathcal{S}\to\complex$. This means that for
every pair of elements $\mk,\mq\in\mathcal{S}$ one has the composition
rule
\beq
\weyl(\mk)\cdot\weyl(\mq)=\e^{\frac\ii2\,\omega(\mk,\mq)\,\T}~
\weyl(\mk\comp\mq)
\label{Weylcomprule}\eeq
in the completed complexified universal enveloping algebra of
$\mfn_6$. The associativity of $\comp$ and the relation
(\ref{Weylcomprule}) imply that the subalgebra
$\weyl(\mathcal{S})\subset U(\mfn_6)^c$ is associative if and only if
\beq
\omega(\mk\comp\mbf p,\mq)=\omega(\mk,\mbf p\comp\mq)+\omega(\mbf p,\mq)-
\omega(\mk,\mbf p)
\label{cocyclecond}\eeq
for all vectors $\mk,\mq,\mbf p\in\mcS$. This condition means that
$\omega$ defines a cocycle in the group cohomology
of $(\mathcal{S},\comp)$. It is automatically satisfied if
$\omega$ is a bilinear form with respect to $\comp$. We will in
addition require that $\omega(\mk,\mq)=O(\theta)~~\forall
\mk,\mq\in\mathcal{S}$ for consistency with
(\ref{compsmalltheta}). The identity element of $\weyl(\mathcal{S})$
is $\weyl(\mbf0)$ while the inverse of $\weyl(\mk)$ is given by
\beq
\weyl(\mk)^{-1}=\weyl(\,\underline{\mk}\,) \ .
\label{weylinverse}\eeq
The standard Weyl system on $\real^{2n}$ takes $\comp$ to be ordinary
addition and $\omega$ to be the canonical symplectic two-form, so that
$\weyl(\real^{2n})$ is a projective representation of the translation
group, as is appropriate to the Moyal product.

Given a Weyl system defined as above, we can now introduce
another isomorphism
\beq
\Omega\,:\,\CC^\infty\left(\real^5\right)~\longrightarrow~
\weyl\bigl(\mathcal{S}\bigr)
\label{Omegaquantmap}\eeq
defined by the symbol
\beq
\Omega(f):=\int\limits_{\real^5}\frac{\dd\mk}{(2\pi)^5}~
\tilde{f}(\mk)~\weyl(\mk)
\label{Omegafdef}\eeq
where as before $\tilde f$ denotes the Fourier transform of
$f\in\CC^\infty(\real^5)$. This definition implies that
\beq
\Omega\left(\e^{\ii\mk^\top\mx}\right)=\weyl(\mk) \ ,
\label{weylmkDelta}\eeq
and that we may introduce a $*$-involution $\dag$ on both algebras
$\CC^\infty(\real^5)$ and $\weyl(\mcS)$ by the formula
\beq
\Omega\bigl(f^\dag\bigr)=\Omega\bigl(f\bigr)^\dag:=
\int\limits_{\real^5}\frac{\dd\mk}{(2\pi)^5}~\overline{
\tilde f(\,\underline{\mk}\,)}~\weyl(\mk) \ .
\label{involdef}\eeq
The compatibility condition
\beq
\bigl(\Omega(f)\cdot\Omega(g)\bigr)^\dag=\Omega(g)^\dag\cdot
\Omega(f)^\dag
\label{compconddag}\eeq
with the product in $U(\mfn_6)^c$ imposes further constraints on the
group composition law $\comp$ and cocycle $\omega$. From
(\ref{Weylcomprule}) we may thereby define a $\dag$-hermitean
star-product of $f,g\in\CC^\infty(\real^5)$ by the formula
\beq
f\star g:=\Omega^{-1}\bigl(\Omega(f)\cdot\Omega(g)\bigr)
=\int\limits_{\real^5}\frac{\dd\mk}{(2\pi)^5}~
\int\limits_{\real^5}\frac{\dd\mq}{(2\pi)^5}~\tilde f(\mk)\,
\tilde g(\mq)~\e^{\frac\ii2\,\omega(\mk,\mq)}~
\Omega^{-1}\bigl(\weyl(\mk\comp\mq)\bigr) \ ,
\label{fstargweyl}\eeq
and in this way we have constructed a quantization of the Nappi-Witten
algebra solely from the formal notion of a Weyl system.
The associativity of $\star$ follows from associativity of
$\comp$. We may also rewrite the star-product (\ref{fstargweyl}) in
terms of a bi-differential operator as
\beq
f\star g=f~\e^{\frac\ii2\,\omega(\,-\ii\overleftarrow{\mbf\d}\,,\,
-\ii\overrightarrow{\mbf\d}\,)+\ii\mx^\top(-\ii\overleftarrow{\mbf\d}\,\comp
-\ii\overrightarrow{\mbf\d}+\ii\overleftarrow{\mbf\d}+\ii
\overrightarrow{\mbf\d}\,)}~g \ .
\label{bidiffweyl}\eeq

This deformation is completely characterized in terms of the
new algebraic structure and its projective representation provided by
the Weyl system. It is straightforward to show
that the Lie algebra of $(\mathcal{S},\comp)$ coincides precisely with
the original subalgebra $\mfs\subset{\rm iso}(4)$, while the cocycle $\omega$
generates the central extension of $\mfs$ to $\mfn_6$ in the
usual way. From (\ref{fstargweyl}) one may compute the
star-products of coordinate functions on $\real^5$ as
\beq
x_a\star x_b=x_a\,x_b-\ii\mx^\top\left.\frac\d{\d k^a}\frac\d
{\d q^b}(\mk\comp\mq)\right|_{\mk=\mq=\mbf0}-\left.\frac\ii2\,
\frac\d{\d k^a}\frac\d
{\d q^b}\omega(\mk,\mq)\right|_{\mk=\mq=\mbf0} \ .
\label{xastarxbweyl}\eeq
The corresponding star-commutator may thereby be written as
\beq
[x_a,x_b]_\star=\theta\,C_{ab}^{~~c}\,x_c+\theta\,\xi_{ab} \ ,
\label{xaxbstarcommxi}\eeq
where the relation
\beq
\theta\,C_{ab}^{~~c}=-\ii\left.\left(\frac\d{\d k^a}\frac\d
{\d q^b}-\frac\d{\d k^b}\frac\d
{\d q^a}\right)(\mk\comp\mq)^c\right|_{\mk=\mq=\mbf0}
\label{Cabccomp}\eeq
gives the structure constants of the Lie algebra defined by the Lie
group $(\mathcal{S},\comp)$, while the cocycle term
\beq
\theta\,\xi_{ab}=-\frac\ii2\,\left.\left(\frac\d{\d k^a}\frac\d
{\d q^b}-\frac\d{\d k^b}\frac\d
{\d q^a}\right)\omega(\mk,\mq)\right|_{\mk=\mq=\mbf0}
\label{cocycleterm}\eeq
gives the usual form of a central extension of this Lie
algebra. Demanding that this yield a deformation quantization of the
Kirillov-Kostant Poisson structure on $\mfn_6^*$ requires that
$C_{ab}^{~~c}$ coincide with the structure constants of the subalgebra
$\mfs\subset{\rm iso}(4)$ of $\mfn_6$, and also that
$\xi_{\mz\,\overline{\mz}}=-\xi_{\overline{\mz}\,\mz}=2t$ be the only
non-vanishing components of the central extension.

It is thus possible to define a broad class of deformation quantizations of
$\mfn_6^*$ solely in terms of an abstract Weyl system
$(\mcS,\comp,\weyl,\omega)$, without explicit realization of the
operators $\weyl(\mk)$. In the remainder of this subsection we will
set $\Omega=\Delta$ above and describe the Weyl systems
underpinning the various products that we constructed previously. This
entails identifying the appropriate maps (\ref{genweylmapdef}), which
enables the calculation of the projective representations
(\ref{Weylcomprule}) and hence explicit realizations of the group
composition laws $\comp$ in the various instances. This unveils a
purely algebraic description of the star-products which will be
particularly useful for our later constructions, and enables one to
make the equivalences between these products explicit.

\subsubsection{Time Ordering\label{TOPGWS}}

Setting $v=v'=0$ in (\ref{TOgpprodexpl}), we find the ``time-ordered''
non-abelian group composition law $\compa$ for any two elements of the
form (\ref{Svectors}) to be given by
\beq
\mk\compa\mk'=\begin{pmatrix}u+u'\,\\\mz+\e^{-\ii\theta\,u}\,
\mz'\,\\\overline{\mz}+\e^{\ii\theta\,u}\,\overline{\mz}^{\,\prime}
\,\end{pmatrix} \ .
\label{TOcomplaw}\eeq
From (\ref{TOcomplaw}) it is straightforward to compute the inverse
$\underline{\mk}$ of a group element (\ref{Svectors}), satisfying
(\ref{compinverse}), to be
\beq
\underline{\mk}=-\begin{pmatrix}u\\\e^{\ii\theta\,u}\,\mz\\
\e^{-\ii\theta\,u}\,\overline{\mz}\,\end{pmatrix} \ .
\label{TOinverse}\eeq
The group cocycle is given by
\beq
\omega_*(\mk,\mk'\,)=-4\theta~{\rm Im}\left(\e^{-\ii\theta\,u}\,
\overline{\mz}^{\,\top}\,\mz'\,\right)
\label{TOcocycle}\eeq
and it defines the canonical symplectic structure on the $u={\rm
  constant}$ subspaces $\complex^2\subset\mathcal{S}$. Note that in
  this representation, the central coordinate function $t$ is not
  written explicitly and is simply understood as the unit element of
  $\complex(\real^5)$, as is conventional in the case of the Moyal
  product. For $\mk\in\mathcal{S}$ and $\X\in\mfs$ the projective
  representation (\ref{Weylcomprule}) is generated by the time-ordered
  group elements
\beq
\weyl_*(\mk)=\NOa\,\e^{\ii\mk^\top\X}\,\NOa
\label{TOweylop}\eeq
defined in (\ref{eq:time:defn}).

\subsubsection{Symmetric Time Ordering\label{TSOPGWS}}

In a completely analogous manner, inspection of
(\ref{TOsymgpprodexpl}) reveals the ``symmetric time-ordered''
non-abelian group composition law $\compb$ defined by
\beq
\mk\compb\mk'=\begin{pmatrix}u+u'\,\\\e^{\frac{\ii\theta}2\,u'}\,\mz+
\e^{-\frac{\ii\theta}2\,u}\,\mz'\,\\\e^{-\frac{\ii\theta}2\,u'}\,
\overline{\mz}+
\e^{\frac{\ii\theta}2\,u}\,\overline{\mz}^{\,\prime}\,\end{pmatrix} \ ,
\label{TOsymcomplaw}\eeq
for which the inverse $\underline{\mk}$ of a group element
(\ref{Svectors}) is simply given by
\beq
\underline{\mk}=-\mk \ .
\label{TOsyminverse}\eeq
The group cocycle is
\beq
\omega_\bullet(\mk,\mk'\,)=-4\theta~{\rm Im}\left(\e^{-\frac{\ii\theta}2\,
(u+u'\,)}\,\overline{\mz}^\top\,\mz'\,\right)
\label{TOsymcocycle}\eeq
and it again induces the canonical symplectic structure on
$\complex^2\subset\mcS$. The corresponding projective representation
of $(\mcS,\compb)$ is generated by the symmetric time-ordered group
elements
\beq
\weyl_\bullet(\mk)=\NOb\,\e^{\ii\mk^\top\X}\,\NOb
\label{TOsymweylop}\eeq
defined in (\ref{TOsymgpprods}).

\subsubsection{Weyl Ordering\label{WOPGWS}}

Finally, we construct the Weyl system
$(\mcS,\compc,\weyl_\star,\omega_\star)$ associated with the
Weyl-ordered star-product of Section~\ref{WOP}. Starting from
(\ref{Weylgpprodexpl}) we introduce the non-abelian group composition
law $\compc$ by
\beq
\mk\compc\mk'=\begin{pmatrix}u+u'\,\\[2mm]
\frac{\overline{\phi_\theta(u)}\,{\mz}+
\e^{-\ii\theta\,u}\,\overline{\phi_\theta(u'\,)}\,\mz'}
{\overline{\phi_\theta(u+u'\,)}}\\[3mm]
\frac{\phi_\theta(u)\,\overline{\mz}+
\e^{\ii\theta\,u}\,\phi_\theta(u'\,)\,\overline{\mz}^{\,\prime}}
{\phi_\theta(u+u'\,)}\end{pmatrix} \ ,
\label{Weylcomplaw}\eeq
from which we may again straightforwardly compute the inverse
$\underline{\mk}$ of a group element (\ref{Svectors}) simply as
\beq
\underline{\mk}=-\mk \ .
\label{Weylinverse}\eeq
When combined with the definition (\ref{involdef}), one has
$f^\dag=\overline{f}~~\forall f\in\CC^\infty(\real^5)$ and this
explains the hermitean property (\ref{Weylstarherm}) of the
Weyl-ordered star-product $\star$. This is also true of the product
$\bullet$, whereas $*$ is only hermitean with respect to the modified
involution $\dag$ defined by (\ref{involdef})
and~(\ref{TOinverse}). The group cocycle is given by
\bea
\omega_\star(\mk,\mk'\,)&=&4\theta\,\left[\,{\rm Im}\bigl(\,
\overline{\phi_\theta(u)\,\phi_\theta(u'\,)}\,
\overline{\mz}^{\,\top}\mz'\,\bigr)+2\ii\left(\gamma_\theta(u+u'\,)
\,\bigl|\,\overline{\phi_\theta(u)}\,
\mz+\e^{-\ii\theta\,u}\,\overline{\phi_\theta(u'\,)}\,\mz'\,\bigr|^2
\right.\right.\nonumber\\&&\qquad\qquad\qquad\qquad-\left.\left.
\gamma_\theta(u)\,\bigl|\phi_\theta(u)\bigr|^2\,|\mz|^2-
\gamma_\theta(u'\,)\,\bigl|\phi_\theta(u'\,)\bigr|^2\,|\mz'\,|^2
\right)\right] \ .
\label{Weylcocycle}\eea
In contrast to the other cocycles, this does {\it not} induce any
symplectic structure, at least not in the manner described before. The
corresponding projective representation (\ref{Weylcomprule}) is
generated by the completely symmetrized group elements
\beq
\weyl_\star(\mk)=\e^{\ii\mk^\top\X}
\label{Weylweylop}\eeq
with $\mk\in\mcS$ and $\X\in\mfs$.

The Weyl system $(\mcS,\compc,\weyl_\star,\omega_\star)$ can be used
to generate the other Weyl systems that we have found. From
(\ref{1ststudyexpl}) and (\ref{Weylgpprodexpl}) one has the identity
\beq
\weyl_*(u,\mz,\overline{\mz}\,)=\Delta_\star\left(
\e^{\ii(\,\overline{\mz}^{\,\top}q^+
+\mz^\top q^-)}\star\e^{\ii u\,j}\right)
\label{weylTOWeylDelta}\eeq
with $(q^\pm)^\top:=(p_1^\pm,p_2^\pm)\in\complex^2$, which implies
that the time-ordered star-product $*$ can be expressed
by means of a choice of different Weyl system generating the product
$\star$. Since $\Delta_\star$ is an algebra isomorphism, one has
\beq
\weyl_*(u,\mz,\overline{\mz}\,)=\weyl_\star(0,\mz,\overline{\mz}\,\,)
\cdot\weyl_\star(u,\mbf0,\mbf0) \ .
\label{weylTOWeylprods}\eeq
This explicit relationship between the Weyl systems for the
star-products $*$ and $\star$ is another formulation of the statement
of their equivalence, as established by other means in
Section~\ref{WOP}. Similarly, the symmetric time-ordered star-product
$\bullet$ can be expressed in terms of $\star$ through the identity
\beq
\weyl_\bullet(u,\mz,\overline{\mz}\,)=\Delta_\star\left(
\e^{\ii u\,j/2}\star\e^{\ii(\,\overline{\mz}^{\,\top}q^+
+\mz^\top q^-)}\star\e^{\ii u\,j/2}\right) \ ,
\label{WeylTOsymWeylDelta}\eeq
which implies the relationship
\beq
\weyl_\bullet\bigl(u,\mz,\overline{\mz}\,\bigr)=\weyl_\star
\bigl(\mbox{$\frac u2$}
,\mbf0,\mbf0\bigr)\cdot\weyl_\star\bigl(0,\mz,\overline{\mz}\,\,\bigr)
\cdot\weyl_\star\bigl(\mbox{$\frac u2$},\mbf0,\mbf0\bigr)
\label{WeylTOsymWeylprods}\eeq
between the corresponding Weyl systems. This shows explicitly that the
star-products $\bullet$ and $\star$ are also equivalent.

\newsection{Noncommutative Worldvolume Geometries\label{NCWG}}

\subsection{Coadjoint Orbit Quantization\label{DiscrNCGGWS}}

\newsection{Noncommutative D-Branes in NW Spacetimes\label{NCD-BNWS}}

\subsection*{Acknowledgments}

We thank J.~Figueroa-O'Farrill, J.~Gracia-Bondia, P.-M.~Ho, G.~Landi,
F.~Lizzi, R.~Myers, N.~Obers, S.~Philip, V.~Schomerus and K.~Zarembo
for helpful discussions and correspondence. The work of S.H. was
supported in part by the Engineering and Physical Sciences Research
Council~(U.K.). The work of R.J.S. was supported in part by an
Advanced Fellowship from the Particle Physics and Astronomy Research
Council~(U.K.).

\begin{thebibliography}{99}

\baselineskip=12pt

\bibitem{ALZ1} A.~Agostini, F.~Lizzi and A.~Zampini, ``Generalized
  Weyl Systems and $\kappa$-Minkowski Space'', Mod. Phys. Lett. {\bf
  A17} (2002) 2105--2126 [{\tt hep-th/0209174}].
%%CITATION = HEP-TH 0209174;%%

\bibitem{BP1} C.~Bachas and M.~Petropoulos, ``Anti-de~Sitter
  D-Branes'', J. High Energy Phys. {\bf 0102} (2001) 025 [{\tt
  hep-th/0012234}].
%%CITATION = HEP-TH 0012234;%%

\bibitem{BDS1} C.~Bachas, M.R.~Douglas and C.~Schweigert, ``Flux
  Stabilization of D-Branes'', J. High Energy Phys. {\bf 0005} (2000)
  048 [{\tt hep-th/0003037}].
%%CITATION = HEP-TH 0003037;%%

\bibitem{BOL1} M.~Blau and M.~O'Laughlin, ``Homogeneous Plane Waves'',
  Nucl. Phys. {\bf B654} (2003) 135--176 [{\tt hep-th/0212135}].
%%CITATION = HEP-TH 0212135;%%

\bibitem{BFP1} M.~Blau, J.M.~Figueroa-O'Farrill and G.~Papadopoulos,
  ``Penrose Limits, Supergravity and Brane Dynamics'',
  Class. Quant. Grav. {\bf 19} (2002) 4753--4805 [{\tt
  hep-th/0202111}].
%%CITATION = HEP-TH 0202111;%%

\bibitem{BFHP1} M.~Blau, J.M.~Figueroa-O'Farrill, C.~Hull and
  G.~Papadopoulos, ``Penrose Limits and Maximal Supersymmetry'',
  Class. Quant. Grav. {\bf 19} (2002) L87--L95 [{\tt hep-th/0201081}].
%%CITATION = HEP-TH 0209081;%%

\bibitem{Brink1} H.W.~Brinkmann, Proc. Natl. Acad. Sci. (U.S.) {\bf 9}
  (1923) 1.

\bibitem{CW1} M.~Cahen and N.~Wallach, ``Lorentzian Symmetric
  Spaces'', Bull. Am. Math. Soc. {\bf 76} (1970) 585--591.

\bibitem{DAK1} G.~D'Appollonio and E.~Kiritsis, ``String Interactions
  in Gravitational Wave Backgrounds'', Nucl. Phys. {\bf B674} (2003)
  80--170 [{\tt hep-th/0305081}].
%%CITATION = HEP-TH 0305081;%%

\bibitem{Dito1} G.~Dito, ``Kontsevich Star-Product on the Dual of a
  Lie Algebra'', Lett. Math. Phys. {\bf 48} (1999) 307--322 [{\tt
  math.QA/9905080}]
%%CITATION = MATH.QA 9905080;%%

\bibitem{DN1} L.~Dolan and C.R.~Nappi, ``Noncommutativity in a
  Time-Dependent Background'', Phys. Lett. {\bf B551} (2003) 369--377
  [{\tt hep-th/0210030}].
%%CITATION = HEP-TH 0210030;%%

\bibitem{FS1} J.M.~Figueroa-O'Farrill and S.~Stanciu, ``More D-Branes
  in the Nappi-Witten Background'', J. High Energy Phys. {\bf 0001}
  (2000) 024 [{\tt hep-th/9909164}].
%%CITATION = HEP-TH 9909164;%%

\bibitem{Gutt1} S.~Gutt, ``An Explicit $*$-Product on the Cotangent
  Bundle of a Lie Group'', Lett. Math. Phys. {\bf 7} (1983) 249--258.

\bibitem{Guven1} R.~G\"uven, ``Plane Wave Limits and T-Duality'',
  Phys. Lett. {\bf B482} (2000) 255--263 [{\tt hep-th/0005061}].
%%CITATION = HEP-TH 0005061;%%

\bibitem{HY1} P.-M.~Ho and Y.-T. Yeh, ``Noncommutative D-Brane in
  Non-Constant NS-NS $B$-Field Background'', Phys. Rev. Lett. {\bf 85}
  (2000) 5523--5526 [{\tt hep-th/0005159}].
%%CITATION = HEP-TH 0005159;%%

\bibitem{Kath1} V.~Kathotia, ``Kontsevich's Universal Formula for
  Deformation Quantization and the Campbell-Baker-Hausdorff Formula
  I'', {\tt math.QA/9811174}.
%%CITATION = MATH.QA 9811174;%%

\bibitem{LP1} D.~Lerner and R.~Penrose, {\it Techniques of
    Differential Topology in Relativity} (SIAM, Philadelphia, 1971).

\bibitem{NW1} C.R.~Nappi and E.~Witten, ``Wess-Zumino-Witten Model Based on a
  Non-Semisimple Group'', Phys. Rev. Lett. {\bf 71} (1993) 3751--3753
  [{\tt hep-th/9310112}].
%%CITATION = HEP-TH 9310112;%%

\bibitem{Penrose1} R.~Penrose, ``Any Spacetime has a Plane Wave as a
  Limit'', in: {\it Differential Geometry and Relativity} (Reidel,
  Dordrecht, 1976), pp.~271--275.

\bibitem{Rosen1} N.~Rosen, Phys. Z. Sowjetunion {\bf 12} (1937) 366.

\bibitem{Schom1} V.~Schomerus, ``Lectures on Branes in Curved
  Backgrounds'', Class. Quant. Grav. {\bf 19} (2002) 5781--5847 [{\tt
  hep-th/0209241}].
%%CITATION = HEP-TH 0209241;%%

\bibitem{SW1} N.~Seiberg and E.~Witten, ``String Theory and
  Noncommutative Geometry'', J. High Energy Phys. {\bf 9909} (1999)
  032 [{\tt hep-th/9908142}].
%%CITATION = HEP-TH 9908142;%%

\bibitem{Shoik1} B.~Shoikhet, ``On the Kontsevich and the
  Campbell-Baker-Hausdorff Deformation Quantizations of a Linear
  Poisson Structure'', {\tt math.QA/9903036}.
%%CITATION = MATH.QA 9903036;%%

\bibitem{SF1} S.~Stanciu and J.M.~Figueroa-O'Farrill, ``Penrose Limits
  of Lie Branes and a Nappi-Witten Braneworld'', J. High Energy
  Phys. {\bf 0306} (2003) 025 [{\tt hep-th/0303212}].
%%CITATION = HEP-TH 0303212;%%

\end{thebibliography}

\end{document}